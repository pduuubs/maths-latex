\documentclass[a4paper,10pt]{book}
\usepackage[]{paul}

\usepackage{eso-pic} %pour filigrane
\usepackage{minitoc}

\usepackage[marginratio=1:1,textwidth=15cm,textheight=24.5cm,marginparwidth=12pt,headheight=3cm]{geometry}

\usepackage[dvips,ps2pdf,colorlinks=false]{hyperref} %gestions des hyperliens dans un document .dvi

\pagestyle{fancy}
\renewcommand{\thechapter}{\Roman{chapter}}
\renewcommand{\thesection}{\Alph{section}}

\usepackage{etoolbox}
\makeatletter
\patchcmd{\chapter}{\if@openright\cleardoublepage\else\clearpage\fi}{}{}{}

\newcommand{\coordp}[2]{%
	\left(\begin{array}{c}
	#1 \\ #2
	\end{array}\right)%
}
\newsavebox{\fmbox}
\newenvironment{encadrer}{%
\smallskip \par \noindent
    \begin{lrbox}{\fmbox}\begin{minipage}{.98\linewidth}}
    {%\vspace{1em}
    \end{minipage}\end{lrbox}\fbox{\usebox{\fmbox}}
	\vspace{0.2em}
	%\vspace{1em minus .5em }
	}
	\renewcommand{\sectionmark}[1]%
	   {\markboth{\MakeUppercase{\thechapter.\ #1}}{}}
	\renewcommand{\sectionmark}[1]%
	   {\markright{\MakeUppercase{\thesection.\ #1}}}
	\renewcommand{\headrulewidth}{0.5pt}
	\renewcommand{\footrulewidth}{0pt}
	\newcommand{\Police}{%
	   \fontfamily{%
	pag%style des hauts de page en avant garde small caps
	}\fontseries{m}\fontshape{sc}\fontsize{9}{11}\selectfont}
	\fancyhf{}
	\fancyhead[LE,RO]{\Police \thepage}
	\fancyhead[LO]{\Police \rightmark}
	\fancyhead[RE]{\Police \leftmark}
\usepackage{epstopdf}

	\title{%
	{\bf Restitutions organisees des Connaissances en \term}}
	\author{Paul PLANCHON}
	\date{{\small {\Police Version du \today}}}
	\begin{document}
	%%%%%%%%%%%%%%%%%%%%%%%%%%%%%%%%%%%%%%%%%%%%%%%%%%%%%%%%%%%%%
	% 				          DEBUT DU DOCUMENT                       %
	%%%%%%%%%%%%%%%%%%%%%%%%%%%%%%%%%%%%%%%%%%%%%%%%%%%%%%%%%%%%%

	\lfoot{\Pdpweblink}
	\rfoot{{\small {\Police Version du \today}}}

	\maketitle
  \tableofcontents
	\mainmatter

	\chapter{Les suites, rappelles de premiere}
  \section{Generalitees sur les suites}
  \label{sec:Generalitees sur les suites}

	\subsection{Monotonie d'une suite}
	\label{sub:Monotonie d'une suite}


  \begin{proof}
      $U_{n+1} - U_n = U_n(\frac{U_{n+1}}{U_n} - 1)$, donc le signe de $UI_{n+1} - U_n$ est alors le signe de $\frac{U_{n+1}}{U_n} - 1$ :
      \begin{enumerate}
        \item si on a : $\frac{U_{n+1}}{U_n} - 1 \geq 0$, la suite est alors croissante
        \item si on a : $\frac{U_{n+1}}{U_n} - 1 \leq 0$, la suite est alors decroissante
      \end{enumerate}
  \end{proof}

  \subsection{Theoreme de la suite monotone}
  \label{sub:Theoreme de la suite monotone}

  \begin{proof}
    On part de $\forall n \in \mathbb{N}$ et $U_n \in Df \subset \mathbb{R}^+, n \leq n+1$,
    \begin{enumerate}
      \item alors $f(n) \leq f(n+1)$ car $f$ est croissante sur $\mathbb{R}^+$. ($\mathbb{R}^+ \subset \mathbb{N}$), donc $U_n \leq U_{n+1}$ donc la suite est croissante.
      \item alors $f(n) \geq f(n+1)$ car $f$ est decroissante sur $\mathbb{R}^+$. ($\mathbb{R}^+ \subset \mathbb{N}$), donc $U_n \geq U_{n+1}$ donc la suite est decroissante.
    \end{enumerate}
  \end{proof}

  \subsection{Suites minoree et majoree}
  \label{sub:Suites minoree et majoree}

  \begin{proof}
    \begin{enumerate}
      \item Prendre $U_0 \leq U_1 \leq U_2 \leq U_3 \leq ... \leq U_n$, donc $\forall n \in \mathbb{N}, U_n \geq U_0$.
      \item Prendre $U_0 \geq U_1 \geq U_2 \geq U_3 \geq ... \geq U_n$, donc $\forall n \in \mathbb{N}, U_n \leq U_0$.
    \end{enumerate}
  \end{proof}
  \newpage
  \section{Suites arithmetiques et geometriques}
  \label{sec:Suites arithmetiques et geometriques}

  \subsection{Somme des termes d'une suite}
  \label{sub:Somme des termes d'une suite}

  \begin{proof}
    (par recurrence)\\
    \begin{enumerate}
      \item Methode 1 :
      \begin{enumerate}
        \item Initialisation : \\pour $n=0, U_n + nr = U_0 + 0r = U_0$ donc vrai pour $U_0$.
        \item Heredite :
          \begin{description}
            \item On suppose qu'il $\forall k \in \mathbb{N}, U_k = U_0 +kr$ (HR).
            \item $U_{n+1} = U_k + r$
            \item $U_{n+1} = (U_0 + kr) + r$
            \item $U_{n+1} = U_0 + (k+1)r$
          \end{description}
        \item donc vrai pour $n+1$.
      \end{enumerate}
      \item Methode 2 :
      \begin{enumerate}
        \item Initialisation : \\pour $\sum_{k=0}^{0} U_k = U_0$, et le terme de droite prend $\frac{(0+1)(U_0 + U_0)}{2} = \frac{2U_0}{2} = U_0$. Donc vrai pour n = 0.
        \item Heredite :
          \begin{description}
            \item On suppose qu'il $\forall p \in \mathbb{N}, \sum_{p=0}^{0} U_p = \frac{(p+1)(U_0 + U_p)}{2}$ (HR).
            \item $S_n = U_0 + U_1 + ... + U_p + U_{p+1}$
            \item $S_n = \frac{(p+1)(U_0 + U_p)}{2} + U_{p+1}$
            \item $S_n = \frac{(p+1)(U_0 + U_{p+1} - r)}{2} + U_{p+1}$
            \item $S_n = \frac{(p+1)(U_0 + U_{p+1} - r)(2)(U_{p+1})}{2}$
            \item $S_n = \frac{(p+1)U_0 + (p+3)(U_{p+1} -(p+1))}{2}$
            \item $S_n = \frac{(p+2)U_0 - U_0 + (p+2)U_{p+1} -(p+1) + U_{p+1}}{2}$
            \item $S_n = \frac{(p+2)U_0 + (p+2)U_{p+1} + U_{p+1} - U_0 -(p+1)}{2}$
            \item $S_n = \frac{(p+2)(U_0 + U_p+1)}{2}$
          \end{description}
        \item donc vrai pour $n+1$.
      \end{enumerate}
    \end{enumerate}
    \end{proof}

    \newpage
  \subsection{Definition de la suite geometrique}
  \label{sub:Definition de la suite geometrique}

  \begin{proof}
    \textit{(par recurrence):}\\
    \begin{enumerate}
      \item Methode A:
        \begin{enumerate}
          \item Init : pour $n=0$, $U_0 * q^n = U_0 * q^0 = U_0 * 1 = U_0$, donc vrai pour $n=0$.
          \item Heredite :
            \begin{description}
              \item On suppose que $\forall p \in \mathbb{N} / U_p = U_0 * q^p$
              \item alors, $U_{p+1} = q * U_p = q * U_0 * q^p = U_0 * q^{p+1}$, vrai pour $p+1$.
            \end{description}
        \end{enumerate}
      \item Methode B:
        \begin{enumerate}
          \item Initialisation : pour $k=0, \sum_{k=0}^{p} U_k = U_0 * \frac{1-q^{p+1}}{1-q} = U_0 * 1 = U_0$ donc vrai pour $k=0$
          \item Heredite :
            \begin{description}
              \item On suppose $\forall p \in \mathbb{N}, \sum_{k=0}^{p} = U_0 * \frac{1-q^{p+1}}{1-q}$
              \item or, $U_0 + U_1 + ... U_n + U_{n+1}$
              \item $ = U_0 * \frac{1-q^{p+1}}{1-q} + U_{p+1}$
              \item $ = U_0 * (\frac{1-q^{p+1}}{1-q}) + U_0 * q^{p+1}$
              \item $ = U_0[\frac{1-q^{p+1}}{1-q} + q^{p+1}]$
              \item $ = U_0[\frac{1-q^{p+1} + (1-q)(q^{p+1})}{1-q}]$
              \item $ = U_0[\frac{1-q^{p+1} + q^{p+1} + q^{p+2}}{1-q}]$
              \item $ = U_0[\frac{1-q^{p+2}}{1-q}]$
              \item $ = U_0[\frac{1-q^{(p+1)+1}}{1-q}]$
              \item donc vrai pour $p+1$.
            \end{description}
        \end{enumerate}
      \end{enumerate}
  \end{proof}

  \subsection{Formule importantes des suites}
  \label{sub:Formule importantes des suites}

  \begin{proof}
    Prenons $(U_n)$, une suite geo. avec $U_0 = 1$ et de raison $q$ alors
    \begin{description}
      \item $\forall k \in \mathbb{N}, U_k = U_0 * q^k = q^k$
      \item $U_0 + U_1 + U_2 + ... U_n = 1 + q + q^2 + ... + q^{n-1} + q^n$
      \item $ = U_0 * (\frac{1-q^{n+1}}{1-q})$
      \item $1*(\frac{1-q^{n+1}}{1-q}) = \frac{1-q^{n+1}}{1-q}$
    \end{description}
  \end{proof}

  %%%%%%%%%%%%%%%%%%%%%%%%%%%%%%%%%%%%%%%%%%%%%%%%%%%%%%%%%%%%%
  %                  FIN DU CHAPITRE 1                        %
  %%%%%%%%%%%%%%%%%%%%%%%%%%%%%%%%%%%%%%%%%%%%%%%%%%%%%%%%%%%%%

  \chapter{Convergence et divergence d'une suite}
  \label{chap:Convergence et divergence d'une suite}

  \section{Convergence d'une suite}
  \label{sec:Convergence d'une suite}

  \subsection{Suite convergente vers $l$}
  \label{sub:Suite convergente vers l}

  \begin{proof}
    Suppossons qu'il existe un $n_0 \in \mathbb{N} / u_{n_0} > l$\\
    Comme la suite est croissante, on est assure que $\forall n \geq n_0$, on aura, $u_n \geq u_{n_0} \geq l$ puis la suite est convergente vers $l$ donc :\\
    $(\forall \varepsilon > 0)(\exists n_1 \in \mathbb{N}) / (\forall n \geq n_1, n \in \mathbb{N})$,
    \begin{enumerate}
      \item $|u_n -l| < \varepsilon $
      \item $- \varepsilon < u_n - l < \varepsilon $
      \item $-\varepsilon + l < u_n < \varepsilon + l$\\
    \end{enumerate}
    Notons $n_2$ le maximum de ces deux nombres, c'est a dire : $n_2 > n_0$ et $n_2 > 1$\\
    alors pour $n \geq n_2$ on aura en simultane,
    \begin{enumerate}
      \item $u_n \geq u_{n_0} \geq l$
      \item $l - \varepsilon < u_n < l + \varepsilon$\\
    \end{enumerate}

    Prenons alors $\varepsilon = \frac{u_{n_0} - l}{3}$, remarquons que $u_{n_0} > l$ donc $u_{n_0} - l > 0$, donc $\frac{u_{n_0} - l}{3} > 0$.\\
    Alors, on aura $\forall n \geq n_2$ :
    \begin{enumerate}
      \item $l \leq u_{n_0} \leq u_n \leq l + \frac{u_{n_0} - l}{3}$, donc par transitivite,
      \item $u_{n_0} < l \frac{u_{n_0} - l}{3} \Leftrightarrow u_{n_0} - l < \frac{u_{n_0} - l}{3}$
      \item $3(u_{n_0} - l) < u_{n_0} - l \Leftrightarrow 2(u_{n_0} - l) < 0$ or $2 > 0$ donc,
      \item $u_{n_0} - l < 0 \Leftrightarrow u_{n_0} < l$ ceci est absurde.
    \end{enumerate}

    La supposition faite est donc fausse.
  \end{proof}

  \subsection{Suite constante et convergence}
  \label{subs:Suite constante et convergence}

  \begin{proof}
    $\exists \varepsilon / \forall n \in \mathbb{N}, u_n = k$.\\
    On veut montrer que cette suite est convergente vers $k$. On prend $\varepsilon > 0$ au hasard et on cherche $n_0$ / $n \in \mathbb{N}$ avec $n \geq n_0$, alors, $|u_n - k| < \varepsilon$.\\
    ici, $|u_n - k| < \varepsilon \Leftrightarrow |c - c| < \varepsilon \Leftrightarrow 0 < \varepsilon$, vrai. \\
    Donc, n'importe quel entier jouera le role de $n_0$
  \end{proof}

  \newpage

  \subsection{$u_n < v_n$ donc $l < l'$}
  \label{sub:$u_n < v_n$ donc $l < l'$}

  \begin{proof}
    On suppose que $l > l'$.\\
    Alors, prenons $\varepsilon = \frac{l - l'}{3}$, (comme $l > l'$, $l-l'<0$ donc $\frac{l - l'}{3} > 0$)\\
    Pour cet $\varepsilon$, $\exists n_1 \in \mathbb{N}$ / $\forall n \in \mathbb{N}, n \geq n_1$, $|u_n - l| < \frac{l - l'}{3}$.
    \begin{description}
      \item $\Leftrightarrow -\frac{l - l'}{3} < u_n - l < \frac{l - l'}{3}$
      \item $\Leftrightarrow l-\frac{l - l'}{3} < u_n < \frac{l - l'}{3} + l$
      \item $\Leftrightarrow -\frac{2l - l'}{3} < u_n < \frac{4l - l'}{3}$
    \end{description}

    pour cet $\varepsilon$,  $\exists n_2 \in \mathbb{N}$ / $\forall n \in \mathbb{N}, n \geq n_2$, $|v_n - l| < \frac{l - l'}{3}$.
    \begin{description}
      \item $\Leftrightarrow -\frac{l - l'}{3} < v_n - l' < \frac{l - l'}{3}$
      \item $\Leftrightarrow l'-\frac{l - l'}{3} < v_n < \frac{l - l'}{3}+l'$
      \item $\Leftrightarrow \frac{l - 4l'}{3} < v_n < \frac{l + 2l'}{3}$
    \end{description}

    Notons, $n_3 =$ Max($n_0; n_1; n_2$),\\
    donc, $u_3 \geq n_0$, $n_3 \geq n_1$ et $n_3 \geq n_2$. On aura donc en simulane, $n \geq n_3$
    \begin{description}
      \item $\Leftrightarrow u_n \leq v_n$
      \item $\Leftrightarrow -\frac{2l - l'}{3} < u_n < \frac{4l - l'}{3}$
      \item $\Leftrightarrow \frac{l - 4l'}{3} < v_n < \frac{l + 2l'}{3}$
    \end{description}

    donc, $\forall n \geq n_3$, on aura, (entre autre),
    \begin{description}
       \item $\Leftrightarrow -\frac{2l - l'}{3} < u_n \leq v_n< \frac{l + 2l'}{3}$
       \item $\Leftrightarrow -\frac{2l - l'}{3} < \frac{l + 2l'}{3}$
       \item $\Leftrightarrow 2l - l' < l + 2l'$
       \item donc, $l < l'$ absurde.
    \end{description}
    La supposition faite est donc fausse. On a donc bien $l \leq l'$
  \end{proof}






	\end{document}



\documentclass[a4paper]{article}
\usepackage[]{pan}

\begin{document}
\activite{Repr\'esentation param\'etrique de droite}{\term}
\cfoot{} \lfoot{}
\section{Mobile dans le plan}
On se place dans le plan muni d'un rep\`ere orthonorm\'e $\oij$. Un
point $M$ mobile en fonction du temps $t$  est rep\'er\'e par ses
coordonn\'ees variables en fonction du temps, not\'ees $x(t)$ et
$y(t)$. On a donc: $M\left( x(t)\,;y(t)\right)$. On suppose que ces
coordonn\'ees v\'erifient:
\begin{equation}
  \label{eq:para}
  \left\{
\begin{array}{ll}
x(t)=2t+3\\
y(t)=-t+\frac12
\end{array}
\right.
\end{equation}

\begin{enumerate}
\item Prouver que $M$ appartient \`a une droite $\Delta$ dont on
  donnera une \'equation r\'eduite et que l'on tracera.

\item Et $t$ il varie dans quel ensemble?
  \begin{enumerate}
  \item Prouver que si $t$ d\'ecrit $\R$ alors $M$ d\'ecrit tout
    $\Delta$.

  \item Si $t\in\intfo{0}{+\infty}$ d\'efinir et colorier la partie du
    plan d\'ecrite par~$M$.
  \item Si $t\in\intfo{0}{1}$ d\'efinir et colorier (d'une autre
    couleur!) la partie du plan d\'ecrite par~$M$.
  \end{enumerate}

\item Un second point mobile, not\'e $N$ arrive... ses coordonn\'ees
  v\'erifient pour $t\in\R$:
\begin{equation}
  \label{eq:para2}
  \left\{
\begin{array}{ll}
x(t)=5-\frac{t}{3}\\
y(t)=-\frac12+\frac{t}{6}
\end{array}
\right.
\end{equation}

\begin{enumerate}
\item Prouver que $N$ \'evolue aussi sur $\Delta$
\item $M$ et $N$ vont-ils se rencontrer? Si oui, quand et o\`u ?
\end{enumerate}


\item V\'erifier que la relation \eqref{eq:para} est \'equivalente
  \`a: $\V{AM}=t\vu$ o\`u $A$ et $\vu$ sont respectivement un point et
  un vecteur dont on donnera les  coordonn\'ees.

\item M\^eme question pour: \eqref{eq:para2} $\iff \V{BN}=t\vv$

\item Que repr\'esentent $\vu$ et $\vv$ pour $\Delta$? D\'eterminer un
  r\'eel $k$ tel que: $\vu=k\vv$.

\item \textbf{Conclusion:} On dit que \eqref{eq:para} et
  \eqref{eq:para2} (pour $t\in\R$) sont des repr\'esentations
  param\'etriques de la droite~$\Delta$. \\
\textbf{G\'en\'eralisation:} Soit $A$
  un point d'une droite $Dr$ et $\vu$ un vecteur directeur
  de~$\Dr$. On a l'\'equivalence: $M\in\Dr$ ssi il existe un
  r\'eel $t$ tel que $\V{AM}=t\vu$. En d\'eduire une repr\'esentation
  param\'etrique de $\Dr$ en fonction des coordonn\'ees de $A$ et de~$\vu$.
\end{enumerate}
\section{Droites de l'espace}
On se place maintenant dans l'espace \`a trois dimensions ramen\'e \`a
un rep\`ere orthonormal~$\oijk$. Le m\^eme principe s'applique et on a
donc une repr\'esentation param\'etrique avec trois \'equations
$(x=\ldots,\ y=\ldots,\ z=\ldots)$.

\exercice{Soit les points: $A(1;-2;3)$ et $B(0;0;1)$.
  \begin{enumerate}
  \item D\'eterminer une repr\'esentation param\'etrique de $(AB)$\label{a}
  \item Les points $C(-3;6;-5)$ et $D(2;-5;5)$ appartiennent-ils \label{b}
    \`a~$(AB)$?

  \item Soit $E(-1;3;-1)$ Prouver $(DE)$ et $(AB)$ sont s\'ecantes en
    un point $I$ dont on d\'eterminera les coordonn\'ees. \label{c}

  \end{enumerate}
\hfill \rotatebox{180}{\small{\textsl{Solutions:} \textbf{\ref{a}.} $x=1-t$, $y-2+2t$, $z=3-2t$ avec
  $t\in\R$ \textbf{\ref{b}.} $C\in(AB)$ et $D\notin(AB)$
  \textbf{\ref{c}.} $I(\frac12;-1;2)$}}
}
\exercice{Soit le point: $A(-2;0;3)$ et la droite
    $\Dr$ dont une repr\'esentation param\'etrique est:
\[ 
\left\{
\begin{array}{ll}
x=2+\frac{t}{2}\\
y=-2t\\
z=-3t+\frac{1}{2}
\end{array}
\right.\qquad ,t\in\R
\]
  D\'eterminer une repr\'esentation param\'etrique de la droite
    $\Dr'$ telle que: $\Dr' // \Dr$ et $A\in\Dr'$

}

\end{document}

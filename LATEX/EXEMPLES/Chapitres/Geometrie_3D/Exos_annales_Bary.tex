

\documentclass[a4paper]{article}
\usepackage[]{pan}

\begin{document}
\fexos{Barycentre (Annales Bac)}{\term S}
\lfoot{}
\cfoot{}
%%-----------------------


\textbf{\large Antilles--Guyane juin 2004}

\vspace{0,25cm}

On consid\`ere un t\'etra\`edre ABCD; on note I milieu du segment [AB] et J celui
 de [CD].

\begin{enumerate} \item

\begin{enumerate} \item Soit G$_{1}$ le barycentre du syst\`eme de points pond\'er\'es 

 \[\{(\text{A},~ 1)~;~ (\text{B},~ 1)~;~ (\text{C},~-1)~;~ (\text{D},~1)\}\]

Exprimez $\V{\text{IG}_{1}}$ en fonction de $\V{\text{CD}}$. Placez I, J et G$_{1}$ sur une figure. 

\item Soit G$_{2}$ le barycentre du syst\`eme de points pond\'er\'es 
$\{(A,~ 1)~;~ (B,~ 1)~ ;~ (D,~ 2)\}$.

D\'emontrez que G$_{2}$ est le milieu du segment [ID]. Placez G$_{2}$.

\item D\'emontrez que IG$_{1}$DJ est un parall\'elogramme.

En d\'eduire la position de G$_{2}$ par rapport aux points G$_{1}$ et J.

\end{enumerate}

\item Soit $m$ un r\'eel. On note $G_{m}$ le barycentre du syst\`eme de points pond\'er\'es
\[\{(A,~ 1)~;~(B,~ 1)~;~(C,~m-2)~;~(D,~m)\}\]

\begin{enumerate} \item Pr\'ecisez l'ensemble $\mathcal{E}$ des valeurs de $m$ pour lesquelles le barycentre 
$G_{m}$ existe.

Dans les questions qui suivent, on suppose que le r\'eel $m$ appartient \`a l'ensemble
$\mathcal{E}$.

\item D\'emontrez que $G_{m}$, appartient au plan (ICD).

\item D\'emontrez que le vecteur $m\V{\text{J}G_{m}}$ est constant.

\item En d\'eduire l'ensemble $\mathcal{F}$ des points $G_{m}$ lorsque $m$ d\'ecrit
l'ensemble $\mathcal{E}$.

\end{enumerate}
 
\end{enumerate}

%%---------------------------
\vspace{0,25cm}
\textbf{\large  Pondich\'ery  juin 1999}

\vspace{0,25cm}

On consid\`ere un triangle ABC du plan. 

\begin{enumerate} \item \begin{enumerate} \item D\'eterminer et construire le point G, barycentre de $\{(\text{A} ~;~ 1)~ ;~ (\text{B}~ ;~ -~ 1)~;~ (\text{C}~ ;~ 
1)\}$

\item D\'eterminer et construire le point G$'$, barycentre de $\{(\text{A}~;~ 1)~ ;~ (\text{B}~ ;~ 5)~ ;~ (\text{C}~ ;~ -~ 
2)\}.$

\end{enumerate} 

\item \begin{enumerate} \item Soit J le milieu de [AB]. 

Exprimer $\V{\text{GG}'}$ et $\V{\text{JG}'}$ en 
fonction de $\V{\text{AB}}$ et $\V{\text{AC}}$ et 
en d\'eduire l'intersection des droites (CG$'$) et (AB). 

\item Montrer que le barycentre I de $\{$(B~;~2)~ ;~ (C~;~$-1)\}$ appartient \`a 
(GG$'$). 
\end{enumerate} 
 Soit D un point quelconque du plan. Soient O le milieu 
de [CD] et K le milieu de [GA]. 



\item D\'eterminer trois r\'eels $a,~ d$ et $c$ tels que K soit barycentre de $\{(\text{A}~;~ a)~ ;~ (\text{D}~ ;~ d)~ ;~ (\text{C}~;~ c)\}.$

\item Soit X le point d'intersection de (DK) et (AC). 

D\'eterminer les r\'eels $a'$ et $c'$ tels que X soit barycentre de $\{(\text{A}~;~ a')~ ;~ (\text{C}~ ;~ c')\}.$

\end{enumerate} 

%%--------

\vspace{0,25cm}
\textbf{\large Am\'erique du Sud   novembre 1998}

\vspace{0,25cm}

Dans le plan (P), on consid\`ere le triangle ABC isoc\`ele en A, de 
hauteur [AH] tel que AH=BC=4. On prendra le centim\`etre pour 
unit\'e. 

\begin{enumerate} \item En justifiant la construction, placer le 
point G, barycentre du syst\`eme de points pond\'er\'es \{(A ; 2) ; (B ; 
1) ; (C ; 1)\}. 

\item On d\'esigne le point $M$ un point quelconque 
de (P). 

\begin{enumerate} \item Montrer que le vecteur $\vec{V} = 2\V{M\text{A}} - 
\V{M\text{B}} - \V{M\text{C}}$ est un 
vecteur dont la norme est 8. 

\item D\'eterminer et construire l'ensemble E$_{1}$ des points $M$ du plan 
tels que 

\[\left\|2\V{M\text{A}} - \V{M\text{B}} - \V{M\text{C}}\right\| = \left\|\vec{V}\right\|\] 

\end{enumerate} 

\item On consid\`ere le syst\`eme de points pond\'er\'es 
\{(A ; 2) ; (B ; $n$) ; (C ; $n$)\} o\`u $n$ est un entier naturel 
fix\'e. 

\begin{enumerate} \item Montrer que le barycentre G$_{n}$ de ce syst\`eme de points 
pond\'er\'es existe. Placer G$_{0}$,~G$_{1}$,~G$_{2}$. 

\item Montrer que le point G$_{n}$ appartient au segment [AH]. 

\item Calculer la distance AG$_{n}$ en fonction de $n$ et d\'eterminer la 
limite de AG$_{n}$ quand $n$ tend vers +~$\infty$. 

Pr\'eciser la position limite de G$_{n}$ quand $n$ tend vers +~$\infty$. 

\item Soit E$_{n}$ l'ensemble des points $M$ du plan tels que 
\[\left\|2\V{M\text{A}} + n \V{M\text{B}} + 
n\V{M\text{C}}\right\| = n\left\|\V{V}\right\|.\] 

Montrer que E$_{n}$ est un cercle qui passe par le point A. 

En pr\'eciser le centre et le rayon, not\'e R$_{n}$. 

\item Construire E$_{2}$. 

\end{enumerate} 

\end{enumerate} 
%%%66666666666666666666666666666666666666666666666666666666
\end{document}



\documentclass[a4paper]{article}
\usepackage[]{pan}
\usepackage{pst-math}
\usepackage{pst-plot}
\begin{document}
\fexos{Applications du produit scalaire}{\term S}
%%%---------------------------------------
\exercice
Soit $A(2;1;-1)$. Donner des �quations des plans plans passant par $A$
qui sont parall�les aux plans de coordonn�es. (\ie{} $(xOy)$, $(xOz)$, $(yOz)$)
%%%---------------------------------------
\exercice
Soit une droite $\Dr$ de vecteur directeur $\vu(1;-2;3)$ et la droite
$(AB)$ o� $A(-1;0;2)$ et $B(-3;2;4)$. Prouver que $\Dr$ et $(AB)$ sont orthogonales.
%%%---------------------------------------
\exercice
Soit $\Para$ le plan passant par $B(-1;2;-3)$ et perpendiculaire � la
droite $\Dr$ dont une param�trisation est:
\[\left\{
\begin{array}{ll}
  x=3t-1\\
y=-t \\
z=2t+5
\end{array}\right.\qquad t\in\R\]
\begin{enumerate}
\item Calculer la distance du point $A(0;1;-1)$ au plan $\Para$.
\item Calculer les coordonn�es du point d'intersection de $\Dr$ et $\Para$.
\end{enumerate}

%%%---------------------------------------
\exercice
Soit les plans $\Para$ et $\Para'$ d'�quations respectives:
$x+y+z+2=0$ et $2x-y-2z+1=0$.
\begin{enumerate}
\item Ces plans sont ils s�cants?
\item D�terminer le point $A$ d'abscisse nulle appartenant � $\Para$
  et $\Para'$.
\item D�terminer un autre point $B$ appartenant �
  $\Para\cap\Para'$.

\item Donner une param�trisation de la droite $\Dr=\Para\cap\Para'$.
\end{enumerate}
%%%---------------------------------------
\exercice
Dans un t�tra�dre r�gulier $ABCD$, on consid�re le centre de gravit�
$G$ de la face $ABC$. On veut prouver que $(DG)$ est orthogonale �
$(ABC)$. On note $A'$ le milieu de $[BC]$.
\begin{enumerate}
\item%
  \begin{enumerate}
  \item Justifier rapidement que $(GA')$ et $(A'D)$ sont perpendiculaires � $(BC)$.
  \item En d�duire que $\V{GD}\cdot\V{BC}=0$.
  \end{enumerate}
\item On pourrait prouver de la m�me mani�re que $(GD)$ est
  orthogonale � $(AB)$ mais je vous propose une autre m�thode:
  \begin{enumerate}
  \item Justifier que $(GD)$ est dans le plan m�diateur de $[AB]$.
  \item Conclure.
  \end{enumerate}
\item Conclure.
\end{enumerate}
%%%---------------------------------------
\begin{center}
  \input{Figures/tetraedre.psf}
\includegraphics[width=4cm]{Figures/tetraedre}
\end{center}

\exercice
Dans l'espace muni d'un rep�re orthonorm�, montrer qu'il
existe une unique sph�re passant par les points
$$A(4;5;1)\ B(2;5;-5)\ C(-4;5;1)\ D(4;3;3)$$
On d�terminera le centre et le rayon de cette sph�re.
\\

M�thode: Montrer que les plans m�diateurs de $[AB]$, $[AC]$ et
$[AD]$ ont un point commun apr�s en avoir d�termin� des �quations
cart�siennes. 



\end{document}


\begin{pspicture}*(-7,-4)(7,4)
                          \psaxes{->}(0,0)(-6,-4)(7,4)                         
                          \psplot[linecolor=gray]{-7}{7}{x TAN }
                          \end{pspicture}


 

\begin{pspicture}*(-5,-5)(5,5)
                          \psaxes{->}(0,0)(-5,-5)(5,5)
                          \psplot[linecolor=gray]{-5}{5}{x COSH }
                          \psplot[linecolor=red]{-5}{5}{x SINH }
                          \psplot[linecolor=green]{-5}{5}{x TANH }
                          \end{pspicture}


\begin{pspicture}*(-5,-5)(5,5)
                          \psaxes{->}(0,0)(-5,-5)(5,5)
                          \psgrid[griddots=1,%
                          subgriddiv=2,
                          gridlabels=0pt,%
                          xunit=1]
                          (-5,-5)(5,5)
                          \end{pspicture}
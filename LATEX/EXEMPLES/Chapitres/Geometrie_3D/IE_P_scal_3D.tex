

\documentclass[a4paper]{article}
\usepackage[]{pan}
\usepackage{fancybox}
\usepackage{tabularx}
\usepackage{ulem}
\usepackage{dcolumn}
\usepackage{textcomp}
\usepackage{pstricks,pst-plot,pst-text,pst-tree}
\newcommand{\vect}[1]{\mathchoice%
{\overrightarrow{\displaystyle\mathstrut#1\,\,}}%
{\overrightarrow{\textstyle\mathstrut#1\,\,}}%
{\overrightarrow{\scriptstyle\mathstrut#1\,\,}}%
{\overrightarrow{\scriptscriptstyle\mathstrut#1\,\,}}}
\begin{document}
\ds{\term S}
\lhead{\textbf{Nom:}}

%\textbf{\textsc{Exercice 3} \hfill 4 points}\\
%\textbf{Commun \`a tous les candidats}\\
Ce devoir comporte deux exercices. Dans le premier  une r\'eponse par
\og VRAI \fg{} ou \og FAUX \fg{}, sans justification, est demand\'ee
au candidat en regard d'une liste d'affirmations. Toute r\'eponse
conforme \`a la r\'ealit\'e math\'ematique donne 1,5 points. Toute
r\'eponse erron\'ee enl\`eve 0,5 point. Dans le deuxi�me exercice, on
demande d'inscrire la r�ponse sur les pointill�s. Une r�ponse correcte
rapporte alors 2,5 points et une r�ponse incorrecte enl�ve 0,5
point. Dans les deux exercices, l'absence de r\'eponse n'est pas
comptabilis\'ee. Le total des points � ce devoir ne saurait  \^etre n\'egatif.

\vspace{0,5cm}

\noindent\parbox{0,5\textwidth}{On donne le cube ABCDEFGH, d'ar\^ete de longueur 1, et les milieux I et J des ar\^etes [AB] et [CG]. Les \'el\'ements utiles de la figure sont donn\'es ci-contre.
 } \hfill
\parbox{0,45\textwidth}{\begin{pspicture}(4.5,5)
\pspolygon(0.4,1.05)(2.7,0.4)(4.4,2.2)(2,2.8)%ABCD
\pspolygon(0,3.2)(2.4,2.6)(4,4.4)(1.7,5)%EFGH
\pspolygon(0,3.2)(1.55,0.725)(4.2,3.3)%EIJ
\psline(0.4,1.05)(0,3.2) \psline(2.7,0.4)(2.4,2.6) \psline(4.4,2.2)(4,4.4)
\psline(2,2.8)(1.7,5)

\put(0.2,0.8){A}  
\put(2.5,0.05){B}  
\put(4.5,2.2){C}  
\put(1.65,2.8){D}  
\put(-0.3,3){E}  
\put(2.1,2.25){F}  
\put(4,4.5){G}  
\put(1.7,5.15){H}  
\put(1.35,0.375){I}  
\put(4.35,3.3){J}  
\end{pspicture} }

%\vspace{0,5cm}
\exods{}{7,5}
%\noindent \textbf{On utilisera pour r\'epondre la feuille annexe, qui sera rendue avec la copie.}
\begin{center}\begin{tabularx}{\linewidth}{|c|X|c|}\hline
&Affirmation&	VRAI ou FAUX\\\hline\hline
\textbf{1.}&$\vect{\text{AC}}\cdot \vect{\text{AI}} = \dfrac{1}{2}$&\rule[-3mm]{0mm}{8mm} \\ \hline
\textbf{2.}&$\vect{\text{AC}}\cdot \vect{\text{AI}} = \vect{\text{AI}}\cdot \vect{\text{AB}}$&\rule[-3mm]{0mm}{8mm}\\ \hline
\textbf{3.}&$\vect{\text{AB}}\cdot \vect{\text{IJ}} = \vect{\text{AB}}\cdot \vect{\text{IC}}$&\rule[-3mm]{0mm}{8mm} \\ \hline
\textbf{4.}&$\vect{\text{AB}}\cdot \vect{\text{IJ}} = \text{AB} \times
\text{IC} \times \cos \dfrac{\pi}{3}$\hspace{4,3cm}
&\rule[-3mm]{0mm}{8mm} \\ \hline
\textbf{5.}&(HI)$\bot$(AB)\hspace{4,3cm} &\rule[-3mm]{0mm}{8mm} \\ \hline
\end{tabularx}
%\vspace{0,5cm}
\exods{}{12,5}
\noindent On utilise \`a pr\'esent le rep\`ere orthonormal $\left(\text{A}~;~\vect{\text{AB}},~\vect{\text{AD}},~\vect{\text{AE}}\right)$.

\begin{tabularx}{\linewidth}{|c|X|}\hline
&R�ponse � compl�ter\\\hline\hline
\textbf{6.}&\rule[-6.5mm]{0mm}{15mm}Donner une repr\'esentation param\'etrique de la droite (IJ).
$\left\{\begin{array}{l c l}
x&=&\qquad\qquad\\
y& =& \\
z&=&\\
\end{array}\right. \qquad ,t\in\R$ \\ \hline
\textbf{7.}& \rule[-3mm]{0mm}{8mm}Les plans d'�quation $z=1$ et $x=0$ sont s�cants selon la droite$\phantom{\displaystyle \sum}$\dotfill \\\hline
\textbf{8.}&\rule[-3mm]{0mm}{8mm}Donner une \'equation cart\'esienne du plan passant par F
et dont $\vect{\text{EI}}$ est un vecteur normal.\\
&\rule[-3mm]{0mm}{8mm}\dotfill \\ 
 \hline
\textbf{9.}&\rule[-3mm]{0mm}{8mm} 
%L'intersection des plans (FIJ) et (ABC) est elle la droite   passant
%par I et par le milieu de l'ar\^ete [DC]? 
Parmi les droites (AC), (BD), (AG) et (EC) laquelle est parrall�le au
plan d'�quation: $\ x+y+z-2=0$ \dotfill\\ \hline
\textbf{10.}&\rule[-6.5mm]{0mm}{15mm}Le vecteur de coordonn\'ees
$\left(\begin{array}{l}-4\\ 1\\2\\ \end{array}\right)$ 	est-il un vecteur 
 normal  au plan (FIJ)? \dotfill \\ \hline
%\textbf{11.}&	 \\ \hline
\end{tabularx}
\end{center}


\end{document}

% Pour mettre un filigrane Obligatoire

% \begin{center}
% Session 2007\\
% \raisebox{-6cm}{{\LARGE Mathematiques}}\\
% {\Large S�rie: S}\\
% \end{center}
% \fontsize{100}{100pt}\selectfont
% \framebox[1cm][l]{
% \rotatebox{30}{\makebox[1.1\textwidth][s]{{\bf{\gray O
%           b l i g a t o i r e}}}}%
% }
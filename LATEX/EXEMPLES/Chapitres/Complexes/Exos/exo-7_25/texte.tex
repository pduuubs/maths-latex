\begin{exercice}
\noindent On consid\`ere le plan complexe $\mathcal{P}$ rapport\'e \`a un rep\`ere orthononnal direct $\ouv$. Dans tout l'exercice, $\mathcal{P}\verb+\+ \{\text{O}\}$ d\'esigne le plan $\mathcal{P}$ priv\'e du point origine O.
\begin{enumerate}
%%%%%%%%%%%%%%% Question de cours qu'on peut zapper
\begin{comment}
\item  \textbf{Question de cours}\\
On prend comme pr\'e-requis les r\'esultats suivants :
\begin{itemize}
\item  Si $z$ et $z'$ sont deux nombres complexes non nuls, alors : arg$(zz') = \text{arg}(z) + \text{arg}(z')$ \`a $2k\pi$ pr\`es, avec $k$ entier relatif
\item Pour tout vecteur $\vect{w}$ non nul d'affixe $z$ on a : arg$(z) = \left(\vect{u}~;~\vect{w}\right)$ \`a $2k\pi$ pr\`es, avec $k$ entier relatif
\end{itemize}
\begin{enumerate}
\item  Soit $z$ et $z'$ des nombres complexes non nuls, d\'emontrer que\\ arg$\left(\dfrac{z}{z'}\right) = \text{arg}(z)- \text{arg}(z')$ \`a $2k\pi$
pr\`es, avec $k$ entier relatif.
\item D\'emontrer que si A, B, C sont trois points du plan, deux \`a deux distincts, d'affixes respectives $a,~ b,~ c$, on a : arg$\left(\dfrac{c - a}{b - a}\right)  = \left(\vect{\text{AB}},~ \vect{\text{AC}}\right)$ \`a $2k\pi$ pr\`es, avec $k$ entier relatif.
\end{enumerate}
\end{comment}
%%%%%%%%%%%%%%%%%%%%%%%%  END ZAP

\item On consid\`ere l'application $f$ de $\mathcal{P}\verb+\+ \{\text{O}\}$ dans $\mathcal{P}\verb+\+ \{\text{O}\}$ qui, au point $M$ du plan d'affixe $z$, associe le point $M'$ d'affixe $z'$ d\'efinie par :  $z'= \dfrac{1}{\overline{z}}$. On appelle U et V les points du plan d'affixes respectives $1$ et i.
\begin{enumerate}
\item  D\'emontrer que pour $z \neq 0$, on a arg$\left(z'\right) =  \text{arg}(z)$ \`a $2k\pi$ pr\`es, avec $k$ entier relatif.\\
 En d\'eduire que, pour tout point $M$ de $\mathcal{P}\verb+\+ \{\text{O}\}$ les points $M$ et $M' = f(M)$ appartiennent \`a une m\^eme demi-droite d'origine O. 
\item D\'eterminer l'ensemble des points $M$ de $\mathcal{P}\verb+\+ \{\text{O}\}$ tels que $f(M) =  M$. 
\item $M$ est un point du plan $\mathcal{P}$ distinct de O, U et V, on admet que $M'$ est aussi distinct de O, U et V.\\
\'Etablir l'\'egalit\'e $\dfrac{z' - 1}{z' - \text{i}}= \dfrac{1}{\text{i}}\left(\dfrac{\overline{z} - 1}{\overline{z} + \text{i}}  \right) = -\text{i}\overline{\left(\dfrac{z - 1}{z - \text{i}} \right)}$.\\
En d\'eduire une relation entre arg$\left(\dfrac{z' - 1}{z' - \text{i}}\right)$ et  arg$\left(\dfrac{z - 1}{z - \text{i}} \right)$
\end{enumerate}
\item \begin{enumerate}
\item  Soit $z$ un nombre complexe tel que $z \neq 1$ et $z \neq \text{i}$ et soit $M$ le point d'affixe $z$. D\'emontrer que $M$
est sur la droite (UV) priv\'ee de U et de V si et seulement si  $\dfrac{z - 1}{z - \text{i}}$ est un nombre r\'eel non nul.
\item D\'eterminer l'image par $f$ de la droite (UV) priv\'ee de U et de V.
\end{enumerate}
\end{enumerate}

 
\end{exercice}

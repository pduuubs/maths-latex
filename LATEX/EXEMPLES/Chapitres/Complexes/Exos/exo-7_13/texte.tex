\begin{exercice}
$\bullet$\textbf{Type bac!}$\bullet$ Le plan est rapport\'e \`a un rep\`ere
  orthonormal $\ouv$ d'unit\'e graphique 8cm. On appelle $A$ le point
  d'affixe $-1$ et $B$ le point d'affixe $1$. Soit $\Er$ l'ensemble
  des points du plan distincts de $A$, $O$ et~$B$. \`A tout point $M$
  d'affixe $z$ on associe le point $N$ d'affixe $z^2$ et le point $P$
  d'affixe $z^3$.
  \begin{enumerate}
  \item Prouver que $M$, $N$ et $P$ sont deux \`a deux distincts.

  \item On se propose de d\'eterminer l'ensemble $\Cr$ des points $M$
    appartenant \`a $\Er$ tels que $MNP$ soit rectangle en~$P$.
    \begin{enumerate}
    \item Justifier que $MNP$ est rectangle en $P$ ssi: $|z+1|^2+|z|^2=1$
    \item Prouver l'\'equivalence: $|z+1|^2+|z|^2=1\iff \left(z+\dfrac12\right)\left(\overline{z+\dfrac12}\right)=\dfrac14$
    \item En d\'eduire l'ensemble $\Cr$ cherch\'e.
    \end{enumerate}

\end{enumerate}
 
\end{exercice}

\begin{exercice}
%Peut se faire sans les rotations complexes
Le plan complexe est ramen\'e \`a un rep\`ere orthonormal $\ouv$.
 On consid\`ere un nombre complexe $\omega$ tel que $\omega^3=1$ et
$\omega\neq 1$ (Un tel nombre existe, on en a vu en exercice, mais ici
aucune expression de $\omega$ sous forme alg\'ebrique n'est \`a
utiliser.) On consid\`ere alors la transformation complexe~$g$ de $\C$
dans $\C$ d\'efinie par: 
\[g(z)=\omega z\]
Soit un point $M$ du plan, d'affixe $z$, on note $M'$ le point
d'affixe $g(z)$.
\begin{enumerate}

\item Prouver que $1+\omega+\omega^2=0$. %(On pourra factoriser $\omega^3-1$)
\item Prouver que $|\omega|=1$
%\item Prouver que $\omega^2\neq 1$
%\item D\'emontrer que $OM'=OM$.
\item On note $z'=g(z)$ et $z''=g(z')$ de m\^eme on note $M''$ le
  point d'affixe $z''$. Prouver que $MM'M''$ est un triangle
  \'equilat\'eral dont $O$ est le centre  de gravit\'e.

\item  Quelle peut �tre la nature de la transformation du plan qui \`a $M$
  d'affixe $z$ associe $M'$ d'affixe $\omega z$?
\end{enumerate}
\end{exercice}

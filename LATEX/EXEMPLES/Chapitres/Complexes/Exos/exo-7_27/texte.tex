\begin{exercice}
Le plan complexe est ramen\'e \`a un rep\`ere orthonormal $\ouv$.
  On consid\`ere la fonction
complexe $f$ de $\C$ dans $\C$ d\'efinie par:
\[f(z)=iz\]
Soit un point $M$ du plan, d'affixe $z$, on note $M'$ le point
d'affixe $f(z)$.
\begin{enumerate}
\item Prouver que: $\forall z \in\C,\ |f(z)|=|z|$. Qu'en d\'eduit-on
  pour le triangle $OMM'$?
\item Prouver que: $\forall z \in\C,\ |f(z)-z|^2=|f(z)|^2+|z|^2$.
\item Quelle est la nature du triangle $OMM'$?

\item Quelle est la nature de la transformation du plan qui \`a $M$
  d'affixe $z$ associe $M'$ d'affixe $iz$?
\item Application: Soit un carr\'e $ABCD$ de centre $O$ tel que $A$ a
  pour coordonn\'ees cart\'esiennes:~$A(1;2)$. D\'eterminer les
  coordonn\'ees des trois autres sommets du carr\'e.
\end{enumerate}
\end{exercice}

\begin{exercice}
{\bf D\'etermination d'une rotation}\\
On consid\`ere les points $A$, $B$, $A'$ et $B'$ d'affixes respectives
$a$, $b$, $a'$ et $b'$.
\[a=2-i\qquad b=-2+3i\qquad a'=2+\sqrt3\qquad b'=-\sqrt3+2i(1-\sqrt3)\]
Le but de l'exercice est de prouver qu'il existe une rotation qui
transforme $A$ en $A'$ et $B$ en~$B'$ et d'en d\'eterminer le centre et l'angle.

\begin{enumerate}
\item Faire une figure. Expliquer la construction du centre de la
  rotation cherch\'ee.
\item D\'etermination de l'angle.
  \begin{enumerate}
  \item Prouver que $\dfrac{b'-a'}{b-a}=\e^{i\frac{\pi}{3}}$
  \item Interpr\'eter g\'eom\'etriquement ce r\'esultat en terme d'angle et longueurs.
  \end{enumerate}
\item D\'etermination du centre. On note $\Fr$ la rotation d'angle
  $\dfrac{\pi}{3}$ qui transforme $A$ en~$A'$. On note~$\omega$
  l'affixe de son centre~$\Omega$.
  \begin{enumerate}
  \item \'Ecrire la relation v\'erifi\'ee par $a$, $a'$ et
    $\omega$. Exprimer $\omega$ en fonction de $a$ et~$a'$.
  \item En    d\'eduire le centre de la rotation~$\Fr$. Pour all\'eger
    les calculs on prouvera avant que:
    \begin{equation}
      \label{eq:1}
      \e^{i\frac{\pi}{3}}-1=\e^{\frac{2i\pi}{3}}
    \end{equation}
\begin{equation}
      \label{eq:2}
      \omega=\e^{-i\frac{\pi}{3}}a-\e^{-\frac{2i\pi}{3}}a'
    \end{equation}
        

  \item V\'erifier que $\Fr(B)=B'$.
  \item Conclure.
  \end{enumerate}
\end{enumerate}
 
\end{exercice}

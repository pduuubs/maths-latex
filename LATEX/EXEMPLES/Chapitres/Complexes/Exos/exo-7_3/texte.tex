\begin{exercice}
Un grand classique du rire. On veut r\'esoudre une \'equation
  de degr\'e trois dans $\C$. On ne vous donnera pas de formule, mais
  on vous pr\'esentera toujours les choses \`a peu pr\`es ainsi: On exhibe
  ou fait trouver une \og solution \'evidente\fg{} $z_0$, on factorise
  par $(z-z_0)$ le polynome de d\'epart, et on est alors ramen\'e \`a
  une \'equation de degr\'e deux. Voici deux exemples:
  \begin{enumerate}
  \item Pour tout complexe $z$, on pose: $P(z)=z^3-12z^2+48z-128$.
    \begin{enumerate}
    \item Calculer $P(8)$.
    \item  D\'eterminer trois
    r\'eels $a$, $b$, $c$ tels que pour tout complexe $z$:
\[ P(z)=(z-8)(az^2+bz+c)\]

\item R\'esoudre dans $\C$ l'\'equation $P(z)=0$.
    \end{enumerate} 


  \item Pour tout complexe $z$, on pose: $P(z)=2z^3-3z^2+4z-3$
\begin{enumerate}
    \item Trouver une racine \'evidente de $P$, not\'ee $z_0$.
    \item  D\'eterminer trois
    r\'eels $a$, $b$, $c$ tels que pour tout complexe $z$:
\[ P(z)=(z-z_0)(az^2+bz+c)\]

\item R\'esoudre dans $\C$ l'\'equation $P(z)=0$.
    \end{enumerate} 

  \end{enumerate}
  
\end{exercice}

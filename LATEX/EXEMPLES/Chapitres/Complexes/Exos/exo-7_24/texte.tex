\begin{exercice}
{\large \textbf{Baccalaur\'eat S Centres \'etrangers juin 2006}}
 
%\vspace{1cm}
%
%\noindent \textbf{\textsc{ Exercice 1} \hfill 3 points}\\

\noindent \textbf{Partie A.} Restitution organis\'ee de connaissances\\
\fbox{\begin{minipage}{0.9\textwidth}Pr\'erequis : On rappelle les deux r\'esultats suivants :

\textbf{i.}  Si $z$ est un nombre complexe non nul, on a l'\'equivalence suivante :

\[\left\{\begin{array}{l cl}
|z|&=&r\\
\text{arg}~z &=& \theta~\text{\`a}~2\pi~\text{pr\`es}\\
\end{array}\right. \iff 
\left\{\begin{array}{l cl}z&=&r(\cos \theta + \text{i}\sin \theta)\\
r & > & 0\\ \end{array}\right.\]
\textbf{ii.} Pour tous nombres r\'eels $a$ et $b$ :
\[\left\{\begin{array}{l cl}
\cos (a + b)&=&\cos a\cos b - \sin a\sin b\\
\sin (a + b)&=&\sin a\cos b + \sin b \cos a\\
\end{array}\right.\]
\end{minipage}}

\noindent Soient $z_{1}$ et $z_{2}$ deux nombres complexes non nuls.\\
D\'emontrer les relations :

\[\left|z_{1}z_{2}\right|  = \left|z_{1}\right|\left|z_{2}\right|\quad\text{et}\quad \text{arg}\left(z_{1}z_{2}\right) = \text{arg}\left(z_{1}) + \text{arg}(z_{2}\right)~\text{\`a}~2\pi~\text{pr\`es}\]

\medskip

\noindent \textbf{Partie B.}

\noindent  Pour chaque proposition, indiquer si elle est vraie ou fausse et proposer une d\'emonstration pour la r\'eponse indiqu\'ee. Dans le cas d'une proposition fausse, la d\'emonstration consistera \`a fournir un contre-exemple. Une r\'eponse sans d\'emonstration ne rapporte pas de point.\\
On rappelle que si $z$ est un nombre complexe, $\overline{z}$ d\'esigne le conjugu\'e de $z$ et $|z|$ d\'esigne le module de $z$.

\begin{enumerate}
\item Si $z = - \dfrac{1}{2} + \dfrac{1}{2}\text{i}$, alors $z^4$ est un nombre r\'eel.
\item Si $z + \overline{z} = 0$, alors $z = 0$.
\item Si $z + \dfrac{1}{z} = 0$, alors $z = \text{i}$ ou $z = - \text{i}$.
\item Si $|z| = 1$ et si $|z + z'| = 1$, alors $z' = 0$.
\end{enumerate}

 
\end{exercice}

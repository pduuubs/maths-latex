\begin{exercice}

 Le plan est rapport\'e \`a un rep\`ere
  orthonormal $\ouv$. On consid\`ere l'application $f$ du plan dans
  lui m\^eme qui au point $M$ d'affixe $z$ associe le point $M'$
  d'affixe $z'$ telle que $z'=z^2-4z$. Soit $A$ et $B$ et $I$ les points
  d'affixe respective:
\[z_A=1-i \qquad z_B=3+i \qquad z_I=-3\] 
\begin{enumerate}
\item Calculer les affixes de $A'$ et $B'$ images des points $A$ et
  $B$ par~$f$.
\item On suppose que deux points ont la m\^eme image par $f$. Prouver
  qu'ils sont confondus ou image l'un de l'autre par une sym\'etrie
  centrale que l'on pr\'ecisera.
\item D\'emontrer que $OMIM'$ est un parall\'elogramme ssi
  $z^2-3z+3=0$. 
\end{enumerate}
\end{exercice}

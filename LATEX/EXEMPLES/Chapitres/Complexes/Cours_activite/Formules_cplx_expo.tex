

\documentclass[a4paper]{article}
\usepackage[]{pan}
\usepackage{pifont}
\begin{document}
\newcommand{\T}{\theta}
\fcours{Exponentielle complexe $\e^{i\T}$}%{\term}
\rhead{}
%%-----------------------
\begin{list}{}{}
\item[{\bf D�finition.}] Le complexe non nul $z$, de module $|z|$ et
  d'argument $\theta$ est mis sous forme exponentielle lorsque l'on �crit:
  $z=|z|\e^{i\theta}$ o�:
\[ \e^{i\theta}=\cos\theta+i\sin\theta\]
\end{list}
\noindent Les propri�t�s suivantes sont valables pour tous r�els $\theta$ et $\theta'$,
$r$ et $r'$, pour tous $n$ et~$k$ dans $\Z$
\begin{list}{}{}

\item[{\bf Addition d'angles}]
{\large\[ \e^{i(\theta+\theta')}=\e^{i\theta}\e^{i\theta'}\]}
\item[{\bf Conjugu�}]
{\large\[ \overline{r\e^{i\theta}}=r\e^{-i\theta} \]}
\item[{\bf Module unitaire}]
{\large\[ \left|\e^{i\theta}\right|=1\]}
\item[{\bf Inverse}]
{\large\[ \frac{1}{\e^{i\theta}}=\e^{-i\theta}\]}
\item[{\bf Quotient}]
{\large\[ \frac{\e^{i\theta}}{\e^{i\theta'}}=\e^{i(\theta-\theta')}\]}
\item[{\bf Puissance-Formule de De Moivre}]
{\large\[ \left( \e^{i\theta}\right)^n=\e^{in\theta}\]}
\item[{\bf Produit, quotient de deux complexes}]
{\large\[
\left(r\e^{i\theta}\right)\times\left(r'\e^{i\theta'}\right)=rr'\e^{i(\theta+\theta')}\qquad
et \qquad
\frac{r\e^{i\theta}}{r'\e^{i\theta'}}=\frac{r}{r'}\e^{i(\theta-\theta')}
\]}
\item[{\bf Valeurs remarquables}]
{\Large\[ \e^{0}=1\quad;\quad \e^{i\frac{\pi}{2}}=i\quad;\quad
  \e^{i\pi}=-1 \quad;\quad \e^{2i\pi}=1 \]}
\item[{\bf Oppos� d'un complexe}]
{\large\[ -r\e^{i\theta}=r\e^{i(\theta+\pi)}\]}
\item[{\bf $\theta\mapsto\e^{i\theta}$ est $2\pi$ p�riodique}]
{\large\[ \e^{i(\theta+2k\pi)}=\e^{i\theta}\]}
{\large\[ \e^{i\theta}=\e^{i\theta'} \iff \theta=\theta'\ [2\pi]\]}
\item[{\bf Formules d'Euler }]
{\large\[ \cos\theta=\frac{\e^{i\theta}+\e^{-i\theta}}{2}\quad\et{}\quad
\sin\theta=\frac{\e^{i\theta}-\e^{-i\theta}}{2i}\]}
Ces formules proviennent de ce que pour tout complexe $z$:
\[Re(z)=\frac{z+\overline{z}}{2}\quad\et{}\quad Im(z)=\frac{z-\overline{z}}{2i}\]
\item[{\bf Attention!}] $\e^{i\theta}$ est un nombre complexe, on ne peut donc pas parler de son signe\ldots 
%\[ \]
%\ding{63} 
%exp est positif ne veut rien dire

\end{list}




%%-----------------------
\end{document}

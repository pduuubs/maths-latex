

\documentclass[a4paper]{article}
\usepackage[]{pan}
\renewcommand{\i}{\textbf{i}}

\begin{document}
\activite{Nombres complexes}{\term}

\section{R\'esolution historique}
Au \bsc{XVI}$^e$ si\`ecle les alg\'ebristes italiens apprennent
\`a r\'esoudre les \'equations du troisi\`eme degr\'e, en les ramenant
\`a des \'equations du deuxi\`eme degr\'e dont la r\'esolution est
connue depuis le \bsc{IX}$^e$ si\`ecle gr\^ace auxs math\'ematiciens arabes.
On attribue \`a \bsc{Cardan} (Girolamo Cardano: Pavie 1501-Rome 1576) la
formule \eqref{eq:form} donnant une solution \`a l'\'equation du troisi\`eme
degr\'e d'inconnue~$x$:
\begin{equation}
  \label{eq:deg3}
  x^3=px+q
\end{equation}
En fait, \bsc{Tartaglia} (Niccol\`o Tartaglia:
Brescia 1500 - Venise 1557)  autodidacte aurait d\'ecouvert la formule
en 1539 et l'aurait expos\'ee au professeur respect\'e \bsc{Cardan}
qui l'a publi\'e dans son \emph{Ars magna} en
1545 comme \'etant sa propre d\'ecouverte, le reste de l'histoire est
romanesque.
\begin{equation}
  \label{eq:form}
  x=\sqrt[3]{\frac{q}{2}+\sqrt{\left(\frac{q}{2}\right)^2-\left(\frac{p}{3}\right)^3}}+\sqrt[3]{\frac{q}{2}-\sqrt{\left(\frac{q}{2}\right)^2-\left(\frac{p}{3}\right)^3}}
\end{equation}
\textbf{D\'efinition:} Pour $a\in\R$, on note $\sqrt[3]{a}$ appel\'e \emph{racine
    cubique de $a$} l'unique r\'eel $x$ v\'erifiant $x^3=a$.

La racine cubique, contrairement \`a la racine carr\'ee est d\'efinie
pour tout r\'eel car $x\mapsto x^3$ est strictement croissante sur
$\R$, et varie de $-\infty$ \`a $+\infty$. Par exemple,
$\sqrt[3]{-8}=-2$ car $(-2)^3=-8$.
\begin{enumerate}
\item Prouver que toute \'equation de degr\'e trois donc de la forme
  $ax³+bx²+cx+d=0$ avec $a$, $b$, $c$, $d$ trois r\'eels et $a\neq 0$
  peut se mettre sous la forme $x^3+b'x^2+c'x+d'=0$
\item Prouver que toute \'equation de degr\'e trois de la forme
  $x³+bx²+cx+d=0$ peut se mettre sous la forme de l'\'equation
  \eqref{eq:deg3} \`a l'aide du changement de variable:
  $x=X-\frac{b}{3}$.

\item Voyons sur un exemple comment ils trouv\`erent la formule
  improbable \eqref{eq:form}. On consid\`ere l'\'equation:
  \begin{equation}
    \label{eq:eq1}
    x³=6x+20
  \end{equation}
  \begin{enumerate}
  \item On pose $x=u+v$. Que devient l'\'equation \eqref{eq:eq1}?
  \item Quelle valeur suffit-il d'imposer au produit $uv$ pour que
    \eqref{eq:eq1} s'\'ecrive $u³+v³=20$?
  \item On pose $U=u³$ et $V=v³$. Former et r\'esoudre une \'equation
    de degr\'e deux ayant pour solutions $U$ et~$V$.

  \item En d\'eduire que:
\[ \sqrt[3]{10+2\sqrt{23}} + \sqrt[3]{10-2\sqrt{23}}\]
est une solution de \eqref{eq:eq1}.

\item Optionnel: Etudier $x\mapsto x³-6x-20$ sur $\R$ pour voir qu'il
  n'y a qu'une solution \`a l'\'equation \eqref{eq:eq1}.
  \end{enumerate}

\end{enumerate}
La m\'ethode de \bsc{Tartaglia-Cardan} conduit cependant, dans certains cas,
\`a un paradoxe que \bsc{Bombelli}(1526-1572) va essayer de surmonter. Il publie en
1572 dans son \emph{l'Algebra opera} l'exemple suivant:
\section{Des nombres impossibles}
On consid\`ere l'\'equation:
  \begin{equation}
    \label{eq:eq2}
    x³=15x+4
  \end{equation}
  \begin{enumerate}
  \item Combien vaut alors la quantit\'e:
    $\left(\frac{q}{2}\right)^2-\left(\frac{p}{3}\right)^3$ qui
    appara\^it dans la formule \eqref{eq:form}?
  \item Prouver cependant que 4 est solution de \eqref{eq:eq2}\\
L'id\'ee et l'audace de \bsc{Bombelli} est de faire comme si $-121$
\'etait le carr\'e d'un nombre imaginaire qui s'\'ecrirait
$11\sqrt{-1}$. Il appella $\sqrt{-1}$ \emph{piu di meno} qui sera not\'e
\textbf{i} plus tard par \bsc{Euler}(1734-1810) en 1777. Gr\^ace \`a ce stratag\`eme,\bsc{Bombelli} retrouve la solution r\'eelle 4:

\item D\'evelopper $(2+\i)^3$ et $(2-\i)^3$ en tenant compte de $\i^2=-1$.
\item En d\'eduire que $(2+\i)+(2-\i)$ est solution de
  \eqref{eq:eq2}.\\
Ce nombre \og $\sqrt{-1}$ \fg{} qualifi\'e d'impossible va susciter beaucoup de m\'efiance et
de pol\'emique pendant deux si\`ecles jusqu'\`a \bsc{Argand} en 1806 qui
proposera une repr\'esentation g\'eom\'etrique de ces nombres
imaginaires (de la forme $a+\i b$ avec $a\in\R$, $b\in\R$) et \bsc{Gauss}(1777-1855) qui les
nommera \textbf{Nombres complexes}.
  \end{enumerate}
\end{document}



\documentclass[a4paper]{article}
\usepackage[]{pan}

\begin{document}
\fcours{Complexes et transformations}%{\term}
\chead{} \rhead{}
%%-----------------------
Soit $\Fr$ une transformation du plan qui au point $M$ associe le
point $M'$. \ie{} $\Fr(M)=M'$. On associe � cette transformation la
fonction complexe $f$ qui � l'affixe $z$ de $M$ associe l'affixe $z'$
de~$M'$. On dit que {\bf $z'=f(z)$ est l'�criture complexe de la
transformation~$\Fr$.} On va donner les �critures complexes de trois
transformations usuelles.
\section{Translation}
\begin{prop} Soit $\vw$ un vecteur d'affixe $b$. 
L'�criture complexe de la translation de vecteur $\vw$ est:
\[\boxed{z'=z+b}\]
\end{prop}

\begin{proof} Ici $\Fr$ est la translation de vecteur
  $\vw$.  
\[\Fr(M)=M' \iff \V{MM'}=\vw\iff z'-z=b\iff z'=z+b\]
\end{proof}
\section{Homoth�tie}
\begin{prop} Soit $\Omega$ un point d'affixe $z_{\Omega}$ et
  $k\in\R^*$. L'�criture complexe de l'homoth�tie de centre $\Omega$ et de rapport $k$ est:
\[\boxed{z'-z_{\Omega}=k(z-z_{\Omega})}\]
\end{prop}

\begin{proof}Ici $\Fr$ est l'homoth�tie de centre $\Omega$ et de
  rapport $k$.
\[\Fr(M)=M' \iff \V{\Omega M'}=k \V{\Omega M} \iff z'-z_{\Omega}=k(z-z_{\Omega})\]
\end{proof}
\section{Rotation}
\begin{prop} Soit $\theta\in\R$.
L'�criture complexe de la rotation de  centre $\Omega$ et d'angle
$\theta$ est:
\[\boxed{z'-z_{\Omega}=\e^{i\theta}(z-z_{\Omega})}\]
\end{prop}

\begin{proof}Ici $\Fr$ est la rotation de  centre $\Omega$ et d'angle
$\theta$. 
\begin{list}{$\bullet$}{}
\item Si $M=\Omega$ alors $\Fr(M)=M' \iff M'=\Omega \iff z'=z_{\Omega}$. Car le centre
  d'une rotation est l'unique point invariant par cette
  rotation. L'�galit� voulue est donc vraie (0=0)
\item Si $M\neq \Omega$ alors $\Fr(M)=M'$ signifie:
\[ \Omega M'=\Omega M\quad \et{}\quad (\V{\Omega M};\V{\Omega
  M'})=\theta\]
Ce qui se traduit en module et argument par:
\[ |z'-z_{\Omega}|=|z-z_{\Omega}|\quad \et{}\quad
arg(\frac{z'-z_{\Omega}}{z-z_{\Omega}})=\theta [2\pi]\]
C'est � dire que le complexe $\frac{z'-z_{\Omega}}{z-z_{\Omega}}$ est
de module 1 et d'argument $\theta$. Il est donc �gal �
$\e^{i\theta}$. D'o� le r�sultat:
\[ \frac{z'-z_{\Omega}}{z-z_{\Omega}}=\e^{i\theta}\iff z'-z_{\Omega}=\e^{i\theta}(z-z_{\Omega})\]
\end{list}
\end{proof}
\noindent
{\bf Cas particulier.} $z'=iz$ est l'�criture complexe de la rotation
de centre $O$ et d'angle $\frac{\pi}{2}$ car
$i=\e^{i\frac{\pi}{2}}$. Donc le triangle $OMM'$ est isoc�le rectangle
en~ $O$.

%%-----------------------
\end{document}



\documentclass[a4paper]{article}
\usepackage[]{pan}

\begin{document}
\activite{Nombres complexes 3}{\term}

\Titre{\'Equation du second degr\'e}
\hline
\medskip

L'objectif est de r\'esoudre dans $\C$ toutes les \'equations du
second degr\'e \`a coefficients r\'eels. \ie{} de la forme:
\[az^2+bz+c=0\quad \text{o\`u}\ a\in\R^*,\ b\in\R,\ c\in\R\]
\section{Forme canonique}
On commence \`a la main, sans formule, en utilsant la forme canonique.
\exemple On veut r\'esoudre: $z^2+2z+5=0$:
\[z^2+2z+5=(z+1)^2-1+5=(z+1)^2+4=(z+1)^2-(2i)^2\]
D'o\`u la factorisation et la r\'esolution:
\[ z^2+2z+5=0 \iff \left( z+1-2i\right)\cdot\left(z+1+2i\right)=0 \iff
z\in\{-1+2i\,;\,-1-2i\}\]
R\'esoudre de la m\^eme mani\`ere les \'equations du
second degr\'e suivantes dans $\C$:
\qcmabc{$z^2-4z+5=0$}{$z^2+6z+11=0$}{$2z^2-4z+3=0$}
\section{Formules}
Cette technique vue par factorisation gr\^ace \`a la forme canonique est
syst\'ematique et on peut donc obtenir des formules, similaires \` a
celles des racines r\'eelles.
\begin{enumerate}
\item Mettre $az^2+bz+c$ sous forme canonique. On fera appara\^itre $\Delta=b^2-4ac$
\item On suppose que $\Delta <0$. \'Ecrire $\dfrac{\Delta}{4a^2}$
  comme le carr\'e d'un nombre complexe.


\item En d\'eduire une factorisation de $az^2+bz+c$ sous la forme:
\[ az^2+bz+c= a(z-z_1)(z-\overline{z_1})\quad \text{o\`u}\ z_1\in\C \]

\item En d\'eduire les formules donnant les racines complexes de
  $az^2+bz+c$ dans le cas o\`u le discriminant est n\'egatif.
\end{enumerate}
\textbf{Remarque:} Comme dans $\R$, si $b$, le coefficient de $z$, est nul on
n'applique pas ces formules comme un \^ane. On proc\`ede ainsi: 
\exemple $z^2+3=0\iff z^2-(i\sqrt3)^2=0 \iff z \in  \{i\sqrt3\,;\,-i\sqrt3\}$
\section{Application}
\Exo[]{
R\'esoudre dans $\C$ les \'equations suivantes en appliquant les
formules pr\'ec\'edentes.
\begin{enumerate}
\item \qcmabc{$2z^2+4z+5=0$}{$-2z^2+6z-5=0$}{$-5z^2+2z+2=0$}
\item \qcmabc{$z^2+z+1=0$}{$(z^2+2)(z^2-4z+4)=0$}{$(z+1)^2=-(2z+1)^2$}
\item \qcmabc{$2z^4-9z^2+4=0$}{$\left(z^2+1\right)^2=1$}{$\dfrac{z-1}{z+1}=z+2$}
\end{enumerate}
}

\Exo[]{\textbf{\'Equation \`a coefficients sym\'etriques}. Soit l'\'equation $(E)$:
  $z^4-5z^3+6z^2-5z+1=0$\\
Prouver que $(E)$ est \'equivalente au syst\`eme:
$\left\{\begin{array}{ll}
  u^2-5u+4=0\\
u=z+\dfrac{1}{z} 
\end{array}\right.$\\

R\'esoudre $u^2-5u+4=0$, puis r\'esoudre l'\'equation $(E)$ dans $\C$.
}
\end{document}



\Exo[]{Un grand classique du rire. On veut r\'esoudre une \'equation
  de degr\'e trois dans $\C$. On ne vous donnera pas de formule, mais
  on vous pr\'esentera toujours les choses \`a peu pr\`es ainsi: On exhibe
  ou fait trouver une \og solution \'evidente\fg{} $z_0$, on factorise
  par $(z-z_0)$ le polynome de d\'epart, et on est alors ramen\'e \`a
  une \'equation de degr\'e deux.
  \begin{enumerate}
  \item Pour tout complexe $z$, on pose
    $P(z)=z^3-12z^2+48z-128$. Calculer $P(8)$. D\'eterminer trois
    r\'eels $a$, $b$, $c$ tels que pour tout complexe $z$:
\[ P(z)=(z-8)(az^2+bz+c)\]
R\'esoudre dans $\C$ l'\'equation $P(z)=0$.


  \end{enumerate}
  

}
\end{document}

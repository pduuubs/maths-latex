

\documentclass[a4paper]{article}
\usepackage[]{pan}
\newtheorem{Exercice}{Exercice}
      \newtheorem{solution}{Solution}
      \setlength{\parindent}{0pt}
\begin{document}
\fexos{Complexes-Transformations}{\term S}
\hrule
\begin{minipage}{0.55\linewidth}
\vspace{0.5cm}
 \begin{center}
\subsection*{Expression complexe d'une rotation}
\end{center}%\hrule
\vspace{0.5cm}
%\subsubsection*{Sujets}
Le plan est muni d'un rep\`ere orthonormal direct $\left(\mathrm{O};\vec{u},\vec{v}\right)$.\\D\'eterminez l'affixe $\omega$ du centre $\Omega$ et la mesure principale $\theta$ de l'angle de la rotation qui \`a tout point $M$ d'affixe $z$ associe le point $M'$ d'affixe $z'$.
\begin{Exercice}\[
\displaystyle z'=\left(-\frac{1}{2}+\frac{\sqrt{3}}{2}\mathrm{i}\right)z+\frac{15}{4}+\frac{1}{2 \sqrt{3}}+\left(\frac{1}{2}-\frac{5 \sqrt{3}}{4}\right)\mathrm{i}\]
\end{Exercice}\begin{Exercice}\[
\displaystyle z'=\left(-\frac{\sqrt{3}}{2}+\frac{1}{2}\mathrm{i}\right)z-\frac{8}{3}-\sqrt{3}+\left(-\frac{1}{3}-\frac{2}{\sqrt{3}}\right)\mathrm{i}\]
%\end{Exercice}\begin{Exercice}\[
%\displaystyle z'=\left(-\frac{\sqrt{3}}{2}+\frac{1}{2}\mathrm{i}\right)z+\frac{17}{6}+\sqrt{3}+\left(\frac{2}{3}+\frac{5}{2 \sqrt{3}}\right)\mathrm{i}\]
\end{Exercice}
\begin{Exercice}\[
 z'=\left(\frac{1}{2}+\frac{\sqrt{3}}{2}\mathrm{i}\right)z-\frac{1}{2}+\frac{1}{\sqrt{3}}+\left(\frac{1}{3}+\frac{\sqrt{3}}{2}\right)\mathrm{i}\]
\end{Exercice}

%\begin{Exercice}\[
%\displaystyle z'=\left(\frac{1}{2}+\frac{\sqrt{3}}{2}\mathrm{i}\right)z-\frac{5}{4}-\frac{5}{2 \sqrt{3}}+\left(-\frac{5}{6}+\frac{5 \sqrt{3}}{4}\right)\mathrm{i}\]
%\end{Exercice}
\end{minipage}\hfill%
%%%%%%%%%%%%%%%%%%%%%%%%%%%%%%%%%%%%%
\begin{minipage}{0.40\linewidth}
 % \hrule
\vspace{0.5cm}
\begin{center}
\subsection*{Image d'un point par une rotation}
\end{center}%\hrule
\vspace{0.5cm}
%\subsubsection*{Sujets}
Le plan est muni d'un rep\`ere orthonormal direct $\left(\mathrm{O};\vec{u},\vec{v}\right)$.\\D\'eterminez l'affixe $z'$ de l'image $M'$ du point $M$ d'affixe $z$ par la rotation de centre $\Omega$ d'affixe $\omega$ et d'angle de mesure~$\theta$.
\begin{Exercice}
$\displaystyle \omega=-\frac{5}{2}-\mathrm{i}$ et $\displaystyle\theta=\frac{2 \pi }{3}$.
\end{Exercice}\begin{Exercice}
$\displaystyle \omega=-\frac{3}{2}+4\mathrm{i}$ et $\displaystyle\theta=\frac{3 \pi }{4}$.
\end{Exercice}\begin{Exercice}
$\displaystyle \omega=-\frac{3}{2}$ et $\displaystyle\theta=-\frac{\pi }{4}$.
\end{Exercice}\begin{Exercice}
$\displaystyle \omega=\frac{5}{3}+5\mathrm{i}$ et $\displaystyle\theta=\frac{\pi }{6}$.
\end{Exercice}\begin{Exercice}
$\displaystyle \omega=1-\frac{1}{3}\mathrm{i}$ et $\displaystyle\theta=\frac{5 \pi }{6}$.
\end{Exercice}
\end{minipage}
\hfill%
%%%%%%%%%%%%%%%%%%%%%%%%%%%%%%%%%%%%%

\vspace{0.5cm}
\hrule
\medskip
\begin{minipage}{0.45\linewidth}
  \vspace{0.5cm}
\begin{center}
\subsection*{Expression complexe d'une homoth\'etie}
\end{center}%\hrule
\vspace{0.5cm}
%\subsubsection*{Sujets}
Le plan est muni d'un rep\`ere orthonormal direct $\left(\mathrm{O};\vec{u},\vec{v}\right)$.\\D\'eterminez l'affixe $z'$ de l'image $M'$ du point $M$ d'affixe $z$ par l'homoth\'etie de centre $\Omega$ d'affixe $\omega$ et de rapport $k$.
\begin{Exercice}
$\displaystyle \omega=2-3\mathrm{i}$ et $\displaystyle k=2$.
\end{Exercice}\begin{Exercice}
$\displaystyle \omega=-3$ et $\displaystyle k=-1$.
\end{Exercice}\begin{Exercice}
$\displaystyle \omega=1+3\mathrm{i}$ et $\displaystyle k=\frac{5}{2}$.
\end{Exercice}\begin{Exercice}
$\displaystyle \omega=3$ et $\displaystyle k=\frac{3}{2}$.
\end{Exercice}\begin{Exercice}
$\displaystyle \omega=-2$ et $\displaystyle k=-1$.
\end{Exercice}
\end{minipage}
\hfill%
%%%%%%%%%%%%%%%%%%%%%%%%%%%%%%%%%%%%%
\begin{minipage}{0.45\linewidth}
  %\hrule
\vspace{0.5cm}
\begin{center}
\subsection*{Expression complexe d'une translation}
\end{center}%\hrule
\vspace{0.5cm}%\vspace{1cm}
%\subsubsection*{Sujets}
Le plan est muni d'un rep\`ere orthonormal direct $\left(\mathrm{O};\vec{u},\vec{v}\right)$.\\D\'eterminez l'affixe $z'$ de l'image $M'$ du point $M$ d'affixe $z$ par la translation de vecteur $\vec{w}$ d'affixe $\omega$.
\begin{Exercice}
$\displaystyle \omega=-1-\frac{1}{2}\mathrm{i}$.
\end{Exercice}\begin{Exercice}
$\displaystyle \omega=-\frac{9}{2}+4\mathrm{i}$.
\end{Exercice}\begin{Exercice}
$\displaystyle \omega=2+3\mathrm{i}$.
\end{Exercice}\begin{Exercice}
$\displaystyle \omega=-4-4\mathrm{i}$.
\end{Exercice}\begin{Exercice}
$\displaystyle \omega=-\frac{1}{2}+4\mathrm{i}$.
\end{Exercice}
\end{minipage}




\newpage
\subsection*{Solutions}
\begin{solution}
La rotation qui \`a tout point $M$ d'affixe $z$ associe le point $M'$ d'affixe $z'$ d\'efinie par $$z'=\left(-\frac{1}{2}+\frac{\sqrt{3}}{2}\mathrm{i}\right)z+\frac{15}{4}+\frac{1}{2 \sqrt{3}}+\left(\frac{1}{2}-\frac{5 \sqrt{3}}{4}\right)\mathrm{i}$$ admet pour centre le point $\Omega$ d'affixe $\displaystyle \omega=\frac{5}{2}+\frac{1}{3}\mathrm{i}$ et pour mesure d'angle $\displaystyle \frac{2 \pi }{3}$.
\end{solution}\begin{solution}
La rotation qui \`a tout point $M$ d'affixe $z$ associe le point $M'$ d'affixe $z'$ d\'efinie par $$z'=\left(-\frac{\sqrt{3}}{2}+\frac{1}{2}\mathrm{i}\right)z-\frac{8}{3}-\sqrt{3}+\left(-\frac{1}{3}-\frac{2}{\sqrt{3}}\right)\mathrm{i}$$ admet pour centre le point $\Omega$ d'affixe $\displaystyle \omega=-2-\frac{4}{3}\mathrm{i}$ et pour mesure d'angle $\displaystyle \frac{5 \pi }{6}$.
\end{solution}\begin{solution}
La rotation qui \`a tout point $M$ d'affixe $z$ associe le point $M'$ d'affixe $z'$ d\'efinie par $$z'=\left(-\frac{\sqrt{3}}{2}+\frac{1}{2}\mathrm{i}\right)z+\frac{17}{6}+\sqrt{3}+\left(\frac{2}{3}+\frac{5}{2 \sqrt{3}}\right)\mathrm{i}$$ admet pour centre le point $\Omega$ d'affixe $\displaystyle \omega=2+\frac{5}{3}\mathrm{i}$ et pour mesure d'angle $\displaystyle \frac{5 \pi }{6}$.
\end{solution}\begin{solution}
La rotation qui \`a tout point $M$ d'affixe $z$ associe le point $M'$ d'affixe $z'$ d\'efinie par $$z'=\left(-\frac{\sqrt{3}}{2}+\frac{1}{2}\mathrm{i}\right)z+\frac{5}{6}+\frac{\sqrt{3}}{2}+\left(-\frac{5}{6}-\frac{1}{2 \sqrt{3}}\right)\mathrm{i}$$ admet pour centre le point $\Omega$ d'affixe $\displaystyle \omega=1-\frac{1}{3}\mathrm{i}$ et pour mesure d'angle $\displaystyle \frac{5 \pi }{6}$.
\end{solution}\begin{solution}
La rotation qui \`a tout point $M$ d'affixe $z$ associe le point $M'$ d'affixe $z'$ d\'efinie par $$z'=\left(\frac{1}{2}+\frac{\sqrt{3}}{2}\mathrm{i}\right)z-\frac{5}{4}-\frac{5}{2 \sqrt{3}}+\left(-\frac{5}{6}+\frac{5 \sqrt{3}}{4}\right)\mathrm{i}$$ admet pour centre le point $\Omega$ d'affixe $\displaystyle \omega=-\frac{5}{2}-\frac{5}{3}\mathrm{i}$ et pour mesure d'angle $\displaystyle \frac{\pi }{3}$.
\end{solution}
%%%%%%%%%%%%%%%%%%
\begin{solution}
La rotation de centre $\Omega$ d'affixe $\omega=-\frac{5}{2}-\mathrm{i}$ et d'angle de mesure $\frac{2 \pi }{3}$ associe \`a tout point $M$ d'affixe $z$ le point $M'$ d'affixe $z'$ d\'efinie par $$z'=\left(-\frac{1}{2}+\frac{\sqrt{3}}{2}\mathrm{i}\right)z-\frac{15}{4}-\frac{\sqrt{3}}{2}+\left(-\frac{3}{2}+\frac{5 \sqrt{3}}{4}\right)\mathrm{i}.$$
\end{solution}\begin{solution}
La rotation de centre $\Omega$ d'affixe $\omega=-\frac{3}{2}+4\mathrm{i}$ et d'angle de mesure $\frac{3 \pi }{4}$ associe \`a tout point $M$ d'affixe $z$ le point $M'$ d'affixe $z'$ d\'efinie par $$z'=\left(-\frac{1}{\sqrt{2}}+\frac{1}{\sqrt{2}}\mathrm{i}\right)z-\frac{3}{2}+\frac{5}{2 \sqrt{2}}+\left(4+\frac{11}{2 \sqrt{2}}\right)\mathrm{i}.$$
\end{solution}\begin{solution}
La rotation de centre $\Omega$ d'affixe $\omega=-\frac{3}{2}$ et d'angle de mesure $-\frac{\pi }{4}$ associe \`a tout point $M$ d'affixe $z$ le point $M'$ d'affixe $z'$ d\'efinie par $$z'=\left(\frac{1}{\sqrt{2}}-\frac{1}{\sqrt{2}}\mathrm{i}\right)z-\frac{3}{2}+\frac{3}{2 \sqrt{2}}-\frac{3}{2 \sqrt{2}}\mathrm{i}.$$
\end{solution}\begin{solution}
La rotation de centre $\Omega$ d'affixe $\omega=\frac{5}{3}+5\mathrm{i}$ et d'angle de mesure $\frac{\pi }{6}$ associe \`a tout point $M$ d'affixe $z$ le point $M'$ d'affixe $z'$ d\'efinie par $$z'=\left(\frac{\sqrt{3}}{2}+\frac{1}{2}\mathrm{i}\right)z+\frac{25}{6}-\frac{5}{2 \sqrt{3}}+\left(\frac{25}{6}-\frac{5 \sqrt{3}}{2}\right)\mathrm{i}.$$
\end{solution}\begin{solution}
La rotation de centre $\Omega$ d'affixe $\omega=1-\frac{1}{3}\mathrm{i}$ et d'angle de mesure $\frac{5 \pi }{6}$ associe \`a tout point $M$ d'affixe $z$ le point $M'$ d'affixe $z'$ d\'efinie par $$z'=\left(-\frac{\sqrt{3}}{2}+\frac{1}{2}\mathrm{i}\right)z+\frac{5}{6}+\frac{\sqrt{3}}{2}+\left(-\frac{5}{6}-\frac{1}{2 \sqrt{3}}\right)\mathrm{i}.$$
\end{solution}
%%%%%%%%%%%%%%%%%%
\begin{solution}
L'homoth\'etie de centre $\Omega$ d'affixe $\omega=2-3\mathrm{i}$ et de rapport $2$ associe \`a tout point $M$ d'affixe $z$ le point $M'$ d'affixe $z'$ d\'efinie par $$z'=2z-2+3\mathrm{i}.$$
\end{solution}\begin{solution}
L'homoth\'etie de centre $\Omega$ d'affixe $\omega=-3$ et de rapport $-1$ associe \`a tout point $M$ d'affixe $z$ le point $M'$ d'affixe $z'$ d\'efinie par $$z'=-z-6.$$
\end{solution}\begin{solution}
L'homoth\'etie de centre $\Omega$ d'affixe $\omega=1+3\mathrm{i}$ et de rapport $\frac{5}{2}$ associe \`a tout point $M$ d'affixe $z$ le point $M'$ d'affixe $z'$ d\'efinie par $$z'=\frac{5}{2}z-\frac{3}{2}-\frac{9}{2}\mathrm{i}.$$
\end{solution}\begin{solution}
L'homoth\'etie de centre $\Omega$ d'affixe $\omega=3$ et de rapport $\frac{3}{2}$ associe \`a tout point $M$ d'affixe $z$ le point $M'$ d'affixe $z'$ d\'efinie par $$z'=\frac{3}{2}z-\frac{3}{2}.$$
\end{solution}\begin{solution}
L'homoth\'etie de centre $\Omega$ d'affixe $\omega=-2$ et de rapport $-1$ associe \`a tout point $M$ d'affixe $z$ le point $M'$ d'affixe $z'$ d\'efinie par $$z'=-z-4.$$
\end{solution}
%%%%%%%%%%
\begin{solution}
La translation de vecteur $\vec{w}$ d'affixe $\omega=-1-\frac{1}{2}\mathrm{i}$ associe \`a tout point $M$ d'affixe $z$ le point $M'$ d'affixe $z'$ d\'efinie par $$\displaystyle z'=z-1-\frac{1}{2}\mathrm{i}.$$
\end{solution}\begin{solution}
La translation de vecteur $\vec{w}$ d'affixe $\omega=-\frac{9}{2}+4\mathrm{i}$ associe \`a tout point $M$ d'affixe $z$ le point $M'$ d'affixe $z'$ d\'efinie par $$\displaystyle z'=z-\frac{9}{2}+4\mathrm{i}.$$
\end{solution}\begin{solution}
La translation de vecteur $\vec{w}$ d'affixe $\omega=2+3\mathrm{i}$ associe \`a tout point $M$ d'affixe $z$ le point $M'$ d'affixe $z'$ d\'efinie par $$\displaystyle z'=z+2+3\mathrm{i}.$$
\end{solution}\begin{solution}
La translation de vecteur $\vec{w}$ d'affixe $\omega=-4-4\mathrm{i}$ associe \`a tout point $M$ d'affixe $z$ le point $M'$ d'affixe $z'$ d\'efinie par $$\displaystyle z'=z-4-4\mathrm{i}.$$
\end{solution}\begin{solution}
La translation de vecteur $\vec{w}$ d'affixe $\omega=-\frac{1}{2}+4\mathrm{i}$ associe \`a tout point $M$ d'affixe $z$ le point $M'$ d'affixe $z'$ d\'efinie par $$\displaystyle z'=z-\frac{1}{2}+4\mathrm{i}.$$
\end{solution}
\end{document}



\documentclass[a4paper]{article}
\usepackage[]{pan}
\newtheorem{Exercice}{Exercice}
      \newtheorem{solution}{Solution}
      \setlength{\parindent}{0pt}
\begin{document}
\fexos{Complexes forme exponentielle}{\term S}
\begin{minipage}{0.45\linewidth}
  \section*{Forme alg\'ebrique d'un nombre complexe sous forme exponentielle}%\subsection*{Sujets}
     D\'eterminez la forme alg\'ebrique de chacun des nombres $a$ suivants.\begin{Exercice}
$\displaystyle a=5 \sqrt{3}\mathrm{e}^{0}$.
\end{Exercice}
\begin{Exercice}
$\displaystyle a=4 \sqrt{3}\mathrm{e}^{\mathrm{i}\frac{\pi }{6}}$.
\end{Exercice}
\begin{Exercice}
$\displaystyle a=\sqrt{3}\mathrm{e}^{-\mathrm{i}\frac{\pi }{2}}$.
\end{Exercice}
\begin{Exercice}
$\displaystyle a=5\mathrm{e}^{-\mathrm{i}\frac{\pi }{3}}$.
\end{Exercice}
\begin{Exercice}
$\displaystyle a=2 \sqrt{2}\mathrm{e}^{-\mathrm{i}\frac{\pi }{4}}$.
\end{Exercice}
\end{minipage}\hfill
\begin{minipage}{0.45\linewidth}
  \section*{Forme exponentielle du produit de deux nombres complexes}%\subsection*{Sujets}
     D\'eterminez une forme exponentielle de chacun des nombres $a$ suivants.\begin{Exercice}
$\displaystyle a=\frac{3}{2}\mathrm{e}^{0}\times 4\mathrm{e}^{\mathrm{i}\frac{3 \pi }{4}}$.
\end{Exercice}
\begin{Exercice}
$\displaystyle a=\frac{3}{8}\mathrm{e}^{\mathrm{i}\pi }\times \frac{5}{7}\mathrm{e}^{\mathrm{i}\frac{3 \pi }{4}}$.
\end{Exercice}
\begin{Exercice}
$\displaystyle a=\frac{10}{9}\mathrm{e}^{-\mathrm{i}\frac{3 \pi }{4}}\times \frac{1}{3}\mathrm{e}^{-\mathrm{i}\frac{3 \pi }{4}}$.
\end{Exercice}
\begin{Exercice}
$\displaystyle a=\frac{7}{3}\mathrm{e}^{\mathrm{i}\frac{\pi }{4}}\times \frac{3}{2}\mathrm{e}^{0}$.
\end{Exercice}
\begin{Exercice}
$\displaystyle a=\frac{1}{3}\mathrm{e}^{-\mathrm{i}\frac{\pi }{6}}\times \frac{5}{4}\mathrm{e}^{\mathrm{i}\frac{\pi }{6}}$.
\end{Exercice}

\end{minipage}\\
\medskip

\begin{minipage}{0.45\linewidth}
 \section*{Forme exponentielle du quotient de deux nombres complexes}%\subsection*{Sujets}
     D\'eterminez une forme exponentielle de chacun des nombres $a$ suivants.\begin{Exercice}
$\displaystyle a=\frac{\frac{9}{2}\mathrm{e}^{0}}{\mathrm{e}^{-\mathrm{i}\frac{2 \pi }{3}}}$.
\end{Exercice}
\begin{Exercice}
$\displaystyle a=\frac{\frac{6}{7}\mathrm{e}^{\mathrm{i}\frac{\pi }{2}}}{\frac{5}{2}\mathrm{e}^{\mathrm{i}\frac{3 \pi }{4}}}$.
\end{Exercice}
\begin{Exercice}
$\displaystyle a=\frac{2\mathrm{e}^{\mathrm{i}\frac{\pi }{6}}}{8\mathrm{e}^{-\mathrm{i}\frac{\pi }{2}}}$.
\end{Exercice}
\begin{Exercice}
$\displaystyle a=\frac{\frac{5}{7}\mathrm{e}^{-\mathrm{i}\frac{\pi }{4}}}{5\mathrm{e}^{-\mathrm{i}\frac{\pi }{2}}}$.
\end{Exercice}
\begin{Exercice}
$\displaystyle a=\frac{\mathrm{e}^{\mathrm{i}\frac{\pi }{3}}}{\frac{7}{2}\mathrm{e}^{\mathrm{i}\frac{\pi }{2}}}$.
\end{Exercice} 
\end{minipage}\hfill%
\begin{minipage}{0.45\linewidth}
  \section*{Forme exponentielle d'une puissance enti\`ere d'un nombre complexe}%\subsection*{Sujets}
     D\'eterminez une forme exponentielle de chacun des nombres $a$ suivants.\begin{Exercice}
$\displaystyle a=\left(\frac{2}{5}\mathrm{e}^{-\mathrm{i}\frac{5 \pi }{6}}\right)^{-4}$.
\end{Exercice}
\begin{Exercice}
$\displaystyle a=\left(3\mathrm{e}^{-\mathrm{i}\frac{3 \pi }{4}}\right)^{3}$.
\end{Exercice}
\begin{Exercice}
$\displaystyle a=\left(\frac{1}{6}\mathrm{e}^{-\mathrm{i}\frac{\pi }{4}}\right)^{5}$.
\end{Exercice}
\begin{Exercice}
$\displaystyle a=\left(\frac{9}{8}\mathrm{e}^{\mathrm{i}\frac{2 \pi }{3}}\right)^{5}$.
\end{Exercice}
\begin{Exercice}
$\displaystyle a=\left(\frac{1}{2}\mathrm{e}^{-\mathrm{i}\frac{\pi }{4}}\right)^{2}$.
\end{Exercice}
\end{minipage}







\newpage
\subsection*{Solutions}
\begin{solution}
$\displaystyle 5 \sqrt{3}\mathrm{e}^{0}=5 \sqrt{3}$.\end{solution}
\begin{solution}
$\displaystyle 4 \sqrt{3}\mathrm{e}^{\mathrm{i}\frac{\pi }{6}}=6+2 \sqrt{3}\mathrm{i}$.\end{solution}
\begin{solution}
$\displaystyle \sqrt{3}\mathrm{e}^{-\mathrm{i}\frac{\pi }{2}}=-\sqrt{3}\mathrm{i}$.\end{solution}
\begin{solution}
$\displaystyle 5\mathrm{e}^{-\mathrm{i}\frac{\pi }{3}}=\frac{5}{2}-\frac{5 \sqrt{3}}{2}\mathrm{i}$.\end{solution}
\begin{solution}
$\displaystyle 2 \sqrt{2}\mathrm{e}^{-\mathrm{i}\frac{\pi }{4}}=2-2\mathrm{i}$.\end{solution}

\subsection*{Solutions}
\begin{solution}
$\displaystyle \frac{3}{2}\mathrm{e}^{0}\times 4\mathrm{e}^{\mathrm{i}\frac{3 \pi }{4}}=6\mathrm{e}^{\mathrm{i}\frac{3 \pi }{4}}$.\end{solution}
\begin{solution}
$\displaystyle \frac{3}{8}\mathrm{e}^{\mathrm{i}\pi }\times \frac{5}{7}\mathrm{e}^{\mathrm{i}\frac{3 \pi }{4}}=\frac{15}{56}\mathrm{e}^{-\mathrm{i}\frac{\pi }{4}}$.\end{solution}
\begin{solution}
$\displaystyle \frac{10}{9}\mathrm{e}^{-\mathrm{i}\frac{3 \pi }{4}}\times \frac{1}{3}\mathrm{e}^{-\mathrm{i}\frac{3 \pi }{4}}=\frac{10}{27}\mathrm{e}^{\mathrm{i}\frac{\pi }{2}}$.\end{solution}
\begin{solution}
$\displaystyle \frac{7}{3}\mathrm{e}^{\mathrm{i}\frac{\pi }{4}}\times \frac{3}{2}\mathrm{e}^{0}=\frac{7}{2}\mathrm{e}^{\mathrm{i}\frac{\pi }{4}}$.\end{solution}
\begin{solution}
$\displaystyle \frac{1}{3}\mathrm{e}^{-\mathrm{i}\frac{\pi }{6}}\times
\frac{5}{4}\mathrm{e}^{\mathrm{i}\frac{\pi
  }{6}}=\frac{5}{12}\mathrm{e}^{0}$.\end{solution}
\subsection*{Solutions}
\begin{solution}
$\displaystyle \frac{\frac{9}{2}\mathrm{e}^{0}}{\mathrm{e}^{-\mathrm{i}\frac{2 \pi }{3}}}=\frac{9}{2}\mathrm{e}^{\mathrm{i}\frac{2 \pi }{3}}$.\end{solution}
\begin{solution}
$\displaystyle \frac{\frac{6}{7}\mathrm{e}^{\mathrm{i}\frac{\pi }{2}}}{\frac{5}{2}\mathrm{e}^{\mathrm{i}\frac{3 \pi }{4}}}=\frac{12}{35}\mathrm{e}^{-\mathrm{i}\frac{\pi }{4}}$.\end{solution}
\begin{solution}
$\displaystyle \frac{2\mathrm{e}^{\mathrm{i}\frac{\pi }{6}}}{8\mathrm{e}^{-\mathrm{i}\frac{\pi }{2}}}=\frac{1}{4}\mathrm{e}^{\mathrm{i}\frac{2 \pi }{3}}$.\end{solution}
\begin{solution}
$\displaystyle \frac{\frac{5}{7}\mathrm{e}^{-\mathrm{i}\frac{\pi }{4}}}{5\mathrm{e}^{-\mathrm{i}\frac{\pi }{2}}}=\frac{1}{7}\mathrm{e}^{\mathrm{i}\frac{\pi }{4}}$.\end{solution}
\begin{solution}
$\displaystyle \frac{\mathrm{e}^{\mathrm{i}\frac{\pi
    }{3}}}{\frac{7}{2}\mathrm{e}^{\mathrm{i}\frac{\pi
    }{2}}}=\frac{2}{7}\mathrm{e}^{-\mathrm{i}\frac{\pi
  }{6}}$.\end{solution}

\subsection*{Solutions}
\begin{solution}
$\displaystyle \left(\frac{2}{5}\mathrm{e}^{-\mathrm{i}\frac{5 \pi }{6}}\right)^{-4}=\frac{625}{16}\mathrm{e}^{-\mathrm{i}\frac{2 \pi }{3}}$.\end{solution}
\begin{solution}
$\displaystyle \left(3\mathrm{e}^{-\mathrm{i}\frac{3 \pi }{4}}\right)^{3}=27\mathrm{e}^{-\mathrm{i}\frac{\pi }{4}}$.\end{solution}
\begin{solution}
$\displaystyle \left(\frac{1}{6}\mathrm{e}^{-\mathrm{i}\frac{\pi }{4}}\right)^{5}=\frac{1}{7776}\mathrm{e}^{\mathrm{i}\frac{3 \pi }{4}}$.\end{solution}
\begin{solution}
$\displaystyle \left(\frac{9}{8}\mathrm{e}^{\mathrm{i}\frac{2 \pi }{3}}\right)^{5}=\frac{59049}{32768}\mathrm{e}^{-\mathrm{i}\frac{2 \pi }{3}}$.\end{solution}
\begin{solution}
$\displaystyle \left(\frac{1}{2}\mathrm{e}^{-\mathrm{i}\frac{\pi }{4}}\right)^{2}=\frac{1}{4}\mathrm{e}^{-\mathrm{i}\frac{\pi }{2}}$.\end{solution}
\end{document}

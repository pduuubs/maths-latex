\begin{exercice}

 Rappel sur \og Factorielle $n$\fg{}:
$n!=1\times2\times3\times\cdots\times n$ Par convention, $0!=1$
La suite $(u_n)$ est d\'efinie pour $n\in\N^*$ par:
\[ u_n=1+\frac{1}{1!}+\frac{1}{2!}+\frac{1}{3!}+\cdots+\frac{1}{n!}\]
  \begin{enumerate}
  \item {\bf Convergence}
\begin{enumerate}
  \item D\'emontrer par r\'ecurrence que: $\forall n\in\N^*,\quad    \dfrac{1}{n!}\leq \dfrac{1}{2^{n-1}}$

\item D\'eduisez en que la suite $(u_n)$ est major\'ee par 3.
\item D\'emontrer que la suite $(u_n)$ est convergente
  \end{enumerate}
\item {\bf Irrationalit\'e de la limite}\\
On d\'efinit de plus la suite $(v_n)$ $n\in\N^*$ par: $v_n=u+\dfrac{1}{n!}$
\begin{enumerate}
\item Prouvez que $(u_n)$ et $(v_n)$ sont adjacentes.
\item Calculer $u_7$ et $v_7$ puis d\'eterminer un encadrement de la
  limite commune $L$ des deux suites. En d\'eduire une valeur approch\'ee
  de $L$ de pr\'ecision maximale.
\item Prouver par l'absurde que: $L\notin \Q$. \\
\emph{Indication: On suppose que $L=\dfrac{p}{q}$, o\`u $p$ et $q$
  sont entiers. On pourra
    utiliser l'encadrement: $\forall n\in\N^*,\ L
    \in\into{u_n}{v_n}$ puis remarquer que: $q!\times u_q\in\N$.}
\end{enumerate}

\end{enumerate}
\end{exercice}

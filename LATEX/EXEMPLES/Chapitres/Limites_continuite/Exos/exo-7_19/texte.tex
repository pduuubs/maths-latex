\begin{exercice}
Une fonction tr�s importante en th�orie des nombres est la fonction
not�e $\pi$ (rien � voir avec $3,14\ldots$), on la d�finit sur
$\mathbb{R}^+$ par:
\[\pi(x)\ \text{est le nombre de nombre premiers inf�rieurs � } x.\]
%Si vous pr�f�rez, en notant $\mathscr{P}$ l'ensemble des nombres
%premiers, alors $\pi(x)$ est le nombre d'�l�ments dans l'intersection:
%$\mathscr{P}\cap \left[0\,;x\right] $\\\noindent
Tracer la courbe de la fonction $\pi$ sur l'intervalle
$\left[0\,;47\right[$.\\
%\noindent \includegraphics{NewCourbe.24}
%\includegraphics{/home/pan/Desktop/TeX_MiX/Figures/Metapost_fig/NewCourbe.24}
\end{exercice}

\begin{exercice}
Soit $f$ la fonction d�finie sur $\mathbb{R}$ par:
\[ f(x)=x^5-5x\]
  \begin{enumerate}
  \item \'Etudier les variations de~$f$ sur $\mathbb{R}$. On
    compl�tera le tableau de variations avec les limites de~$f$
    en~$+\infty$ et~$-\infty$
  \item Justifier que l'�quation suivante  n'a aucune solution
    sur~$\left]-\infty\,;1\right[$.
    \begin{equation}
      \label{eq:tviexo24}
      f(x)=5
    \end{equation}
  \item Prouver que l'�quation \eqref{eq:tviexo24} a au moins une solution
    $\alpha$ sur~$\left]1\,;2\right[$.
  \item Justifier l'implication: $x> \alpha \implies f(x)> 5$
  \item En utilisant les variations de $f$, prouver alors que
    l'�quation  \eqref{eq:tviexo24} a exactement une solution sur $\mathbb{R}$
    dont on donnera une valeur approch�e  � $10^{-2}$ pr�s.
  \item En utilisant uniquement le tableau de variation de $f$, donner
    le nombre de solutions de $f(x)=\lambda$ en fonction de la valeur
    du r�el~$\lambda$.
   
  \end{enumerate}
\end{exercice}

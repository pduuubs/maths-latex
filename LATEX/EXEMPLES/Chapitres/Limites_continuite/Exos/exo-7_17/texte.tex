\begin{exercice}

 {\bf La R\'eunion sept. 2004}\\
{\bf Partie A: �tude d'une fonction auxiliaire}\\
Soit $\varphi$ la fonction d\'efinie sur $\R$ par 

\[\varphi(x) = \left(x^2 + x + 1\right)\text{e}^{-x} - 1.\]

\begin{enumerate} \item	\begin{enumerate} \item D\'eterminer les  limites de $\varphi$ en $- \infty$ et en $+ \infty$.

 \item �tudier le sens de variation de $\varphi$ puis dresser son tableau de variations sur $\R$.
 
\end{enumerate} 

 \item D\'emontrer que l'\'equation $\varphi(x) =  0$ admet deux solutions dans $\R$, dont l'une dans l'intervalle $[1~;~ +\infty[$, qui sera not\'ee $\alpha$.
 
  D\'eterminer un encadrement d'amplitude $10^{-2}$ de $\alpha$.

 \item En d\'eduire le signe de $\varphi(x)$ sur $\R$ et le
   pr\'esenter dans un tableau.
\end{enumerate}
 {\bf Partie B: �tude de la position relative de deux courbes}\\
On consid�re les fonctions $f$et $g$ sont d\'efinies sur $\R$ par :

\[ f(x)= (2x + 1)\text{e}^{-x}\quad \text{et} \quad  g(x) = \dfrac{2x
  + 1}{x^2 + x + 1}.\]
\begin{enumerate} \item D\'emontrer que, pour tout nombre r\'eel $x,~ f(x) - g(x) = 
\dfrac{(2x + 1)\varphi(x)}{x^2 + x + 1}$ o� $\varphi$ est la fonction \'etudi\'ee dans la \textbf{partie A}.

\item � l'aide d'un tableau, \'etudier le signe de $f(x) -g(x)$ sur $\R$.

\item  En d\'eduire la position relative des courbes  $\Cr_f$  et  $\Cr_g$.

\end{enumerate}

\end{exercice}

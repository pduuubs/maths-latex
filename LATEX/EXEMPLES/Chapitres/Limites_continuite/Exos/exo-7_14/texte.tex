\begin{exercice}

 On pourrait penser que la continuit� est une condition essentielle
pour obtenir la propri�t� des valeurs interm�diaires. Et pourtant,
trouver des contre exemples v�rifiant:
\begin{list}{$\bullet$}{}
\item $f$ d�finie sur $\R$, non continue en $0$ telle que:
\[ \forall y_0 \in\R, \exists x_0 \in\R; f(x_0)=y_0\]
\rotatebox{180}{{\small Indication: On pourra partir d'une fonction c�l�bre non d�finie en z�ro}}
\item $f$ d�finie sur $\intf{0}{1}$ non continue affine par morceaux (deux morceaux
  suffisent) telle que $f(0)=0 $ et $f(1)=2 $ et 
\[ \forall y_0 \in\intf{0}{2}, \exists x_0 \in\intf{0}{1};
f(x_0)=y_0\]
Tracer la courbe possible d'une telle fonction.
\end{list}
\end{exercice}

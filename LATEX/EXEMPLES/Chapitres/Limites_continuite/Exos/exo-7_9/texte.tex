\begin{exercice}
  \begin{enumerate}
  \item On d\'efinit les deux suites $(u_n)$ et $(v_n)$ pour $n\in\N^*$ par:
\[ u_n=\sum_{k=1}^{n}\dfrac{1}{k^2}\quad \et{}\quad v_n=u_n+\frac{1}{n} \]

Prouver que ces suites sont adjacentes. Donner une valeur approch\'ee
\`a $10^{-1}$ pr\`es de leur limite commune. Comparer � la valeur exacte: 
\[\lim_{n\to+\infty}u_n=  \sum_{k=1}^{+\infty}\dfrac{1}{k^2}=\dfrac{\pi^2}{6\phantom{^2}}\]
  


\item La suite $(u_n)$ est d\'efinie pour $n\in\N^*$ par:
\[ u_n=\sum_{k=1}^{n}\dfrac{1}{k}=\frac{1}{1}+\frac{1}{2}+\frac{1}{3}+\cdots+\frac{1}{n}\]
  \begin{enumerate}
  \item Prouver que $(u_n)$ est croissante.
  \item Prouver que: $\forall n\in\N^*,\quad u_{2n}-u_n\geq\dfrac12$
  \item Prouver par r\'ecurrence que pour tout $n$ dans $\N^*$: $u_{2^n}\geq \dfrac{n}{2}$
\item Que peut-on en d\'eduire pour la limite \'eventuelle de la suite $(u_n)$?
  \end{enumerate}
\end{enumerate}
\end{exercice}

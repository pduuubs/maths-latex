\begin{exercice}
On consid�re l'�quation $(E)$: $x^3 - 15 x - 4 = 0$ et l'in�quation:
\begin{equation}
  \label{eq:ineqexo21}
   x^3 - 15 x - 4 > 0 
\end{equation}

  \begin{enumerate}

  \item \textbf{R�solution graphique.}

    \begin{enumerate}
    \item D�montrer que l'�quation $(E)$ est �quivalente � 
l'�quation: 
\[x^2-15=\frac{4}{x}\]
\item Tracer dans un m�me rep�re, les courbes repr�sentatives des
  fonctions: 
\[x \longmapsto x^2-15 \quad\text{et}\quad x \longmapsto \frac{4}{x}\]
\item D�terminer graphiquement le nombre de solutions 
de l'�quation $(E)$. Une des solutions est un nombre entier, quelle
est sa valeur? Encadrer chacune des autres solutions $\alpha$ et~$\beta$ (avec $\alpha < \beta$) par deux entiers cons�cutifs.
\item D�montrer que l'in�quation \eqref{eq:ineqexo21} s'�crit:
  $\left]0\,;+\infty\right[$:
 \[x^2-15 >\frac{4}{x}\ \text{sur}\ \left]0\,;+\infty\right[
 \quad\text{et}\quad%
x^2-15 < \frac{4}{x}\ \text{sur}\ \left]-\infty\,;0\right[
 \]

\item R�soudre l'in�quation \eqref{eq:ineqexo21}.
    \end{enumerate}

  \item \textbf{�tude d'une fonction.}\\
Soit $f$ la fonction d�finie sur $\mathbb{R}$ par $f (x) = x^3 - 15 x -
4$, et $\mathscr{C}_f$ est sa courbe repr�sentative dans un rep�re.
\begin{enumerate}
\item Justifier la continuit� de $f$ sur $\mathbb{R}$.
\item �tudier les limites de $f$ en $-\infty$ et~$+\infty$.
\item D�terminer les variations de $f$ et dresser son tableau de variation.
\item Tracer $\mathscr{C}_f$.
\item D�montrer que l'�quation $f (x) = 0$ admet exactement trois
  solutions dans~$\mathbb{R}$.
\item L'une des solutions est un nombre entier, donner un encadrement
  d'amplitude $10^{-3}$ de chacune des autres solutions  $\alpha$ et
$\beta$ (avec $\alpha < \beta$)
\item �tudier le signe de la fonction $f$. En d�duire l'ensemble des solutions de l'in�quation~\eqref{eq:ineqexo21}.
\end{enumerate}

\item \textbf{M�thode alg�brique.}
  \begin{enumerate}
  \item D�terminer les r�els $a$, $b$ et $c$ tels que pour tout
    r�el~$x$:
\[ x^3 - 15 x - 4 = (x - 4) (ax^2 + bx + c)\]
\item R�soudre l'�quation~$(E)$.
\item R�soudre l'in�quation \eqref{eq:ineqexo21}.
  \end{enumerate}
  \end{enumerate}
\end{exercice}

\begin{exercice}
Soit $f$ la fonction d�finie sur $\left]0\,;+\infty\right[$ par:
\[ f(x)=\sin\left(\frac{1}{x}\right)\]
  \begin{enumerate}
  \item Prouvons que cette fonction n'a pas de limite en z�ro:
    \begin{enumerate}
    \item Soit $(x_n)$ la suite d�finie sur $\mathbb{N}^*$ par:
      $x_n=\frac{1}{n\pi}$. Donner la limite de $(x_n)$ et celle de $f(x_n)$.
    \item Si $f$ admet une limite en z�ro, quelle devrait-elle �tre?
    \item Reprendre les deux derni�res questions avec: $x_n'=\frac{2}{(2n+1)\pi}$
    \item Conclure. Pour rigoler, vous pouvez demander l'impossible �
      votre calculette graphique: Tracer la courbe de~$f$.
    \end{enumerate}
  \item On consid�re maintenant la fonction $g$ d�finie sur
    $\left[0\,;+\infty\right[$ par: $ g(x)=
\begin{cases} 
x\sin\left(\frac{1}{x}\right) & \text{si $x>0$,}\\
0 &\text{si $x=0$.}
\end{cases}%
$\\
\begin{enumerate}
\item Justifier que: $\forall x \in\mathbb{R}^+,\ |g(x)|\leq x$. En
d�duire que $g$ est continue en~0.
\item Prouver que $g$ n'est pas d�rivable en~0.
\item Prouver que $x \longmapsto xg(x)$ est d�rivable en~0.
\end{enumerate}

  \end{enumerate}
\end{exercice}

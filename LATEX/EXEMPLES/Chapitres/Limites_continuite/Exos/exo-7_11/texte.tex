\begin{exercice}
\noindent
{\large \textbf{\gray  Antilles-Guyane juin 2004 (4pts)}}\\
%%-----------------------
\noindent On d\'efinit les suites $\left(a_n\right)$ et $\left(b_n\right)$ par $a_0 =1,~ 
b_0 =7$ et $\left\{\begin{array}{l c l}
a_{n+1}& =&\cfrac{1}{3}\left(2a_n +b_n\right)\\
b_{n+1}& =&\cfrac{1}{3}\left(a_n +2b_n\right)\\
\end{array}\right.$

\noindent Soit D une droite munie d'un rep\`ere $\oi$. Pour tout $n \in \N$, on consid\`ere
 les points $A_n$ et $B_n$ d'abscisses respectives $a_n$ et $b_n$.

\begin{enumerate} \item Placez les points A$_0$, B$_0$, A$_1$, B$_1$, A$_2$ et B$_2$.

\item Soit $\left(u_n\right)$ la suite d\'efinie par $u_n = b_n - a_n$ pour 
tout $n \in \N$. D\'emontrez que $\left(u_n\right)$ est une
suite g\'eom\'etrique  dont on pr\'ecisera la raison et le premier terme.

\noindent Exprimez $u_n$ en fonction de $n$.

\item Comparez $a_n$ et $b_n$. \'Etudiez le sens de variation des suites $\left(a_n\right)$
 et $\left(b_n\right)$. Interpr\'etez g\'eom\'etriquement ces r\'esultats.

\item D\'emontrez que les suites $\left(a_n\right)$ et $\left(b_n\right)$ sont adjacentes.

\item Soit $\left(v_n\right)$ la suite d\'efinie par $v = a_n + b_n$ pour tout 
$n \in \N$.  D\'emontrez que $\left(v_n\right)$ est une suite constante. En d\'eduire
 que les segments $\left[A_nB_n\right]$ ont tous le m\^eme milieu I.

\item Justifiez que les suites $\left(a_n\right)$ et $\left(b_n\right)$ sont convergentes et 
calculez leur limite. Interpr\'etez g\'eom\'etriquement ce r\'esultat.

\end{enumerate}
 
\end{exercice}

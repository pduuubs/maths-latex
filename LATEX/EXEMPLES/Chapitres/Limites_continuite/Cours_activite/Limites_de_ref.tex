\documentclass[a4paper]{article}
\usepackage[]{panpan}

\begin{document}
\fcours{Limites de r�f�rence}
\chead{} \rhead{}
\lhead{}
\rule{0cm}{0.3cm}\\
\noindent 

%\chead{ \begin{LARGE} \textbf{Formulaire sur les suites} \end{LARGE}}
%\lhead{}
%\rhead{}
%\cfoot{}
\noindent Voici les courbes des fonctions carr�, cube et inverse. Leurs limites sont � conna�tre. \\\rule{0cm}{0.3cm}\\
\psfrag{O}{$O$} \psfrag{i}{$\vi$}   \psfrag{j}{$\vj$}
\resizebox{4cm}{!}{\includegraphics{Courbe.2} }%\includegraphics[totalheight=6cm]{Courbe.2} %Para
\hfill \resizebox{2.8cm}{!}{\includegraphics{Courbe.6} }%\includegraphics[totalheight=6cm]{Courbe.6} %Cube
\hfill \resizebox{6cm}{!}{\includegraphics{Courbe.3} }
%\includegraphics[totalheight=6cm]{Courbe.3} %Hyper



\begin{minipage}{0.22\textwidth}
\begin{align*}
 \lim_{x\to +\infty}\,x^2&=+\infty\\
\lim_{x\to -\infty}\,x^2&=+\infty
\end{align*}
\end{minipage} \hfill
\begin{minipage}{0.32\textwidth}
\begin{align*}
 \lim_{x\to +\infty}\,x^3&=+\infty\\
\lim_{x\to -\infty}\,x^3&=-\infty
\end{align*}
\end{minipage} \hfill
\begin{minipage}{0.18\textwidth}
\begin{align*}
\lim_{x\to +\infty}\,\frac{1}{x}&=0\\ 
\lim_{x\to -\infty}\,\frac{1}{x}&=0
\end{align*}
\end{minipage}\hfill 
\begin{minipage}{0.18\textwidth}
\begin{align*}
\lim_{x\to 0^+}\,\frac{1}{x}&=+\infty\\ 
\lim_{x\to 0^-}\,\frac{1}{x}&=-\infty
%\lim_{\stackrel{x \to 0}{x<0}}\frac{1}{x}&=-\infty
\end{align*}
\end{minipage} 


\end{document}  

\par \noindent
\renewcommand{\arraystretch}{2.5}
\begin{tabular}{||c|p{5.45cm}|p{5.5cm}||}
\hline 
%Ligne 1
  Suite $(u_n)$, $n\in \N$ & \hspace{\stretch{1}} Suite \textbf{arithm�tique} de raison $r$  \hspace{\stretch{1}} & \hspace{\stretch{1}}  Suite \textbf{g�om�trique} de raison $q$ \hspace{\stretch{1}} \\\hline \hline
%Ligne 2
D�finition & On passe de chaque terme au suivant en ajoutant le m�me r�el $r$ \[u_{n+1}=u_n+r \] & On passe de chaque terme au suivant en multipliant par le m�me r�el $q$ \[u_{n+1}=q\times u_n \]\\ \hline
%Ligne 3
 Terme g�n�ral& \[u_n=u_0+nr\]\[u_n=u_1+(n-1)r\] & \[u_n=u_0\times q^n\]\[u_n=u_1\times q^{n-1}\] \\\hline
%Ligne 4 
Somme de termes & ``Nombre de termes''$\times$ ``Moyenne du premier et dernier terme'' 
\[ \sum_{k=0}^{n}u_k=(n+1)\times \frac{u_0+u_n}{2}\]& ``Premier terme''$\times \dfrac{1-q^{\text{nb de termes}}}{1-q}$ \[ \sum_{k=0}^{n}u_k=u_0\times \frac{1-q^{n+1}}{1-q}\] 
\\\hline

 Cas particuliers& \[1+2+\cdots+n=\dfrac{n(n+1)}{2}\]& \[1+q+q^2+\cdots+q^n=\frac{1-q^{n+1}}{1-q}\]\\\hline
\end{tabular}


 



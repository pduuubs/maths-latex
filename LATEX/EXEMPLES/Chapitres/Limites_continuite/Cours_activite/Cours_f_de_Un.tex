

\documentclass[a4paper]{article}
\usepackage[]{pan}
\usepackage{pstricks}
\usepackage{pst-node} 
\begin{document}
\fcours{Continuit�-Application � $u_{n+1}=f(u_n)$}%{\term}

%%-----------------------
{\tiny
\section{Continuit�}
\section{Th�or�me des valeurs interm�diaires} 
\section{Applications}
\subsection{Racine d'un polyn�me de degr� impair}
\subsection{Fonctions r�ciproques} }
\vspace{0.2cm}
\hrule
\vspace{0.2cm}
\subsection{$u_{n+1}=f(u_n)$}
On se donne une fonction $f$ d�finie, continue sur un intervalle $I$ de
$\R$. Soit $u_0\in I$.
\subsubsection{Intervalle stable}
\Def{On dit que $I$ est stable par $f$ si $f(I)\subset I$}
\exemples{
  \begin{enumerate}
  \item $f(x)=x(1-x)$ sur $I=\intf{0}{1}$. $I$ est stable par f, mais
pas par $g=5f$. Il suffit d'�tudier $f$.
\item $f(x)=\frac{1}{x-1}$ et $u_0=\frac32$ V�rifier que la relation
  $u_{n+1}=f(u_n)$ ne d�finit pas une suite. 
  \end{enumerate}
}
La condition $I$ stable par $f$ permet de garantir que la suite est
d�finie et que tous les termes de la suite sont dans l'intervalle $I$.
(r�currence imm�diate)

\subsubsection{Sens de variation}
\begin{prop}
  \begin{enumerate} 
  \item $ \forall x \in I,\ f(x)-x\geq 0 \implies$ $(u_n)$ croissante
  \item $\iff \forall x \in I,\ f(x)-x\leq 0 \implies$ $(u_n)$ d�croissante 
  \end{enumerate}
\end{prop}

\proof{Facile}
\begin{prop}
\begin{enumerate} 
  \item $f$ croissante $\implies$ $(u_n)$ monotone. Le sens de var est
    donn� par le signe de $u_1-u_0$
  \item $f$ d�croissante $\implies$ $(u_{2n})$ et $(u_{2n+1})$ sont
    monotones de monotonies contraires
\end{enumerate}
\end{prop}
\proof{ Supposons $u_1>u_0$.
\begin{enumerate} 
  \item Par r�currence. $\Para (n)$: \og $u_{n+1}-u_n\geq 0$\fg{}
  \item On pose $p_n=u_{2n}$ et $i_n=u_{2n+1}$. Alors:
\[p_{n+1}=fof(p_n) \quad \et{}\quad i_{n+1}=fof(i_n)\]
Or $f$ dec. implique $fof$ croissante donc par le 1. $p$ et $i$ sont
monotones.
Supposons $p$ croissante, alors $u_2\geq u_0$ donc, en appliquant $f$
qui est dec. on a $u_3\leq u_1$ donc $i_1 \leq i_0$ donc $i$ est dec.
\end{enumerate}
}

\subsubsection{Convergence}
\Def{On appelle point fixe d'une fonction $f$ un r�el tel que $f(x)=x$}
\begin{theo}Si $f$ est continue sur $I$ et que $(u_n)$ converge, alors la
  limite est n�cessairement un point fixe de $f$
\end{theo}
Attention: L'existence d'un point fixe ne garantit pas la convergence
de $(u_n)$. Mais l'absence de point fixe suffit � justifier que $(u_n)$
ne converge pas.
En pratique: On justifie que $(u_n)$ CV (croissante major�e par
exemple) puis on d�termine la limite en cherchant les points fixes de
$f$. Si le point fixe est unique, c'est facile, sinon il faut
raisonner avec le sens de variation de $(u_n)$.

\exemple{$u_{n+1}=\frac{u_n}{2-u_n}$ et $u_0 \in \intf{0}{1}$}





\end{document}
\label{sec:para1}

\begin{pspicture}(0,-0.5)(5,4.5)
\rput(2.2,4){{\it diagramme de Venn}}
\rput(2.2,3){$A\cup B$}
\psframe(0,0)(4.5,3.6)
\pscircle[fillstyle=hlines](1.5,1.5){1}
\pscircle[fillstyle=hlines](3,1.5){1}
\rput(1.5,1.5){A}
\rput(3,1.5){B}
\end{pspicture}
\begin{pspicture}(0,-0.5)(5,4.5)
\rput(2.2,4){{\it diagramme de Venn}}
\rput(2.2,3){$A\cap B$}
\psclip{% bloc colorier
\pscustom{% 1re partie colorier
\pscircle(1.5,1.5){1}}
\pscustom{% 2me partie
\pscircle(3,1.5){1}}
\psframe[fillstyle=hlines](0,0)(4,3)} % hachurer l'intersection des 2
% parties qui se trouvent dans le frame indiqué
\endpsclip % fin bloc colorier (pas de } fermant)  %% style=hachured

\psframe(0,0)(4.5,3.6)
\pscircle(1.5,1.5){1}
\pscircle(3,1.5){1}
\rput(1.5,1.5){A}
\rput(3,1.5){B}
\end{pspicture}\\
Bla bla
%%-----------------------
\end{document}



\documentclass[a4paper]{article}
\usepackage[]{pan}
\usepackage{pstricks}
\usepackage{pst-node} 
\begin{document}
\fcours{D�rivabilit�}%{\term}

%%-----------------------

\section{D�finition}
\Def{Soit $a\in I$ o� $I$ est un intervalle \emph{ouvert}. Soit $f$
  une fonction d�finie sur $I$. On dit que $f$ est d�rivable en $a$
  si:
\[\frac{f(a+h)-f(a)}{h}\quad \text{admet une limite finie quand $h$ tend
  vers 0}\]
Si cette limite finie existe, on l'appelle le nombre d�riv� de $f$ en
$a$, not�~$f'(a)$}
Remarque: Cette def signifie que
$\frac{f(a+h)-f(a)}{h}=f'(a)+\varepsilon(h)$ o� $\varepsilon(h)$ tend vers 0 en~0
\begin{prop}{\bf Approximation affine} On en d�duit que si $f$ est d�rivable
en $a$, et que $h$ est proche de z�ro: $f(a+h)\approx f(a)+hf'(a)$
\end{prop}
\exemples Pour $x$ proche de 0: $\e^x\approx x+1$ 
$\sin x \approx x$, $(1+x)^n\approx 1+nx$\\

Cette propri�t� doit son nom au fait que faire cette approximation
revient � confondre au point d'abscisse $(a+h)$ la courbe avec sa
tangente en $a$: 
\proof{Eq de la tangente. ordonn�e pour $x=a+h$}
\Def{Soit $I$  un intervalle \emph{ouvert}. Soit $f$
  une fonction d�finie sur $I$. On dit que $f$ est d�rivable sur $I$
  si elle est d�rivable en tout en $a$ de $I$.
Pour prouver qu'une fonction est d�rivable en un point, il faut donc
calculer une limite. En particulier lorsque ce point est aux bornes
d'un intervalle ferm�.
Pour prouver que $f$ est d�rivable sur un intervalle ouvert, en
g�n�ral on utilise que:

\prop{La compos�e de fonctions d�rivables est d�rivable sur tout
  intervalle ouvert o� la fonction est d�finie.}
\proof{Admis}
\section{Propri�t�s}
\prop{Une fonction d�rivable en $a$ est continue en $a$}
\proof{{\bf \`A conna�tre!}$f$ d�rivable en $a$ signifie que:
\[\frac{f(a+h)-f(a)}{h}=f'(a)+\varepsilon(h)\]
o� $\varepsilon(h)$ tend vers 0 en 0. D'o� pour $h$ non nul:
\[ f(a+h)=hf'(a)+h\varepsilon(h)+f(a)\]
D'o� la limite voule, et $f$ est continue en $a$.
}


\section{Exemples, contre exemples}
{\bf Attention!}
La r�ciproque de la propri�t� pr�c�dente est fausse:
Valeur absolue et racine.
On a m�me construit des fonctions continues sur [0;1] nulle part
d�rivable.

Soit $f(x)=x\sin(\frac{1}{x})$ pour $x\in\into{0}{+\infty}$ et
$f(0)=0$
Prouver que $f$ est continue sur $\R^+$. Est-elle d�rivable en 0?
M�me question avec $g(x)=xf(x)$. 
Retour aux exos de la feuille avec $\e^{-\frac{1}{x}}$

%%-----------------------
\end{document}



\documentclass[a4paper]{article}
\usepackage[]{panpan}
\renewcommand{\O}{\Omega}
\begin{document}
\setcounter{Aff}{1}
\fexost{Exponentielle}
\chead{\Large{\textbf{Fonction exponentielle}}}
\rhead{}
%\lfoot{}
%\rfoot{}
%\Titre{}
%%%%%____________________________________________________________
\Exo[]{En posant $X=\e^x$ r�soudre dans $\R$ les �quations suivantes:
\begin{enumerate}
\item \alignetrois{a) $3\e^{2x}-\e^x-2=0$}{b) $\e^{2x}+\e^x-2=0$}{c) $\e^{2x}-\e^{x+1}-\e^x+\e=0$}
\item \alignetrois{a) $\e^{3x}-\e^x=0$}{b) $2\e^{x}+2\e^{-x}+5=0$}{c) $\e^{x}-\e^{-x}=0$}
\end{enumerate}}
%%%%%____________________________________________________________
\Exo[]{\textbf{Exponentielle et tangentes}\\
On note $\Cr$ la courbe de la fonction exponentielle dans un rep�re orthonormal.
  \begin{enumerate}
  \item D�terminer l'�quation de $\Dr$, la tangente � $\Cr$ au point d'abscisse~$0$.
  \item On veut prouver que $\Cr$ est au dessus de $\Dr$. On pose pour $x\in\R$:
\[ \Xi(x)=\e^x -x-1\]
\begin{enumerate}
\item Etudier les variations de $\Xi$ sur $\R$.
\item En d�duire pour conclure l'in�galit� suivante (� retenir):
  \begin{equation}
    \label{eq:tang}
    \forall x \in\R,\quad \e^x \geq x+1
  \end{equation}


\end{enumerate}

\item On passe au cas g�n�ral. Soit $a\in\R$. On note $\Dr_a$ la
  tangente � $\Cr$ au point d'abscisse~$a$. On veut prouver que $\Cr$ est au dessus de $\Dr_a$.
  \begin{enumerate}
  \item Prouver que l'assertion \og $\Cr$ est au dessus de $\Dr_a$\fg{} �quivaut �:
\[ \forall x \in\R,\quad \e^{x-a} \geq x-a+1\]
\item Conclure � l'aide de \eqref{eq:tang}.
  \end{enumerate}
    
  \end{enumerate}
La courbe de la fonction exponentielle est donc au dessus de chacune
de ses tangentes. On dit que la fonction exponentielle (comme la
fonction carr�) est \emph{convexe}.
}
%%%%%____________________________________________________________
%\Exo[]{ R�soudre les �quations diff�rentielles suivantes
%  \begin{enumerate}
%  \item $y'=2y$ et $y(0)=1$
%  \end{enumerate}
%}
%%%%%____________________________________________________________
\Exo[]{ 
\textbf{Application aux limites}
 \begin{enumerate}

 \item Limites de $\e^x$.
    \begin{enumerate}
    \item D�duire de \eqref{eq:tang} une preuve de la limite de
    $\e^x$ en $+\infty$.

     \item D�duire de \eqref{eq:tang} que pour tout r�el $x$ tel que
    $x<1$:
       \begin{equation}
         \label{eq:mino}
\e^x\leq\frac{1}{1-x}
       \end{equation}

     \item En d�duire la limite de $\e^x$ en $-\infty$.
    \end{enumerate}

\item Encadrement de $\e$.\label{enc}
    \begin{enumerate}
    \item D�duire de \eqref{eq:tang} que pour tout $n\in\N^*$:
    \[\left( 1+\frac{1}{n}\right)^n\leq \e\]
    \emph{Indication: }$\frac{1}{n}\times n=1$
    \item D�duire de \eqref{eq:mino} que pour tout $n\in\N^*$:
    \[\e\leq\left( 1+\frac{1}{n}\right)^{n+1}\]
    \end{enumerate}
\item Une suite qui converge vers $\e$ \\
    On consid�re la suite $(u_n)$ d�finie par: 
    \[ n\in\N^*,\quad u_n=\left( 1+\frac{1}{n}\right)^n \]
    \begin{enumerate}
    \item D�montrer que pour tout entier $n\in\N^*$:
    \[ 0\leq \e- u_n\leq \frac{\e}{n}\]
    \item En d�duire que la suite $(u_n)$ converge vers $\e$
    \item Calculer � l'aide de la calculatrice et de la suite $(u_n)$ une
      valeur approch�e � $10^{-6}$ pr�s du nombre r�el~$\e$.
    \end{enumerate}%
\end{enumerate}%
}

\end{document}




\begin{exercice}
\textbf{Caract\'erisation de  l'exponentielle.}\\
Le but de cet exercice est de d\'eterminer toutes les fonctions $f$
(diff\'erentes de la fonction nulle)
d\'efinies et d\'erivables sur $\R$ v\'erifiant la relation fonctionnelle\footnote{On dit qu'une telle fonction $f$ est un \emph{morphisme} du groupe additif
$\R$ dans le groupe multiplicatif~$\R^*$}:
\begin{equation}
  \label{eq:fonc}
  \forall a\in\R,\forall b\in\R,\ f(a+b)=f(a)f(b) 
\end{equation}

On consid\`ere une fonction $f$ non identiquement nulle v\'erifiant \eqref{eq:fonc}.
 Soit $b\in\R$. On consid\`ere la fonction $\Psi_b$ de variable~$x$ et de param\`etre r\'eel~$b$ d\'efinie par:
\[\forall x\in\R,\ \Psi_b(x)=f(x+b)\]
\begin{enumerate}
\item Prouver que si $f(0)=0$, alors $f$ est la fonction
  nulle. Combien vaut donc $f(0)$?
\item Calculer $\Psi_b'(x)$ de deux mani\`eres.
\item En d\'eduire que pour tout r\'eel $b$ la fonction $f$ v\'erifie:
  $ f'(b)=f'(0)f(b)$

\item Justifier alors que $f$ a n\'ecessairement une
  expression de la forme: $f(x)=\e^{\alpha x}$
o\`u $\alpha$ est une constante que l'on pr\'ecisera.

\item R\'eciproquement, justifier que toute fonction $g$ dont
  l'expression est de la forme \mbox{$g(x)=\e^{\alpha x}$} o\`u
  $\alpha$ est une constante r\'eelle v\'erifie la relation~\eqref{eq:fonc}
\item Enoncer la propri\'et\'e d\'emontr\'ee.
\end{enumerate}
  %\item Quelles sont les deux valeurs possibles pour $f(0)$?
  %\item Prouver que si $f(0)=0$ alors $f$ est la fonction nulle. ($\forall x\in\R,\ f(x)=0$)
  %\item On suppose \`a partir de maintenant que $f$ n'est pas la
   % fonction nulle. Prouver que $f$ ne s'annule jamais.


 
\end{exercice}

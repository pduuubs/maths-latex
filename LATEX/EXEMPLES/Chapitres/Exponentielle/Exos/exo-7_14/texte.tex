\begin{exercice}

 \textbf{Exponentielle et tangentes}\\
\noindent On note $\Cr$ la courbe de la fonction exponentielle dans un rep�re orthonormal.
  \begin{enumerate}
  \item D�terminer l'�quation de $\Dr$, la tangente � $\Cr$ au point d'abscisse~$0$.
  \item On veut prouver que $\Cr$ est au dessus de $\Dr$. On pose pour $x\in\R$:
\[ \Xi(x)=\e^x -x-1\]
\begin{enumerate}
\item Etudier les variations de $\Xi$ sur $\R$.
\item En d�duire pour conclure l'in�galit� suivante (� retenir):
  \begin{equation}
    \label{eq:tang}
    \forall x \in\R,\quad \e^x \geq x+1
  \end{equation}


\end{enumerate}

\item On passe au cas g�n�ral. Soit $a\in\R$. On note $\Dr_a$ la
  tangente � $\Cr$ au point d'abscisse~$a$. On veut prouver que $\Cr$ est au dessus de $\Dr_a$.
  \begin{enumerate}
  \item Prouver que l'assertion \og $\Cr$ est au dessus de $\Dr_a$\fg{} �quivaut �:
\[ \forall x \in\R,\quad \e^{x-a} \geq x-a+1\]
\item Conclure � l'aide de \eqref{eq:tang}.
  \end{enumerate}
    
  \end{enumerate}
La courbe de la fonction exponentielle est donc au dessus de chacune
de ses tangentes. On dit que la fonction exponentielle (comme la
fonction carr�) est \emph{convexe}.
\end{exercice}

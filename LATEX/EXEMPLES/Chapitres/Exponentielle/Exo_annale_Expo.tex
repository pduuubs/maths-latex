

\documentclass[a4paper]{article}
\usepackage[]{pan}

\begin{document}
\fexos{Exponentielle (Annales Bac)}{\term S}
\lfoot{}
\cfoot{}
%%-----------------------
\Exo[]{
\textsl{Nov. 2005 Am\'erique du Sud} \textbf{Ex.\no 4 Partie A}\\
On consid\`ere les fonctions $f$ et $g$ d\'efinies sur $\R$ par
\[ f(x) = \text{e}^{-x^2} \quad  \text{et}\quad  g(x) = x^2\text{e}^{-x^2}.\]

On note respectivement $\mathcal{C}_{f}$ et $\mathcal{C}_{g}$ les courbes repr\'esentatives de $f$ et $g$ dans un rep\`ere orthogonal $\oij$, dont les trac\'es se trouvent sur la feuille annexe.

\begin{enumerate}
\item  Identifier $\mathcal{C}_{f}$ et $\mathcal{C}_{g}$ sur la figure fournie. (Justifier la r\'eponse apport\'ee).
\item \'Etudier la parit\'e des fonctions $f$ et $g$.
\item \'Etudier le sens de variation de $f$ et de $g$. \'Etudier les limites \'eventuelles de $f$ et de $g$ en $+\infty$.
\item \'Etudier la position relative de $\mathcal{C}_{f}$ et $\mathcal{C}_{g}$.
 \end{enumerate}
} 


%%-----------------------
\Exo[]{\textsl{Mai 2006 Am\'erique du Nord}
\textbf{Ex. \no 4 Partie B. \'Etude d'une fonction}

\noindent Soit $g$ la fonction d\'efinie sur $[0~;~ + \infty[$ par $g(x) =  2\left(\dfrac{\text{e}^{4x} - 1}{\text{e}^{4x} + 1}\right)$  et $\left(\mathcal{C}_{g}\right)$ sa courbe repr\'esentative.
\begin{enumerate}
\item \begin{enumerate}
\item Montrer que $\left(\mathcal{C}_{g}\right)$ admet une asymptote $\Delta$ dont on donnera une \'equation.
\item \'Etudier les variations de $g$ sur $[0~;~ + \infty[$.
\end{enumerate}
\item D\'eterminer l'abscisse $\alpha$ du point d'intersection de $\Delta$ et de la tangente \`a $\left(\mathcal{C}_{g}\right)$ \`a l'origine.
\item Tracer, dans le rep\`ere de l'annexe,  la courbe $\left(\mathcal{C}_{g}\right)$ et les \'el\'ements mis en \'evidence dans les questions pr\'ec\'edentes de cette partie B.
\end{enumerate}
}
%-------------------------
\Exo[]{\textsl{Juin 2006 Asie} \textbf{Ex. \no 4} \\
\noindent\textbf{Partie A}

\noindent On consid\`ere l'\'equation diff\'erentielle 
\[(\text{E}) : \quad  y'+ y = \text{e}^{- x}.\]

\begin{enumerate}
\item  D\'emontrer que la fonction $u$ d\'efinie sur l'ensemble $\R$ des nombres r\'eels par $u(x) = x\text{e}^{- x}$ est une solution de (E).
\item R\'esoudre l'\'equation diff\'erentielle (E$_{0}) : \quad  y'+ y = 0$.
\item D\'emontrer qu'une fonction $v$, d\'efinie et d\'erivable sur $\R$, est solution de (E) si et seulement si $v - u$ est solution de (E$_{0}$).
\item En d\'eduire toutes les solutions de (E).
\item D\'eterminer la fonction $f_{2}$, solution de (E), qui prend la valeur $2$ en $0$.
\end{enumerate}
 
\medskip

\noindent \textbf{Partie B}

\noindent  $k$ \'etant un nombre r\'eel donn\'e, on note $f_{k}$ la fonction d\'efinie sur l'ensemble $\R$ par :
\[f_{k}(x) = (x + k)\text{e}^{- x}.\]
\noindent  On note $\mathcal{C}_{k}$ la courbe repr\'esentative de la fonction $f_{k}$ dans un rep\`ere orthonormal $\oij$.
\begin{enumerate}
\item  D\'eterminer les limites de $f_{k}$ en $-\infty$ et $+ \infty$.
\item Calculer $f_{k}'(x)$ pour tout r\'eel $x$.
\item En d\'eduire le tableau de variations de $f_{k}$.
}
%%-----------------------

\hfill Annexe Exo 1: \includegraphics[totalheight=5cm]{courbe_bac_2006_jpg}
\end{document}

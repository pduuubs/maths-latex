\begin{exercice}

 Pour tout r\'eel $k$ strictement positif, on consid\`ere la fonction $f_{k}$ d\'efinie 
sur $[0~ ;~ + \infty[$ par : 

\[f_{k}(x) = \ln (\text{e}^x + kx) - x.\] 

\noindent Soit $\mathcal{C}_{k}$ la courbe repr\'esentative 
de la fonction $f_{k}$ dans le plan muni d'un rep\`ere orthogonal 
$\Oij$, (unit\'es graphiques : 5 cm sur l'axe des abscisses et 10 cm sur l'axe des 
ordonn\'ees).\\ 
\textbf{\'Etude pr\'eliminaire : mise en place d'une in\'egalit\'e.}\\ 
On consid\`ere la fonction $g$ d\'efinie sur [0 ; + $\infty$[ par : 
\[g(x) = \ln (1 + x) - x.\] 

\begin{enumerate} \item \'Etudier le sens de variation de $g$. 

\item En d\'eduire que pour tout r\'eel $a$ positif ou nul 
$\ln (1 + a) \leqslant a$. 

\end{enumerate} 

\vspace{0,25cm} 

\noindent \textbf{Partie A : \'Etude de la fonction \boldmath $f_{1}$ 
\unboldmath d\'efinie sur 
\boldmath $[0~;~ + \infty[$ \unboldmath par~ \boldmath $f_{1}(x) = \ln \left(\mathrm{e}^x + x\right) - x$\unboldmath.} 

\begin{enumerate} \item Calculer $f'_{1}(x)$ pour tout r\'eel $x$ appartenant 
\`a l'intervalle [0 ; + $\infty$[ et en d\'eduire le sens de variation de la 
fonction $f_{1}$. 

\item Montrer que pour tout r\'eel $x$ appartenant \`a 
l'intervalle [0 ; + $\infty[,~\\
 f_{1}(x) = \ln \left(1 + 
\dfrac{x}{\text{e}^x}\right)$. En 
d\'eduire la limite de $f_{1}$ en + $\infty$. 

\item Dresser le tableau de variation de $f_{1}$. 

\end{enumerate} 

\vspace{0,25cm} 

\noindent \textbf{Partie B : \'Etude et propri\'et\'es des fonctions 
\boldmath $f_{k}$\unboldmath .} 

\begin{enumerate} \item Calculer $f_{k}(x)$ pour tout r\'eel $x$ appartenant \`a 
l'intervalle [0 ; + $\infty$[ et en d\'eduire le sens de variation de la fonction 
$f_{k}$. 

\item Montrer que pour tout r\'eel $x$ appartenant \`a 
l'intervalle [0 ; + $\infty[,~\\f_{k}(x) = \ln \left(1 + 
k\dfrac{x}{\text{e}^x}\right)$. En d\'eduire la limite de $f_{k}$, en + 
$\infty$. 

\item \begin{enumerate} \item Dresser le tableau de variation 
de $f_{k}$. 

\item Montrer que pour tout r\'eel $x$ de l'intervalle [0 ; + $\infty[$, 
on a $f_{k}(x) \leqslant \dfrac{k}{\text{e}}$. 

\end{enumerate} 

\item D\'eterminer une \'equation de la tangente $T_{k}$ 
\`a $\mathcal{C}_{k}$ au point O. 

\item Soit $p$ et $m$ deux r\'eels strictement positifs tels 
que $p < m$. \'Etudier la position relative de $\mathcal{C}_{p}$ et 
$\mathcal{C}_{m}$. 

\item Tracer les courbes $\mathcal{C}_{1}$ et 
$\mathcal{C}_{2}$ ainsi que leurs tangentes respectives $T_{1}$ 
et $T_{2}$ en O. 

\end{enumerate} 
\end{exercice}

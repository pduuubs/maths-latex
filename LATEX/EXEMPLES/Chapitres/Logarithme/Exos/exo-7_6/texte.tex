\begin{exercice}

 \begin{enumerate}
\item \'Etablir que pour tout $n$ de $\N^*$ et tout $x$ de $\intf{0}{1}$ :

\[\dfrac{1}{n+1} \leqslant \dfrac{1}{x + n} \leqslant \dfrac{1}{n}.\]

%\item 
%\begin{enumerate}\item Calculer $\displaystyle\int_{0}^1 \dfrac{1}{x + n}\,\text{d}x$.

\item En d\'eduire  que :

\[\text{pour}~n \in \N^* \quad \dfrac{1}{n+1} \leqslant \ln
\left(\dfrac{n+1}{n}\right)\leq \dfrac{1}{n}\]

%\[ \text{puis que}\quad \ln \left(\dfrac{n+1}{n}\right) \leqslant \dfrac{1}{n}.\]

%\end{enumerate}

\item On appelle $U$ la suite d\'efinie pour $n \in \N^*$ par :

\[U(n) = \displaystyle\sum_{k=1}^{k=n} \dfrac{1}{k} - \ln (n) = 1 + \dfrac{1}{2} + \dfrac{1}{3} + \cdots + \dfrac{1}{n} - \ln (n).\]

D\'emontrer que $U$ est d\'ecroissante.% (on pourra utiliser \textbf{2. b.}.)

\item On d\'esigne par $V$ la suite de terme g\'en\'eral :

\[V(n) = \displaystyle\sum_{k=1}^{k=n} \dfrac{1}{k} - \ln (n + 1)  = 1 + \dfrac{1}{2} + \dfrac{1}{3} + \cdots + \dfrac{1}{n} - \ln (n + 1).\]
D\'emontrer que $V$ est croissante.

\item  D\'emontrer que $U$ et $V$ convergent vers une limite commune not\'ee $\gamma$.

\item Pour avoir un encadrement de $\gamma$ d'amplitude au plus �gal �
  $10^{-2}$  � l'aide de $U_n$ et $V_n$, quelle est la plus petite valeur de $n$ qu'il faut choisir?
D\'eterminer une valeur approch\'ee de $\gamma$  \`a $10^{-2}$ pr\`es par la m\'ethode de votre choix.

\end{enumerate}

\end{exercice}

\begin{exercice}
\noindent \textbf{Partie A}\\
Soit $f$ la fonction d\'efinie sur l'intervalle $]1 ~;~ +\infty[$ par 
\[f (x) = \dfrac{x}{\ln x}\]

\begin{enumerate}
\item  \begin{enumerate}
\item D\'eterminer les limites de la fonction $f$ en $1$ et en $+\infty$.
\item \'Etudier les variations de la fonction $f$.
\end{enumerate}
\item Soit $\left(u_{n}\right)$ la suite d\'efinie par $u_{0} = 5$ et $u_{n+1}  =f\left(u_{n}\right)$ pour tout entier naturel $n$.
\begin{enumerate}
\item  On a trac\'e ci-dessous $\mathscr{C}$, la courbe repr\'esentative  de la
  fonction $f$ .
% sur la figure donn\'ee en annexe qui sera rendue avec la copie. 
Construire la droite d'\'equation $y = x$ et les points $M_{1}$ et $M_{2}$ de la courbe $\mathscr{C}$ d'abscisses respectives $u_{1}$ et~$u_{2}$. Proposer une conjecture sur le comportement de la suite $\left(u_{n}\right)$.
\item D\'emontrer que pour tout entier naturel $n$, on a: $u_{n} \geqslant  \text{e}$. \\ (\textit{On pourra utiliser la question {\bf 1.b.}}).
\item D\'emontrer que la suite $\left(u_{n}\right)$ converge vers un r\'eel  $\ell$ de l'intervalle $[\text{e} ~�;~ +\infty[$.
\end{enumerate}
\end{enumerate}

\bigskip

\noindent \textbf{Partie B}\\
On rappelle que la fonction $f$ est continue sur l'intervalle $]1 ~;~ +\infty[$.
\begin{enumerate}
\item  En \'etudiant de deux mani\`eres la limite de la suite $\left(f\left(u_{n}\right)\right)$, d\'emontrer que $f(\ell) = \ell$.
\item En d\'eduire la valeur de $\ell$.
\end{enumerate}
 
\vspace{1cm}
\begin{center}

\begin{pspicture}(0,0)(12,12)

  \psgrid[griddots=10,gridlabels=0pt, subgriddiv=0]
\psset{xunit=2cm,yunit=2cm}
  \psaxes(0,0)(0,0)(6,6)
  \psplot[%linecolor=gris, 
linewidth=0.5pt]%
     {1.23}{6}{x dup ln div}
\put(2.2,9.2){$\Cr$}
%neg -1 atan 180 sub 180 div 3.1416 mul}
\end{pspicture}
\end{center}
 
\end{exercice}

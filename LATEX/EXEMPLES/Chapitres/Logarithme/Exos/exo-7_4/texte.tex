\begin{exercice}
{\bf R.O.C.}\\
\noindent
\framebox[\width]{
\parbox{0.5\textwidth}{%
{\bf Pr�requis�:}\\
La fonction $\ln$ est la r�ciproque de $\exp$. \\ \noindent Propri�t�s alg�briques
de l'exponentielle.
}
}\vspace{0.4cm}\\
Soient $a$ et $b$ deux r�els strictement positifs.
\begin{enumerate}
\item D�montrer que:
\[ \ln(ab)=\ln(a)+\ln(b) \]

\item Utiliser ce r�sultat pour prouver que:
  \[\ln(\frac{1}{b})=-\ln(b)\]

 %les deux suivants:
  % \begin{eqnarray}
%     \label{eq:1}
%     \ln(\frac{1}{b})&=&-\ln(b)\\
%     \ln(\frac{a}{b})&=&\ln(a)-\ln(b)
%   \end{eqnarray}
\item On donne l'encadrement: $0,69<\ln(2)<0,7$. %  et $1,09<\ln(3)<1,10$.
En d�duire des encadrements de $\ln(4)$  et $\ln(\frac18)$
  \end{enumerate}
 
\end{exercice}

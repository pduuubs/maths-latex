\begin{exercice}

 Soit $f$ la fonction d�finie sur $I=\into{0}{+\infty}$ par:
\[ f(x)=x\ln x-2x\]
On notera $\Cr$ la courbe de $f$ dans un rep�re orthonorm� $\oij$.
\begin{enumerate}
\item \'Etudier les variations de $f$ sur $I$.
\item D�terminer les limites de $f$ aux bornes de son ensemble de d�finition. \\
\emph{Indication: On pourra factoriser par $x$}
\item Prouver que l'�quation $f(x)=0$ admet une unique solution $x_0$ sur $\intf{7}{8}$. On donnera une valeur approch�e � $10^{-2}$ pr�s de $x_0$.
\item D�terminer l'�quation de $T$, la tangente � $\Cr$ en $1$.
\item Tracer $\Cr$ et $T$.
\item D�terminer une primitive de la fonction logarithme n�perien sur $I$.
\end{enumerate}
\end{exercice}

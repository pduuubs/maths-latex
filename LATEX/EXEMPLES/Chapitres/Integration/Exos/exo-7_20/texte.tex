\begin{exercice}
\noindent
\`A l'aide d'int�gration(s) par parties, calculer :\\

$\displaystyle \int_{0}^{\frac{\pi}{2}}  t \sin(t) \,dt$
\hfill
\encadre{
$1$
}


$\displaystyle \int_{0}^{1} (2-t)e^{t} \, dt$
\hfill
\encadre{
$2e-3$
}


$\displaystyle \int_{0}^{\pi}  (2t+3) \sin(t) \, dt$
\hfill
\encadre{
$2 \pi+6$
}


$\displaystyle \int_{1}^{x}  t\ln(t) \, dt$
\hfill
\encadre{
$\frac{1}{2} {x}^{2}  \ln (x) - \frac{1}{4} {x}^{2}$
}



%%%%%%%%%%%$\displaystyle \int_{0}^{-1}  t^2 e^t \, dt$

$\displaystyle \int_{0}^{-1}  (3t^2-t+1) e^t \, dt$
\hfill
\encadre{
$-8+18\,{e^{-1}}$
}



$\displaystyle \int_{0}^{1} t^2 e^{3t}  \, dt$
\hfill
\encadre{
${\frac {5}{27}}\,{e^{3}}-{\frac {2}{27}}$
}



$\displaystyle \int_{0}^{\pi}  e^t \cos(t)  \, dt$
\hfill
\encadre{
$-\frac{1}{2} {e^{\pi }} - \frac{1}{2}$
}



$\displaystyle \int_{0}^{\pi}  e^{-t} \sin(t)  \, dt$
\hfill
\encadre{
$\frac{1}{2} {e^{-\pi }}+ \frac{1}{2}$
}

 
\end{exercice}

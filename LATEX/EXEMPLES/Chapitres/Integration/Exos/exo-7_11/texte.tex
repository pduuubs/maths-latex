\begin{exercice}
{\bf Int�gration par parties (I.P.P.)}
\begin{enumerate}
\item Soit $f$ et $g$ deux fonctions continues sur un intervalle
  $\intf{a}{b}$. Prouver que:
\[ \int_{a}^{b}f(x)+g(x)\ud x=\int_{a}^{b}f(x)\ud x+\int_{a}^{b}g(x)\ud x\]
\item Soit $u$ et $v$ deux fonctions d�rivables sur un intervalle
  $\intf{a}{b}$. En utilisant la d�riv�e de $uv$, �tablir la formule
  d'int�gration par parties:
\[ \boxed{\int_{a}^{b}u'(x)v(x)\ud x =\big[u(x)v(x)\big]_a^b-\int_{a}^{b}u(x)v'(x)\ud x}\]
\item {\bf Applications.} 
  \begin{enumerate}
  \item En choisissant intelligement un bon candidat
  pour $u'$ et pour $v$, appliquer la formule de l' I.P.P. pour
  calculer les int�grales (tr�s classiques) suivantes:
\[ I_1=\int_{0}^{1}x\e^x\ud x ,\quad  I_2=\int_{1}^{\e}x\ln x\ud x ,\quad
 I_3(x)=\int_{1}^{x}\ln t\ud t ,\quad  I_4=\int_{0}^{1}x^2\e^x\ud x\]

\item Donner l'expression de $I_3'(x)$. Que repr�sente $I_3(x)$?
\item \`A l'aide d'une I.P.P. prouver que: 
\[ \int_{0}^{\pi}\sin^2x\ud x =\int_{0}^{\pi}\cos^2x\ud x \]

\item En d�duire la valeur de ces int�grales.
\item Calculer la m�me int�grale en utilisant la formule de
  lin�arisation: $\sin^2(x)=\frac{1-\cos 2x}{2}$
  \end{enumerate}
\end{enumerate}
 
\end{exercice}

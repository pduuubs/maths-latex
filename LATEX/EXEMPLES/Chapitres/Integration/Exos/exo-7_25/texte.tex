\begin{exercice}
On pose $I_0 = \int_0^{\frac{\pi}{6}}\sin(3x)\  dx$ et, pour tout nombre $n$
entier naturel non nul, $I_n=\int_0^{\frac{\pi}{6}}x^n\sin(3x)\ dx$.
  \begin{enumerate}
  \item 
    \begin{enumerate}
    \item Calculer $I_0$.
    \item En utilisant une int�gration par parties, calculer $I_1$.
    \end{enumerate}
\item
    \begin{enumerate}
    \item En effectuant deux int�grations par parties successives,
      prouver que pour $n \geqslant 1$, 
      $$I_{n+2}=\frac{n+2}{9}\times\left(\frac{\pi}{6}\right)^{n+1}-\frac{(n+2)(n+1)}{9}I_n$$
    \item Calculer $I_3$.
    \end{enumerate}
  \item Sans calculer l'int�grale $I_n$ :
\begin{enumerate}
  \item Prouver que la suite $(I_n)_{n\in\mathbb{N}}$ est monotone.
  \item Prouver que pour tout nombre  $n$ entier naturel non nul, 
    $$0 \leqslant I_n \leqslant \int_0^{\frac{\pi}{6}}x^n\ dx$$
  \item D�terminer $\lim\limits_{n\to+\infty} I_n$.
  \end{enumerate}
\end{enumerate}
 
\end{exercice}

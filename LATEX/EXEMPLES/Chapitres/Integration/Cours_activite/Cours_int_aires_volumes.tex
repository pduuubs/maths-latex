

\documentclass[a4paper]{article}
\usepackage[]{pan}
\usepackage{pstricks}
\usepackage{pst-node} 
\begin{document}
\fcours{Int�grales, aires et volumes}%{\term}
\rhead{}
%%-----------------------

\section{Aires}

\label{sec:para1}
\noindent
\begin{minipage}{0.5\linewidth}
{\bf Aire alg�brique.} %
L'aire est compt�e n�gativement lorsque la fonction est n�gative,
\ie{} lorsque sa courbe se trouve sous l'axe des abscisses. Ainsi,
pour prouver que le domaine hachur� et le domaine gris� ont la m�me
aire, il suffit de prouver que:
\[ \int_0^{\pi} \sin x\cos x\ud x=0\]  
\end{minipage}\hfill%
\begin{minipage}{0.5\linewidth}
 % \input{Courbes/aire_int_2.psf}
\psfrag{L1}{\makebox(0,0)[r]{\small $0$}}
\psfrag{L2}{\makebox(0,0)[l]{\normalsize $f(x)=\sin x \cos x$}}
\psfrag{L3}{\makebox(0,0)[t]{\small $\pi$}}
\hfill \includegraphics{Courbes/aire_int_2}
\end{minipage}\\
\vspace{0.3cm}
\noindent
\begin{minipage}{0.5\linewidth}
{\bf Unit� d'aire.} %
L'unit� pour l'aire obtenue grace au calcul d'une int�grale est
l'unit� d'aire (u.a.). Dans le rep�re orthogonal ci-contre (2cm pour une unit�
en abscisse et 1cm en ordonn�e) l'unit�
d'aire mesure: $2\times 1=2$cm$^2$. Calculer alors l'aire du domaine
hachur� en cm$^2$ (penser � la relation de Chasles).
\end{minipage}\hfill%
\begin{minipage}{0.5\linewidth}
 % \input{Courbes/aire_int_1.psf}
\psfrag{L1}{\makebox(0,0)[t]{\footnotesize $0$}}
\psfrag{L2}{\makebox(0,0)[t]{\footnotesize $1$}}
\psfrag{L3}{\makebox(0,0)[t]{\footnotesize $2$}}
\psfrag{L4}{\makebox(0,0)[t]{\footnotesize $-1$}}
\psfrag{L5}{\makebox(0,0)[r]{\footnotesize $0$}}
\psfrag{L6}{\makebox(0,0)[r]{\footnotesize $1$}}
\psfrag{L7}{\makebox(0,0)[r]{\footnotesize $2$}}
\psfrag{L8}{\makebox(0,0)[r]{\footnotesize $-1$}}
\psfrag{L9}{\makebox(0,0)[l]{\small $f(x)=(x+1)(x-1)(x-2)$}}
\hfill \includegraphics{Courbes/aire_int_1}
\end{minipage}
\\
\vspace{0.3cm}
\noindent
\begin{minipage}{0.5\linewidth}
{\bf Aire entre deux courbes}
Si sur un intervalle $\intf{a}{b}$ on a $f(x)\geq g(x)$ alors l'aire
du domaine compris entre les deux courbes sur 
$\intf{a}{b}$ est:
 \[\int_a^{b} f(x)-g(x)\ud x\]
 

 En d�duire l'aire en u.a. du domaine hachur� sur la figure
   ci-contre apr�s avoir prouv� que:
\[ f(x)\geq g(x)  \iff
   \sin(x-\frac{\pi}{6})\geq 0 \]

\end{minipage}\hfill%
\begin{minipage}{0.5\linewidth}
  %\input{Courbes/aire_int_3.psf}
\psfrag{L1}{\makebox(0,0)[t]{\small $0$}}
\psfrag{L2}{\makebox(0,0)[t]{\small $1$}}
\psfrag{L3}{\makebox(0,0)[t]{\small $2$}}
\psfrag{L4}{\makebox(0,0)[t]{\small $3$}}
\psfrag{L5}{\makebox(0,0)[t]{\small $-1$}}
\psfrag{L6}{\makebox(0,0)[t]{\small $-2$}}
\psfrag{L7}{\makebox(0,0)[t]{\small $-3$}}
\psfrag{L8}{\makebox(0,0)[r]{\small $0$}}
\psfrag{L9}{\makebox(0,0)[r]{\small $1$}}
\psfrag{L10}{\makebox(0,0)[r]{\small $2$}}
\psfrag{L11}{\makebox(0,0)[r]{\small $-1$}}
\psfrag{L12}{\makebox(0,0)[r]{\small $-2$}}
\psfrag{L13}{\makebox(0,0){\normalsize $\mathscr{C}_f$}}
\psfrag{L14}{\makebox(0,0){\normalsize $\mathscr{C}_g$}}
\psfrag{L15}{\makebox(0,0){\normalsize $f(x)=\sqrt3\sin x$}}
\psfrag{L16}{\makebox(0,0){\normalsize $g(x)=\cos x$}}
\hfill \includegraphics{Courbes/aire_int_3}
\end{minipage}
\section{Volume}
\noindent
\begin{minipage}{0.5\linewidth}
{\bf Volume obtenu par rotation autour d'un axe.} Si on a une fonction
$f$ positive sur un intervalle $\intf{a}{b}$, et que l'on \og fait
tourner \fg{} la courbe de $f$ autour de $(Ox)$ on d�limite alors un
volume qui vaut:
 \[\int_a^{b}\pi f(x)^2\ud x\]
On somme en fait l'aire des disques de rayon $f(x)$ (qui est positif) pour $x$
variant de $a$ �~$b$.\\
La courbe repr�sent�e dans le plan $(xOy)$ est celle de la fonction
$f:\ x\mapsto \frac{1}{4}x^2$. Calculer le volume alors d�fini sur la
figure ci-contre. O� $A(2;1;0)$.

\end{minipage}\hfill%
\begin{minipage}{0.5\linewidth}
\psfrag{L1}{\makebox(0,0){\footnotesize 0}}
\psfrag{L2}{\makebox(0,0){\footnotesize 1}}
\psfrag{L3}{\makebox(0,0){\footnotesize 2}}
\psfrag{L4}{\makebox(0,0){\footnotesize 0}}
\psfrag{L5}{\makebox(0,0){\footnotesize 1}}
\psfrag{L6}{\makebox(0,0){\footnotesize -5}}
\psfrag{L7}{\makebox(0,0){\footnotesize -4}}
\psfrag{L8}{\makebox(0,0){\footnotesize -3}}
\psfrag{L9}{\makebox(0,0){\footnotesize -2}}
\psfrag{L10}{\makebox(0,0){\footnotesize -1}}
\psfrag{L11}{\makebox(0,0){\footnotesize 0}}
\psfrag{L12}{\makebox(0,0){\footnotesize 1}}
\psfrag{L13}{\makebox(0,0){\footnotesize 2}}
\psfrag{L14}{\makebox(0,0)[r,b]{\small $A$}}
\psfrag{L15}{\makebox(0,0){\small $x$}}
\psfrag{L16}{\makebox(0,0){\small $y$}}
\psfrag{L17}{\makebox(0,0){\small $z$}}
  %\input{Courbes/volume_int.psf}
\hfill \includegraphics{Courbes/volume_int}
\end{minipage}



%%-----------------------
\end{document}

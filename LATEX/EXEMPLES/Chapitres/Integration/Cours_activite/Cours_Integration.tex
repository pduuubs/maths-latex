

\documentclass[a4paper]{article}
\usepackage[]{pan}
\usepackage{pstricks}
\usepackage{pst-node} 
\begin{document}
\fcours{Int�gration}%{\term}

%%-----------------------

\section{Notion d'int�grale et aire}
Rep�re orthogonal, $\oij$ unit� d'aire petit dessin.
\begin{enumerate}
\item Aire et int�grale d'une fct cont positive.\\
\Def{Soit $f$ cont positive sur un intervalle $[a;b]$ et $\Cr$ sa
  courbe dans
$\oij$. l'int�grale ... est l'aire du domaine d�limit� par $\Cr$,
$(Ox)$ et les droites $x=a$ et $x=b$}
\rem La variable est muette, on met ce qu'on veut.
\rem On lit int�grale de... ou somme de...
\rem a et b sont les bornes de l'int�grale.
On a vu que l'aire s'obtient comme somme d'aires de rectangles de
largeur infinit�simale $\ud x$ et de hauteur $f(x)$ ce qui explique la
notation due � Leibniz (1646-1716)
\exemple $\int_{-2}^{0}3 \ud x=6$ (rectangle) et $\int_{1}^{3} t \ud
t=4$ (trap�ze)
\item Valeur approch�e d'int�grale.\\
Il suffit de d�composer l'aire cherch�e en rectangles, triangles et ou
trap�zes pour obtenir une valeur approch�e voire un encadrement de
l'int�grale cherch�e. 
La calculatrice permet d'obtenir graphiquement une valeur approch�e de
l'int�grale et un joli coloriage.
L'unit� est unit� d'aire (u.a.) Si on veut l'aire en ${\rm cm^2}$ il
faut calculer combien de ${\rm cm^2}$ mesure 1 u.a.


\end{enumerate}
\section{Int�grale et primitive}
On a �tabli le lien entre primitive et int�grale hier.
\begin{enumerate}
\item Primitive d'une fonction continue.\\
\Def{Soit $f$ une fonction continue sur un intervalle $I$. On appelle
  primitive de $f$ sur $I$ une fonction $F$ d�rivable sur $I$ telle
  que: $F'=f$}
\exemple $(x^3)'=3x^2$ donc une primitive de $3x^2$ est $x^3$.
\begin{prop}
  Si F et G sont deux prim de $f$ alors elles diff�rent d'une constante.
\end{prop}

\begin{proof}
  $(F-G)'=0$
\end{proof}

\begin{prop}
  Unicit�. Soit $f$  continue positive sur un intervalle $\intf{a}{b}$
  alors $f$ admet une unique primitive qui s'annule en $a$:
\[  F(x)=\int_a^x f(t)\ud t\]
\end{prop}
\begin{proof}
 On l'a prouv� dans le cas d'une fonction monotone en TD. On l'admet
 dans le cas g�n�ral.
\end{proof}
\rem La condition de continuit� est suffisante mais pas n�cessaire...
\item G�n�ralisation de l'int�grale � l'aide de primitive\\
Ici on se d�barrasse de la condition de positivit� et de l'ordre de
$a$ et $b$:
\Def{$f$ cont sur $I$ contenant $a$ et $b$, $F$ une prim de $f$ alors:
\[ \int_a^b f(x)\ud x=F(b)-F(a)\]}
\end{enumerate}
%%-----------------------
\end{document}

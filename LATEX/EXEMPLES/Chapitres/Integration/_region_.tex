\message{ !name(Exos_Integrales.tex)}

\documentclass[a4paper]{article}
\usepackage[]{panpan}
\newcommand{\ud}{\,\mathrm{d}}
\begin{document}

\message{ !name(Exos_Integrales.tex) !offset(-3) }

\fexost{Intégrales}
\chead{} \rhead{}
\Titre{Intégrales}
\Exo[]{Dériver la fonction $F$ définie sur $\R^{+*}$ par:
  \[ F(x)=x\ln x-x\]
En déduire la valeur exacte de l'intégrale:
\[ \int_1^{\e} \ln x \ud x\]
}
\Exo[]{On considère la fonction $f$ définie sur $\R$ par $f(x)=\e^{-x}$. On note $\Cr $ sa courbe représentative dans un repère orthonormal, unités  2cm. 
  \begin{enumerate}
  \item Calculer l'aire (en unité d'aire puis en cm$^2$) entre $\Cr$ et $(Ox)$ et limitée par $(Oy)$ et la droite verticale d'équation $x=5$
  \item Soit $b$ un réel positif. On note $I_b=\int_1^{b} f(x) \ud x$. 
    \begin{enumerate}
    \item Combien vaut $I_5$?
    \item Calculer $I_b$ puis sa limite quand $b$ tend vers l'infini. Qu'est-ce que cela signifie en terme d'aire?
    \end{enumerate}

  \end{enumerate}
}
\Exo[]{On appelle valeur moyenne d'une fonction intégrable $f$ sur un intervalle $\intf{a}{b}$ le réel $M(f)$:
\[ M(f)=\frac{1}{b-a} \int_a^{b} f(x) \ud x\]
Calculer les valeurs moyennes des fonctions suivantes sur l'intervalle $\intf{a}{b}$ donné.
\begin{enumerate}
\item $f(x)=x$ sur $\intf{0}{10}$.
\item $f(x)=x^2$ sur $\intf{-1}{1}$.
\item $f(x)=\e^{2x}$ sur $\intf{-1}{1}$
\item $f(x)=\sin (3x+5)$ sur $\intf{0}{\pi}$.
\item $f(x)=\sin ^2(x)$ sur $\intf{0}{\pi}$. On rappelle les formules de linéarisation:
\[ \sin ^2x=\frac{1-\cos 2x}{2} \quad \et{} \quad \cos ^2x=\frac{1+\cos 2x}{2}\]
\end{enumerate}
}
\Exo[]{Soit $f(x)=(x^2-3x+1)\e^x$. Déterminer une primitive $F$ de $f$ sous la forme:
\[ F(x)=(ax^2+bx+c)\e^x\]
En déduire  la valeur de $\int_{-1}^{2} f(x) \ud x$
}
\Exo[]{Soit $i(t)$ une intensité de courant sinusoïdale, variable en fonction du temps $t$. On note $T$ la période de la fonction $i$. Un ampèremètre mesure $I_{\text{eff}}$ l'intensité efficace de cette intensité, définie par:
\[ I_{\text{eff}}^2=\frac{1}{T}  \int_0^{T} i^2(t) \ud t \]
L'intensité efficace au carré est donc la moyenne sur une période du carré de l'intensité. On a la même définition pour la tension efficace. Déterminer $I_{\text{eff}}$ pour les intensités $i(t)$ données, après avoir déterminé la période $T$. ( $i(t+T)=i(t)$ et on rappelle que $\sin$ et $\cos$ sont de période $2\pi$)
  \begin{enumerate}
  \item $i(t)=\sin t$
  \item $i(t)=7\sin (2t-\frac{\pi}{4})$
  \item $i(t)=2\cos (6\pi t+\frac{\pi}{3})$
    \end{enumerate}
}



 
\end{document}




\message{ !name(Exos_Integrales.tex) !offset(-63) }

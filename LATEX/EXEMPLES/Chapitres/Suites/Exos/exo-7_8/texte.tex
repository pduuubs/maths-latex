\begin{exercice}
Soit un entier naturel $ n\geq 2$. Sur un cercle sont donn�s  $n$  points distincts  $A_ 1$  ,  $A_ 2$  , \ldots ,  $A_ n$. On d�signe  par  $S(n)$  le nombre de segments obtenus en prenant ces points pour extr�mit�s. 
\begin{enumerate}
\item � l'aide de figures, donner  $S(2)$  ,  $S(3)$  ,  $S(4)$  ,  $S(5)$. 
\item Si �  $n$  points du cercle on ajoute  un point suppl�mentaire, combien de segments rajoute-t-on ?
\item En d�duire une relation entre $S(n + 1)$  et  $S(n)$.
\item D�montrer par r�currence que l'on a : pour tout entier  $n\geq 2$  ,    $S(n) = \dfrac{ n (n - 1)}{ 2}$
\end{enumerate} 
\end{exercice}

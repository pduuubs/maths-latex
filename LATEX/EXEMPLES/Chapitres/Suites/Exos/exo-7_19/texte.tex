\begin{exercice}
\noindent
La suite de \textsc{Fibonacci}\footnote{Tr�s c�l�bre, particuli�rement depuis le
  succ�s du livre \textsl{Da Vinci Code}.} est la suite $(U_n)$ d�finie par $U_1=U_2=1$ et
la relation de r�currence pour tout $n$ dans~$\mathbb{N}^*$:
\begin{equation}
  \label{eq:rel_fonk_fibo}
  U_{n+2}=U_{n+1}+U_n
\end{equation}

Je vous propose une d�monstration de l'�tonnante formule� d�couverte
par \textsc{De Moivre}:
\begin{equation}
  \label{eq:fibo_tg}
 \forall n\in \mathbb{N}^*,\  U_n=\frac{1}{\sqrt{5}}\left[
   \left(\frac{1+\sqrt{5}}{2} \right)^n- \left(\frac{1-\sqrt{5}}{2} \right)^n\right]
\end{equation}
\begin{enumerate}
\item  Calculer $U_n$ pour $n$ allant de $1$ � $6$ � l'aide de~\eqref{eq:rel_fonk_fibo}.
\item Comme la suite de Fibonacci v�rifie~\eqref{eq:rel_fonk_fibo} on
  r�sout d'abord dans $\mathbb{R}$ l'�quation associ�e\footnote{C'est
    une technique qui se g�n�ralise � toutes les suites r�currentes lin�aires
    d'ordre deux, \emph{i.e.} v�rifiant une relation du type:
    $U_{n+2}=aU_{n+1}+bU_n$ pour $a\in\R$, $b\in\R$.}:
  \begin{equation}
    \label{eq:trino_fibo}
    x^2=x+1
  \end{equation}
On notera $\phi$ (le nombre d'or) la racine positive et $\bar{\phi}$ la racine n�gative
de~\eqref{eq:trino_fibo}.
\item \label{avecphi1} En utilisant uniquement l'�quation~\eqref{eq:trino_fibo} v�rifi�e par
  $\phi$ , exprimer $\phi^n$ pour $n$ valant $2$, $3$, $4$, $5$, puis
  $6$ sous la forme $a\phi +b$ o� � chaque fois $a$ et $b$ sont des entiers naturels. Que remarque-t-on sur les coefficients $a$ et $b$�?
\item \label{avecphi2} Prouver alors par r�currence que pour tout entier $n$ sup�rieur �~2:
\[\phi^n=U_n\times\phi+U_{n-1}\]

\item En d�duire la formule~\eqref{eq:fibo_tg}. \hfill
  \rotatebox{180}{{\footnotesize Indication: Peut--on traiter les questions
  \textbf{\ref{avecphi1}.} et  \textbf{\ref{avecphi2}.} en rempla�ant
  $\phi$ par  $\bar{\phi}$?}}

\item En d�duire la limite de $U_n$.

\item $\bigstar$ Quelle est la limite de $(\frac{\bar{\phi}}{\phi})^n$? En d�duire
  que la limite du quotient $\frac{U_{n+1}}{U_n}$ vaut~$\phi$.
\end{enumerate}

 \end{exercice}

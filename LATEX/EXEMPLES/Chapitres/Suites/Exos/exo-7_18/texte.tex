\begin{exercice}
\noindent
On cherche si il existe deux r�els $a$ et $b$ tels que le polyn�me $P$
d�fini par: 
\[P(x) = \frac{1}{3}x^3 + ax^2 + bx\]
v�rifie pour tout nombre r�el $x$ la relation:
\begin{equation}
  \label{eq:rela_fonk_exo17}
  P(x+1) - P(x) = x^2
\end{equation}

 
\begin{enumerate} 

\item On suppose dans cette question qu'un tel polyn�me existe.
  \begin{enumerate}
 
\item Sans d�terminer $a$ et $b$, calculer $P(0)$, $P(1)$ et $P(-1)$. 
 
\item En d�duire les seules valeurs possibles pour $a$ et $b$. \label{deterab}

  \end{enumerate}
\item Prouver l'existence et l'unicit� du polyn�me~$P$ v�rifiant les conditions voulues. On v�rifiera en
  particulier que l'expression obtenue au~\textbf{\ref{deterab}.} pour $P$ v�rifie bien~\eqref{eq:rela_fonk_exo17} pour tout r�el~$x$.
 
\item Applications :  
 \begin{enumerate} 
 \item Montrer par r�currence que pour tout entier naturel $n$:  $P(n)\in\mathbb{N}$.
 
 \item On pose pour $n\in\N^*$: $\displaystyle S_n = \sum_{k=1}^n
   k^2=1+2^2+3^2+\cdots +n^2.\ $
 D�montrer les �galit�s : 
 \[ S_n=P(n+1)=\frac{n(n+1)(2n+1)}{6} \]
 
 \end{enumerate}  
\end{enumerate}  

 \end{exercice}

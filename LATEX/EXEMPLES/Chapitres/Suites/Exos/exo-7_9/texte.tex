\begin{exercice}
On d\'efinit deux suites $u$ et $v$ par  $u_0 = 1,~ v_0 = 12$ et pour tout 
entier naturel~$n$ :

\[%
\left\{ %
\renewcommand{\arraystretch}{2}
%\raisebox{0.5ex}{
\begin{array}[l]{r c r}
u_{n+1}& =& \dfrac{1}{3} \left(u_n  +2v_n\right)\\
v_{n+1} & = & \dfrac{1}{4}\left(u_n  +3v_n\right)
\end{array}%
%}%
\right.%
\]
\renewcommand{\arraystretch}{1}
\begin{enumerate} \item	On appelle $w$ la suite d\'efinie pour tout entier
 naturel $n$ par :  $w_n =  v_n - u_n$.

\begin{enumerate} \item Montrer que $w$ est une suite g\'eom\'etrique \`a termes positifs, dont
 on pr\'ecisera la raison.

\item D\'eterminer la limite de la suite $w$.

\end{enumerate}

\item \begin{enumerate} \item Montrer que la suite $u$ est croissante.

\item Montrer que la suite $v$ est d\'ecroissante.

%\item En d\'eduire que, pour tout entier naturel $n,~ u_0 \leqslant u_n  \leqslant v_n \leqslant  v_0$.

\end{enumerate}

%\item On admet que les suites $u$ et $v$ convergent. Montrer qu'elles ont alors m\^eme limite que l'on appellera $\ell$.

\item On appelle $t$ la suite d\'efinie pour tout entier
 naturel $n$ par : $t_n = 3u_n + 8v_n$.

\begin{enumerate} \item Montrer que $t$ est une suite constante.
 D\'eterminer cette constante.
\item D\'eterminer alors l'expression de $u_n$ et de $v_n$ en fonction
  de~$n$.
\item Prouver que $u_n$ et $v_n$ ont la m�me limite~$\ell$ que l'on d�terminera.

%\item D\'eterminer alors la valeur de $\ell$.

\end{enumerate}

\end{enumerate}

\end{exercice}

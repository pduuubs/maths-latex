\begin{exercice}
\noindent
La suite de \textsc{Fibonacci}\footnote{Tr�s c�l�bre, particuli�rement depuis le
  succ�s du livre \textsl{Da Vinci Code}. \textsc{Fibonacci} (Pise 1170--1245)} est la suite $(U_n)$ d�finie par $U_1=U_2=1$ et
la relation de r�currence pour tout $n$ dans~$\mathbb{N}^*$:
\begin{equation}
  \label{eq:rel_fonk_fibo}
  U_{n+2}=U_{n+1}+U_n
\end{equation}
\begin{enumerate}
\item  Calculer les dix premiers termes de la suite de Fibonacci.\\

Il a fallu attendre 500 ans pour trouver l'expression du terme g�n�ral
de la suite de Fibonacci. Je vous propose une d�monstration de
l'�tonnante\footnote{Pourquoi peut--on trouver cette formule �tonnante
  au premier abord?} formule� d�couverte
par \textsc{De Moivre}\footnote{Abraham \textsc{De Moivre} (Vitry le Fran�ois
  1667-- Londres 1754) Math�maticien fran�ais, ami d'Isaac
  \textsc{Newton}. }:
\begin{equation}
  \label{eq:fibo_tg}
 \forall n\in \mathbb{N}^*,\  U_n=\frac{1}{\sqrt{5}}\left[
   \left(\frac{1+\sqrt{5}}{2} \right)^n- \left(\frac{1-\sqrt{5}}{2} \right)^n\right]
\end{equation}
%\begin{enumerate}
\item Supposons un instant que $(V_n)$ soit une suite g�om�trique non nulle
  v�rifiant la m�me relation de r�currence que $(U_n)$. Prouver alors
  que la raison de $(V_n)$ est n�cessairement solution de:
\begin{equation}
    \label{eq:trino_fibo}
    x^2=x+1
  \end{equation}
\item R�soudre dans $\mathbb{R}$ l'�quation~\eqref{eq:trino_fibo}. On notera $\phi$ (le nombre d'or) la racine positive et $\bar{\phi}$ la racine n�gative
de~\eqref{eq:trino_fibo}.
%\item  Calculer $U_n$ pour $n$ allant de $1$ � $6$ � l'aide de~\eqref{eq:rel_fonk_fibo}.
\item La suite de Fibonacci est--elle une suite  g�om�trique?

\item \label{avecphi1} 
 
    En utilisant uniquement l'�quation~\eqref{eq:trino_fibo} v�rifi�e par
  $\phi$ , exprimer $\phi^n$ pour $n$ valant $2$, $3$, $4$, $5$, puis
  $6$ sous la forme $a\phi +b$ o� � chaque fois $a$ et $b$ sont des
  entiers naturels. On reproduira et compl�tera alors le tableau
  ci--dessous avec les valeurs obtenues pour $a$ et $b$. Que remarque-t-on sur les coefficients $a$ et $b$?%\smallskip\\
  \begin{center}
    
 
\begin{tabular}{|c|c|c|c|c|c|c|}
    \hline
$n$ &$1$&$2$&$3$&$4$&$5$&$6$ \\ \hline 
$a$ & $1$  & $1$ &&&&\\ \hline
$b$ & $0$  & $1$ &&&& \\ \hline
  \end{tabular}
 \end{center}
% \begin{tabular}{|c|c|c|}
%     \hline
% $n$ &$a$&$b$\\ \hline \hline
% $1$ & $1$  & $0$  \\ \hline
% $2$ & $1$  & $1$  \\ \hline
% $3$ &   &  \\ \hline
% $\vdots$ & $\vdots$  & $\vdots$ %\\ 
% %\hline
% %$5$ &   &  \\ \hline
% %$6$ &   &  \\ \hline
%   \end{tabular}
%\end{minipage}
  


\item \label{avecphi2} Prouver alors par r�currence que pour tout entier $n$ sup�rieur �~2:
\[\phi^n=U_n\times\phi+U_{n-1}\]

\item En d�duire la formule~\eqref{eq:fibo_tg}. \hfill
 \raisebox{0.8ex}{ \rotatebox{180}{{\footnotesize Indication: Peut--on traiter la question
  \textbf{\ref{avecphi1}.} en rempla�ant
  $\phi$ par  $\bar{\phi}$?}}}

\item En d�duire la limite de $U_n$.
\item On s'int�resse maintenant aux quotients de deux termes
  successifs de la suite de Fibonacci. On consid�re donc la suite $(Q_n)$ d�finie par:
  $Q_n=\frac{U_{n+1}}{U_n}$ pour tout $n$ de $\mathbb{N}^*$. Prouver que: 
\[\forall n\in\mathbb{N}^*,\ 
Q_n=\frac{\phi-\bar{\phi}\left(\frac{\bar{\phi}}{\phi}\right)^n} {1-\left(\frac{\bar{\phi}}{\phi}\right)^n}\]

\item Prouver que\footnote{Cette limite  signifie que $(U_n)$ se
    comporte  pour $n$ grand \og comme\fg{} une
  suite g�om�trique de raison~$\phi$.}:
 $\lim\limits_{ n \to +\infty }Q_n=\phi$.\\\noindent


\item Prouver  en utilisant~\eqref{eq:rel_fonk_fibo} que
  $\left(Q_{n}\right)_{n\in\mathbb{N}^*}$ est d�termin�e par:
\[\forall n\in\mathbb{N}^*,\ Q_{n+1}=1+\frac{1}{Q_n} \quad\text{et}\quad Q_1=1\]
\item Expliquer l'�criture de $\phi$ dite en \og fraction
  continue\fg{}: $\phi=1+\cfrac{1}{1+\cfrac{1}{1+\cfrac{1}{1+\cdots}}}$
\end{enumerate}

 \end{exercice}

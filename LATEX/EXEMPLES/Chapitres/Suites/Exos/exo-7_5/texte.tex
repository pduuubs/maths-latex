\begin{exercice}
Quand  $n\in\N^*$ On note  $S(n)$  la somme des  $n$  premiers entiers naturels \emph{impairs} ; 
par exemple:  $S(1) = 1$  ,  $S(2) = 1 + 3$  ,  $S(3) = 1 + 3 + 5$ 
\begin{enumerate}
\item Calculer  $S(1)$  ,  $S(2)$  ,  $S(3)$  ,  $S(4)$  ,  $S(5)$. Quelle conjecture �mettre ?  
\item On pose  $T(n) = n^2$. D�montrer par r�currence que pour tout entier naturel non nul: \[S(n) = T(n)\]
\end{enumerate}

\end{exercice}

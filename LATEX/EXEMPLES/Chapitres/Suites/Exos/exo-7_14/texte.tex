\begin{exercice}
Soit $a$ et $b$ les deux suites r\'eelles d\'efinies par : 
$$ 
a_0=2, \ b_0=4 \textrm{ et pour tout entier } n : 
\left\{ 
\begin{array}{ccc} 
a_{n+1} & = & \dfrac{1}{4} (a_n+3b_n)\\ 
\\ 
b_{n+1} & = & \dfrac{1}{4} (3a_n+b_n)\\ 
\end{array} 
\right. 
$$ 
\noindent 
La droite $(\Delta)$ \'etant rapport\'ee au rep\`ere $(O;\vec{\imath})$, on 
consid\`ere pour tout entier $n$ les points A$_n$ et B$_n$ de $(\Delta)$ 
d'abscisses respectives $a_n$ et $b_n$. 
\begin{enumerate} 
\item Placer les points A$_0$, B$_0$, A$_1$ B$_1$, A$_2$ et B$_2$ sur 
 l'axe $(\Delta)$. 
\item Soit $u$ la suite de terme g\'en\'eral $ u_n=a_n+b_n$.\\ 
D\'emontrer que la suite $u$ est constante. Que peut on en d\'eduire pour le 
segment $[A_n;B_n]$ ? 
\item Soit $v$ la suite de terme g\'en\'eral $v_n=a_n-b_n$. 
 \begin{enumerate}
 \item Prouver que la suite $v$ est g\'eom\'etrique et convergente.\\ 
 Que peut on en d\'eduire pour la distance $A_nB_n$ ? 
 \item Exprimer $v_n$ en fonction de $n$. 
 \end{enumerate} 
\item Exprimer $a_n$ et $b_n$ en fonction de $n$.\\ 
 D\'emontrer que les suites $a$ et $b$ sont convergentes et qu'elles ont 
 la m\^eme limite. 
\end{enumerate} 
 \end{exercice}

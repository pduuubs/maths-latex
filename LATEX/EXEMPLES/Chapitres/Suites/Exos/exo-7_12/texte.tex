\begin{exercice}
  \begin{enumerate}
  \item Montrer\footnote{On pourra montrer par r�currence que $u_n>-1$ pour
tout $n\in\N$ en �tudiant une fonction bien choisie.} qu'il existe une
suite $(u_n)_{n\in\mathbb{N}}$ telle que:
\[ u_0=32\quad\text{et}\quad \forall n\in\mathbb{N}\ \
u_{n+1}=\frac{2u_n-1}{u_n+4}.\] 

\item Montrer que l'�quation $ x=\frac{2x-1}{x+4}$ admet une seule solution
  que l'on notera $\alpha$. 
\item  Montrer que $u_n\neq\alpha$ pour tout $n\in\mathbb{N}$. 
\item Pour
$n\in\mathbb{N}$, on pose \[ v_n=\frac{1}{u_n-\alpha}.\] Montrer que  la suite
$(v_n)$ est arithm�tique. 
\item Exprimer $u_n$ en fonction de $n$
pour tout $n\in\mathbb{N}$. 
\item \'Etudier la convergence de la suite $(u_n)$.
  \end{enumerate}
 \end{exercice}

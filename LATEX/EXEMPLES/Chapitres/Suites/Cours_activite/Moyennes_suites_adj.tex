

\documentclass[a4paper]{article}
\usepackage[]{pan}

\begin{document}
\activite{Diff\'erentes moyennes}{\term}
\cfoot{}\lfoot{}
\section{Trois moyennes}
 Soit $x$ et $y$ deux r\'eels strictement positifs.\\
{\bf D\'efinition:} On d\'efinit $m$, $g$ et $h$ respectivement les moyennes arithm\'etique, g\'eom\'etrique et
harmonique de $x$ et $y$ par:
\[ m=\frac{x+y}{2}\qquad \quad g=\sqrt{xy}\qquad \quad h=\frac{2}{\frac{1}{x}+\frac{1}{y}}\]
\begin{enumerate}
\item Prouver que: $g\leq m$ avec \'egalit\'e ssi $x=y$
\rotatebox{180}{On pourra factoriser $m-g$ avec une I.R.}
\item Exprimer $h$ en fonction de $m$ et $g$.
\item En d\'eduire que: 
  \begin{enumerate}
  \item $h\leq g\leq m$
  \item $g$ est aussi la moyenne  g\'eom\'etrique de $m$ et~$h$.
  \end{enumerate}
\end{enumerate}

\section{Suites adjacentes}
Maintenant on va s'amuser \`a calculer des moyennes successives. Soit
$u_0$ et $v_0$ deux r\'eels strictement positifs. On d\'efinit alors les
deux suites $(u_n)$ et $(v_n)$ sur $\N$ par:
\[ u_{n+1}=\frac{u_n+v_n}{2} \qquad \et{} \qquad v_{n+1}=\frac{2}{\frac{1}{u_n}+\frac{1}{v_n}}\]

\begin{enumerate}
\item On part de $u_0=1$ et $v_0=16$, calculer et placer sur l'axe
  gradu\'e les valeurs $u_1$, $v_1$, $u_2$ et~$v_2$.

%%%%%%%%%%%%%%%%

\setlength{\unitlength}{9.5mm}
\begin{picture}(15,1)\noindent
%\linethickness{0.2mm}

%\multiput(0,0)(1,0){161}% graduations sur (Ox) fines (1mm)
%{\line(0,1){1}}

%\multiput(0,0)(0,1){101}% graduations sur (Oy) fines (1mm)
%{\line(1,0){1}}

\linethickness{0.2mm}
\put(0,0.5){\line(1,0){15}}
\multiput(0,0.4)(1,0){16}% graduations sur (Ox)
{\line(0,1){0.2}}

\linethickness{0.1mm}
\multiput(0,0.5)(0.1,0){150}% graduations sur (Ox)
{\line(0,1){0.1}}

\put(-0.1,0.1){$u_0$}
\put(-0.1,0.4){$\bullet$}
\put(-0.1,0.7){$1$}

\put(14.9,0.1){$v_0$}
\put(14.9,0.4){$\bullet$}
\put(14.9,0.7){$16$}

\put(2.9,0.7){$4$}
\put(6.9,0.7){$8$}
\put(10.8,0.7){$12$}


%\multiput(0,0)(0,4){26}% graduations sur (Oy) 
%{\line(-1,0){1}}

%\linethickness{0.3mm}
%\put(0,0){\vector(1,0){165}} %Axe (Ox)
%\put(0,0){\vector(0,1){105}} %Axe (Oy)

%\put(-5,-5){O}

%\put(38,-5){10}   %unités (Ox)
\end{picture}

%%%%%%%%%%%%%%%%


\item Prouver par r\'ecurrence que les deux suites sont \`a termes
  strictement positifs.
\item D\'emontrer que $(v_n)$ est croissante.
\item Justifier \`a l'aide de la premi\`ere partie que pour $n\in\N$ on a:  $v_{n+1}\leq u_{n+1}$
% $v_1\leq  u_1$. D\'emontrer alors par r\'ecurrence que pour $n\in\N^*$ on a:  $v_n\leq u_n$
\item En d\'eduire que $(u_n)$ est d\'ecroissante sur $\N^*$
\item Prouver que pour $n\in\N$ on a:
\[ u_{n+1}-v_{n+1}=\frac{(u_n-v_n)^2}{2(u_n+v_n)}\]
\item On pose $d_n=u_n-v_n$. Prouver que pour $n\in\N$:  $d_{n+1}\leq\frac12 d_n$
\item En d\'eduire que: $n\in\N \implies d_n\leq \left(\dfrac12\right)^nd_0$
\item Prouver que $(u_n)$ et $(v_n)$ sont adjacentes. 
\item On note pour $n$ dans $\N$, $g_{n+1}$ la moyenne g\'eom\'etrique de $u_n$
  et $v_n$. Traduire les r\'esultats du {\bf I.3.} avec les suites
  $(u_n)$, $(v_n)$ et~$(g_n)$. 
%Justifier \`a l'aide de la premi\`ere partie que pour tout $n$  dans $\N$:  $g_{n+1}=g_n$. Qu'en d\'eduit-on pour la suite $(g_n)$?


\item En d\'eduire que: $\forall n \in\N^*,\ v_n\leq \sqrt{u_0v_0}\leq u_n$
\item En d\'eduire la limite commune de $(u_n)$ et $(v_n)$.
\item \'Enoncer sous forme de propri\'et\'e ce que l'on vient de d\'emontrer
  concernant les moyennes arithm\'etiques, g\'eom\'etriques et harmoniques.
\end{enumerate}
\end{document}

\begin{exercice}
\textbf{\Large Polyn\'esie juin 2004}

\vspace{0,5cm}

Le laboratoire de physique d'un lyc�e dispose d'un parc d'oscilloscopes
 identiques. La dur�e de vie en ann�es d'un oscilloscope est une variable
 al�atoire not�e $X$ qui suit la \og loi de dur�e de vie sans vieillissement \fg{}
 (ou encore loi exponentielle de param�tre $\lambda$ avec $\lambda > 0$.

\noindent Toutes les probabilit�s seront donn�es � $10^{-3}$ pr�s.

\begin{enumerate}\item Sachant que $p(X > 10) = 0,286$, montrer qu'une valeur approch�e � $10^{-3}$
pr�s de $\lambda$ est $0,125$.

\noindent On prendra $0,125$ pour valeur de $\lambda$ dans la suite de l'exercice.

\item Calculer la probabilit� qu'un oscilloscope du mod�le �tudi� ait une
 dur�e de vie inf�rieure � 6 mois.

\item Sachant qu'un appareil a d�j� fonctionn� huit ann�es, quelle est la
 probabilit� qu'il ait une dur�e de vie sup�rieure dix ans ?

\item On consid�re que la dur�e de vie d'un oscilloscope est ind�pendante
 de celle des autres appareils. Le responsable du laboratoire d�cide de
 commander 15 oscilloscopes. Quelle est la probabilit� qu'au moins un 
oscilloscope ait une dur�e de vie sup�rieure � 10 ans ?

\item Combien l'�tablissement devrait-il acheter d'oscilloscopes pour que
 la probabilit� qu'au moins l'un d'entre eux fonctionne plus de 10 ans
 soit sup�rieure � 0,999 ?

\vspace{0.4cm}

\noindent Rappel :

\noindent \textit{Loi exponentielle de param�tre $\lambda$ sur $[0~;~ + \infty [$,
 dite aussi loi de dur�e de vie sans vieillissement }:

\noindent pour $0 \leqslant a \leqslant  b,~ p([a~;~b]) = \displaystyle\int_a^b  
\lambda \text{e}^{-\lambda t}\:\text{d}t \qquad$  et 

\noindent pour $c \geqslant  0,~ 
p([c~;~+ \infty[) = 1 -  \displaystyle\int_0^c \lambda \text{e}^{-\lambda t}
\:\text{d}t$.

\end{enumerate}
 
\end{exercice}

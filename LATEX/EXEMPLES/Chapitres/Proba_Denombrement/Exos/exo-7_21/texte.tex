\begin{exercice}
On joue encore � la roulette, mais maintenant on veut gagner. On va
utiliser une martingale:
On mise $m$ au premier coup, si on perd, on double la mise au coup
suivant. Quand on a gagn�, la partie s'arr�te (mais rien n'emp�che de
recommencer). 
\begin{enumerate}
\item Si on gagne en un coup, quel est le gain?
\item Si on gagne en deux coups exactement, quel est le gain?
\item Si on gagne en trois coups exactement, quel est le gain?
\item Prouver que quel que soit le nombre de coups perdus avant le
  $n$\ieme{} o� l'on gagne, on a toujours au final un gain �gal � $m$.

\item Cette technique est parfaite sauf qu'il y a une r�gle
  suppl�mentaire au casino pour �viter ce type de martingale: la mise
  est limit�e. Au minimum on doit miser 2\euro{} et au maximum 360
  fois cette mise minimale. Compte tenu de cette r�gle, combien de
  fois a-t-on la possibilit� de doubler la mise initiale? Il faut donc
  gagner en combien de coups au maximum pour pouvoir appliquer la martingale?
\item $\bigstar$ On suppose que si on ne gagne pas en neuf coup, on arr�te de
  jouer. Prouver que l'esp�rance de gain est alors strictement
  n�gative, �gale �: $m\left(1-\left[2(1-p)\right]^9\right)$


\end{enumerate}
 
\end{exercice}

\begin{exercice}

 \noindent
\textbf{\Large France juin 2004}

\vspace{0,5cm}

On s'int\'eresse \`a la dur\'ee de vie, exprim\'ee en semaines, d'un composant \'electronique.
 On mod\'elise cette situation par une loi de probabilit\'e $p$ de dur\'ee de vie sans
 vieillissement d\'efinie sur l'intervalle $[0~;~ + \infty[$ : la probabilit\'e que le 
composant ne soit plus en \'etat de marche au bout de $t$ semaines est:

\[p([0~;~ t[)  = \displaystyle\int_0^t \lambda \text{e}^{-\lambda 
x}\:\text{d}x.\]

\noindent Une \'etude statistique, montrant qu'environ $50\:\%$ d'un lot important de 
ces composants sont encore en \'etat de marche au bout de 200 semaines, permet de 
poser $p([0~;~ 200[) = 0,5$.

\begin{enumerate} \item Montrer que $\lambda = \cfrac{\ln 2}{200}$.

\item Quelle est la probabilit\'e qu'un de ces composants pris au hasard ait une dur\'ee
 de vie sup\'erieure ? 300 semaines ? On donnera la valeur exacte et une valeur 
approch\'ee d\'ecimale au centi\`eme pr\`es.

\item On admet que la dur\'ee de vie moyenne $d_m$ de ces composants est la limite quand 
$A$ tend vers $+ \infty$ de $\displaystyle\int_0^A \lambda x\text{e}^{-\lambda 
x}\:\text{d}x$.

\begin{enumerate} \item Montrer que $\displaystyle\int_0^A \lambda x\text{e}^{-\lambda 
x}\:\text{d}x = \cfrac{- \lambda A\text{e}^{-\lambda A} - \text{e}^{- 
\lambda A} + 1}{\lambda}$.

\item En d\'eduire $d_m$. On donnera la valeur exacte et une valeur approch\'ee d\'ecimale \`a
la semaine pr\`es.

\end{enumerate}

\end{enumerate}
\end{exercice}

\begin{exercice}
On joue � la roulette. La boule tombe au hasard sur un des num�ros de
0 � 36. Le 0 est consid�r� comme �tant ni pair ni impair. On mise $m$
euros sur pair (par exemple). Si on gagne, on r�cup�re $2m$ et sinon
on ne r�cup�re rien.
\begin{enumerate}
\item Calculer la probabilit� que l'on notera $p$ de gagner en un coup.
\item On note $X$ le gain du joueur. Donner la loi et l'esp�rance de $X$.
\item Maintenant on joue encore et encore. Soit $n\in\N^*$. Calculer la probabilit� $p_n$
  de
  gagner en $n$ coups exactement (\ie{} de perdre les premiers coups
  et de gagner le $n$\ieme{}).
\item On note $N$ la variable al�atoire qui donne le nombre de coups
  qu'on a  jou� pour gagner . 
  \begin{enumerate}
  \item Que vaut $P(N=n)$?
  \item Calculer $P(N\leq n)$ et sa limite quand $n$ tend vers l'infini.

  \item D�terminer le rang $n$ � partir duquel $P(N>n) < 10^{-6}$.

  \item $\bigstar$ Prouver que
    $\displaystyle E(N)=p\times\lim_{M\to +\infty}\sum_{k=1}^{M}(k+1)x^k$ o�
    $x=1-p$. \'Etablir que  $E(N)=\frac{1}{p}$ en remarquant que
    $\displaystyle \sum_{k=1}^{M}(k+1)x^k$ est la d�riv�e d'une fonction $f$ dont on
    donnera une expression sans somme.
  \end{enumerate}
\end{enumerate}

 
\end{exercice}

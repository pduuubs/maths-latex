\begin{exercice}
Le Mastermind est un jeu o� un joueur cache une combinaison de quatre
pions de couleur en ligne (donc dans un ordre pr�cis) qu'on nommera code. Il existe six couleurs. L'autre joueur
qu'on appelera Maurice tente de deviner le code cach�.\\
\noindent
{\bf Partie A}
\begin{enumerate}
\item Si on s'autorise � mettre des pions de couleurs identiques,
  combien y a-t-il de codes possibles?
\item On joue dans la suite une version simplifi�e o� le code ne comporte pas de
  couleurs identiques. Combien y a-t-il de tels codes?
\end{enumerate}
On va �tudier deux strat�gies pour le premier coup. On dispose d'une
urne comportant six jetons, un de chaque couleur.\\
\noindent
{\bf Partie B - Strat�gie 1}
On choisit un jeton de l'urne, on pose un pion de cette couleur sur
la premi�re case, on remet le jeton dans l'urne et on recommence ainsi
pour chacun des trois pions restant. On note $Y$ le nombre de pions
ayant la bonne couleur, � la bonne place.
\begin{enumerate}
\item Quelle est la probabilit� de deviner directement le code?
\item Calculer $P(Y=2)$ et $P(Y\geq 2)$. %D�terminer la loi de $Y$, calculer
\end{enumerate}
\noindent
{\bf Partie C - Strat�gie 2}
On choisit simultan�ment quatre jetons de l'urne. On note $X$ le
nombre de couleurs correctes. Ensuite on pose au hasard quatre pions ayant les
couleurs tir�es. 
\begin{enumerate}
\item D�terminer la loi, l'esp�rance et l'�cart type de $X$.
\item Justifier que si $X=4$ est r�alis�, $Y=3$ est de probabilit� nulle.
\item Calculer: $P(Y=4)$,  $P(Y=3)$,  $P(Y=2)$, puis $P(Y\geq
  2)$. Comparer les deux strat�gies.
\end{enumerate}
 
\end{exercice}

\begin{exercice}
\begin{enumerate}
\item On choisit au hasard un nombre dans l'intervalle $[0;7]$; le choix est mod�lis� par une loi uniforme.
\begin{enumerate}
\item Calculer la probabilit� que ce nombre soit dans $[2;5]$.
\item Calculer la probabilit� que ce nombre soit dans $[4;6]$.
\item Calculer $P_{[2;5]}([4;6])$.
\end{enumerate}
\item On choisit un nombre au hasard de $[0;1]$.\\
Calculer la probabilit� que le premier chiffre apr�s la virgule soit impair , sachant que ce nombre
est strictement sup�rieur � $0,7$.
\item La fonction \emph{Random} d'une calculatrice est sens�e donner
  un nombre au hasard suivant la loi uniforme dans l'intervalle
  $\intf{0}{1}$. 
  \begin{enumerate}
  \item Est-ce possible? Comment la calculatrice fait-elle?
  \item Compte tenu de la pr�cision de l'affichage, quelle est la probabilit� que la machine affiche~$\frac12$?
  \item Et si la loi �tait uniforme, quelle serait cette probabilit�?
  \end{enumerate}
\item On rappelle que la fonction de r�partition d'une variable
  al�atoire $X$ est d�finie sur $\R$ par: $F(x)=P(X\leq  x)$. D�terminer celle d'une variable suivant la loi uniforme sur
  $\intf{-1}{2}$. Tracer rapidement sa courbe.
\item Si $X$ suit une loi uniforme sur $[a;b]$, calculer son esp�rance
  math�matique qui est: 
\[E(X)=\int_{a}^{b} \frac{x}{b-a}\ud x\]
\end{enumerate}
 
\end{exercice}

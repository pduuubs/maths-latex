\begin{exercice}
Un Q.C.M. comporte quatre questions comportant chacune quatre r�ponses
possibles dont une seule est correcte. 
\begin{enumerate}
\item Une r�ponse correcte rapporte un point. On veut oter un nombre
  $x$ de point pour chaque r�ponse fausse de mani�re � ce qu'une personne
  r�pondant au hasard � une question ait en moyenne z�ro � cette question. Trouver $x$.
\item Maurice choisit de r�pondre au hasard � toutes les questions. On
  note $N$ la variable al�atoire qui donne son nombre de r�ponses
  correctes. Quelle loi suit $N$? Donner sous forme de tableau la loi
  de probabilit� de $N$.
\item On note $X$ la note de Maurice sachant qu'on applique le bar�me
  de la premi�re question et que si le total est n�gatif, la note est
  ramen�e � z�ro. Calculer l'esp�rance math�matique de $X$. Que
  repr�sente ce nombre?
\end{enumerate}

\end{exercice}

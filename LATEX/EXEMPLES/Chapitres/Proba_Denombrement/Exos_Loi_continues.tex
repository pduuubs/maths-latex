

\documentclass[a4paper]{article}
\usepackage[]{pan}
\usepackage{pst-math}
\usepackage{pst-plot}
\begin{document}
\fexos{Loi exponentielle-Annales}{\term S}
\cfoot{}
%%%---------------------------------------
%\exercice
\noindent
\textbf{\Large France juin 2004}

\vspace{0,5cm}

On s'int\'eresse \`a la dur\'ee de vie, exprim\'ee en semaines, d'un composant \'electronique.
 On mod\'elise cette situation par une loi de probabilit\'e $p$ de dur\'ee de vie sans
 vieillissement d\'efinie sur l'intervalle $[0~;~ + \infty[$ : la probabilit\'e que le 
composant ne soit plus en \'etat de marche au bout de $t$ semaines est:

\[p([0~;~ t[)  = \displaystyle\int_0^t \lambda \text{e}^{-\lambda 
x}\:\text{d}x.\]

\noindent Une \'etude statistique, montrant qu'environ $50\:\%$ d'un lot important de 
ces composants sont encore en \'etat de marche au bout de 200 semaines, permet de 
poser $p([0~;~ 200[) = 0,5$.

\begin{enumerate} \item Montrer que $\lambda = \cfrac{\ln 2}{200}$.

\item Quelle est la probabilit\'e qu'un de ces composants pris au hasard ait une dur\'ee
 de vie sup\'erieure ? 300 semaines ? On donnera la valeur exacte et une valeur 
approch\'ee d\'ecimale au centi\`eme pr\`es.

\item On admet que la dur\'ee de vie moyenne $d_m$ de ces composants est la limite quand 
$A$ tend vers $+ \infty$ de $\displaystyle\int_0^A \lambda x\text{e}^{-\lambda 
x}\:\text{d}x$.

\begin{enumerate} \item Montrer que $\displaystyle\int_0^A \lambda x\text{e}^{-\lambda 
x}\:\text{d}x = \cfrac{- \lambda A\text{e}^{-\lambda A} - \text{e}^{- 
\lambda A} + 1}{\lambda}$.

\item En d\'eduire $d_m$. On donnera la valeur exacte et une valeur approch\'ee d\'ecimale \`a
la semaine pr\`es.

\end{enumerate}

\end{enumerate}
\vspace{0,5cm}
%%%---------------------------------------
%\exercice
\textbf{\Large Polyn\'esie juin 2004}

\vspace{0,5cm}

Le laboratoire de physique d'un lyc�e dispose d'un parc d'oscilloscopes
 identiques. La dur�e de vie en ann�es d'un oscilloscope est une variable
 al�atoire not�e $X$ qui suit la \og loi de dur�e de vie sans vieillissement \fg{}
 (ou encore loi exponentielle de param�tre $\lambda$ avec $\lambda > 0$.

\noindent Toutes les probabilit�s seront donn�es � $10^{-3}$ pr�s.

\begin{enumerate}\item Sachant que $p(X > 10) = 0,286$, montrer qu'une valeur approch�e � $10^{-3}$
pr�s de $\lambda$ est $0,125$.

\noindent On prendra $0,125$ pour valeur de $\lambda$ dans la suite de l'exercice.

\item Calculer la probabilit� qu'un oscilloscope du mod�le �tudi� ait une
 dur�e de vie inf�rieure � 6 mois.

\item Sachant qu'un appareil a d�j� fonctionn� huit ann�es, quelle est la
 probabilit� qu'il ait une dur�e de vie sup�rieure dix ans ?

\item On consid�re que la dur�e de vie d'un oscilloscope est ind�pendante
 de celle des autres appareils. Le responsable du laboratoire d�cide de
 commander 15 oscilloscopes. Quelle est la probabilit� qu'au moins un 
oscilloscope ait une dur�e de vie sup�rieure � 10 ans ?

\item Combien l'�tablissement devrait-il acheter d'oscilloscopes pour que
 la probabilit� qu'au moins l'un d'entre eux fonctionne plus de 10 ans
 soit sup�rieure � 0,999 ?

\vspace{0.4cm}\\

\noindent Rappel :

\noindent \textit{Loi exponentielle de param�tre $\lambda$ sur $[0~;~ + \infty [$,
 dite aussi loi de dur�e de vie sans vieillissement }:

\noindent pour $0 \leqslant a \leqslant  b,~ p([a~;~b]) = \displaystyle\int_a^b  
\lambda \text{e}^{-\lambda t}\:\text{d}t \qquad$  et 

\noindent pour $c \geqslant  0,~ 
p([c~;~+ \infty[) = 1 -  \displaystyle\int_0^c \lambda \text{e}^{-\lambda t}
\:\text{d}t$.

\end{enumerate}

\end{document}
%%%---------------------------------------
\exercice
%%%---------------------------------------
\exercice




\end{document}


\begin{pspicture}*(-7,-4)(7,4)
                          \psaxes{->}(0,0)(-6,-4)(7,4)                         
                          \psplot[linecolor=gray]{-7}{7}{x TAN }
                          \end{pspicture}


 

\begin{pspicture}*(-5,-5)(5,5)
                          \psaxes{->}(0,0)(-5,-5)(5,5)
                          \psplot[linecolor=gray]{-5}{5}{x COSH }
                          \psplot[linecolor=red]{-5}{5}{x SINH }
                          \psplot[linecolor=green]{-5}{5}{x TANH }
                          \end{pspicture}


\begin{pspicture}*(-5,-5)(5,5)
                          \psaxes{->}(0,0)(-5,-5)(5,5)
                          \psgrid[griddots=1,%
                          subgriddiv=2,
                          gridlabels=0pt,%
                          xunit=1]
                          (-5,-5)(5,5)
                          \end{pspicture}
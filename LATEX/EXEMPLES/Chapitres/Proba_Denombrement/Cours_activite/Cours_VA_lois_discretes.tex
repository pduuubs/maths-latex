

\documentclass[a4paper]{article}
\usepackage[]{pan}
\usepackage{pstricks}
\usepackage{pst-node}
\usepackage{pst-tree} 
\begin{document}
\fcours{Variables al�atoires-Lois discr�tes}%{\term}
\rhead{}
\chead{}
%%-----------------------
\section{D�finitions}
Soit $\Omega$ un univers fini � $n$ �l�ments ($n\in\N$). On note
$(\omega_i)_{1\leq i\leq n}$ les $n$ �v�nements �l�mentaires qui
constituent $\Omega$. Donc $\Omega=\{\omega_1;\,\omega_2;\ldots ;\omega_n\,\}$
\begin{Def}
  On appelle \emph{variable al�atoire} sur $\Omega$ toute fonction $X$ qui va de
  $\Omega$ dans $\R$. Pour chaque $i=1,2,\ldots, n$ on note
  $x_i=X(\omega_i)$.
  \begin{eqnarray}
    X:&&\Omega \to \R\notag\\
       &&\omega_i \mapsto X(\omega_i)=x_i \notag
  \end{eqnarray}
\end{Def}
L'ensemble des $x_i$ est donc l'ensemble des valeurs possibles pour
$X$. 
\exemple {\sl Une urne contient 3 jetons noirs et 2 jetons blancs. On tire
sans remise deux jetons dans une urne. On note $X$ le nombre de jetons
noirs tir�s.}\\
On peut choisir: $\Omega=\{ \{N;N \};\{N;B \};\{B;B \} \}$\\
 Alors:
$X\left(\{N;N \}\right)=2$ ; $X(\{N;B \})=1$ ; $X(\{B;B\})=0$. \\
On peut aussi choisir: $\Omega'=\{ (N;N);(N;B);(B;N);(B;B) \}$. On
aurait alors: \\
 $X((N;B))=X((B;N))=1$\ldots

\begin{Def} On appelle \emph{loi de probabilit�} d'une variable al�atoire $X$
  la donn�e pour chaque valeur $x_i$ possible pour $X$ de
  $p_i=P(X=x_i)$. On la pr�sente en g�n�ral sous forme de tableau.
 
\end{Def}
 Ce tableau est � rapprocher de ceux qu'on peut dresser en
 statistiques lorsqu'on a une s�rie de valeurs et leur fr�quence
 d'aparition.
 \begin{prop}
   La somme des $p_i=P(X=x_i)$ vaut 1.
 \end{prop}
Cela vient de ce que les \og$X=x_i$\fg{} forment une partition de $\Omega$.
\exemple On reprend l'exemple pr�c�dent. On dresse un arbre. 

%\psset{radius=6pt,dotsize=4pt,treefit=loose}
%\begin{center}
% \pstree[treemode=R]
%     {\Tcircle{0}}
%     {\pstree{\Tcircle{N}^{$\frac{3}{5}$}}%_{$\frac{1}{2}$}}
%             {\pstree{\Tcircle[name=XU2,linestyle=dashed]{N}}%^{U}}
                    %{\Tdot~{(N,N)}^{L}
                     %\Tdot~{(4,1)}_{R}}
%\pstree{\Tcircle[name=XD2,linestyle=dashed]{B}}
                    %{\Tdot~{(3,0)}^{L}
                     %\Tdot~{(1,1)}_{R}}}
%     \pstree{\Tcircle{B}_{$\frac{1}{2}$}}%_{Y}}
%            {\pstree{\Tcircle[name=YU2,linestyle=dashed]{N}}
                    %{\Tdot~{(1,2)}^{L}
                    % \Tdot~{(2,0)}_{R}}
%             \pstree{\Tcircle[name=YD2,linestyle=dashed]{B}}
                    %{\Tdot~{(3,3)}^{L}
                    % \Tdot~{(0,1)}_{R}}}}
%\end{center}
\vspace{4cm}
On trouve
$P({N;N})=0,3$; $P({N;B})=0,6$; $P({B;B})=0,1$. D'o� la loi de $X$.
\[
\begin{array}{|c|c|c|c|}
\hline
  x_i&0&1&2 \\ \hline
P(X=x_i)&0,1&0,6&0,3 \\
\hline
\end{array}
\]
\begin{Def} On appelle \emph{fonction de r�partition} d'une variable al�atoire $X$
  la fonction $F$ d�finie sur $R$ par:  $F(x)=P(X\leq x)$. 
\end{Def}
Cette fonction est croissante sur $\R$, varie de 0 � 1 mais n'est
\emph{a priori} pas continue. En effet dans notre cas pr�sent, $X$ ne prend qu'un nombre
fini de valeurs. En chacune de ces valeurs $x_i$, $F$ pr�sente une
discontinuit� et m�me plus pr�cis�ment un saut de hauteur $P(X=x_i)$. Entre deux
valeurs cons�cutives, $F$ est constante.
\exemple On reprend notre exemple. Pour $x<0$, $P(X<x)=0$ et
$P(X=0)=0,1$ donc $P(X\leq 0)=0,1$. Soit $F(0)=0,1$ etc \ldots

La connaissance de $F$ permet de rouver la loi de probabilit� de
$X$. Plus tard, on d�finira des variables al�atoires continues, par leur fonction de r�partition.
\section{Esp�rance math�matique, variance}

\begin{Def} $E(X)=\sum \ldots$
\end{Def}
Lin�arit� 

\begin{prop}
  $ E(X+Y)=E(X)+E(Y)$
 \end{prop}
\begin{prop}
$   E(aX+b)=aE(X)+b$
 \end{prop}

\begin{Def} $V(X)=E\left(\left(X-E(X)\right)^2\right)$
\end{Def}
\begin{prop}
$V(X)=E(X^2)-E(X)^2$
 \end{prop}
\end{document}


%%%%%%%%%%%%%%%%%%%%%%%%%%%%%%%%%%%%%%%%%%%%%%%%%%%%%%%%%%%%%%%%%%%%%%%%%%%%%%%%%%%%%%%%%%%%%%%%%%%%%%%%%%%%%%%%%%%%%%%%%%%%%%%%%%%%%%%%%%%%%%%%%%%%%%%%%%%%%%%%%%%%%%%%%%%%%%%%%%%%%%%%%%%%%%%%%%%%%%%%%%%%%%%%%%%%%%%%%%%%%%%%%%%%%%%%%%%%%%%%%%%%%%%%%%%%%%%%%%%%%%%%%%%%%%%%%%%%%%%%%%%%%%%%%%%%%%%%%%%%%%%%%%%%%%%%%%%%%%%%%%%%%%%%%%%%%%%%%%%%%%%%%%%%%%%%%%%%%%%%%%%%%%%%%%%%%%%%%%%%%%%%%%%%%%%%%%%%%%%%%%%%%%%%%%%%%%%%%%%%%%%%%%%%%%%%%%%%%%%%%%%%%%%%%%%%%%%%%%%%%%%%%%%%%%%%%%%%%%%%(%%%%%%%%%%%%%%%%%%%%%%%%%%%%%%%%%%%%%%%%%%%%%%%%%%%%%%%%%%%%%%%%%%%%%%
\begin{pspicture}(0,-0.5)(5,4.5)
\rput(2.2,4){{\it diagramme de Venn}}
\rput(2.2,3){$A\cup B$}
\psframe(0,0)(4.5,3.6)
\pscircle[fillstyle=hlines](1.5,1.5){1}
\pscircle[fillstyle=hlines](3,1.5){1}
\rput(1.5,1.5){A}
\rput(3,1.5){B}
\end{pspicture}
\begin{pspicture}(0,-0.5)(5,4.5)
\rput(2.2,4){{\it diagramme de Venn}}
\rput(2.2,3){$A\cap B$}
\psclip{% bloc colorier
\pscustom{% 1re partie colorier
\pscircle(1.5,1.5){1}}
\pscustom{% 2me partie
\pscircle(3,1.5){1}}
\psframe[fillstyle=hlines](0,0)(4,3)} % hachurer l'intersection des 2
% parties qui se trouvent dans le frame indiqué
\endpsclip % fin bloc colorier (pas de } fermant)  %% style=hachured

\psframe(0,0)(4.5,3.6)
\pscircle(1.5,1.5){1}
\pscircle(3,1.5){1}
\rput(1.5,1.5){A}
\rput(3,1.5){B}
\end{pspicture}\\
Bla bla
%%-----------------------
\end{document}

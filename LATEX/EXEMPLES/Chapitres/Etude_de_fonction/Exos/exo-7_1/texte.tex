\begin{exercice}
On consid�re la fonction $f$ d�finie sur l'intervalle $I=\into{-3}{+\infty}$ par l'expression ci-dessous, de courbe dans un rep�re orthonorm� $\oij$ (unit� 2cm) not�e $\Cr$.
\[ f(x)=\frac{3x+1}{x+3}\]

\begin{enumerate}
\item D�terminer le sens de variation de $f$ sur $I$.
\item D�terminer ses limites aux bornes de l'intervalle $I$.
\item Quelles asymptotes �ventuelles pouvez vous en d�duire. On donnera une �quation pour chaque asymptote.
\item D�terminer par le calcul les coordonn�es des points d'intersection de $\Cr$ avec la droite $\Dr$ d'�quation:~$y=x$
\item D�terminer l'�quation de $T$ la tangente � $\Cr$ au point d'abscisse 1.
\item R�soudre sur $I$: $f(x)\geq x$. En d�duire la position relative de $\Cr$ et $\Dr$ sur $I$.
\item Tracer dans un m�me rep�re $\Cr$, $\Dr$ et les asymptotes. \emph{On fera varier $x$ jusqu'� � 5,5 environ}
\end{enumerate}
 
\end{exercice}

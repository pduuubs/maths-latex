\documentclass[a4paper,12pt]{book}
\usepackage[]{pan_pan} 
\usepackage{eso-pic} %pour filigrane
%\setlength{\textheight}{24cm}
%\setlength{\textwidth}{16cm}
\usepackage[marginratio=1:1,textwidth=16cm,textheight=24.5cm,marginparwidth=0pt,headheight=3cm]{geometry}

\usepackage[dvips,ps2pdf,colorlinks=true]{hyperref} %gestions des hyperliens dans un document .dvi
\pagestyle{fancy}
\renewcommand{\thechapter}{\Roman{chapter}}
\renewcommand{\thesection}{\arabic{section}}
%%
%% Pour en faire un BOOKLET A5
%%
% Suppose your source code is in the file Tous_les_exos_de_VP.tex, then the following
% commands typed into the command prompt will convert the file into an
% A5 booklet: 
%
% latex Tout_le_cours_TS.tex
% latex Tout_le_cours_TS.tex
% dvips -o Tout_le_cours_TS.ps Tout_le_cours_TS.dvi
% psbook Tout_le_cours_TS.ps Tout_le_cours_TS.bk.ps
% psnup -2 Tout_le_cours_TS.bk.ps Tout_le_cours_TS.booklet.ps
% En pdf ensuite avec ps2pdf:
% ps2pdf Tout_le_cours_TS.booklet.ps Tout_le_cours_TS.booklet.pdf

%  The resulting file exam.booklet.ps should then be printed doubled-sided, stapled down the middle, and folded to produce an A5 booklet. The paper size can be changed by using different command line options, see the psutils documentation for more information.

%% Chemins des images
%\includegraphics{/home/pan/Desktop/TeX_MiX/Figures/Metapost_fig/NewCourbe.10}
%penser � Command/File pour creer un .ps pour Metapost
%\includegraphics{/home/pan/Desktop/TeX_MiX/Figures/Fig_num/FIGURE_(99)}}


%%% Reste � faire:
%
%  1) chercher et remplacer les includegraphics par un environnement
%  figure puis \tableoffigures
%  2) Donner les ref des exos concern�s � chaque partie... du livre
%  aussi (?)
%  3) R�pertorier les exos s�par�ment ? En faire une toc? Corrections
%  en fin d'ouvrage?
%  4) Faire un index et donc r�f�rencer des mots clefs
%  
%


%%%%%%%%%%%%%%%%%%%%%%%%%%%%
% Style du livre AMS-math
%%%%

\renewcommand{\chaptermark}[1]%
   {\markboth{\MakeUppercase{\thechapter.\ #1}}{}}
\renewcommand{\sectionmark}[1]%
   {\markright{\MakeUppercase{\thesection.\ #1}}}
\renewcommand{\headrulewidth}{0.5pt}
\renewcommand{\footrulewidth}{0pt}
\newcommand{\Police}{%
   \fontfamily{%
%lmr
pag%style des hauts de page en avant garde small caps
}\fontseries{m}\fontshape{sc}\fontsize{9}{11}\selectfont}
\fancyhf{}
\fancyhead[LE,RO]{\Police \thepage}
\fancyhead[LO]{\Police \rightmark}
\fancyhead[RE]{\Police \leftmark}
%%%%%%%%%%%
\title{%
{\bf Cours de Math�matiques en \term S}}
%\\rule{0cm}{1cm}\noindent\begin{center}
%includegraphics[width=7cm]{petite_frise_bague_viking}
%end{center}}
\author{Vincent PANTALONI}
\date{{\small {\Police Version du \today}}}
\renewcommand{\iff}{\Longleftrightarrow}
%%%%
%\makeindex
%------------- Fin pr�ambule ----
\begin{document}
\frontmatter
%\newcommand{\Pdpweb}{{\footnotesize {\tt
%http://prof.pantaloni.free.fr}}}
%\newcommand{\Pdpweblink}{\url{http://prof.pantaloni.free.fr}}
\lfoot{\Pdpweblink}
\rfoot{{\small {\Police Version du \today}}}
\AddToShipoutPicture*{\AtTextUpperLeft{\rotatebox{270}{\
      \includegraphics[scale=0.5]{/home/pan/Desktop/TeX_MiX/Figures/Fig_num/FIGURE_(111)%
}}}}
\maketitle

\tableofcontents

%\AddToShipoutPicture*{\AtTextLowerLeft{\includegraphics[width=16cm]{petite_frise_bague_viking}}}
%\listoffigures
\mainmatter
\lfoot{\Pdpweb}%\lfoot{}
\rfoot{}
%%___________________________________
%%
%%------------- Debut des INPUT  ----
%%___________________________________
\chapter{Suites, raisonnement par r�currence.}% suites adjacentes}
\section{Rappels}
\par \noindent
\renewcommand{\arraystretch}{2.5}
\begin{tabular}{||c|p{5.2cm}|p{5.5cm}||}
\hline
%Ligne 1
  Suite $(u_n)$, $n\in \N$ & \hspace{\stretch{1}} Suite \textbf{arithm�tique} de raison $r$  \hspace{\stretch{1}} & \hspace{\stretch{1}}  Suite \textbf{g�om�trique} de raison $q$ \hspace{\stretch{1}} \\\hline \hline
%Ligne 2
D�finition & On passe de chaque terme au suivant en ajoutant le m�me r�el $r$ \[u_{n+1}=u_n+r \] & On passe de chaque terme au suivant en multipliant par le m�me r�el $q$ \[u_{n+1}=q\times u_n \]\\ \hline
%Ligne 3
 Terme g�n�ral& \[u_n=u_0+nr\]\[u_n=u_1+(n-1)r\] & \[u_n=u_0\times q^n\]\[u_n=u_1\times q^{n-1}\] \\\hline
%Ligne 4
Somme de termes & ``Nombre de termes''$\times$ ``Moyenne du premier et dernier terme''
\[ \sum_{k=0}^{n}u_k=(n+1)\times \frac{u_0+u_n}{2}\]& ``Premier terme''$\times \dfrac{1-q^{\text{nb de termes}}}{1-q}$ \[ \sum_{k=0}^{n}u_k=u_0\times \frac{1-q^{n+1}}{1-q}\]
\\\hline

 Cas particuliers& \[1+2+\cdots+n=\dfrac{n(n+1)}{2}\]& \[1+q+q^2+\cdots+q^n=\frac{1-q^{n+1}}{1-q}\]\\\hline
\end{tabular}
\renewcommand{\arraystretch}{1.5}
\subsection{Exercices corrig�s}
%-------------------
\setcounter{Aff}{1}  %<<<<<<<<<<<<<<<<<<<
%-------------------
%
%Si ``Aff'' vaut 0 on n'affiche pas les solutions sinon, on les affiche, s'il y en a... ( Aff=\theAff )
\Exo[C'est une somme g�om�trique du type $1+q+q^2+\cdots+q^n $ avec $q=x^2$. D'o�: $S=\dfrac{1-(x^2)^{n+1}}{1-x^2}$ si $x^2\neq1$. Si $x=1$ alors $S=1+1+\cdots+1=n+1$]{Calculer $S = 1 + x^2 + x^4 +\cdots + x^{2n}$}
\Exo[On a $u_n=u_0+nr$. En rempla�ant $n$ par 34, on d�termine $r$. Puis on calcule $u_{100}$]{Soit $(u_n)$ une suite arithm�tique de raison $r$. On donne: $u_0=3$ et $u_{34}=321$. D�terminer $u_{100}$.}
\Exo[C'est la somme des termes de la suite $(u_n)$ g�om�trique de raison 2 et de premier terme 5. Il reste � d�terminer le nombre de termes de la somme. $u_n=5\times 2^n \implies 5\times 2^n =5120 \implies 2^n=1024 \implies n=10$. Donc $S=u_0+\cdots+u_{10}$ qui comporte 11 termes. D'o�: $S=5\times \dfrac{1-2^{11}}{1-2}=5(2^{11}-1)=10235$ ]{Calculer la somme: $S=5+10+20+40+\cdots+5120$}
\Exo[$v_{n+1}=u_{n+1}-1=-2u_n+3-1=-2(u_n-1)$. Donc $v_{n+1}=-2v_n$, $(v_n)$ est g�om�trique de raison -2 et de premier terme $v_0=u_0-1=4$. D'o� $v_n=4\times (-2)^n$. Ainsi $u_n=4\times (-2)^n+1$ ]{Soit la suite d�finie par $\forall n \in \N, u_{n+1}=-2u_n+3$ et $u_0=5$. On pose aussi $v_n=u_n-1$. Montrer que $(v_n)$ est g�om�trique, d�terminer le terme g�n�ral de $(u_n)$.}
%%
\setcounter{Aff}{0}
%%
\begin{prop}
  Limite de $q^n$.
\end{prop}
\section{Comportement global d'une suite}
\subsection{Monotonie}
\begin{Def}
  Croissante, d�croissante, constante, monotone
\end{Def}
\begin{rem}
  Pareil avec strict.
\end{rem}
\subsection{Major�e, minor�e, born�e}
\begin{Def}
  Maj, min, born�e.
\end{Def}
\begin{rem}
Pas unicit� d'un maj.
\end{rem}
\exemple $u_n=4-\frac{1}{n}$
\section{Raisonnement par r�currence}
Le raisonnement par r�currence (on dit aussi par induction) est un principe de raisonnement (comme le raisonnement par l'absurde) qui sert � �tablir une propri�t� valable pour une infinit� d'entiers naturels. Ce raisonnement comporte quatre �tapes: L'initialisation, l'hypoth�se de r�currence, l'h�r�dit� et la conclusion.
\subsection{Principe}
\noindent Soit $\Para (n)$ une propri�t� d�pendant d'un entier~$n$ dont on veut prouver qu'elle est vraie pour tout $n$ dans $\N$. (Parfois pour tout $n$ dans $\N^*$ ou $n\geq 3$ \dots etc)\medskip\\
\fbox{%{0.5\linewidth}{

\parbox{0.9\linewidth}{
%\noindent Soit $\Para (n)$ une propri�t� d�pendant d'un entier~$n$ dont on veut prouver qu'elle est vraie pour tout $n$ dans $\N$. (Parfois pour tout $n$ dans $\N^*$ ou $n\geq 3$ \dots etc)
\begin{itemize}
\item[\bf Initialisation.] On v�rifie que $\Para(0)$ est vraie.
\item[\bf Hypoth�se de r�currence.] On suppose que pour un certain $k\in\N$, $\Para (k)$ est vraie.
\item[\bf H�r�dit�.] On montre que sous l'hypoth�se de r�currence, $\Para (k+1)$ est aussi vraie.
\item[\bf Conclusion.] Alors par r�currence, la propri�t� $\Para (n)$ est vraie pour tout $n\in\N$.
\end{itemize}
}}\medskip\\
Toutes les phases du raisonnement sont n�c�ssaires, mais l'essentiel de la difficult� provient en g�n�ral dans la phase de l'h�r�dit�. \`A cet endroit il faut trouver un lien entre $\Para (k)$ et $\Para (k+1)$.
Pour expliquer ce raisonnement on utilise parfois l'analogie suivante:
\subsection{Analogie: gravir un escalier}
\noindent On se trouve sur la premi�re marche d'un escalier infini � gravir. Si on sait monter une marche, alors on peut acc�der � n'importe quelle marche de l'escalier.
\begin{itemize}
\item[\bf Initialisation.] On est sur la premi�re marche, donc on peut commencer.
\item[\bf Hypoth�se de r�currence.] On suppose que pour un certain $k\in\N$, on soit sur la $k$-i�me marche.
\item[\bf H�r�dit�.] On v�rifie qu'on peut alors passer � la marche suivante, la $(k+1)$-i�me.
\item[\bf Conclusion.] Alors par r�currence, on peut acc�der � la marche num�ro $n$ pour tout $n$ dans~$\N^*$.
\end{itemize}
\subsection{Exemples}
\begin{dingautolist}{172}
\item
Prouver la formule suivante donnant la somme des premiers entiers naturels:
\[ 1+2+3+\cdots+n=\frac{n(n+1)}{2}\]
\begin{proof}

On note pour tout $n\in\N^*$: $S_n=1+2+3+\cdots+n$. La propri�t� � prouver par recurrence pour tout $n$ de $\N^*$ est donc:
\[ \Para (n): \qquad S_n=\frac{n(n+1)}{2}\]
\begin{itemize}
\item[\bf Initialisation.] $S_1=1$ et pour $n=1$, $\frac{n(n+1)}{2}$ vaut $\frac{1\times 2}{2}=1$. Donc $\Para (1)$ est vraie.
\item[\bf Hypoth�se de r�currence.] On suppose que pour un certain $k\in\N$, $\Para (k)$ est vraie \ie{}  $S_k=\frac{k(k+1)}{2}$.
\item[\bf H�r�dit�.] Ici on veut prouver que $S_{k+1}= \frac{(k+1)(k+2)}{2}$. Or\footnote{L'argument qui suit est la clef pour passer du rang $k$ au rang suivant~$k+1$.}: $S_{k+1}=S_k+(k+1)$. Comme par hypoth�se de r�currence $S_k=\frac{k(k+1)}{2}$ on en d�duit que:
\[S_{k+1}=\frac{k(k+1)}{2}+(k+1)=(k+1)(1+\frac{k}{2})=\frac{(k+1)(k+2)}{2}\]
Ainsi $\Para (k+1)$ est vraie.
\item[\bf Conclusion.] Par r�currence, on a prouv� que  pour tout $n$ dans~$\N^*$,  $S_n=\frac{n(n+1)}{2}$.

\end{itemize}
\end{proof}

\item Prouver par r�currence que pour $n$ entier assez grand, $2^n \geq  n^2$.
\begin{proof} On commence par chercher � la main ou avec une calculette un rang � partir duquel la propri�t� semble vraie. On pense � $n=4$. Soit $\Para (n)$ la propri�t�: $2^n \geq  n^2$. On veut prouver qu'elle est vraie pour tout $n$ dans $\N$ sup�rieur ou �gal �~$4$.
\begin{itemize}
\item[\bf Initialisation.] $2^4=16$ et $4^2=16$ donc $2^4 \geq  4^2$. Donc $\Para (4)$ est vraie.
\item[\bf Hypoth�se de r�currence.] On suppose que pour un certain
  $k\in\N$, avec  $k\geq 4$, $\Para (k)$ est vraie \ie{} $2^k \geq  k^2$.

\item[\bf H�r�dit�.] Ici on veut prouver que $2^{k+1} \geq  (k+1)^2$.
\[2^{k+1}=2\times 2^{k} \geq 2k^2  \]
Il suffit donc de prouver que pour $k\geq 4$ on a $2k^2\geq
(k+1)^2$. Les deux membres sont positifs, cette �galit� est
�quivalente � $\sqrt{2}k\geq k+1$ soit $k\geq \frac{1}{\sqrt{2}-1}$,
ou encore $k\geq \sqrt{2}+1$. Or $\sqrt{2}+1\leq 4$.
Ainsi $\Para (k+1)$ est vraie.
\item[\bf Conclusion.] Par r�currence, on a prouv� que  pour tout
  entier $n$ sup�rieur � 4, $2^n \geq  n^2$.

\end{itemize}
\end{proof}
\end{dingautolist}



\section{\'Etude des suites du type $u_{n+1}=f(u_n)$}
On se donne une fonction $f$ d�finie, continue sur un intervalle $I$ de
$\R$. Soit $u_0\in I$.
\subsection{Intervalle stable}
\Def{On dit que $I$ est stable par $f$ si $f(I)\subset I$}
\exemples
  \begin{enumerate}
  \item $f(x)=x(1-x)$ sur $I=\intf{0}{1}$. $I$ est stable par f, mais
pas par $g=5f$. Il suffit d'�tudier $f$.
\item $f(x)=\frac{1}{x-1}$ et $u_0=\frac32$ V�rifier que la relation
  $u_{n+1}=f(u_n)$ ne d�finit pas une suite.
  \end{enumerate}

La condition $I$ stable par $f$ permet de garantir que la suite est
d�finie et que tous les termes de la suite sont dans l'intervalle $I$.
(r�currence imm�diate)

\subsection{Sens de variation}
\begin{prop}
  \begin{enumerate}
  \item $ \forall x \in I,\ f(x)-x\geq 0 \implies$ $(u_n)$ croissante
  \item $\iff \forall x \in I,\ f(x)-x\leq 0 \implies$ $(u_n)$ d�croissante
  \end{enumerate}
\end{prop}

\proof{Facile}
\begin{prop}
\begin{enumerate}
  \item $f$ croissante $\implies$ $(u_n)$ monotone. Le sens de var est
    donn� par le signe de $u_1-u_0$
  \item $f$ d�croissante $\implies$ $(u_{2n})$ et $(u_{2n+1})$ sont
    monotones de monotonies contraires
\end{enumerate}
\end{prop}
\proof{ Supposons $u_1>u_0$.
\begin{enumerate}
  \item Par r�currence. $\Para (n)$: \og $u_{n+1}-u_n\geq 0$\fg{}
  \item On pose $p_n=u_{2n}$ et $i_n=u_{2n+1}$. Alors:
\[p_{n+1}=f\circ f(p_n) \quad \et{}\quad i_{n+1}=f\circ f(i_n)\]
Or $f$ dec. implique $f\circ f$ croissante donc par le 1. $p$ et $i$ sont
monotones.
Supposons $p$ croissante, alors $u_2\geq u_0$ donc, en appliquant $f$
qui est dec. on a $u_3\leq u_1$ donc $i_1 \leq i_0$ donc $i$ est dec.
\end{enumerate}
}

\subsection{Convergence}
\Def{On appelle point fixe d'une fonction $f$ un r�el tel que $f(x)=x$}
\begin{theo}Si $f$ est continue sur $I$ et que $(u_n)$ converge, alors la
  limite est n�cessairement un point fixe de $f$
\end{theo}
Attention: L'existence d'un point fixe ne garantit pas la convergence
de $(u_n)$. Mais l'absence de point fixe suffit � justifier que $(u_n)$
ne converge pas.
En pratique: On justifie que $(u_n)$ CV (croissante major�e par
exemple) puis on d�termine la limite en cherchant les points fixes de
$f$. Si le point fixe est unique, c'est facile, sinon il faut
raisonner avec le sens de variation de $(u_n)$.

\exemple $u_{n+1}=\frac{u_n}{2-u_n}$ et $u_0 \in \intf{0}{1}$

\chapter{Limites et continuit�}
\section{Rappels: Limites de r�f�rence}
\noindent Voici les courbes des fonctions carr�, cube et inverse. Leurs limites sont � conna�tre. \\\rule{0cm}{0.3cm}\\
\psfrag{O}{$O$} \psfrag{i}{$\vi$}   \psfrag{j}{$\vj$}
\resizebox{4cm}{!}{\includegraphics{/home/pan/Desktop/TeX_MiX/Figures/Metapost_fig/Courbe.2} }%\includegraphics[totalheight=6cm]{Courbe.2} %Para
\hfill \resizebox{2.8cm}{!}{\includegraphics{/home/pan/Desktop/TeX_MiX/Figures/Metapost_fig/Courbe.6} }%\includegraphics[totalheight=6cm]{Courbe.6} %Cube
\hfill \resizebox{6cm}{!}{\includegraphics{/home/pan/Desktop/TeX_MiX/Figures/Metapost_fig/Courbe.3} }
%\includegraphics[totalheight=6cm]{Courbe.3} %Hyper


\begin{minipage}{0.22\textwidth}
\begin{align*}
 \lim_{x\to +\infty}\,x^2&=+\infty\\
\lim_{x\to -\infty}\,x^2&=+\infty
\end{align*}
\end{minipage} \hfill
\begin{minipage}{0.32\textwidth}
\begin{align*}
 \lim_{x\to +\infty}\,x^3&=+\infty\\
\lim_{x\to -\infty}\,x^3&=-\infty
\end{align*}
\end{minipage} \hfill
\begin{minipage}{0.18\textwidth}
\begin{align*}
\lim_{x\to +\infty}\,\frac{1}{x}&=0\\ 
\lim_{x\to -\infty}\,\frac{1}{x}&=0
\end{align*}
\end{minipage}\hfill 
\begin{minipage}{0.18\textwidth}
\begin{align*}
\lim_{x\to 0^+}\,\frac{1}{x}&=+\infty\\ 
\lim_{x\to 0^-}\,\frac{1}{x}&=-\infty
%\lim_{\stackrel{x \to 0}{x<0}}\frac{1}{x}&=-\infty
\end{align*}
\end{minipage} 
%\includegraphics[totalheight=6cm]{/home/pan/Desktop/TeX_MiX/Figures/Metapost_fig/Courbe.2} %Para
%\includegraphics[totalheight=6cm]{/home/pan/Desktop/TeX_MiX/Figures/Metapost_fig/Courbe.6} %Cube
%\includegraphics[totalheight=6cm]{/home/pan/Desktop/TeX_MiX/Figures/Metapost_fig/Courbe.3}
% Hyper
%\section{Fonctions continues}

%\section{Limites, par la d�finition}
\section{Rappels sur la d�rivation, notation diff�rentielle}
\begin{Def}
 
Soit $a\in I$ o� $I$ est un intervalle \emph{ouvert}. Soit $f$
  une fonction d�finie sur $I$. On dit que $f$ est d�rivable en $a$
  si le taux de variation de $f$ entre $a$ et $a+h$:
\[\frac{f(a+h)-f(a)}{h}\]
admet une limite finie quand $h$ tend  vers~0.
Si cette limite finie existe, on l'appelle le nombre d�riv� de $f$ en
$a$, not�~$f'(a)$.
   \end{Def}
\begin{rem}
En posant $x=a+h$, cette d�finition �quivaut �: $ \boxed{\lim\limits_{ x \to a } \frac{f(x)-f(a)}{x-a}=f'(a)}$
\end{rem}

\centrage{Interpr�tation graphique}
\begin{multicols}{2}
  
  Avec les notations du graphique, on a: 
\[\frac{\Delta y}{\Delta x}=\frac{f(a+h)-f(a)}{h}\]
Quand  $h$ tend vers z�ro  le point $M_h$ se rapproche du point
$A$. Dire que $f$ est d�rivable en $a$ signifie que la s�cante
$(AM_h)$ admet une position limite. On appelle alors tangente  � la courbe de~$f$ au point d'abscisse~$a$,
la droite~$T_a$ passant par~$A$ et de coefficient directeur~$f'(a)$.
Ainsi l'�quation r�duite de $T_a$ est: 
\[ \boxed{y=f'(a)(x-a)+f(a)}\]

  \includegraphics{/home/pan/Desktop/TeX_MiX/Figures/Metapost_fig/Courbe.18}
\end{multicols}
  Autour du point $A$, la courbe de $f$ et sa tangente $T_a$ sont \og
  tr�s proches \fg{}. D'o� la propri�t�:%Cette constatation donne lieu �:
\begin{prop}[Approximation affine] Si $f$ est d�rivable
en $a$, et que $h$ est proche de z�ro: \\
$\boxed{f(a+h)\approx f(a)+hf'(a)}$. Ou encore, si $x$ est proche de $a$, on a: $\boxed{f(x)\approx f(a)+(x-a)f'(a)}$
\end{prop}
\begin{proof}
  La d�finition 1 signifie que $\frac{f(a+h)-f(a)}{h}=f'(a)+\varepsilon(h)$
o� $\varepsilon$ est une fonction qui tend vers 0 en~0. D'o� le
r�sultat, qui consiste � n�gliger le terme $h\varepsilon(h)$. L'autre
expression vient en posant $x=a+h$.
\end{proof}
%\exemples Pour $x$ proche de 0: %$\e^x\approx x+1$ 
%$\sin x \approx x$, $(1+x)^n\approx 1+nx$\\

\centrage{Notation diff�rentielle et sciences physiques}
\begin{multicols}{2}
Soit $f$ une fonction d�rivable sur un intervalle $I$, c'est � dire
d�rivable en tout r�el $a$ � l'int�rieur de $I$. Alors en
math�matiques, on note $f'$ la fonction d�riv�e de $f$, mais pas en
physique\dots{} Comme:
\[ f'(a)=\lim\limits_{\Delta x \to 0 } \frac{\Delta y}{\Delta x}\]
Lorsque le $\Delta x$ est infiniment petit, on le note ${\rm d}x$.  De
m�me le $\Delta y$ devient ${\rm d}y$. En physique, au lieu de noter
$f'$ on notera alors $\frac{{\rm d}y}{{\rm d}x}$ ou $\frac{{\rm
    d}f}{{\rm d}x}$. De plus, en physique, la variable est souvent le
temps, donc not� $t$ au lieu de $x$. On a donc: $f:\ t \longmapsto
f(t)$. Si on d�rive � nouveau $f'$ on obtient la d�riv�e
\emph{seconde} de $f$ not�e $f''$. Si on red�rive cette derni�re on
obtient $f'''$ que l'on note aussi $f^{(3)}$, ainsi de suite,
$f^{(4)}$ d�signe la d�riv�e quatri�me de~$f$. Avec la notation
diff�rentielle utilis�e en physique, cel� donne:
\[ f'=\frac{{\rm d}f}{{\rm d}t}\qquad f''=(f')'=\frac{{\rm d}}{{\rm d}t}\left(\frac{{\rm d}f}{{\rm d}t}\right)=\frac{{\rm d}^2f}{{\rm d}t^2}\]
Remarquez comme l'op�ration \og d�river par rapport � $t$\fg{} se
comporte avec cette notation comme si on multipliait par l'op�rateur:
$\frac{{\rm d}}{{\rm d}t}$.
\par Pour embrouiller un peut plus, la fonction
en physique est souvent not�e $x$. $x(t)$ repr�sente typiquement
l'abscisse d'un point mobile en fonction du temps~$t$. La vitesse
moyenne du point entre les temps $t_1$ et $t_2$ est alors:
\[\frac{\Delta x}{\Delta t}=\frac{x(t_2)-x(t_1)}{t_2-t_1}\]
 La vitesse
instantan�e est alors la limite de $\frac{\Delta x}{\Delta t}$ quand
$\Delta t$ tend vers z�ro, c'est donc: $x'(t)$, euh\ldots pardon
$\frac{{\rm d}x}{{\rm d}t}$. Mais les physiciens aussi feignants que les math�maticiens
on trouv� une notation plus courte~$\dot{x}$ qui d�signe donc la
vitesse $v$. De m�me vous savez (ou apprendrez) que l'acc�l�ration est la d�riv�e
de la vitesse. On a alors:
\[\dot{x}=\frac{{\rm d}x}{{\rm d}t}=x'\quad\text{et}\quad
\ddot{x}=\frac{{\rm d}^2x}{{\rm d}t^2}\]
Comment note--t--on alors la vitesse instantan�e au temps~$t_1$? En
maths ce serait $x'(t_1)$, en physique il y a trois notations
possibles:

\[\dot{x}(t_1)=\left(\frac{{\rm d}x}{{\rm d}t}\right)(t_1)=\left(\frac{{\rm d}x}{{\rm d}t}\right)_{t=t_1}\]
\end{multicols}
\section{Continuit�}
\subsection{D�finitions, exemples et contre exemples}
La continuit� est une notion de \emph{r�gularit�} des fonctions. En
voici une d�finition heuristique:
\begin{Def}
On dit qu'une fonction $f$ est continue sur un intervalle $I$ lorsque
$f$ est d�finie sur $I$ et que sa courbe sur $I$ peut se tracer \og sans lever le crayon \fg{}.
\end{Def}

\begin{rem}
Cette d�finition est �videment uniquement intuitive, une \emph{vraie}
d�finition utilise les limites en un point (cf d�finition suivante).
\end{rem}

\begin{Def}
On dit qu'une fonction $f$ est continue en $a$ ($a\in\R$) si les trois
conditions suivantes sont v�rifi�es:
\begin{dingautolist}{172}
\item $f$ est d�finie en $a$.
\item $f(x)$ admet une limite quand $x$ tend vers $a$.
\item $\displaystyle \lim_{x\to a}f(x)=f(a)$
\end{dingautolist}
\end{Def}
\noindent Si l'une quelconque de ces trois conditions n'est pas v�rifi�e, on dit
que $f$ n'est pas continue en $a$, ou qu'elle pr�sente une
discontinuit� en~$a$. 
\begin{rem}
Parfois le comportement d'une fonction $f$ est diff�rent � gauche et �
droite d'un r�el $a$. Il pourra alors �tre utile pour �tudier
l'existence d'une limite en $a$ d'�tudier les �ventuelles limites �
gauche et � droite de $a$, \emph{i.e.}:
\[ \lim\limits_{ x \to a \atop x < a } f(x) \quad\text{et}\quad
\lim\limits_{ x \to a \atop x > a } f(x)\]
Si ces limites existent, alors, dire que  $f$ est continue en $a$
revient � dire que ces deux limites sont �gales � $f(a)$.
\end{rem}
\noindent Les quatre fonctions $f$ repr�sent�es ci--dessous pr�sentent
une discontinuit� en~$1$. Dire lequel des trois points de la
d�finition est mis en d�faut et pourquoi.\\\noindent
\includegraphics{/home/pan/Desktop/TeX_MiX/Figures/Metapost_fig/NewCourbe.20}%
\hspace{\stretch{1}}%
\includegraphics{/home/pan/Desktop/TeX_MiX/Figures/Metapost_fig/NewCourbe.21}%
\hspace{\stretch{1}}%
\includegraphics{/home/pan/Desktop/TeX_MiX/Figures/Metapost_fig/NewCourbe.22}%
\hspace{\stretch{1}}%
\includegraphics{/home/pan/Desktop/TeX_MiX/Figures/Metapost_fig/NewCourbe.23}%
\\\noindent
\fbox{\begin{minipage}{3.3cm}\[\]\[\]\[\]
  \end{minipage}}\hspace{\stretch{1}}%
\fbox{\begin{minipage}{3.3cm}\[\]\[\]\[\]
  \end{minipage}}\hspace{\stretch{1}}%
\fbox{\begin{minipage}{3.3cm}\[\]\[\]\[\]
  \end{minipage}}\hspace{\stretch{1}}%
\fbox{\begin{minipage}{3.3cm}\[\]\[\]\[\] \end{minipage}}%

\begin{Def}
On dit qu'une fonction $f$ est continue sur un intervalle $I$ lorsque
$f$ est continue en tout r�el $a$ de~$I$.
\end{Def}
{\bf Convention} Dans un tableau de variation, lorsqu'on note une
fl�che pour une fonction croissante (ou d�croissante) sur un
intervalle, cette fl�che signifie aussi la continuit� de la fonction
sur l'intervalle. On mettrait une double barre en un point de discontinuit�.
\begin{prop}
  Une fonction d�rivable sur un intervalle est continue sur cet intervalle.
\end{prop}
\Danger La r�ciproque est fausse. Contre--exemple: La fonction valeur
absolue. Elle est continue sur~$\mathbb{R}$ mais non d�rivable en~0.
La d�rivabilit� est donc une notion de r�gularit� plus forte que
la continuit�.
\begin{proof}(Du contre--exemple)
  $|x|= 
\begin{cases} x & \text{if $x\geq 0$,}
\\
-x &\text{if $x\leq 0$.}
\end{cases}\implies%
\frac{|x|}{x}= 
\begin{cases} 1 & \text{if $x> 0$,}
\\
-1 &\text{if $x< 0$.}
\end{cases}$\\
Le taux de variation de la valeur absolue en 0 a une limite � gauche
qui vaut -1, et une limite � droite qui vaut 1. Donc la limite en 0
n'existe pas.

\end{proof}
\begin{proof}
  On �crit l'approximation affine avec une fonction $\epsilon$ telle
  que: $\lim\limits_{ x \to a } \epsilon(x) =0$:
\[f(x)=f(a)+(x-a)f'(a)+\epsilon(x)\]
\end{proof}
\begin{prop} (Admise)
Les fonctions usuelles: polyn�mes, racine carr�e, valeur absolue,
fractions rationnelles, $\sin$, $\cos$,(bient�t  $\exp$ et $\ln$) et leurs compos�es sont continues sur les intervalles o� elles sont d�finies.
\end{prop}
Cette propri�t� permet de justifier la continuit� d'une fonction en un
point ou sur un intervalle.
\exemple Soit $f$ d�finie par: $f(x)=\sqrt{x^2-1}$. $f$ est d�finie
d�s que $x^2-1$ est positif donc sur les intervalles
$\intof{-\infty}{-1}$ et $\intfo{1}{+\infty}$. Sur chacun de ces
intervalles, $f$ est la compos�e de la fonction racine (continue sur
$\R^+$) et d'un polyn�me (donc continu sur $\R$). Ainsi $f$ est
continue sur les intervalles $\intof{-\infty}{-1}$ et
$\intfo{1}{+\infty}$. La phrase \og{\sl $f$ est continue sur les intervalles $\intof{-\infty}{-1}$ et
$\intfo{1}{+\infty}$ comme compos�e  de fonctions continues}\fg{} est plus
vague mais suffira.\\

Cependant il existe des fonctions qui ne sont pas continues:

\subsection{La fonction partie enti�re}
\begin{Def}
On appelle partie enti�re d'un r�el $x$, not�e $E(x)$, le plus grand entier inf�rieur ou �gal � $x$. \ie{} $E(x)$ est l'entier $n$ tel que: $n\leq x < n+1$
 \end{Def}
\exemples $E(2,36)=2$;  $E(2)=2$; $E(\pi)=3$; $E(-2,36)=-3$
\begin{prop}
Pour tout entier $n$, sur l'intervalle $\intfo{n}{n+1}$, la fonction partie enti�re est constante �gale �~$n$. Elle n'est pas continue sur $\R$, elle admet une discontinuit� pour chaque entier.%\ie{} $\forall x \in \intfo{n}{n+1},\ E(x)=n$
\end{prop}
\begin{proof} Soit $n\in\Z$. Par d�finition de $E$, pour
  $x\in\intfo{n}{n+1}$ on a $E(x)=n$. Prouvons la discontinuit� de $E$
  en~$n+1$. Par passage � la limite (en restant dans l'intervalle
  $\intfo{n}{n+1}$): 
\[ %\displaystyle 
\lim_{x\rightarrow n+1\atop x<n+1}  E(x)=n\neq E(n+1)=n+1 \]

\end{proof}
On en d�duit la courbe de la fonction partie enti�re, on dit que c'est une fonction en escalier.
\begin{center}
\includegraphics{/home/pan/Desktop/TeX_MiX/Figures/Metapost_fig/Courbe.9} %\hfill  \includegraphics{courbe.8}
\end{center}
\subsection{Prolongement par continuit�}
\exemple La fonction $f$ d�finie sur $\R\setminus \{3\}$ par:
\[f(x)=\frac{x^2-9}{x-3}\] n'est pas continue sur $\R$ car non d�finie
en $3$, mais:
\[ x\neq 3\implies f(x)=\frac{(x-3)(x+3)}{x-3}=x+3\]
Ainsi la courbe de $f$ est la droite d'�quation: $y=x+3$ � laquelle il
manque le point d'abscisse~3. On pose alors la fonction $\widetilde{f}$ d�finie sur
$\R$:
\[ \widetilde{f}(x)=x+3.\quad \text{donc}\quad \widetilde{f}(x)=
\begin{cases} f(x) & \text{si $x\neq 3$,}
\\
6 &\text{si $x=3$.}
\end{cases}
\]
La fonction $\widetilde{f}$ est alors continue sur $\R$, on dit que
c'est un prolongement de $f$ par continuit� en~3.
\exemple La fonction $f$ d�finie sur $\R^*$ par $f(x)=\dfrac{\e^x-1}{x}$ n'est pas d�finie en z�ro, mais on a vu que sa limite en z�ro
existe et vaut 1. On peut \og prolonger la fonction $f$ par
continuit�\fg{} en~0, en posant la fonction $\tilde f$ d�finie sur
$\R$ par:
$\tilde f(x)=f(x)$ si $x\in\R^*$ et $\tilde f(0)=1$
\section{Th�or�me des valeurs interm�diaires}

\begin{theo}{\bf Th�or�me des valeurs interm�diaires.}
Soit $f$ une fonction continue sur un intervalle $\intf{a}{b}$. Soit $\lambda$ un r�el compris strictement entre $f(a)$ et $f(b)$. Alors l'�quation $f(x)=\lambda$ admet au moins une solution dans l'intervalle ouvert $\into{a}{b}$.
\end{theo}

\begin{rem}
Si la fonction n'est pas continue, cette propri�t� n'est pas n�cessairement v�rifi�e. La fontion partie enti�re fournit un contre exemple: $E(0)=0$ et $E(1)=1$ mais l'�quation $E(x)=0,5$ n'a pas de solution dans $\into{0}{1}$ (ni ailleurs).
\end{rem}

\begin{prop}[Cas particulier important: $\lambda=0$]
Soit $f$ une fonction continue sur un intervalle $\intf{a}{b}$ avec $f(a)$ et $f(b)$ de signes contraires, alors l'�quation $f(x)=0$ admet au moins une solution dans l'intervalle ouvert $\into{a}{b}$
\end{prop}
C'est un cas particulier, mais en fait cette propri�t� est �quivalente
au th�or�me des valeurs interm�diaires, et on la d�montre par
dichotomie en utilisant des suites adjacentes qui convergent vers une
racine de $f$. (cf TD)\\
Ces propri�t�s nous garantissent l'existence de solutions pour des
�quations mais ne nous fournissent pas les solutions, ni leur unicit�, juste des intervalles o� elles se trouvent. On peut donc avec la calculatrice trouver des encadrements � la pr�cision souhait�e des solutions cherch�es.

\exemple
Sur le graphique ci-dessous, on a une courbe $\Cr$ repr�sentant une fonction continue $f$ sur l'intervalle $\intf{-3}{2}$. On a $f(-3)=-1$ et $f(2)=$1,5. Comme la valeur $\lambda=1$ est comprise entre $-1$ et 1,5  le th�or�me des valeurs interm�diaires nous assure que l'�quation $f(x)=1$ admet au moins une solution (ici il y en a trois) sur l'intervalle $\into{-3}{2}$. Autrement dit, la courbe $\Cr$ coupe la droite d'�quation $y=1$ sur cet intervalle.

\begin{center}
\psfrag{C}{$\Cr$}
\includegraphics{/home/pan/Desktop/TeX_MiX/Figures/Metapost_fig/Courbe.10} 
\end{center}

En pratique, la simple lecture d'un tableau de variation permet de
r�pondre aux questions du type: ``Combien l'�quation $f(x)=\lambda$
admet elle de solutions? Donner des encadrements de ces solutions.''
Rappel: on admet qu'une fl�che dans un tableau de variation traduit
aussi la continuit� de la fonction sur l'intervalle en question. La
version suivante du th�or�me des valeurs interm�diares permet de
garantir l'unicit� de la solution:

\begin{theo}[Utile en pratique\footnote{On dit dans ce cas que $f$ r�alise une \emph{bijection}
croissante de $\intf{a}{b}$ dans $\intf{f(a)}{f(b)}$. C'est pourquoi
cette propri�t� est parfois appell�e \og th�or�me de la bijection\fg{}.}]
Soit $f$ une fonction continue et strictement croissante sur un
intervalle $\intf{a}{b}$. Soit $\lambda$ un r�el tel que: $f(a)<
\lambda < f(b)$.  Alors l'�quation $f(x)=\lambda$ admet une {\bf unique} solution dans l'intervalle ouvert $\into{a}{b}$.
\end{theo}
\begin{proof}   Laiss� en exercice.
\end{proof}
\begin{rem}
Il existe �videmment un �nonc� similaire pour une fonction strictement d�croissante...
\end{rem}
\exemple Soit $f$ d�finie sur $\R$ par $f(x)=x^5+x+7$. Justifier que $f(x)=0$ admet une unique solution dans l'intervalle $\intf{-2}{1}$. R�ponse d'un bon �l�ve: \\
$f$ est continue sur $\R$ (c'est un polyn�me). Or $f'(x)=5x^4+1$ et $x^4$ est positif donc la d�riv�e de $f$ est strictement positive, donc $f$ est strictement croissante sur $\intf{-2}{1}$. De plus $f(-2)=-33<0 $ et $f(1)=9>0$. Donc $f(x)=0$ admet une unique solution dans l'intervalle $\intf{-2}{1}$.
D'apr�s le tableau de variation de $f$ et ses limites en $-\infty$ et $+\infty$ on en d�duit aussi que $f(x)=0$ admet en fait une unique solution dans $\R$.
\\ Il est par contre imposible de r�soudre par le calcul cette
�quation, mais pouvez vous � l'aide de la calculette d�terminer un
encadrement d'amplitude $10^{-2}$ du r�el $\alpha$ v�rifiant
$f(\alpha)=0$. Trois m�thodes:\\

{\bf M�thodes de r�solution approch�e de $f(x)=\lambda$}
{\small
\begin{list}{$\bullet$}{}
\item Graphiquement, en zoomant de plus en plus sur le point
  d'intersection de la courbe avec l'axe des abscisses (pour $\lambda=0$) ou avec la droite d'�quation $y=\lambda$.
\item Mieux encore, certaines calculettes disposent d'un outil de
  r�solution graphique (G-solv) il faut choisir {\sl isect} pour
  l'intersection de deux courbes que l'on doit s�lectionner (la courbe
  de $f$ et la droite d'�quation $y=\lambda$ si on veut r�soudre
  $f(x)=\lambda$) ou {\sl Root} si on cherche une racine.
\item En utilisant le tableau de valeurs que l'on doit param�trer
  correctement. $pitch$ (Casio) ou $\Delta Tbl$ (TI) donne le pas de
  la variation de $X$,  donc la pr�cision voulue. Mais il faut d'abord
  connaitre un vague encadrement de la solution (par une m�thode
  graphique par exemple)
\item \`A l'aide d'un programme adapt�, comme celui que nous allons
  �tablir qui fonctionne par dichotomie. On peut utiliser la m�thode
  de Newton-Raphson ou m�thode de la tangente\footnote{$x_{n+1}=x_n-\dfrac{f(x_n)}{f'(x_n)}$. Cette suite converge vers une
  racine si elle existe, prendre $x_0$ assez pr�s de la racine.} (plus rapide)
  
\end{list}
}

%%%

\section{Preuve du T.V.I. par dichotomie}
cf chapitre Suites adjacentes.~\ref{sect:preuve.TVI}


\section{Applications}
\subsection{Racine d'un polyn�me de degr� impair}
\subsection{Fonctions r�ciproques} 
%%%%%%%%%%%%% Mis avec les suites
\begin{comment}
\vspace{0.2cm}
\hrule
\vspace{0.2cm}
\subsection{\'Etude des suites du type $u_{n+1}=f(u_n)$}
On se donne une fonction $f$ d�finie, continue sur un intervalle $I$ de
$\R$. Soit $u_0\in I$.
\subsubsection{Intervalle stable}
\Def{On dit que $I$ est stable par $f$ si $f(I)\subset I$}
\exemples{
  \begin{enumerate}
  \item $f(x)=x(1-x)$ sur $I=\intf{0}{1}$. $I$ est stable par f, mais
pas par $g=5f$. Il suffit d'�tudier $f$.
\item $f(x)=\frac{1}{x-1}$ et $u_0=\frac32$ V�rifier que la relation
  $u_{n+1}=f(u_n)$ ne d�finit pas une suite. 
  \end{enumerate}
}
La condition $I$ stable par $f$ permet de garantir que la suite est
d�finie et que tous les termes de la suite sont dans l'intervalle $I$.
(r�currence imm�diate)

\subsubsection{Sens de variation}
\begin{prop}
  \begin{enumerate} 
  \item $ \forall x \in I,\ f(x)-x\geq 0 \implies$ $(u_n)$ croissante
  \item $\iff \forall x \in I,\ f(x)-x\leq 0 \implies$ $(u_n)$ d�croissante 
  \end{enumerate}
\end{prop}

\proof{Facile}
\begin{prop}
\begin{enumerate} 
  \item $f$ croissante $\implies$ $(u_n)$ monotone. Le sens de var est
    donn� par le signe de $u_1-u_0$
  \item $f$ d�croissante $\implies$ $(u_{2n})$ et $(u_{2n+1})$ sont
    monotones de monotonies contraires
\end{enumerate}
\end{prop}
\proof{ Supposons $u_1>u_0$.
\begin{enumerate} 
  \item Par r�currence. $\Para (n)$: \og $u_{n+1}-u_n\geq 0$\fg{}
  \item On pose $p_n=u_{2n}$ et $i_n=u_{2n+1}$. Alors:
\[p_{n+1}=fof(p_n) \quad \et{}\quad i_{n+1}=fof(i_n)\]
Or $f$ dec. implique $fof$ croissante donc par le 1. $p$ et $i$ sont
monotones.
Supposons $p$ croissante, alors $u_2\geq u_0$ donc, en appliquant $f$
qui est dec. on a $u_3\leq u_1$ donc $i_1 \leq i_0$ donc $i$ est dec.
\end{enumerate}
}

\subsubsection{Convergence}
\Def{On appelle point fixe d'une fonction $f$ un r�el tel que $f(x)=x$}
\begin{theo}Si $f$ est continue sur $I$ et que $(u_n)$ converge, alors la
  limite est n�cessairement un point fixe de $f$
\end{theo}
Attention: L'existence d'un point fixe ne garantit pas la convergence
de $(u_n)$. Mais l'absence de point fixe suffit � justifier que $(u_n)$
ne converge pas.
En pratique: On justifie que $(u_n)$ CV (croissante major�e par
exemple) puis on d�termine la limite en cherchant les points fixes de
$f$. Si le point fixe est unique, c'est facile, sinon il faut
raisonner avec le sens de variation de $(u_n)$.

\exemple $u_{n+1}=\frac{u_n}{2-u_n}$ et $u_0 \in \intf{0}{1}$

\end{comment}
\section{D�rivabilit�}
\subsection{D�finition}
\begin{Def}
Soit $a\in I$ o� $I$ est un intervalle \emph{ouvert}. Soit $f$
  une fonction d�finie sur $I$. On dit que $f$ est d�rivable en $a$
  si:
\[\frac{f(a+h)-f(a)}{h}\quad \text{admet une limite finie quand $h$ tend
  vers 0}\]
Si cette limite finie existe, on l'appelle le nombre d�riv� de $f$ en
$a$, not�~$f'(a)$
\end{Def}
{\sl Remarque:} Cette def signifie que
$\frac{f(a+h)-f(a)}{h}=f'(a)+\varepsilon(h)$ o� $\varepsilon(h)$ tend vers 0 en~0
\begin{prop}{\bf Approximation affine} On en d�duit que si $f$ est d�rivable
en $a$, et que $h$ est proche de z�ro: $f(a+h)\approx f(a)+hf'(a)$
\end{prop}
\exemples Pour $x$ proche de 0: $\e^x\approx x+1$ 
$\sin x \approx x$, $(1+x)^n\approx 1+nx$\\

Cette propri�t� doit son nom au fait que faire cette approximation
revient � confondre au point d'abscisse $(a+h)$ la courbe avec sa
tangente en $a$: 
\begin{proof} Eq de la tangente. Ordonn�e pour $x=a+h$.
\end{proof}
\begin{Def} Soit $I$  un intervalle \emph{ouvert}. Soit $f$
  une fonction d�finie sur $I$. On dit que $f$ est d�rivable sur $I$
  si elle est d�rivable en tout en $a$ de $I$.
\end{Def}
Pour prouver qu'une fonction est d�rivable en un point, il faut donc
calculer une limite. En particulier lorsque ce point est aux bornes
d'un intervalle ferm�.
Pour prouver que $f$ est d�rivable sur un intervalle ouvert, en
g�n�ral on utilise que:

\begin{prop} La compos�e de fonctions d�rivables est d�rivable sur tout
  intervalle ouvert o� la fonction est d�finie.
\end{prop}
\begin{proof} Admis \end{proof}
\subsection{Propri�t�s}
\begin{prop} Une fonction d�rivable en $a$ est continue en $a$
\end{prop}
\begin{proof}
{\bf \`A conna�tre!}$f$ d�rivable en $a$ signifie que:
\[\frac{f(a+h)-f(a)}{h}=f'(a)+\varepsilon(h)\]
o� $\varepsilon(h)$ tend vers 0 en 0. D'o� pour $h$ non nul:
\[ f(a+h)=hf'(a)+h\varepsilon(h)+f(a)\]
D'o� la limite voule, et $f$ est continue en $a$.%
\end{proof}


\subsection{Exemples, contre exemples}
{\bf Attention!}
La r�ciproque de la propri�t� pr�c�dente est fausse:
Valeur absolue et racine.
On a m�me construit des fonctions continues sur [0;1] nulle part
d�rivable.

Soit $f(x)=x\sin(\frac{1}{x})$ pour $x\in\into{0}{+\infty}$ et
$f(0)=0$
Prouver que $f$ est continue sur $\R^+$. Est-elle d�rivable en 0?
M�me question avec $g(x)=xf(x)$. 
Retour aux exos de la feuille avec $\e^{-\frac{1}{x}}$

\chapter{Exponentielle}

\section{M�thode d'\textsc{Euler}}
\subsection{Aspect graphique}
On utilise la m�thode d'Euler pour approximer la courbe d'une fonction $f$ (si elle existe) telle que $f$ est d�finie et d�rivable sur $\R$ et:

\[ \left\{ 
\begin{array}{ll}
f'(x)=f(x) \quad \forall x\in\R \\
f(0)=1\\ 
\end{array} \right. 
\]
Cette m�thode utilise l'approximation affine suivante: $f(a+h)\approx f(a)+hf'(a)$ valable pour une fonction d�rivable sur un intervalle ouvert $I$, $a$ dans $I$ et $h$ petit (et $a+h$ dans $I$). Graphiquement cela revient � confondre la courbe de la fonction avec sa tangente sur l'intervalle $\intf{a}{a+h}$. J'ai fait les graphiques sur l'intervalle $\intf{0}{1}$. Soit $n\in\N^*$. On pose $h=\frac{1}{n}$ pour le pas de la m�thode. On fera ensuite tendre le pas vers z�ro en faisant tendre $n$ vers l'infini.
Comme $f(0)=1$ on place le premier point $A_0$ de coordonn�es $A_0(0;1)$.
Je fais l'explication pour $n=2$ donc $h=\frac12$ (cf. premier graphique). Comme $f'=f$ on a donc $f'(0)=f(0)=1$\\
\rule{0cm}{3.4cm}
\begin{minipage}{0.65\textwidth}
\begin{enumerate}
  \item On trace donc le segment partant de $A_0$ joignant $A_1$ tel que l'abscisse de $A_1$ soit $0+h=\frac12$ et tel que $(A_0A_1)$ ait pour coefficient directeur 1. Les coordonn�es de $A_1$ sont donc $A_1(0,5;1,5)$. Par le calcul: 
\[f(0+h)\approx f(0)+hf'(0)=(1+h)f(0)=1,5\]
  \item On a donc $f(0,5)\approx 1,5$. Or $f'=f$ donc 1,5 est aussi une valeur approch�e de $f'(0,5)$. On l'utilise pour obtenir le point suivant $A_2$ d'abscisse $0,5+0,5=1$. Et $(A_1A_2)$ a pour coefficient directeur $1,5$. On trouve donc l'ordonn�e de $A_2$: $1,5+0,5\times 1,5=2,25$
  \item Ainsi 2,25 est une valeur approch�e de $f(1)$. On pourrait continuer ainsi ind�finiment. Le prochain segment a un coefficient directeur de $2,25$.
  \end{enumerate}
\end{minipage}\hfill
\begin{minipage}{0.32\textwidth}
\psfrag{O}{$O$} \psfrag{i}{$\vi$}   \psfrag{j}{$\vj$}
\includegraphics{/home/pan/Desktop/TeX_MiX/Figures/meth_euler.2}
\end{minipage}\\
\rule{0cm}{0.5cm}
  Les deux graphiques suivants s'obtiennent de la m�me mani�re mais avec des valeurs de $n$ plus grandes ($n=4$ et $n=8$ et donc des pas de $\frac14$ et $\frac18$). Le dernier est obtenu avec un pas de $\frac15$ et un pas de $-\frac15$ afin d'obtenir la courbe pour des valeurs n�gatives de $x$. De plus celui-ci est dans un rep�re orthonormal. Tous les graphiques de cette page comportent aussi en pointill� la \og courbe limite \fg{} recherch�e afin que l'on voit mieux la convergence de la m�thode.\\
\rule{0mm}{0.8cm}

\noindent
\begin{minipage}{0.33\textwidth}
\centering
\psfrag{O}{$O$} \psfrag{i}{$\vi$}   \psfrag{j}{$\vj$}
\includegraphics{/home/pan/Desktop/TeX_MiX/Figures/meth_euler.3}
\end{minipage}
\begin{minipage}{0.33\textwidth}
\centering
\psfrag{O}{$O$} \psfrag{i}{$\vi$}   \psfrag{j}{$\vj$}
\includegraphics{/home/pan/Desktop/TeX_MiX/Figures/meth_euler.4}
\end{minipage}
\begin{minipage}{0.33\textwidth}
\centering
\psfrag{O}{$O$} \psfrag{i}{$\vi$}   \psfrag{j}{$\vj$}
\includegraphics{/home/pan/Desktop/TeX_MiX/Figures/meth_euler.6}
\end{minipage}



%\includegraphics{/home/pan/Desktop/TeX_MiX/Figures/meth_euler.6}


\subsection{Aspect calculatoire}
On va d'abord approximer $f$ sur l'intervalle $\intf{0}{1}$.
\begin{list}{$\bullet$}{}
\item Soit $n\in\N^*$. On pose $h=\frac{1}{n}$ pour le pas de la m�thode. On fera ensuite tendre le pas vers z�ro en faisant tendre $n$ vers l'infini.
\item On d�finit une suite d'abscisses. On part de $x_0=0$ et $x_{k+1}=x_k+h$ pour tout entier naturel $k$. Cette suite est arithm�tique de raison $h=\frac{1}{n}$ donc on a le terme g�n�ral: 
\[ \forall k \in \N \quad x_k=\frac{k}{n}\]
\item On d�finit maintenant la suite des ordonn�es des points que l'on place. On part de $y_0=1$ puisque $f(0)=1$. On applique alors l'approximation affine,  $f(x_{k+1})=f(x_k+h)$ d'o�: $f(x_{k+1})=f(x_k+h)\approx f(x_k)+hf'(x_k)$.  Or $f$ v�rifie aussi $f=f'$ sur $\R$ donc: $f(x_{k+1})\approx f(x_k)(1+h)$. On veut que $y_{k}$ soit proche de $f(x_{k})$ et donc que $y_{k+1}$ soit proche de $f(x_{k+1})$. On d�finit donc la suite $(y_k)$ par la relation de r�currence: \[ y_{k+1}=y_k(1+h)\]
Cette suite est g�om�trique de raison $(1+h)$ et comme $y_0=0$ et $h=\frac{1}{n}$ on a donc: 
\[ \forall k \in \N \quad y_k=(1+\frac{1}{n})^k\]
\end{list}

%\noindent
%\includegraphics{/home/pan/Desktop/TeX_MiX/Figures/meth_euler.2}
%\includegraphics{/home/pan/Desktop/TeX_MiX/Figures/meth_euler.3}
%\includegraphics{/home/pan/Desktop/TeX_MiX/Figures/meth_euler.4}\\
%\includegraphics{/home/pan/Desktop/TeX_MiX/Figures/meth_euler.5}
%\includegraphics{/home/pan/Desktop/TeX_MiX/Figures/meth_euler.6}

%\section{Approche Graphique}
%cf m�thode d'Euler.
%\section{La fonction exponentielle}
\section{Equation diff�rentielle y'=y}
\begin{theo}
Il existe une unique fonction $f$ d�finie et d�rivable sur $\R$ telle que: $f'=f$ et $f(0)=1$.
\end{theo}
Existence admise, on va prouver l'unicit�.
\begin{lemme} \label{lemme}
Si il existe une  fonction $f$ d�finie et d�rivable sur $\R$ telle que $f'=$ et $f(0)=1$ alors elle v�rifie:
$\forall x\in\R \quad f(-x)f(x)=1$ et par cons�quent $f$ ne s'annule pas.
\end{lemme}
\begin{proof} 
On pose la fonction $\phi$ d�finie par: $\forall x\in\R \quad \phi(x)=f(-x)f(x)$. On d�rive, $\phi'=0$ et $\phi(0)=1$
\end{proof}
On prouve maintenant l'unicit�:
\begin{proof} 
Supposons qu'il existe deux fonctions $f$ et $g$ qui d�finies et
d�rivables sur $\R$ telle que: $f'=f$, $g'=g$ et
$f(0)=g(0)=1$. D'apr�s le lemme, ces fonctions ne s'annulent pas, et
donc on peut consid�rer la fonction $\psi$ d�finie et d�rivable sur
$\R$ par: $\psi=\frac{f}{g}$. On d�rive, $\psi'=0$ et
$\psi(0)=1$. Donc $\psi$ est constante, �gale �~1, ainsi $f=g$.
\end{proof}
Cette unique fonction s'appelle la fonction exponentielle et on la note $\exp: x\mapsto \exp(x)$
Pour calculer des valeurs approch�es de $\exp(x)$ on peut utiliser la
m�thode d'Euler. On d�duit du lemme\ref{lemme} et de la continuit� de
$\exp$ sur~$\R$ la propri�t� qui sera tr�s utile plus tard pour des
�tudes de signe:% La fonction $\exp$ est {\bf strictement positive sur $\R$}

\begin{prop} \label{positive}
La fonction $\exp$ est {\bf strictement positive} sur $\R$
\[\boxed{ \forall x \in \R,\; \e^x>0}\]
\end{prop}

\section{M�thode d'Euler-valeur approch�e de $\exp(x)$}
\begin{list}{$\bullet$}{}
\item Soit $n\in\N^*$. On pose $h=\frac{1}{n}$ pour le pas de la m�thode. On fera ensuite tendre le pas vers z�ro en faisant tendre $n$ vers l'infini.
\item On d�finit une suite d'abscisses. On part de $x_0=0$ et $x_{k+1}=x_k+h$ pour tout entier naturel $k$. Cette suite est arithm�tique de raison $h=\frac{1}{n}$ donc on a le terme g�n�ral: 
\[ \forall k \in \N \quad x_k=\frac{k}{n}\]
\item On d�finit maintenant la suite des ordonn�es des points que l'on place. On part de $y_0=1$ puisque $f(0)=1$. On applique alors l'approximation affine,  $f(x_{k+1})=f(x_k+h)$ d'o�: $f(x_{k+1})=f(x_k+h)\approx f(x_k)+hf'(x_k)$.  Or $f$ v�rifie aussi $f=f'$ sur $\R$ donc: $f(x_{k+1})\approx f(x_k)(1+h)$. On veut que $y_{k}$ soit proche de $f(x_{k})$ et donc que $y_{k+1}$ soit proche de $f(x_{k+1})$. On d�finit donc la suite $(y_k)$ par la relation de r�currence: \[ y_{k+1}=y_k(1+h)\]
Cette suite est g�om�trique de raison $(1+h)$ et comme $y_0=0$ et $h=\frac{1}{n}$ on a donc: 
\[ \forall k \in \N \quad y_k=(1+\frac{1}{n})^k\]
\end{list}
Application: Si on veut une valeur approch�e de $\exp(1)$ il faut remarquer que $x_k=\frac{k}{n}=1 \iff k=n$. On a donc:
$\exp(1)=\lim_{n\to\infty} (1+\frac{1}{n})^n$ (On admet que cette limite existe \ie{} que la m�thode d'Euler converge) Pour des grandes valeurs de $n$ on obtient les valeurs approch�es suivantes: 
\[(1+\frac{1}{2})^2=2,25 \quad (1+\frac{1}{10})^{10}\simeq 2,59 \quad
(1+10^{-3})^{1000} \simeq 2,717 \quad (1+10^{-6})^{10^6} \simeq 2,71828\]
Cette valeur limite est not�e $\e$
\begin{prop} 
La valeur de $\exp(1)$ est not�e $\e{}$, c'est un r�el transcendant (comme $\pi$).
\[ \exp(1)=\e{}=\lim_{n\to\infty} (1+\frac{1}{n})^n\]
\end{prop}

\section{Propri�t�s alg�briques}

\begin{prop} 
Pour tous r�els $a$ et $b$ et tout entier relatif $n$, on a les formules:
\begin{eqnarray}
\boxed{\exp(-a)=\frac{1}{\exp(a)}}        \label{inv}\\
\boxed{\exp(a+b)=\exp(a)\times \exp(b)}   \label{som}\\
\boxed{\exp(a-b)=\frac{\exp(a)}{\exp(b)}} \label{moins}\\
\boxed{\exp(a\times n)=(\exp(a))^n}       \label{puiss}
\end{eqnarray}
% \boxed{\exp(-a)=\frac{1}{\exp(a)}} \quad \boxed{\exp(a+b)=\exp(a)\times \exp(b)} \quad \boxed{\exp(a-b)=\frac{\exp(a)}{\exp(b)}} \quad \boxed{(\exp(a))^n=\exp(a\times n)} 
\end{prop}
La relation~\eqref{som} s'appelle la relation fonctionnelle de l'exponentielle.
\begin{proof} 
\eqref{inv} d�coule du Lemme~\ref{lemme}. \eqref{moins} d�coule de \eqref{som} et \eqref{inv}. \eqref{puiss} se prouve � partir de \eqref{som} par r�currence sur $n$. L'essentiel est de prouver \eqref{som}:\\
Soit $b\in\R$. On consid�re la fonction $\chi$ de $x$ o� $b$ est un param�tre d�finie par: $\chi(x)=\dfrac{\exp(x+b)}{\exp(x)}$ On d�rive, $\chi'=0$ et $\chi(0)=\exp(b)$. Donc pour tout r�el $x$, $\exp(x+b)=\exp(x)\exp(b)$. Ceci pour tout r�el $b$. D'o�~\eqref{som}\\
Preuve de \eqref{puiss}: Soit $a$ un r�el. Soit $\Para(n)$: \og{}
$\exp(a\times n)=(\exp(a))^n $ \fg{} Prouvons par r�currence sur $n$ que
$\Para(n)$ est vraie pour tout $n\in\N$. Ini: Comme $\exp(a)\neq 0$,  $\exp(a)^0=1=\exp(0)$ ok...
\end{proof}
\section{Notation exponentielle}
La formule \eqref{puiss} donne pour $a=1$:  $\exp(n)=(\exp(1))^n=\e^n$ Par analogie, pour tout r�el $x$ on note $\exp(x)=\e^x$. On a donc les r�gles de calcul usuelles avec les puissances pour tous r�els $a$ et $b$ et tout $n\in\Z$:

\[ \boxed{\e^a\times \e^b=\e^{a+b}} \qquad \boxed{\frac{\e^a}{\e^b}=\e^{a-b}}
\qquad \boxed{(\e^a)^n=\e^{a\times n}} \]
\textbf{Cas particuliers importants:} $\boxed{\e^1=\e} \qquad \boxed{\e^0=1} \qquad \boxed{\e^{-x}=\dfrac{1}{\e^x}} $\\

cf touche de la calculette. Attention selon mod�le on tape ou non le chapeau. Attention aux parenth�ses.
Exercices � base de simplification d'expressions.
\exercice \'Ecrire plus simplement les expressions suivantes en utilisant les r�gles sur l'exponentielle:
  \[ A=\e^2\times \e^3\qquad B=\frac{(\e^2)^4}{\e^3}\qquad C=\Big(\frac{1}{\e^x}\Big)^2\qquad 
D=\frac{(\e^{x})^2\times\e^x}{\e^{-x}} \qquad 
E=\frac{\e^{3x-1}}{\e^{2-x}}\]

\exercice Prouver les �galit�s suivantes:
\[\frac{\e^{2x}-1}{\e^{2x}+1}=\frac{\e^{x}-\e^{-x}}{\e^{x}+\e^{-x}}
\]






%%%%%%%%%%%%%%%%

%%%%%%%%%%%%%%%%
%%%%%%%%%%%%%%%%
%%   COURS DE ES ou GC en partant de \ln
%%%%%%%%%%%%%%%%
\begin{comment}
On a d�j� r�solu des �quations de ce type o� $y$ est l'inconnue et $n$ un entier. On applique la propri�t�: \textit{``Si a et b sont deux r�els strictement positifs, alors: $\ln a =\ln b \iff a=b$''}
\exemples 
\begin{enumerate}
\item $\ln y =1\iff \ln y=\ln\e \iff y=\e$
\item $\ln y =2\iff \ln y=2\ln\e \iff \ln y=\ln\e^2 \iff y=\e^2$
\item $\ln y =-3\iff \ln y=-3\ln\e \iff \ln y=\ln\e^{-3} \iff y=\e^{-3}$
\end{enumerate}
Plus g�n�ralement, on montre de la m�me mani�re que si $n$ est un entier: $\ln y=n \iff y=\e^n$ 
\subsection{\'Equation $\ln y=x$}
On reprend la m�me �quation mais o� $x$ est un r�el. Elle admet une unique solution car la fonction $\ln$ est d�rivable et strictement croissante sur $\into{0}{+\infty}$ et admet les limites:
\[ \lim_{x\to 0^+}{\ln x}=-\infty  \quad \et{} \quad \lim_{x\to +\infty}{\ln x}=+\infty\]
D'apr�s le th�or�me des valeurs interm�diaires, il existe un unique r�el $y$ tel que $\ln y=x$. Par analogie avec le cas o� $x$ est entier, on note ce r�el: $y=\e^x$
\begin{Def}
La fonction exponentielle--not�e \emph{exp}--est la fonction r�ciproque de la fonction logarithme n�p�rien. � tout nombre r�el $x$ elle associe l'unique nombre r�el strictement positif $y$ tel que $\ln y=x$
\end{Def}
\begin{rem}
On utilise l'une ou l'autre des notations $\exp(x)$ ou $\e^x$ pour l'exponentielle de $x$.
\end{rem}
\exemple $\exp(\sqrt2)=\e^{\sqrt2}$ est le r�el (strictement positif) tel que: $\ln(\e^{\sqrt2})=\sqrt2$
\section{Propri�t�s}
\subsection{Cons�quences de la d�finition}
La fonction logarithme est d�finie sur $\into{0}{+\infty}$ et prend ses valeurs dans $\R$ donc:
\begin{prop}\label{positif}
La fonction exponentielle est d�finie sur $\R$ et prend ses valeurs dans $\into{0}{+\infty}$
\[\boxed{ \forall x \in \R,\; \e^x>0}\]
\end{prop}
\begin{prop}
Pour tout r�el $x$ et tout r�el strictement positif $y$: \dotfill $\boxed{\ln y=x\iff y=\e^x }$
\end{prop}
\begin{prop}
Pour tout r�el $x$: \dotfill $\boxed{ \ln(\e^x)=x }$
\end{prop}

\begin{prop}
Pour tout r�el strictement positif $x$ : \dotfill $\boxed{\e^{\ln x}=x}$
\end{prop}
Les deux derni�res traduisent que $\ln$ et $\exp$ sont r�ciproques l'une de l'autre de la m�me mani�re que la fonction carr� et la fonction racine carr�e sont r�ciproques l'une de l'autre:\\
Pour tout r�el positif $x$: $ (\sqrt x)^2=x $ et $\sqrt{ x^2}=x$\\
\subsection{Propri�t�s des puissances}
\begin{prop}
Pour tous r�els $a$ et $b$ on a les formules usuelles avec les puissances: 
\end{prop}
\[ \boxed{\e^a\times \e^b=\e^{a+b}} \qquad \boxed{\frac{\e^a}{\e^b}=\e^{a-b}}
\qquad \boxed{(\e^a)^b=\e^{a\times b}} \]
\textbf{Cas particuliers importants:} $\boxed{\e^1=\e} \qquad
\boxed{\e^0=1} \qquad \boxed{\e^{-x}=\dfrac{1}{\e^x}} $
\end{comment}
\section{\'Etude de la fonction exponentielle}
\subsection{D�riv�e et variation}
\begin{prop}
La fonction exponentielle est sa propre d�riv�e: $\boxed{(\e^x)'=\e^x}$
\end{prop}
D'apr�s la propri�t� \ref{positive}, on en d�duit:
\begin{prop}
La fonction exponentielle est strictement croissante sur $\R$
\end{prop}
\begin{prop}[D�riv�e d'une fonction compos�e avec $\e^x$] 
Si $u$ d�signe une fonction d�rivable sur $\R$ alors $\exp(u)$ est d�rivable et:
\[ \boxed{[\e^{u(x)}]'=u'(x)\times\e^{u(x)}}\]
\end{prop}
\subsection{\'Equations et in�quations}
\begin{prop}
  $\forall a, b \in\R,\quad \boxed{\e^a\leq \e^b\iff a\leq b}$
\end{prop}
\begin{proof}On prouve chaque sens s�par�ment. Soit $a$ et $b$ deux r�els.
  
 \Marge{$(\Longleftarrow )$} %% Utiliser \Impli puis \Recip
$a\leq b\implies\e^a\leq \e^b $ car $\exp$
 est croissante sur~$\R$.
\Marge{$(\Longrightarrow )$} Si $\e^a\leq \e^b $. Supposons que $a$
soit strictement sup�rieur � $b$, alors comme $\exp$ est strictement
croissante sur~$\R$ on aurait $\e^a$ strictement sup�rieur �~$e^b$.
\end{proof}
\begin{prop}
  $\forall a, b \in\R,\quad \boxed{\e^a= \e^b\iff a= b}$
\end{prop}
\begin{proof} On prouve chaque sens s�par�ment. Soit $a$ et $b$ deux r�els.
  \Marge{$(\Longleftarrow )$} $a= b\implies\e^a= \e^b $ car $\exp$
 est une application.
\Marge{$(\Longrightarrow )$} Si on a $\e^a= \e^b $ on proc�de par double in�quation:
\begin{list}{$\bullet$}{}
\item $\e^a= \e^b\implies \e^a\leq \e^b\implies a\leq b$
\item $\e^a= \e^b\implies \e^a\leq \e^b\implies  a\geq b$
\end{list}
On a donc $a=b$.
\end{proof}
\Danger Attention, l'�quation: $\e^x=a$ n'a pas de solution si $a\leq
0$ mais grace au TVI, on peut prouver que:
\exercice R�soudre dans $\R$:\\\noindent
\aligneabcd{$\e^{5x-1}=1$}{$\e^{2x-1}=\e^{3x-2}$}{$\e^{5x-1}=-2$}{$\e\e^{x^2}=\left(\e^{2x+1}\right)^2$}
\exercice p31 \no 17, 18, 24, 25, 26, 27.
\exercice R�soudre dans $\R$: $\e^x-\frac{1}{\left(\e^x\right)^2}\leq 0$
\exercice On d�finit deux fonctions, cosinus hyperbolique et sinus hyperbolique par:
\[\text{ch}(x)=\frac{\e^{x}+\e^{-x}}{2} \quad \et{} \quad \text{sh}(x)=\frac{\e^{x}-\e^{-x}}{2} \]
\begin{enumerate}
\item Prouver que: $\forall x\in\R$, $\text{ch}^2(x)+\text{sh}^2(x)=-1$.
\item D�montrer que pour tout r�el $x$: $\text{ch}(2x)=2\text{ch}^2(x)-1$ \et{} $\text{sh}(2x)=2\text{ch}(x)\text{sh}(x)$
\item Prouver que: $\text{ch}(x+y)=\text{ch}(x)\text{ch}(y)+\text{sh}(x)\text{sh}(y)$ et $\text{sh}(x+y)=\text{sh}(x)\text{ch}(y)+\text{ch}(x)\text{sh}(y)$.
\item On d�finit tangente hyperbolique par th=sh/ch. Prouver que $\text{th}(x)=\frac{\e^{2x}-1}{\e^{2x}+1}$ et $\text{th}'=1-\text{th}^2$

\end{enumerate}
\begin{prop}[Logarithme n�perien]
  Pour
tout r�el strictement positif $a$, l'�quation en $x$:
\begin{equation}
  \label{eq:exy}
  \e^x=a
\end{equation}
a une unique solution que l'on note: $\ln(a)$ appel�e logarithme
n�perien de~$a$.
\end{prop}
On trouve cette fonction sur la calculette sur la m�me touche que
$\exp$. Ne pas confondre avec log.
\exemple $\e^x=2\implies x=\ln(2)$.
\exercice R�soudre dans $\R$: $\e^{2x}+3\e^x-4=0$ \\
\rotatebox{180}{\textbf{Solution:}$\Delta=25,\ X_1=1,\ X_2=-4$  }
\exercice R�soudre les �quations suivantes qui se ram�nent � des �quations du second degr�. \\\emph{On pourra poser $X=\e^x$}
  \begin{align}
  2\e^{2x}-3\e^x+1&=0\\
  \e^{2x}-2\e^x&=0\\
  \e^{x}+12\e^{-x}+7&=0
  \end{align}
\subsection{Courbe, approximation affine}
%Comme la fonction exponentielle est la r�ciproque de la fonction $\ln$, leurs courbes sont sym�triques l'une de l'autre par rapport � la droite $\Dr$ d'�quation $y=x$\\
%\hfill{\stretch{1}}
\begin{center}
% \psfrag{i}{$\vec{\imath}$} \psfrag{j}{$\vec{\jmath}$} \psfrag{O}{$O$}
% \psfrag{F}{$\Cr_{\exp}$}
% \psfrag{L}{$\Cr_{\ln}$}
% \psfrag{D}{$\Dr:\ y=x$}
% \psfrag{e}{$\e$}
\includegraphics{/home/pan/Desktop/TeX_MiX/Figures/Metapost_fig/NewCourbe.5} 
\end{center}
Sur la courbe trac�e ci-dessus, on a aussi trac� la droite d'�quation
$y=x+1$ qui est la tangente en z�ro la courbe de $\exp$.
\begin{prop}
  Pour $x$ proche de z�ro, $\e^x$ est proche de $x+1$. Plus
  pr�cis�ment, il existe une fonction $\epsilon$ qui tend vers z�ro en
  z�ro ($\lim\limits_{ x \to 0  } \epsilon(x)=0$) telle que pour tout
  r�el~$x$:
\[\e^x=x+1+x\epsilon(x)\]
\end{prop}
\begin{proof}
  On applique la formule de l'approximation affine vue en premi�re pour une fonction $f$ d�rivable
  au voisinage de $a$: $f(x+a)=f(a)+xf'(a)+x\epsilon(x)$. Comme
  $\exp(0)=1$ on a le r�sultat souhait� pour $a=0$.
\end{proof}
\begin{prop}
  $\Cr_{\exp}$ est au dessus de sa tangente en z�ro. \ie
\[ \forall x\in\R,\quad \e^x\geq x+1\]
\end{prop}

\begin{proof}
  On pose $\phi$ la fonction d�finie par: $\phi(x)=\e^x-x-1$ et on
  veut proiuver que cette fonction est positive sur $\R$ pour cel�, on
  l'�tudie. $\phi$ est d�rivable sur $\R$ et on a: $\phi '(x)=\e^x
  -1$. Or:
\[ \e^x-1\geq 0\iff \e^x\geq \e^0\iff x\geq 0\]
D'o� les variations de $\phi$ qui admet ainsi un minimum en z�ro qui
vaut: $\phi(0)=0$. Donc $\phi$ est positive sur~$\R$.
\end{proof}
Ceci permet d'�tablir la limite de la fonction exponentielle en~$+\infty$
\subsection{Limites --- Croissance compar�e}
On a d�montr� en T.D. par comparaison les limites suivantes (o� $n\in \N \,)$ %qui d�coulent de celles de $\ln x$:
\[ \boxed{\lim_{x\to -\infty}{\e^x}=0} \qquad
\boxed{\lim_{x\to +\infty}{\e^x}=+\infty} \qquad
\boxed{\lim_{x\to -\infty}{x^n\e^x}=0} \qquad
\boxed{\lim_{x\to +\infty}{\frac{\e^x}{x^n}}=+\infty}  \]
La premi�re limite indique que l'axe de abscisses est asymptote �
$\Cr_{\exp}$ en $-\infty$. Les deux derni�res sont \emph{a priori} des
formes ind�termin�es que l'on retient par \emph{``L'exponentielle
  l'emporte sur les puissances de~$x$.''}
\chapter{Nombres Complexes, g�n�ralit�s}
%\activite{Nombres complexes}{\term}
\section*{Introduction}
\subsection{R\'esolution historique}
Au \bsc{XVI\ieme} si\`ecle les alg\'ebristes italiens apprennent
\`a r\'esoudre les \'equations du troisi\`eme degr\'e, en les ramenant
\`a des \'equations du deuxi\`eme degr\'e dont la r\'esolution est
connue depuis le \bsc{IX\ieme} si\`ecle gr\^ace auxs math\'ematiciens arabes.
On attribue \`a \bsc{Cardan} (Girolamo Cardano: Pavie 1501-Rome 1576) la
formule \eqref{eq:form} donnant une solution \`a l'\'equation du troisi\`eme
degr\'e d'inconnue~$x$:
\begin{equation}
  \label{eq:deg3}
  x^3=px+q
\end{equation}
En fait, \bsc{Tartaglia} (Niccol\`o Tartaglia:
Brescia 1500 - Venise 1557)  autodidacte aurait d\'ecouvert la formule
en 1539 et l'aurait expos\'ee au professeur respect\'e \bsc{Cardan}
qui l'a publi\'e dans son \emph{Ars magna} en
1545 comme \'etant sa propre d\'ecouverte, le reste de l'histoire est
romanesque.
\begin{equation}
  \label{eq:form}
  x=\sqrt[3]{\frac{q}{2}+\sqrt{\left(\frac{q}{2}\right)^2-\left(\frac{p}{3}\right)^3}}+\sqrt[3]{\frac{q}{2}-\sqrt{\left(\frac{q}{2}\right)^2-\left(\frac{p}{3}\right)^3}}
\end{equation}
\begin{Def} Pour $a\in\R$, on note $\sqrt[3]{a}$ appel\'e \emph{racine
    cubique de $a$} l'unique r\'eel $x$ v\'erifiant $x^3=a$.
\end{Def}

La racine cubique, contrairement \`a la racine carr\'ee est d\'efinie
pour tout r\'eel car $x\mapsto x^3$ est strictement croissante sur
$\R$, et varie de $-\infty$ \`a $+\infty$. Par exemple,
$\sqrt[3]{-8}=-2$ car $(-2)^3=-8$.
\begin{enumerate}
\item Prouver que toute \'equation de degr\'e trois donc de la forme
  $ax^3+bx^2+cx+d=0$ avec $a$, $b$, $c$, $d$ trois r\'eels et $a\neq 0$
  peut se mettre sous la forme $x^3+b'x^2+c'x+d'=0$
\item Prouver que toute \'equation de degr\'e trois de la forme
  $x^3+bx^2+cx+d=0$ peut se mettre sous la forme de l'\'equation
  \eqref{eq:deg3} \`a l'aide du changement de variable:
  $x=X-\frac{b}{3}$.

\item Voyons sur un exemple comment ils trouv\`erent la formule
  improbable \eqref{eq:form}. On consid\`ere l'\'equation:
  \begin{equation}
    \label{eq:eq1}
    x^3=6x+20
  \end{equation}
  \begin{enumerate}
  \item On pose $x=u+v$. Que devient l'\'equation \eqref{eq:eq1}?
  \item Quelle valeur suffit-il d'imposer au produit $uv$ pour que
    \eqref{eq:eq1} s'\'ecrive $u^3+v^3=20$?
  \item On pose $U=u^3$ et $V=v^3$. Former et r\'esoudre une \'equation
    de degr\'e deux ayant pour solutions $U$ et~$V$.

  \item En d\'eduire que:
\[ \sqrt[3]{10+2\sqrt{23}} + \sqrt[3]{10-2\sqrt{23}}\]
est une solution de \eqref{eq:eq1}.

\item Optionnel: Etudier $x\mapsto x^3-6x-20$ sur $\R$ pour voir qu'il
  n'y a qu'une solution \`a l'\'equation \eqref{eq:eq1}.
  \end{enumerate}

\end{enumerate}
La m\'ethode de \bsc{Tartaglia-Cardan} conduit cependant, dans certains cas,
\`a un paradoxe que \bsc{Bombelli}(1526-1572) va essayer de surmonter. Il publie en
1572 dans son \emph{l'Algebra opera} l'exemple suivant:
\subsection{Des nombres impossibles}
On consid\`ere l'\'equation:
  \begin{equation}
    \label{eq:eq2}
    x^3=15x+4
  \end{equation}
  \begin{enumerate}
  \item Combien vaut alors la quantit\'e:
    $\left(\frac{q}{2}\right)^2-\left(\frac{p}{3}\right)^3$ qui
    appara\^it dans la formule \eqref{eq:form}?
  \item Prouver cependant que 4 est solution de \eqref{eq:eq2}\\
L'id\'ee et l'audace de \bsc{Bombelli} est de faire comme si $-121$
\'etait le carr\'e d'un nombre imaginaire qui s'\'ecrirait
$11\sqrt{-1}$. Il appella $\sqrt{-1}$ \emph{piu di meno} qui sera not\'e
\textbf{i} plus tard par \bsc{Euler}(1734-1810) en 1777. Gr\^ace \`a ce stratag\`eme,\bsc{Bombelli} retrouve la solution r\'eelle 4:

\item D\'evelopper $(2+\I{})^3$ et $(2-\I{})^3$ en tenant compte de $\I{}^2=-1$.
\item En d\'eduire que $(2+\I{})+(2-\I{})$ est solution de
  \eqref{eq:eq2}.\\
Ce nombre \og $\sqrt{-1}$ \fg{} qualifi\'e d'impossible va susciter beaucoup de m\'efiance et
de pol\'emique pendant deux si\`ecles jusqu'\`a \bsc{Argand} en 1806 qui
proposera une repr\'esentation g\'eom\'etrique de ces nombres
imaginaires (de la forme $a+\I{} b$ avec $a\in\R$, $b\in\R$) et \bsc{Gauss}(1777-1855) qui les
nommera \textbf{Nombres complexes}.
  \end{enumerate}
\section{Introduction-Premi\`eres d\'efinitions}
\label{sec:para1}

\subsection{Historique}
\label{sec:histo}
Cf. activit\'e. Bombelli en voulant r\'esoudre $x^3=15x+4$ a utilis\'e
des racines de nombres reels negatifs, comme $\sqrt{-1}$. On a peu a
peu construit un ensemble plus grand que $\R$ qui contient un nombre
imaginaire note $\I$ solution de l'equation $x^2=1$. La notation
$\sqrt{}$ reste r�serv�e aux r�els positifs. On n'�crira plus~$\sqrt{-1}$.

\subsection{Construction de $\C$}
\label{sec:cons}
\begin{Def} On appelle ens. des nombres complexes note $\C$ l'ens des $a+\I{}b$
  o\`u $a$ et $b$ sont des reels et {\rm i} une solution de $x^2=-1$
\end{Def}
Dans la suite, $a$ et $b$ d�signeront deux r�els, et $z$ un nombre
complexe �gal �~$a+\I{}b$.
\begin{Def}
 Si $b=0$ on dit que $z=a$ est r�el, si $a=0$, on dit
que $z=\I{}b$ est un imaginaire pur.
\end{Def}
Remarque: $\R \subset \C$. On a donc les inclusions d'ensembles de
nombres:
\[ \N\subset\Z\subset\Q\subset\R \subset \C\]
Les r�gles de calcul de $\R$ se prolongent �
$\C$ en tenant compte de $\I{}^2=-1$
On definit l'addition et la multiplication de deux complexes:
\begin{itemize}
\item Addition: La somme de deux complexes est un complexe. $z+z'=...$
  (L'op�ration $+$ est associative et commutative, l'�lement
  neutre est le r�el 0. Tout complexe admet un oppos�. On dit que $\C$
  muni de l'addition est un groupe.
\item Multiplcation\dots idem.%, tout complexe non nul admet un inverse ($zz'=1$)
\end{itemize}

\begin{prop} $\C$ est stable par addition et  multiplication, la multiplication
  est distributive sur l'addition, tout complexe non nul admet un
  inverse qui est un nombre complexe. On dit que $\C$ est un corps.\end{prop}
\begin{prop} Tout complexe $z$ s'�crit de mani�re unique sous la forme
  $a+\I{}b=\Re(z)+\I{} \Im(z)$ appel�e forme algebrique du complexe
  $z$\end{prop}

\begin{proof}
    \begin{list}{$\bullet$}
  \item $a+\I{}b=0 \iff (a=0 \quad\text{et}\quad b=0)$. Car sinon $\I{}\in\R$
  \item $z=z' \iff z-z'=0 \iff ...$
  \end{list}
\end{proof}
\begin{Def}
  Lorsqu'un nombre complexe $z$ est mis sous forme alg�brique $a+\I{}b$,
  alors $a$ s'appelle la partie r�elle de $z$, not�e $\Re (z)$ et $b$ sa
  partie imaginaire, not�e: $\Im (z)$.
\end{Def}
\DangerZ La partie imaginaire comme la partie r�elle est un nombre
r�el!\\
\exemple $\Re (1+2i)=1$ et $\Im (1+2\I{})=2$ et pas $2\I{}$

Consequence: On peut identifier partie r�elle et imaginaire de deux
complexes qui sont �gaux
\begin{prop}
 $z=z' \iff \left(\Re(z)=\Re(z') \quad\text{et}\quad \Im(z)=\Im(z')\right)$ 
\end{prop}
 

\subsection{Conjugu\'e d'un nombre complexe}
\label{sec:conj}
\begin{Def} On appelle conjugu\'e de $z$, not\'e $\bar{z}$ \end{Def}
\begin{prop} Conjugu\'e de: $z+z'$, $zz'$, $z^n$, $\dfrac{z}{z'}$ \end{prop}
\begin{prop} $z\in\R \iff z=\bar{z}$ et $z\in \I{}\R \iff z+\bar{z}=0$\end{prop}
\proof{$z=\bar{z} \iff \cdots \iff 2\I b=0\iff b=0 \iff z\in \R$}
En particulier, $\bar{\lambda z}=\lambda \bar{z}$ pour $\lambda \in \R$.\\

\subsection{Module d'un nombre complexe}
\begin{prop} $z\bar{z}\in\R^+$ Et on a la formule $z\bar{z}=a^2+b^2$\end{prop}
\begin{Def} On appelle module d'un nombre complexe, not\'e $|z|$ le r\'eel
  positif: $\sqrt{z\bar{z}}$
\end{Def}


Remarque, cette notation est la m\^eme que la valeur absolue d'un reel
et c'est bien justifie, les deux coincident sur $\R$.
\begin{prop} Module de $zz'$, $z/z'$\end{prop}

Remarque: $\C$ c'est cool, mais on a perdu qqchose: l'ordre, c'est le
bordel, mais il reste le module...
Formule de l'inverse: $1/z=\bar{z}/z\bar{z}$
\section{G�om�trie dans le plan complexe}
On se place dans un plan muni d'un rep�re orthonorm�
$(O~;\vec{u},\vec{v})$.
\begin{Def}
  On appelle \emph{affixe} d'un point $A$ de coordonn�es cart�siennes
  $(x;y)$ le nombre complexe: $z_A=x+\I y$.
\end{Def}
D'apr�s l'unicit� des coordonn�es cart�siennes d'un point dans un
rep�re donn� et de la forme alg�brique d'un complexe, tout point du
plan a une unique affixe et tout complexe est l'affixe d'un point du
plan. On peut donc identifier $\C$ au plan, on parle alors du
\emph{plan complexe} de la m�me mani�re que l'on parlait de la droite
r�elle.
\begin{prop}
  Dans le plan complexe, les r�els sont repr�sent�s par l'axe des
  abscisses et les imaginaires purs par l'axe des ordonn�es.
\end{prop}
\begin{prop}[Interpr�tation g�om�trique du conjugu�]
Soit un point $A$ d'affixe $z$ et $A'$ d'affixe $z'$. Alors $A$ et
$A'$ sont sym�triques par rapport � $(Ox)$ ssi  $z'=\bar{z}$.
\end{prop}
\begin{Def}
  Soit $\overrightarrow{w}$ un vecteur de coordonn�es $\left(
    \begin{array}[l]{c}
      a\\
     b
    \end{array}
    \right)$ on appelle \emph{affixe du vecteur} $\overrightarrow{w}$ le complexe $z_{\overrightarrow{w}}=a+ib$
\end{Def}
Ainsi on a grace aux r�gles de calcul sur les complexes:
\begin{prop}
  Pour deux points $A$ et $B$ d'affixe $z_A$ et $z_B$, on a: $\boxed{ z_{\overrightarrow{AB}}=z_B-z_A}$.
\end{prop}
\begin{prop}[Interpr�tation g�om�trique du module]
Soit un point $A$ d'affixe $z_A$ et $B$ d'affixe~$z_B$. 
\[\boxed{AB=\|\overrightarrow{AB}\|=|z_{\overrightarrow{AB}}|=|z_B-z_A|}\]
\end{prop}
Cette �galit� est vraie car le rep�re $(O~;\vec{u},\vec{v})$ estorthonorm�.
\exercice Prouver que le triangle $OAB$ est isoc�le rectangle o� $A$ et
$B$ sont les points d'affixes respectives:
\[ z_A=1+2i\quad\text{et}\quad z_B=-2+i\]
\exemple \textbf{(fondamental)} L'ensemble des points d'affixe de module 1 est
le cercle de centre $O$ et de rayon~1.
\begin{prop}
  Deux vecteurs $\overrightarrow{w}$ et $\overrightarrow{w'}$ sont
  colin�aires ssi il existe un r�el $k$ tels que: 
\[ z_{\overrightarrow{w}}=kz_{\overrightarrow{w'}}\]
\end{prop}
\begin{proof}
  On l'�crit avec les coordonn�es de vecteurs. Straight forward.
\end{proof}
\begin{prop}
  Trois points $A$, $B$, $C$ sont align�s ssi il existe un r�el $k$
  tel que: 
\[ z_B-z_A=k(z_C-z_A)\]
\end{prop}
\section{\'Equation du second degr\'e}
L'objectif est de r\'esoudre dans $\C$ toutes les \'equations du
second degr\'e \`a coefficients r\'eels. \ie{} de la forme:
\[az^2+bz+c=0 \quad \text{o\`u}\ a\in\R^*,\ b\in\R,\ c\in\R\]
\subsection{Forme canonique}
On commence \`a la main, sans formule, en utilsant la forme canonique.
\exemple On veut r\'esoudre: $z^2+2z+5=0$:
\[z^2+2z+5=(z+1)^2-1+5=(z+1)^2+4=(z+1)^2-(2\I{})^2\]
D'o\`u la factorisation et la r\'esolution:
\[ z^2+2z+5=0 \iff \left( z+1-2\I{}\right)\cdot\left(z+1+2\I{}\right)=0 \iff
z\in\{-1+2\I{}\,;\,-1-2\I{}\}\]
R\'esoudre de la m\^eme mani\`ere les \'equations du
second degr\'e suivantes dans $\C$:\\\noindent
\qcmabc{$z^2-4z+5=0$}{$z^2+6z+11=0$}{$2z^2-4z+3=0$}
\subsection{Formules}
Cette technique vue par factorisation gr\^ace \`a la forme canonique est
syst\'ematique et on peut donc obtenir des formules, similaires \` a
celles des racines r\'eelles.
\begin{enumerate}
\item Mettre $az^2+bz+c$ sous forme canonique. On fera appara\^itre $\Delta=b^2-4ac$
\item On suppose que $\Delta <0$. \'Ecrire $\dfrac{\Delta}{4a^2}$
  comme le carr\'e d'un nombre complexe.


\item En d\'eduire une factorisation de $az^2+bz+c$ sous la forme:
\[ az^2+bz+c= a(z-z_1)(z-\overline{z_1})\quad \text{o\`u}\ z_1\in\C \]

\item En d\'eduire les formules donnant les racines complexes de
  $az^2+bz+c$ dans le cas o\`u le discriminant est n\'egatif.
\end{enumerate}
\textbf{Remarque:} Comme dans $\R$, si $b$, le coefficient de $z$, est nul on
n'applique pas ces formules comme un \^ane. On proc\`ede ainsi: 
\exemple $z^2+3=0\iff z^2-(\I{}\sqrt3)^2=0 \iff z \in  \{\I{}\sqrt3\,;\,-\I{}\sqrt3\}$
\subsection{Application}
\exercice
R\'esoudre dans $\C$ les \'equations suivantes en appliquant les
formules pr\'ec\'edentes.
\begin{enumerate}
\item \qcmabc{$2z^2+4z+5=0$}{$-2z^2+6z-5=0$}{$-5z^2+2z+2=0$}
\item \qcmabc{$z^2+z+1=0$}{$(z^2+2)(z^2-4z+4)=0$}{$(z+1)^2=-(2z+1)^2$}
\item \qcmabc{$2z^4-9z^2+4=0$}{$\left(z^2+1\right)^2=1$}{$\dfrac{z-1}{z+1}=z+2$}
\end{enumerate}


\exercice \textbf{\'Equation \`a coefficients sym\'etriques}. Soit l'\'equation $(E)$:
  $z^4-5z^3+6z^2-5z+1=0$\\
Prouver que $(E)$ est \'equivalente au syst\`eme:
$\left\{\begin{array}{ll}
  u^2-5u+4=0\\
u=z+\dfrac{1}{z} 
\end{array}\right.$\\

R\'esoudre $u^2-5u+4=0$, puis r\'esoudre l'\'equation $(E)$ dans $\C$.




\exercice Un grand classique du rire. On veut r\'esoudre une \'equation
  de degr\'e trois dans $\C$. On ne vous donnera pas de formule, mais
  on vous pr\'esentera toujours les choses \`a peu pr\`es ainsi: On exhibe
  ou fait trouver une \og solution \'evidente\fg{} $z_0$, on factorise
  par $(z-z_0)$ le polynome de d\'epart, et on est alors ramen\'e \`a
  une \'equation de degr\'e deux.
  \begin{enumerate}
  \item Pour tout complexe $z$, on pose
    $P(z)=z^3-12z^2+48z-128$. Calculer $P(8)$. D\'eterminer trois
    r\'eels $a$, $b$, $c$ tels que pour tout complexe $z$:
\[ P(z)=(z-8)(az^2+bz+c)\]
R\'esoudre dans $\C$ l'\'equation $P(z)=0$.
  \end{enumerate}
  

\chapter{Barycentres et droites de l'espace}
\section{Barycentre: rappels}
Rappel des diff�rentes utilisations du barycentre
en sciences physiques: 
\begin{itemize}
\item Centre d'inertie. Cette notion permet de mod�liser des syst�mes
  solides complexes par des masses ponctuelles, les forces subies par le
  syst�me s'appliquant au centre de masse, ou centre d'inertie. 
\item Point d'�quilibre d'un syst�me.
%\item Moment d'une force.
\item Utilisation en chimie: mol�cules polaires (moment
  dipolaire). Importance pour la question de solubilit� d'une mol�cule
  dans un solvant.
\end{itemize}
Feuille de rappels (math�matiques) lue et comment�e.\\ 
S�ances d'exercices, annales � partir des connaissances de premi�re. 
\centrage{Rappels}
On peut d\'efinir le barycentre d'un nombre quelconque de points
pond\'er\'es, pourvu que la somme des masses soit non nulle. Je traite ici
le cas de trois points.
 \begin{Def}
Soit le syst\`eme de points pond\'er\'es $\left\{(A;a);(B;b);(C;c)\right\}$ avec:
$a+b+c\neq 0$. Il~existe un unique point $G$ appel\'e barycentre de ce syst\`eme v\'erifiant:
  \begin{equation} \label{fonda3}
  \boxed{ a.\V{GA}+b.\V{GB}+c.\V{GC}=\V{0}}
  \end{equation}
 \end{Def}
\begin{prop}[\textbf{Homog\'en\'eit\'e du barycentre}]
 Soit $k$ un r\'eel non nul. Le barycentre reste inchang\'e si on multiplie toutes les masses par $k$. 
\end{prop}
\begin{prop}[\textbf{Position du barycentre de deux points}]
Soit $ G=Bar\{(A;a);(B;b)\}$
\begin{itemize}
\item[$(i)$] $G\in (AB)$. 
\item[$(ii)$] Si $a$ et $b$ sont de m\^eme signe: $G$ appartient \`a $[AB]$ et est plus pr\`es du point qui est affect\'e de la masse la plus grande en valeur absolue. 
\item[$(iii)$] Si $a$ et $b$ sont de signes contraires: $G$ est en dehors du segment $[AB]$ du c\^ot\'e du point qui est affect\'e de la masse la plus grande en valeur absolue.
\end{itemize}
\end{prop}
 \begin{prop}
   Pour tout point $M$, si $a+b+c\neq0$ on note $G=Bar\left\{(A;a);(B;b);(C;c)\right\}$ et on~a:
\begin{equation} \label{MG}
  \boxed{ a.\V{MA}+b.\V{MB}+c.\V{MC}=(a+b+c)\V{MG}}
  \end{equation}
 \end{prop}
On le d\'emontre \`a partir de \eqref{fonda3} en incrustant le point $M$ dans les quatre vecteurs \`a l'aide de la relation de Chasles. En particulier si on est dans un plan muni d'un rep\`ere $\oij$ on en d\'eduit les coordonn\'ees du $G$ en posant $M=O$ dans \eqref{MG}:
\begin{prop}
   Les coordonn\'ees de $G=Bar\left\{(A;a);(B;b);(C;c)\right\}$ dans $\oij$ sont:
\[ \boxed{ x_G=\frac{ax_A+bx_B+cx_C}{a+b+c} \qquad y_G=\frac{ay_A+by_B+cy_C}{a+b+c}} \]
 \end{prop}
En rempla\c{c}ant $M$ par le point $A$ dans \eqref{MG} on a une expression
utile pour placer~$G$:
\[  \boxed{\V{AG}=\frac{b}{a+b+c}\V{AB} + \frac{c}{a+b+c}\V{AC} }\]

La relation ci-dessus prouve que $G$ appartient au plan $(ABC)$ et que
ses coordonn\'ees dans le rep\`ere $(A;\V{AB};\V{AC})$ sont: $G(\dfrac{b}{a+b+c};\dfrac{c}{a+b+c})$

\begin{prop}[\textbf{Th\'eor\`eme d'associativit\'e}]
Soit trois r\'eels $a,\ b$ et $c$ tels que: $a+b+c\neq0$ et $a+b\neq0$. On consid\`ere alors les deux barycentres:
  \[ G=Bar\left\{(A;a);(B;b);(C;c)\right\} \ \text{et} \ H=Bar\left\{(A;a);(B;b)\right\}. \]
Alors: \[ G=Bar\left\{(H;a+b);(C;c)\right\} \]
\end{prop}

Plus g\'en\'eralement, le barycentre $G$ de $n$ points pond\'er\'es
($n\in\N,\, n\geq 3$) est inchang\'e si on remplace $p$ points
pond\'er\'es ($p<n$) dont la somme des masses $m$ est non nulle par leur
barycentre $H$ (dit partiel) affect\'e de ladite masse~$m$.

\subsection{Droites, segments de l'espace}
Dans cette partie, $\mathscr{E}$ d�signe l'ensemble des points de
l'espace � trois dimensions. $A$ et $B$ sont deux points distincts de $\mathscr{E}$.
\subsubsection{Relation vectorielle}% $\overrightarrow{AM}=k\overrightarrow{AB}$}
On consid�re le point $M_k$ de $\mathscr{E}$ v�rifiant la relation
suivante o� $k$ d�signe un r�el:
\begin{equation}
  \label{eq:amkab}
  \overrightarrow{AM_k}=k\overrightarrow{AB}
\end{equation}
Alors on a les r�sultats suivants qui proviennent du rep�rage sur une
droite, car $k$ repr�sente l'abscisse du point $M_k$ dans le rep�re
$(A;\overrightarrow{AB})$ de la droite~$(AB)$.
\begin{dingautolist}{172}
\item Lorsque $k$ d�crit $\mathbb{R}$, $M_k$ d�crit~$(AB)$.
\item Lorsque $k$ d�crit $\mathbb{R}^+$, $M_k$ d�crit~$[AB)$.
\item Lorsque $k$ d�crit $\left[0\,;1\right]$, $M_k$ d�crit~$[AB]$.
\item Lorsque $k$ d�crit $\mathbb{R}^*$, $M_k$ d�crit~$(AB)$ priv�e de~$A$.
\item Lorsque $k$ d�crit $\mathbb{R}\setminus\{a\}$, $M_k$
  d�crit~$(AB)$ priv�e de~$M_a$. (o� $a\in\mathbb{R}$)
\end{dingautolist}

\subsubsection{Caract�risation barycentrique}
\paragraph{Droites}
\begin{prop}
  La droite $(AB)$ est l'ensemble des barycentres de $A$ et
  $B$. \emph{i.e.} \[M\in(AB)\iff\exists (a;b)\in \mathbb{R}^2;\ 
  M=\Bary\{ (A;a);(B;b)\} \quad \text{avec}\quad a+b\neq 0 \]
\end{prop}
\begin{proof}
  On prouve les deux sens s�par�ment, en utilisant la caract�risation:\\
  $M\in(AB)\iff\exists k\in \mathbb{R};\ \overrightarrow{AM}=k\overrightarrow{AB}$
  
\end{proof}
\paragraph{Segment}
\begin{prop}
  Le segment $[AB]$ est l'ensemble des barycentres de $A$ et
  $B$ avec des masses de m�me signe. \emph{i.e.} \[M\in[AB]\iff\exists (a;b)\in \mathbb{R}^2;\ 
  M=\Bary\{ (A;a);(B;b)\}\quad \text{avec}\quad a+b\neq 0 \quad\text{et}\quad ab\geq 0\]
\end{prop}
\begin{proof}
  cf Premi�re, et feuille de rappels pour la position du
  barycentres (plus pr�s de quel point?). Rappels � l'oral sur le cas o� les masses sont de  signes contraires.   
\end{proof}

\subsection{Plans et triangles}
\subsubsection{Caract�risation barycentrique}
\paragraph{Plan}
\begin{prop}
  Le plan $(ABC)$ est l'ensemble des barycentres de $A$, $B$ et
  $C$. \emph{i.e.} \[M\in(ABC)\iff\exists (a;b;c)\in \mathbb{R}^3;\ 
  M=\Bary\{ (A;a);(B;b);(C;c)\}\ \text{avec}\ a+b+c\neq 0 \]
\end{prop}
\begin{proof}
  On prouve les deux sens s�par�ment, en utilisant la caract�risation:\\
  $M\in(ABC)\iff\exists (x;y)\in \mathbb{R}^2;\ \overrightarrow{AM}=x\overrightarrow{AB}+y\overrightarrow{AC}$
  \end{proof}
\paragraph{Int�rieur d'un triangle}
\begin{prop}
  L'int�rieur du triangle $ABC$ est l'ensemble des barycentres de $A$, $B$ et
  $C$ avec des masses strictement positives. 
\end{prop}
\begin{proof}
  On prouve les deux sens s�par�ment, g�om�triquement, par le th�or�me
  d'associativit�. 
  \begin{dingautolist}{172}
  \item Soit $a$, $b$, $c$ trois r�els strictement positifs. On consid�re
    $M=\Bary\{ (A;a);(B;b);(C;c)\}$. Comme $b+c>0$ on peut consid�rer
    $H=\Bary\{(B;b);(C;c)\}$ qui appartient � $[BC]$, alors par associativit� du barycentre:
\[ M=\Bary\{ (A;a);(H;b+c)\} \]
Comme $a$ et $b+c$ sont du m�me signe, non nuls,
$M\in[AH]$ diff�rent de $A$ et $H$. Figure. \og Donc \fg{} $M$ est � l'int�rieur du triangle $ABC$.
\item R�ciproque, on refait la construction dans l'autre sens. On
  appelle $H$ l'intersection de $[AM]$ et $[BC]$. Comme $H\in[BC]$
  (diff�rent de $B$ et $C$) alors il existe $b$
  et $c$ strictement positifs tels que: $H=\Bary\{(B;b);(C;c)\}$. De
  m�me il existe $a'$ et $k$ strictement positifs tels que:
  $M=\Bary\{(A;a');(H;k)\}$. Par homog�n�it� en multipliant les masses
  par $\frac{b+c}{k}$ (strictement positif), on obtient, en notant
  $a=a' \times \dfrac{b+c}{k}$:
\[ M=\Bary\{(A;a);(H;b+c)\} \quad\text{donc}\quad M=\Bary\{
(A;a);(B;b);(C;c)\}\]

  \end{dingautolist}
  
  \end{proof}
  \begin{rem}
    La preuve peut �tre refaite avec des masses strictement
    n�gatives. Par cons�quent, on retiendra qu'un point appartient �
    l'int�rieur du triangle $ABC$ ssi il est barycentre de $A$, $B$ et
    $C$ affect�s de masses non nulles, de m�me signe.
  \end{rem}
  \begin{rem}
    Si une des masse $a$, $b$ ou $c$ est nulle, le point appartient � un c�t� du
    triangle, et si deux masses sont nulles, le point est un sommet du triangle.
  \end{rem}
\medskip Exercice d'application: cf Poly \texttt{Barycentres fp3}
\section{Repr�sentations param�triques d'une droite de l'espace}
\subsection{T.P. sur GeoGebra (cas des droites du plan) }
Compte rendu sur feuille par bin�me, puis oral en classe. Ramasser quelques
comptes rendus.\\
Questions compl�mentaires pour faire le point sur leurs connaissances
et leur compr�hension de la probl�matique soulev�e par le TP (� l'oral): 
\begin{dingautolist}{172}
\item Une repr�sentation param�trique d'une droite est elle unique?
  (Bah non! observer (1) et (2) qui d�finissent toutes deux~$\Delta$)
\item Mais alors, quel lien y a-t-il entre ces deux repr�sentations
  param�triques?
\item Quel nom donne-t-on aux vecteurs $\vec{u}\,$ et $\vec{v}\,$ dont
  on a trouv� les coordonn�es au \textbf{6.c.} et \textbf{6.d.}?
\item Que peut-on dire de ces deux vecteurs (directeurs de $\Delta$) ?

\item Cette condition de colin�arit� est elle suffisante pour �tablir
  que les deux syst�mes repr�sentent la m�me droite?
\item Comment caract�riser une droite? (Deux points, oui\dots{} mais
  encore\dots{} barycentres\dots{} vecteur directeur et un point.)
\end{dingautolist}

\subsection{Caract�risation analytique d'une droite de l'espace}
\subsubsection*{Introduction}
\textsl{En T.P. on a �tudi� le probl�me physique de la trajectoire rectiligne de deux
points mobiles dans le plan. On a vu qu'il fallait distinguer point
d'intersection des trajectoires et \og collision \fg{} des deux
points mobiles. Dans le plan, pour savoir si des trajectoires
rectilignes sont s�cantes, il suffit de savoir si les droites sont parall�les, ce qui
est simple.\\
Que donne le m�me probl�me dans l'espace? L'objectif est de savoir
d�terminer par le calcul si deux droites sont s�cantes, et le cas
�ch�ant savoir calculer les coordonn�es de ce point
d'intersection. Pour cel� il nous faut une caract�risation analytique
d'une droite de l'espace. Malheureusement, il n'y a pas d'�quation de
droite dans l'espace\dots{} On verra ensuite qu'une �quation de la
forme $ax+by+cz+d=0$ est en fait celle d'un plan. \\}

\hrule
\medskip


Une droite~$\mathscr{D}$ de~$\mathscr{E}$ est d�finie:
\begin{itemize}
\item Soit par deux points distincts $A$ et $B$ de~$\mathscr{D}$. (Alors  $\mathscr{D}=(AB)$)
\item Soit par un point $A$ et un vecteur directeur
  $\vec{u}\,$. (Alors  $(A;\vec{u}\,)$ est un rep�re de~$\mathscr{D}$)
\end{itemize}
\textsl{(Rappels oraux sur les vecteurs directeurs.)}
\begin{Def}
 Soit $\vec{u}\,$ un vecteur non nul et $A$ un point
de~$\mathscr{E}$. La droite passant par~$A$ et de vecteur
directeur~$\vec{u}\,$ est l'ensemble des points $M$ de~$\mathscr{E}$ tels que
$\overrightarrow{AM}=t\vec{u}\,$ o� $t$ d�crit~$\mathbb{R}$.
\end{Def}

On veut traduire cel� avec des coordonn�es. Dans la suite, l'espace est rapport� � un rep�re~$(O~;\vec{\imath},\vec{\jmath},\vec{k})$. 
\begin{prop}[Repr�sentation param�trique de droite] Soit $\mathscr{D}$
  une droite de l'espace. On note $A(x_A;y_A;z_A)$ un point
  de~$\mathscr{D}$ et $\vec{u}\,(a;b;c)$%
% \left(
%     \begin{array}[l]{c}
%       a\\
%       b \\
%       c
%     \end{array}
%   \right)$%
 un vecteur directeur de~$\mathscr{D}$. ($x_A,\,y_A,\,z_A,\,a,\,b,\,c$
sont des r�els avec $a,\,b,\,c$ non tous nuls).\\
 La droite~$\mathscr{D}$ est l'ensemble des points $M(x;y;z)$
 de~$\mathscr{E}$ pour lesquels il existe un r�el~$t$ tel
 que~\mbox{$\overrightarrow{AM}=t\vec{u}\,$}. $\mathscr{D}$ est donc caract�ris�e
 par le syst�me:
 \begin{equation}
   \label{eq:represparadroite}
\left\{
\begin{array}{ll}
x=x_A+ta\\
y=y_A+tb\\
z=z_A+tc
\end{array}
\right.\quad ,\text{o�:}\ t\in\R
 \end{equation}
\end{prop}
\begin{proof} Straight forward.
Deux vecteurs sont �gaux ssi ils ont les m�mes coordonn�es. Or:
\[ \overrightarrow{AM}\left(
     \begin{array}[l]{c}
       x-x_A\\
       y-y_A \\
       z-z_A
     \end{array}
   \right) \quad\text{et}\quad
 \vec{u}\, \left(
     \begin{array}[l]{c}
       a\\
       b \\
       c
    \end{array}
   \right)\quad\text{donc}\quad %
t\vec{u}\, \left(
     \begin{array}[l]{c}
       ta\\
       tb \\
       tc
    \end{array}
   \right)
\]
$M\in\mathscr{D}\iff \exists t\in\mathbb{R};\ \overrightarrow{AM}=t\vec{u}\,$. En identifiant les trois coordonn�es, on obtient le r�sultat.
\end{proof}
Un syst�me comme~\eqref{eq:represparadroite} s'appelle une
repr�sentation param�trique de~$\mathscr{D}$. Le param�tre est~$t$. On
peut mettre n'importe quelle lettre � la place de~$t$. Il peut �tre
utile de se repr�senter~$t$ comme le temps (variant dans $\mathbb{R}$)
et le point~$M$ comme un point mobile dans l'espace en fonction du
temps dont les coordonn�es v�rifient le
syst�me~\eqref{eq:represparadroite}.
 \begin{rem}
  Une repr�sentation param�trique comme~\eqref{eq:represparadroite}
  caract�rise la droite passant par $A(x_A;y_A;z_A)$ et de vecteur
  directeur~$\vec{u}\,(a;b;c)$. On peut d�terminer
  $x_A,\,y_A,\,z_A,\,a,\,b,\,c$ � partir du syst�me que constitue~\eqref{eq:represparadroite}.
  \end{rem}
\begin{exercice}
  \begin{enumerate}
  \item Donner une repr�sentation param�trique de la droite~$(IJ)$ o�
  $I(-1;\,0;\,2 )$ et $J(3;-2;\,1)$.\\
Une fois fait: Tout le monde a-t-il la m�me repr�sentation
param�trique? En donner une autre.
\item Le point $A(3;0,1)$ appartient-il � $(IJ)$? Et $B(3;-2;1)$? Et $C(4;4;4)$?
\item Donner une repr�sentation param�trique de la droite parall�le �
  $(IJ)$ passant par~$A$.
\item D�terminer l'intersection de $(IJ)$ avec le plan~$(xOy)$.

  \end{enumerate}
 

\rotatebox{180}{{\bf Solution:} On choisit un vecteur directeur, par
  exemple $\overrightarrow{IJ}(4;-2;-1)$ et un point, $I$ ou~$J$\dots}\\
\end{exercice}
 
Une fois fait: Tout le monde a-t-il la m�me repr�sentation
param�trique? En donner une autre.
\begin{rem}
Une droite n'a pas une unique repr�sentation param�trique. On a le
choix pour le vecteur directeur et pour le point. Alors comment savoir
si deux repr�sentations diff�rentes caract�risent la m�me droite?
\end{rem}

\centrage{Exercices}
cf. Repr�sentations param�triques de droites. fp.1 (fichier \texttt{Exos\_Repres\_par\_droites\_fp1.tex})
\centrage{Comp�tences � avoir.}
\begin{dingautolist}{172}
\item Savoir v�rifier si un point appartient � une droite dont on
  donne une repr�sentation param�trique.
\item Savoir d�terminer les coordonn�es de deux (ou plus) points sur une droite dont on donne une repr�sentation param�trique.
\item Savoir d�terminer une repr�sentation param�trique d'une droite
  $\mathscr{D}$ connaissant:
  \begin{itemize}
  \item Un point et un vecteur directeur de $\mathscr{D}$.
  \item Deux points distincts de $\mathscr{D}$.
  \item Un point et sachant que $\mathscr{D}$ est  parall�le � une droite $\mathscr{D}'$ dont on a une repr�sentation param�trique.
  %\item 
  \end{itemize}

\item Savoir d�cider si deux droites dont on donne des repr�sentations
  param�triques sont:
\begin{itemize}
  \item Parall�les.
  \item Confondues.
  \item S�cantes, et dans ce cas savoir d�terminer les coordonn�es du point d'intersection.
  \end{itemize}
\item Savoir d�terminer les coordonn�es du point d'intersection (si il
  existe) d'une
  droite avec un plan (parall�le � un plan de coordonn�es d'abord).
\end{dingautolist}
 
%%_______________________________Fin 1er trimestre
\chapter{\'Equations diff�rentielles}
\section{D�finition, exemples d'�quations diff�rentielles}
On en a d�j� vu lors de la d�finition de l'exponentielle: $y'=y$. 
\section{\'Equation lin�aire du premier ordre sans second menmbre}
\section{\'Equation lin�aire du premier ordre avec second menmbre}
\chapter{Probabilit�s et d�nombrement}

\section{Probabilit�s conditionnelles}
\section{Combinatoire}
\subsection{D�nombrer des listes}
Dans ce paragraphe, $E$ d�signe un ensemble � $n$ �l�ments ($n\in
\N^*$). On imagine l'existence d'une urne $U$ contenant $n$ jetons sur
lequels sont inscrits les $n$ �l�ments de $E$. 
\begin{Def} On appelle \emph{liste} de $p$ �l�ments de $E$ une �num�ration
  {\bf ordonn�e} de ces $p$ �l�ments. On la note comme des coordonn�es
  de points.
\end{Def}
\exemple Si $a$, $b$, $c$ sont trois �l�ments de $E$, $(a\,;b\,;c)$ et
$(a\,;c\,;b)$ sont deux listes distinctes avec les trois �l�ments $a$,
$b$ et $c$.
Dans ce paragraphe on s'int�resse donc � d�nombrer dans des situations
o� l'{\bf ordre} des �l�ments compte.
\subsubsection{Permutations d'un ensemble}
% On veut r�pondre aux questions du type:
% \begin{list}{$\bullet$}{}
% \item Combien peut-on �crire de mots distincts avec les lettres A, B, C?
% \item Combien y a-t-il de quint�s avec les chevaux 1, 2, 3, 4, 5?
% \item Combien y a-t-il de listes des $n$ �l�ments de $E$?
% \end{list}
Le mod�le est celui du tirage al�atoire dans l'urne $U$ des $n$
jetons, sans remise, et en notant l'ordre de sortie.
\begin{Def} On
  appelle \emph{permutation} de $E$ une liste de ses $n$ �l�ments.
\end{Def}
\exemple Si $E=\left\{A,\,B,\, C\right\}$ il y a six permutations de
$E$ et donc six mots distincts avec les trois lettres $A$, $B$ et $C$. On les
trouve avec un arbre en distinguant les choix pour le 1\ier{}, le 2\ieme{} et
le 3\ieme{} �l�ment: ABC, ACB, BAC, BCA, CAB, CBA.
\begin{Def} Soit $n\in\N^*$. On note $n!$ l'entier appell�
  \emph{factorielle $n$} d�fini par: $n!=1\times 2\times\cdots\times
  n$.
Par convention on pose $0!=1$
\end{Def}
\begin{prop}
Il y a $n!$ \emph{permutations} des $n$ �l�ments de $E$.
\end{prop}
\begin{proof} On le prouve avec un arbre ou en imaginant remplir des
  cases: Il il a $n$ choix pour le 1\ier{} �l�ment, et pour chacun de
  ces choix il y a $n-1$ choix pour le 2\ieme{} �l�ment \ldots \emph{etc}
  puis plus qu'un choix pour le dernier �l�ment. On trouve donc:
  $n\times(n-1)\times\cdots\times 1=n!$
\end{proof}
\exemple Cinq chevaux font la course. Combien y a-t-il d'arriv�es
possibles (on suppose qu'il ne peut pas y avoir
d'\emph{ex-\ae{}quo}). Une arriv�e est une permutation de l'ensemble
des cinq chevaux. Il y a donc $5!=120$ quint�s possibles avec 5 chevaux donn�s.

 
\subsubsection{Liste sans r�p�tition de $p$ �l�ments parmi $n$}
Le mod�le est celui du tirage al�atoire dans l'urne $U$ de $p$
jetons, {\bf sans remise}, et en notant l'ordre de sortie. O� $p$ est entier,
$1\leq p\leq n$.
\begin{Def} Une \emph{liste sans r�p�tition} de $p$ �l�ments de $E$ est une
  liste o� les $p$ �l�ments sont deux � deux distincts.
\end{Def}
\begin{prop}
Il y a $\dfrac{n!}{(n-p)!}=n\times(n-1)\times\cdots\times (n-(p-1))$  listes sans r�p�tition de $p$ �l�ments de $E$.
\end{prop}
\begin{proof} On le prouve avec un arbre ou en imaginant remplir des
  cases: Il il a $n$ choix pour le 1\ier{} �l�ment, et pour chacun de
  ces choix il y a $n-1$ choix pour le 2\ieme{} �l�ment \ldots \emph{etc}
  puis $(n-(p-1))$ choix pour le $p$\ieme{} et dernier �l�ment.
\end{proof}
\exemple Combien y a-t-il de tierc�s possibles avec $10$ chevaux au
d�part? Ici $E$ est l'esemble des 10 chevaux. Un tierc� est une liste
sans r�p�tition de trois de ces chevaux. Il y a donc: $\dfrac{10!}{7!}=10\times 9\times 8=720$ tierc�s
possibles avec $10$ chevaux au d�part.
\subsubsection{Liste avec r�p�tition de $p$ �l�ments parmi $n$}
Le mod�le est celui du tirage al�atoire dans l'urne $U$ de $p$
jetons, {\bf avec remise}, et en notant l'ordre de sortie. Ici
$p\in\N$. On peut donc avoir $p\geq n$.
\begin{Def} Une \emph{liste avec r�p�tition} de $p$ �l�ments de $E$ est une
  liste o� les $p$ �l�ments ne sont pas n�cessairement deux � deux distincts.
\end{Def}
\og{} Avec r�p�tition \fg{} est donc � comprendre au sens o� \og il peut y
avoir r�p�titon\fg{}. Remarquez que si $p>n$ il y a n�c�ssairement
r�p�tition. (C'est le principe des tiroirs: Si il y a plus de
chaussettes que de tiroirs, il y a au moins un tiroir qui comporte au
moins deux chaussettes.)
\begin{prop}
Il y a $n^p$ listes avec r�p�tition de $p$ �l�ments de $E$.
\end{prop}
\exemple Combien y a-t-il de points de l'espace dont les coordonn�es sont $-1$
ou $1$? Ici $p=3$ et $n=2$ car $E=\{-1;\,1\}$. Il y a deux choix pour
l'abscisse, deux pour l'ordonn�e et deux aussi pour la cote. Soit
$2^3=8$ points en tout. Ce sont les coordonn�es des sommets d'un cube. 

\subsection{Combinaisons}
Maintenant on ne d�nombrera plus des listes mais des sous ensembles de
$E$, l'ordre des �l�ments ne comptera pas. $E$ d�signe encore un ensemble � $n$ �l�ments ($n\in
\N^*$) et $p$ d�signe un entier compris entre $0$ et $n$.\\
\noindent Le mod�le est celui du tirage al�atoire dans l'urne $U$ de $p$
jetons, {\bf sans remise}, et sans tenir compte de l'ordre de sortie.
\subsubsection{Choisir $p$ �l�ments parmi $n$}
\begin{Def} Une \emph{combinaison} de $p$ �l�ments de $E$ est un sous
  ensemble (ou partie) de $E$ qui comporte $p$ �l�ments.
\end{Def}
\begin{Def} Le nombre de combinaisons de $p$ �l�ments d'un  ensemble �
  $n$ �l�ments est not�:$\binom{n}{p}$ et se lit \og{}$p$ parmi $n$\fg{}. (En France
  on le  notait autrefois $\mathrm{C}_n^p$)
\end{Def}

\exemple Si $E=\left\{a\,;\,b\,;\,c\right\}$, alors:
  \begin{list}{$\bullet$}{}
  \item $\binom{3}{3}=1$. Il y a une combinaison de  trois �l�ments de $E$, c'est $E$ lui m�me.
  \item $\binom{3}{2}=3$. Il y a trois combinaisons de deux �l�ments de $E$:
    $\left\{a\,;\,b\right\}$; $\left\{a\,;\,c\right\}$; $\left\{b\,;\,c\right\}$
  \item $\binom{3}{1}=3$. Il y a trois combinaisons d'un �l�ment de $E$:
    $\left\{a\right\}$; $\left\{b\right\}$; $\left\{c\right\}$

  \item $\binom{3}{0}=1$. Il y a une combinaison de 0 �l�ment de $E$ c'est l'ensemble vide: $\emptyset$
  \end{list}
{\bf Calculette.} Pour calculer $\binom{5}{3}$ � la calculette, utiliser la fonction not�e nCr (5 nCr 3 {\sc Exe}). Menu Proba (TI:
Math/Prb, Casio: Optn/Prob). 
\subsubsection{Nombre de combinaisons}

\begin{prop}
$\displaystyle \boxed{\binom{n}{p}=\frac{n!}{p!(n-p)!}}$
\end{prop}
\begin{proof} On sait qu'il y a  $\frac{n!}{(n-p)!}$ listes sans
  r�p�tition de $p$ �l�ments de $E$. Quand on a un ensemble � $p$
  �l�ments il y a $p!$ permutations, soit $p!$ listes sans r�p�tition
  avec ces $p$ �l�ments. C'est pourquoi pour obtenir le nombre de
  parties (\ie{} de combinaisons) de $E$ � $p$ �l�ments on doit
  diviser $\frac{n!}{(n-p)!}$ par $p!$
\end{proof}
\exemple Combien y a-t-il de mains diff�rentes au bridge? (On joue �
quatre au bridge avec un jeu de 52 cartes que l'on distribue
enti�rement). On doit donc compter le nombre de fa�ons de choisir 13
cartes parmi 52, soit $\binom{52}{13}=\frac{52!}{13!(39)!}=1905040678800\approx 1,9.10^{12}$
%%%
\section{Loi de probabilit�s discr�tes}
\subsection{D�finitions}
Soit $\Omega$ un univers fini � $n$ �l�ments ($n\in\N$). On note
$(\omega_i)_{1\leq i\leq n}$ les $n$ �v�nements �l�mentaires qui
constituent $\Omega$. Donc $\Omega=\{\omega_1;\,\omega_2;\ldots ;\omega_n\,\}$
\begin{Def}
  On appelle \emph{variable al�atoire} sur $\Omega$ toute fonction $X$ qui va de
  $\Omega$ dans $\R$. Pour chaque $i=1,2,\ldots, n$ on note
  $x_i=X(\omega_i)$.
  \begin{eqnarray}
    X:&&\Omega \to \R\notag\\
       &&\omega_i \mapsto X(\omega_i)=x_i \notag
  \end{eqnarray}
\end{Def}
L'ensemble des $x_i$ est donc l'ensemble des valeurs possibles pour
$X$. 
\exemple {\sl Une urne contient 3 jetons noirs et 2 jetons blancs. On tire
sans remise deux jetons dans une urne. On note $X$ le nombre de jetons
noirs tir�s.}\\
On peut choisir: $\Omega=\{ \{N;N \};\{N;B \};\{B;B \} \}$\\
 Alors:
$X\left(\{N;N \}\right)=2$ ; $X(\{N;B \})=1$ ; $X(\{B;B\})=0$. \\
On peut aussi choisir: $\Omega'=\{ (N;N);(N;B);(B;N);(B;B) \}$. On
aurait alors: \\
 $X((N;B))=X((B;N))=1$\ldots

\begin{Def} On appelle \emph{loi de probabilit�} d'une variable al�atoire $X$
  la donn�e pour chaque valeur $x_i$ possible pour $X$ de
  $p_i=P(X=x_i)$. On la pr�sente en g�n�ral sous forme de tableau.
 
\end{Def}
 Ce tableau est � rapprocher de ceux qu'on peut dresser en
 statistiques lorsqu'on a une s�rie de valeurs et leur fr�quence
 d'aparition.
 \begin{prop}
   La somme des $p_i=P(X=x_i)$ vaut 1.
 \end{prop}
Cela vient de ce que les \og$X=x_i$\fg{} forment une partition de $\Omega$.
\exemple On reprend l'exemple pr�c�dent. On dresse un arbre 
\verb#(Revoir le code pstricks)#

% \psset{radius=6pt,dotsize=4pt,treefit=loose}
% \begin{center}
% \pstree[treemode=R]
%     {\Tcircle{0}}
%     {\pstree{\Tcircle{N}^{$\frac{3}{5}$}}%_{$\frac{1}{2}$}}
%             {\pstree{\Tcircle[name=XU2,linestyle=dashed]{N}}%^{U}}
%                     {\Tdot~{(N,N)}^{L}
%                      \Tdot~{(4,1)}_{R}}
% \pstree{\Tcircle[name=XD2,linestyle=dashed]{B}}
%                     {\Tdot~{(3,0)}^{L}
%                      \Tdot~{(1,1)}_{R}}}
%     \pstree{\Tcircle{B}_{$\frac{1}{2}$}}%_{Y}}
%            {\pstree{\Tcircle[name=YU2,linestyle=dashed]{N}}
%                     {\Tdot~{(1,2)}^{L}
%                     \Tdot~{(2,0)}_{R}}
%             \pstree{\Tcircle[name=YD2,linestyle=dashed]{B}}
%                     {\Tdot~{(3,3)}^{L}
%                     \Tdot~{(0,1)}_{R}}}}
% \end{center}
\vspace{4cm}
On trouve
$P({N;N})=0,3$; $P({N;B})=0,6$; $P({B;B})=0,1$. D'o� la loi de $X$.
\[
\begin{array}{|c|c|c|c|}
\hline
  x_i&0&1&2 \\ \hline
P(X=x_i)&0,1&0,6&0,3 \\
\hline
\end{array}
\]
\begin{Def} On appelle \emph{fonction de r�partition} d'une variable al�atoire $X$
  la fonction $F$ d�finie sur $R$ par:  $F(x)=P(X\leq x)$. 
\end{Def}
Cette fonction est croissante sur $\R$, varie de 0 � 1 mais n'est
\emph{a priori} pas continue. En effet dans notre cas pr�sent, $X$ ne prend qu'un nombre
fini de valeurs. En chacune de ces valeurs $x_i$, $F$ pr�sente une
discontinuit� et m�me plus pr�cis�ment un saut de hauteur $P(X=x_i)$. Entre deux
valeurs cons�cutives, $F$ est constante.
\exemple On reprend notre exemple. Pour $x<0$, $P(X<x)=0$ et
$P(X=0)=0,1$ donc $P(X\leq 0)=0,1$. Soit $F(0)=0,1$ etc \ldots

La connaissance de $F$ permet de rouver la loi de probabilit� de
$X$. Plus tard, on d�finira des variables al�atoires continues, par leur fonction de r�partition.
\subsection{Esp�rance math�matique, variance}

\begin{Def} $E(X)=\sum \ldots$
\end{Def}
Lin�arit� 

\begin{prop}
  $ E(X+Y)=E(X)+E(Y)$
 \end{prop}
\begin{prop}
$   E(aX+b)=aE(X)+b$
 \end{prop}

\begin{Def} $V(X)=E\left(\left(X-E(X)\right)^2\right)$
\end{Def}
\begin{prop}
$V(X)=E(X^2)-E(X)^2$
 \end{prop}
\section{Lois continues} %conditionnelles

\chapter{Nombres Complexes, forme exponentielle et transformations}
\section{Forme trigonom�trique d'un nombre complexe}
\subsection{Rep\'erage polaire dans le plan}
Soit $\oij$ un rep\`ere orthonorm\'e du plan. Un point $M$ y est rep\'er\'e
par ses coordonn\'ees cart\'esiennes (de \textsc{Descartes})   \ie{}
$M(x;y)\iff \V{OM}  = x\vi + y\vj$. Il y a une autre façon de d\'efinir
la position du point $M$ : Par la donn\'ee de la longueur $OM$ et de
l'angle orient\'e $(\vi  ; \V{OM} )$. Cela donne les coordonn\'ees
polaires: $M(\rho;\theta)$ o\`u $x=\rho\cos \theta$ et $y=\rho\sin
\theta$. Donc:
\[
\rho=\sqrt{x^2+y^2}\qquad \et{}\ \theta\ \text{v\'erifie:}
\left\{
\begin{array}[c]{ll}
\cos \theta&= \frac{x}{\rho}\\
\sin \theta&=\frac{y}{\rho}\\
  \end{array} \right.
\]

\noindent
\exemple  Le point $M$ a pour coordonn\'ees cart\'esiennes $(-\sqrt3;3)$. On calcule $\rho$:
\[ \rho=\sqrt{(-\sqrt3)^2+3^2} = \sqrt{12 } = 2\sqrt 3 \]
Il faut toujours simplifier la racine. Puis:

\[  
\Bigl\{
\begin{array}[c]{ll}
\cos \theta&=\frac{-\sqrt3}{2\sqrt 3} \\
\sin \theta&=\frac{3}{2\sqrt 3} 
  \end{array}
\iff 
\Bigl\{
\begin{array}[c]{ll}
\cos \theta&=-\frac12 \\
\sin \theta&=\frac{\sqrt3}{2} 
  \end{array}
 \]
On part de l'angle de r\'ef\'erence qui a les m\^emes valeurs de cos et sin
en valeur absolue. C'est $\frac{\pi}{3}$. On place $\frac{\pi}{3}$ sur
un cercle trigonom\'etrique, ainsi que l'angle $\theta$ cherch\'e et on
observe que $\theta=\pi- \frac{\pi}{3}$. Donc $\theta=\frac{2\pi}{3}$
convient. Ainsi des coordonn\'ees polaires de $M$ sont: $(2\sqrt
3;\frac{2\pi}{3})$.\qed %\hline





%%%%%%%%%%%%%%%%%%%%%%%%%%%%%%%%%%%%%%%%%%%%%%%%%%%%%%%%%%%%%%%%%
\begin{minipage}[l]{6.8cm}
%\begin{table}[htbp]

\renewcommand{\arraystretch}{1.5}
\begin{tabular}{||c|c|c|c|c|c||}
\hline 
%Ligne 1
% $ x$ en degr\'es& 
%0& 30&45&60&90 \\\hline
%Ligne 2
 $\dfrac{}{}$$ x$ en radians& 
0& $\frac{\pi}{6}$&$\frac{\pi}{4}$&$\frac{\pi}{3}$&$\frac{\pi}{2}$ \\\hline \hline
%Ligne 3
 $\dfrac{}{}$$ \cos x$& 
1& $\frac{\sqrt{3}}{2}$&$\frac{\sqrt{2}}{2}$&$\frac{1}{2}$&0 \\\hline

%Ligne 4
 $\dfrac{}{}$$ \sin x$& 
0& $\frac{1}{2}$&$\frac{\sqrt{2}}{2}$&$\frac{\sqrt{3}}{2}$&1 \\\hline 
\end{tabular}
\label{tab1}
\renewcommand{\arraystretch}{1}
 Pour se souvenir de ce tableau il suffit de conna\^itre le quart de
 cercle trigonom\'etrique  et de se souvenir de: 
$\frac{1}{2}<\frac{\sqrt{2}}{2}<\frac{\sqrt{3}}{2}$
\end{minipage}
%%%%%%%%%%%%%%%%%%%%%%%%%%%%%%%%%%%%%%%%%%%%%%%%%%%%%%%%%%%%%%%%%
\begin{minipage}[l]{0.5\textwidth}
\begin{center}
\psfrag{O}{$O$} \psfrag{x}{$x$} \psfrag{y}{$y$} \psfrag{vi}{$\vi$} \psfrag{vj}{$\vj$} 
\includegraphics{/home/pan/Desktop/TeX_MiX/Figures/pol_cart.1}%[totalheight=4cm]{cercle_trigo.3}
\end{center}
\end{minipage}
\subsection{Forme trigonom\'etrique d'un nombre complexe}
Traduisons cel\`a en complexe. Soit M un point d'affixe $z=x+\I y$ (o\`u
$x\in\R$, $y\in\R$). $x+\I y$ est la forme alg\'ebrique de $z$, M a pour
coordonn\'ees cart\'esiennes $(x;y)$. Le rayon polaire $\rho$ est le module de
$z$, et un angle polaire $\theta$ s'appelle  {\bf un argument}
de~$z$.\\
\noindent
\begin{Def} Tout complexe $z$ peut s'\'ecrire sous la forme appel\'ee forme trigonom\'etrique de~$z$:
\[z=|z|\times(\cos \theta +\I \sin\theta)  \]
\end{Def}
Un argument $\theta$ n'est pas unique, car d\'efini modulo $2\pi$. On
note: $\arg (z)=\theta$ un argument de~$z$.\newline
%\hline
\begin{prop}
  Deux nombres complexes sont �gaux ssi ils ont m�me module et m�me
  argument � $2\pi$ pr�s.
\end{prop}
\noindent
\exemple {\sl Cf exemple pr\'ec\'edent.} Soit
$z=-\sqrt3+3\I $. Mettre $z$ sous forme trigonom\'etrique.\\
$|z|=\sqrt{12}=2\sqrt3$. La suite est identique et donc
$\theta=\frac{2\pi}{3}$ d'o\`u: $z=2\sqrt3(\cos\frac{2\pi}{3}  +\I \sin\frac{2\pi}{3})$
\qed
\exercice
Mettre les nombres complexes suivants sous forme trigonom\'etrique:
\begin{enumerate}
\item \qcmabc{$z=1+\I{}$}{$z=5-5\I{}$}{$z=-3-3\I{}$}
\item \qcmabc{$z=-4$}{$z=\I{}$}{$z=-2\I{}$}
\item \qcmabc{$z=2\sqrt3-2\I{}$}{$z=-3+\I{}\sqrt3$}{$z=-1+\I{}\sqrt3$}
\end{enumerate}
\exercice
Mettre les nombres complexes $z$ et $z'$ suivants sous forme
trigonom\'etrique, ainsi que $zz'$ et $\dfrac{z}{z'}$:
\begin{enumerate}
\item %
\begin{minipage}[l]{0.45\textwidth}
(a) $z=4\I \quad$ et $\quad z'=1+\I{}\sqrt3$
\end{minipage}\hfill
\begin{minipage}[l]{0.45\textwidth}
(b) $z=-2+2\I{}\sqrt3\quad$ et $\quad z'=-\sqrt3-\I{}$
\end{minipage}%
\item $z=\cos\theta+\I{}\sin\theta\quad$ et $\quad
  z'=\cos\theta'+\I{}\sin\theta'$ \\
On donnera $\arg{}(zz')$ et
  $\arg{}(\dfrac{z}{z'})$ en fonction de $\theta$ et~$\theta'$.
\item D\'emontrer les formules dans le cas g\'en\'eral:
\[ \arg{}(zz')=\arg{}(z)+\arg{}(z')\quad\et{}\quad \arg{}(\dfrac{z}{z'})=\arg{}(z)-\arg{}(z')\]
\end{enumerate}
\subsection{Propri�t�s des arguments}
\paragraph{ Rappel de trigonom�trie}
$\cos(a+b)=$\\
$\sin(a+b)=$

\begin{prop} Pour $z$ et $z'$ deux complexes non nuls:
  \begin{dingautolist}{172}
  \item $\arg{zz'}=\arg{z}+\arg{z'}\quad [2\pi]$
  \item $\arg{\frac{1}{z}}=-\arg{z}\quad [2\pi]$
  \item $\arg{\frac{z}{z'}}=\arg{z}-\arg{z'}\quad [2\pi]$
  \item $\arg{z^n}=n\arg{z}\quad [2\pi]$
  \item $\arg{\bar{z}}=-\arg{z}\quad [2\pi]$
  \end{dingautolist}
\end{prop}

\paragraph{Interpr�tation g�om�trique}
\begin{list}{$\bullet$}{}
\item Argument de l'affixe d'un vecteur: $\arg{(z_B-z_A)}=( \vec{u}\, ;\overrightarrow{AB})$ 
\item $\arg\dfrac{z_B-z_A}{z_D-z_C}=(\overrightarrow{CD};\overrightarrow{AB})$
\end{list}
\paragraph{Ensembles de points}

\begin{list}{$\bullet$}{}
\item $z\in\R^* \iff \arg z =0 \quad [\pi]$
\item $z\in\R^{+*} \iff \arg z =0 \quad [2\pi]$
\item $z\in\R^{-*} \iff \arg z =\pi \quad [2\pi]$
\item $z\ \text{est imaginaire pur} \iff (\arg z =\dfrac{\pi}{2} \quad
  [\pi] \quad\text{ou}\quad z=0)$

\end{list}
\exercice
D�terminer et tracer  l'ensemble $\mathscr{E}_i$ des points $M$ d'affixe
$z$ v�rifiant une des conditions suivantes:
\begin{dingautolist}{172}
\item $\mathscr{E}_1$: $(1+i)z\in \mathbb{R}$
\item $\mathscr{E}_2$: $\arg iz =\dfrac{\pi}{4} \quad [2\pi]$
\item  $\mathscr{E}_3$: $\arg\left(\dfrac{z-z_B}{z-z_A}\right)=\pi \quad [2\pi]$ o�
  $z_A=2-\I$ et $z_B=-1+\I$
\item  $\mathscr{E}_4$: $\arg\left(\dfrac{z-z_B}{z-z_A}\right)=\dfrac{\pi}{2} \quad [2\pi]$ o�
  $z_A=2-\I$ et $z_B=-1+\I$
\item  $\mathscr{E}_5$: $\arg\left(\dfrac{z-z_B}{z-z_A}\right)=\dfrac{\pi}{2} \quad [\pi]$ o�
  $z_A=-2-2\I$ et $z_B=-\I$
\item  $\mathscr{E}_6$: $\dfrac{z-z_B}{z-z_A}\in \mathbb{R}^{-*}$ o�
  $z_A=3$ et $z_B=1+2\I$
\end{dingautolist}


\section{Forme exponentielle d'un nombre complexe}
Vues les propri�t�s des arguments, on pose pour $\theta$ r�el :
\[ \e^{\I{}\theta}=\cos\theta+\I{}\sin\theta\]
On peut v�rifier la coh�rence de cette notation avec l'�qua diff
$y'=iy$ v�rifi�e par $\theta \longmapsto \cos\theta+\I{}\sin\theta$.
\begin{Def} Le complexe non nul $z$, de module $|z|$ et
  d'argument $\theta$ est mis sous forme exponentielle lorsque l'on �crit:
  $z=|z|\e^{\I{}\theta}$ o�:
\[ \e^{\I{}\theta}=\cos\theta+\I{}\sin\theta\]
\end{Def}
Ainsi $\e^{\I{}\theta}$ d�signe un nombre complexe de module
1. R�ciproquement, tout nombre complexe de module 1 peut s'�crire sous
cette forme. 
\noindent Les propri�t�s suivantes sont valables pour tous r�els $\theta$ et $\theta'$,
$r$ et $r'$, pour tous $n$ et~$k$ dans $\Z$. Elles d�coulent des
propri�t�s vues sur le module et les arguments d'un nombre complexe
non nul.

\begin{list}{}{}

\item[{\bf Addition d'angles}]
{\large\[ \e^{\I{}(\theta+\theta')}=\e^{\I{}\theta}\e^{\I{}\theta'}\]}
\item[{\bf Conjugu�}]
{\large\[ \overline{r\e^{\I{}\theta}}=r\e^{-\I{}\theta} \]}
\item[{\bf Module unitaire}]
{\large\[ \left|\e^{\I{}\theta}\right|=1\]}
\item[{\bf Inverse}]
{\large\[ \frac{1}{\e^{\I{}\theta}}=\e^{-\I{}\theta}\]}
\item[{\bf Quotient}]
{\large\[ \frac{\e^{\I{}\theta}}{\e^{\I{}\theta'}}=\e^{\I{}(\theta-\theta')}\]}
\item[{\bf Puissance-Formule de De Moivre}]
{\large\[ \left( \e^{\I{}\theta}\right)^n=\e^{in\theta}\]}
\item[{\bf Produit, quotient de deux complexes}]
{\large\[
\left(r\e^{\I{}\theta}\right)\times\left(r'\e^{\I{}\theta'}\right)=rr'\e^{\I{}(\theta+\theta')}\qquad
et \qquad
\frac{r\e^{\I{}\theta}}{r'\e^{\I{}\theta'}}=\frac{r}{r'}\e^{\I{}(\theta-\theta')}
\]}
\item[{\bf Valeurs remarquables}]
{\Large\[ \e^{0}=1\quad;\quad \e^{\I{}\frac{\pi}{2}}=\I{}\quad;\quad
  \e^{\I{}\pi}=-1 \quad;\quad \e^{2\I{}\pi}=1 \]}
\item[{\bf Oppos� d'un complexe}]
{\large\[ -r\e^{\I{}\theta}=r\e^{\I{}(\theta+\pi)}\]}
\item[{\bf $\theta\mapsto\e^{\I{}\theta}$ est $2\pi$ p�riodique}]
{\large\[ \e^{\I{}(\theta+2k\pi)}=\e^{\I{}\theta}\]}
{\large\[ \e^{\I{}\theta}=\e^{\I{}\theta'} \iff \theta=\theta'\ [2\pi]\]}
\item[{\bf Formules d'Euler }]
{\large\[ \cos\theta=\frac{\e^{\I{}\theta}+\e^{-\I{}\theta}}{2}\quad\et{}\quad
\sin\theta=\frac{\e^{\I{}\theta}-\e^{-\I{}\theta}}{2\I{}}\]}
Ces formules proviennent de ce que pour tout complexe $z$:
\[Re(z)=\frac{z+\overline{z}}{2}\quad\et{}\quad Im(z)=\frac{z-\overline{z}}{2\I{}}\]
\item[{\bf Attention!}] $\e^{\I{}\theta}$ est un nombre complexe, on ne peut donc pas parler de son signe\ldots \DangerZ
%\[ \]
%\ding{63} 
%exp est positif ne veut rien dire

\end{list}
\exercice
R�soudre dans $\mathbb{C}$:\quad $z^3=8\I$
\subsection{Param�trisation d'un cercle}
Le cercle de centre $O$ et de rayon $r$ est l'ensemble des points
d'affixe de module $R$, donc de la forme: $R\e^{\I{}\theta}$ o�
$\theta$ d�crit $\mathbb{R}$ (ou tout intervalle de longueur au moins
�gale � $2\pi$). Plus g�n�ralement:
\begin{prop}
  Le cercle de centre $\Omega$ d'affixe $z_{\Omega}$ et de rayon $R$
  ($R\in\mathbb{R}^+$) est l'ensemble des points d'affixe:
\[ z=z_{\Omega}+R\e^{\I{}\theta}\]
o� $\theta$ d�crit $\mathbb{R}$ (ou tout intervalle de longueur au moins
�gale � $2\pi$)
\end{prop}
\begin{proof}
 $M\in C\iff |z-z{\Omega}|=R$
\end{proof}
On a ainsi une fonction de $\mathbb{R}$ dans le plan qui fournit
l'enroulement de la droite r�elle sur le cercle que l'on a vu \og avec
les mains\fg{} en seconde.
\section{Transformations complexes}

Soit $\Fr$ une transformation du plan qui au point $M$ associe le
point $M'$. \ie{} $\Fr(M)=M'$. On associe � cette transformation la
fonction complexe $f$ qui � l'affixe $z$ de $M$ associe l'affixe $z'$
de~$M'$. On dit que {\bf $z'=f(z)$ est l'�criture complexe de la
transformation~$\Fr$.} On va donner les �critures complexes de trois
transformations usuelles.
\subsection{Translation}
\begin{prop} Soit $\vw$ un vecteur d'affixe $b$. 
L'�criture complexe de la translation de vecteur $\vw$ est:
\[\boxed{z'=z+b}\]
\end{prop}

\begin{proof} Ici $\Fr$ est la translation de vecteur
  $\vw$.  
\[\Fr(M)=M' \iff \V{MM'}=\vw\iff z'-z=b\iff z'=z+b\]
\end{proof}
\subsection{Homoth�tie}
\begin{prop} Soit $\Omega$ un point d'affixe $z_{\Omega}$ et
  $k\in\R^*$. L'�criture complexe de l'homoth�tie de centre $\Omega$ et de rapport $k$ est:
\[\boxed{z'-z_{\Omega}=k(z-z_{\Omega})}\]
\end{prop}

\begin{proof}Ici $\Fr$ est l'homoth�tie de centre $\Omega$ et de
  rapport $k$.
\[\Fr(M)=M' \iff \V{\Omega M'}=k \V{\Omega M} \iff z'-z_{\Omega}=k(z-z_{\Omega})\]
\end{proof}
\subsection{Rotation}
\begin{prop} Soit $\theta\in\R$.
L'�criture complexe de la rotation de  centre $\Omega$ et d'angle
$\theta$ est:
\[\boxed{z'-z_{\Omega}=\e^{\I\theta}(z-z_{\Omega})}\]
\end{prop}

\begin{proof}Ici $\Fr$ est la rotation de  centre $\Omega$ et d'angle
$\theta$. 
\begin{list}{$\bullet$}{}
\item Si $M=\Omega$ alors $\Fr(M)=M' \iff M'=\Omega \iff z'=z_{\Omega}$. Car le centre
  d'une rotation est l'unique point invariant par cette
  rotation. L'�galit� voulue est donc vraie (0=0)
\item Si $M\neq \Omega$ alors $\Fr(M)=M'$ signifie:
\[ \Omega M'=\Omega M\quad \et{}\quad (\V{\Omega M};\V{\Omega
  M'})=\theta\]
Ce qui se traduit en module et argument par:
\[ |z'-z_{\Omega}|=|z-z_{\Omega}|\quad \et{}\quad
\arg(\frac{z'-z_{\Omega}}{z-z_{\Omega}})=\theta [2\pi]\]
C'est � dire que le complexe $\frac{z'-z_{\Omega}}{z-z_{\Omega}}$ est
de module 1 et d'argument $\theta$. Il est donc �gal �
$\e^{i\theta}$. D'o� le r�sultat:
\[ \frac{z'-z_{\Omega}}{z-z_{\Omega}}=\e^{\I\theta}\iff
z'-z_{\Omega}=\e^{\I\theta}(z-z_{\Omega})\]
\end{list}
\end{proof}
\noindent
{\bf Cas particulier.} $z'=\I z$ est l'�criture complexe de la rotation
de centre $O$ et d'angle $\frac{\pi}{2}$ car
$\I =\e^{\I \frac{\pi}{2}}$. Donc le triangle $OMM'$ est isoc�le rectangle
en~ $O$. %Forme trigo, expo, transfos.
%%_______________________________Fin 2e trimestre
%\input{Limites.par.def}
\chapter*{Suites adjacentes}
\label{chap:Suites.adjacentes}
%\section{Limites, par la d�finition}
\section{Convergence monotone}
\subsection{Limites par la d�finition, rappels}
On va �noncer des r�sultats sur les suites, ils se traduisent aussi en
�nonc�s valables pour les fonctions. Dans la suite, $(u_n)$ d�signe
une suite r�elle, d�finie sur~$\mathbb{N}$.
\begin{Def}
  On dit que $(u_n)$ tend vers $+\infty$, si pour tout r�el $M$ fix�
  il existe un rang~$p$ � partir duquel tous les termes de la suite
  sont sup�rieurs � $M$. \emph{i.e.} 
\[\forall M\in\R, \exists p\in\N, n\geq p\implies u_n\geq M\]
\end{Def}
Le rang $p$ d�pend de $M$ et \emph{a priori}, plus $M$ est grand, plus $p$ sera grand.
\begin{rem}
  Ce n'est pas pareil que de dire que \og la suite n'est major�e par
  aucun r�el\fg{}. En effet la suite d�finie sur $\N^*$ par
  $u_n=n\times (-1)^n $ n'est major�e par aucun r�el mais elle ne
  tend pas vers $+\infty$.\DangerZ
\end{rem}
\begin{Def}
On dit que $(u_n)$ converge vers un r�el $\ell$ si tout intervalle
ouvert contenant~$\ell$ contient aussi tous les termes de la suite �
partir d'un certain rang.
\end{Def}

\begin{rem}
  On peut dans cette d�finition se restreindre aux intervalles ouverts
  centr�s en~$\ell$, c'est � dire de la forme
  $\left]\,\ell-\epsilon;\ell+\epsilon\right[$ o�~$\epsilon$ est un
  r�el strictement positif. De plus dire
  $u_n\in\left]\,\ell-\epsilon;\ell+\epsilon\right[$ signifie que
  $u_n-\ell\in\left]\,-\epsilon;\epsilon\right[$ soit encore
  $|u_n-\ell| <\epsilon$. Cette d�finition est donc �quivalente �:\\
\og Aussi petit que soit $\epsilon$ strictement positif, au bout d'un
moment, tous les termes de la suite sont � une distance strictement inf�rieure �
$\epsilon$ du r�el~$\ell$.\fg{} Ce qui se note ainsi:
\end{rem}
\noindent%
\begin{minipage}{1.0\linewidth - 8.2cm}
  La suite  $(u_n)$ converge vers un r�el $\ell$ si:
\[ \forall \epsilon\in\R_+^*, \exists p\in\N, n\geq p\implies
|u_n-\ell| <\epsilon  \]
Le rang $p$ d�pend de $\epsilon$ et \emph{a priori}, plus $\epsilon$ est petit, plus $p$ sera grand.
\end{minipage}\hfill
\begin{minipage}{7.6cm}
  \begin{center}
  \includegraphics[width=7.5cm]{/home/pan/Desktop/TeX_MiX/Figures/Metapost_fig/Limites.1}  
  \end{center}
  
\end{minipage}


\begin{Def}
  Une suite non convergente est dit \emph{divergente}. 
\end{Def}
Dire $(u_n)$ diverge ne signifie pas \og $(u_n)$ n'a pas de
limite\fg{}. L'ensemble des suites divergentes contient les suites qui tendent vers  $+\infty$, celles qui tendent vers $-\infty$ et celles qui n'ont pas
  de limite. On peut par exemple dire: \og $(u_n)$ diverge vers~$+\infty$. \fg{}

\subsection{ L'axiome de la convergence monotone.}
Une suite croissante ne tend pas n�c�ssairement vers $+\infty$, elle
peut aussi converger. On admet le r�sultat suivant (c'est un axiome) qui est li� �
la structure de l'ensemble des nombres r�els:
\begin{theo}[Convergence monotone] 
  Une suite croissante et major�e (ou d�croissante et minor�e) converge.
\end{theo}
%Ainsi pour une suite croissante, il peut se passer deux choses, soit
%elle est major�e et alors elle converge, soit elle n'est pas major�e,
%et dans ce cas elle diverge vers $+\infty$. (Preuve laiss�e en exercice)

Ce th�or�me est un th�or�me d'existence, il ne permet pas de calculer
une limite.
Si $(u_n)$ est croissante est major�e par 3 par exemple, elle
converge, mais pas n�c�ssairement vers 3. Certainement vers un r�el
inf�rieur (ou �gal) �~3.\DangerZ \\

\section{\'Etude des suites du type $u_{n+1}=f(u_n)$}
On se donne une fonction $f$ d�finie, continue sur un intervalle $I$ de
$\R$. Soit $u_0\in I$.
\subsection{Intervalle stable}
\Def{On dit que $I$ est stable par $f$ si $f(I)\subset I$}
\exemples{
  \begin{enumerate}
  \item $f(x)=x(1-x)$ sur $I=\intf{0}{1}$. $I$ est stable par f, mais
pas par $g=5f$. Il suffit d'�tudier $f$.
\item $f(x)=\frac{1}{x-1}$ et $u_0=\frac32$ V�rifier que la relation
  $u_{n+1}=f(u_n)$ ne d�finit pas une suite. 
  \end{enumerate}
}
La condition $I$ stable par $f$ permet de garantir que la suite est
d�finie et que tous les termes de la suite sont dans l'intervalle $I$.
(r�currence imm�diate)

\subsection{Sens de variation}

Il est faut de dire que $f$ croissante donne $(u_n)$ croissante, on a
en fait:
\begin{prop}
\begin{enumerate} 
  \item $f$ croissante $\implies$ $(u_n)$ monotone. Le sens de var est
    donn� par le signe de $u_1-u_0$
  \item $f$ d�croissante $\implies$ $(u_{2n})$ et $(u_{2n+1})$ sont
    monotones de monotonies contraires
\end{enumerate}
\end{prop}
\begin{proof}
   Supposons $u_1>u_0$.
\begin{enumerate} 
  \item Par r�currence. $\Para (n)$: \og $u_{n+1}-u_n\geq 0$\fg{}
  \item On pose $p_n=u_{2n}$ et $i_n=u_{2n+1}$. Alors:
\[p_{n+1}=f\circ f(p_n) \quad \et{}\quad i_{n+1}=f\circ f(i_n)\]
Or $f$ dec. implique $fof$ croissante donc par le 1. $p$ et $i$ sont
monotones.
Supposons $p$ croissante, alors $u_2\geq u_0$ donc, en appliquant $f$
qui est dec. on a $u_3\leq u_1$ donc $i_1 \leq i_0$ donc $i$ est dec.
\end{enumerate}
\end{proof}
Si $f-Id$ est de signe constant, on a un r�sultat sur le sens de
variation de~$(u_n)$:
\begin{prop} Un crit�re pour obtenir le sens de variation de $(u_n)$:
  \begin{enumerate} 
  \item $ \forall x \in I,\ f(x)-x\geq 0 \implies$ $(u_n)$ croissante
  \item $\forall x \in I,\ f(x)-x\leq 0 \implies$ $(u_n)$ d�croissante 
  \end{enumerate}
\end{prop}

\begin{proof}
  
\end{proof}

\subsection{Convergence}
\Def{On appelle point fixe d'une fonction $f$ un r�el $x$  tel que $f(x)=x$}
\begin{theo}Si $f$ est continue sur un intervalle ferm� $I$ et que $(u_n)$ converge, alors la  limite est n�cessairement un point fixe de~$f$.
\end{theo}
\begin{proof}
  
\end{proof}
Attention: L'existence d'un point fixe ne garantit pas la convergence
de $(u_n)$. Mais l'absence de point fixe suffit � justifier que $(u_n)$
ne converge pas.\DangerZ
En pratique: On justifie que $(u_n)$ CV (croissante major�e par
exemple) puis on d�termine la limite en cherchant les points fixes de
$f$. Si le point fixe est unique, c'est facile, sinon il faut
raisonner avec le sens de variation de $(u_n)$.

\exemple $u_{n+1}=\frac{u_n}{2-u_n}$ et $u_0 \in \intf{0}{1}$
\newpage

\subsection{Diff�rents comportements asymptotiques.}
Je vous ai trac� ci-dessous diff�rentes repr�sentations graphiques en
\og toile d'araign�e \fg{} d'une suite $(u_n)$ d�finie par une
relation de r�currence $u_{n+1}=f(u_n)$. J'ai donc trac� $C_f$ la
courbe de $f$ ainsi que la droite d'�quation~$y=x$.\\
% \noindent
% \includegraphics{/home/pan/Desktop/TeX_MiX/Figures/Metapost_fig/Suites-rec.1}\quad% 
% \includegraphics{/home/pan/Desktop/TeX_MiX/Figures/Metapost_fig/Suites-rec.2}\quad% 
% \includegraphics{/home/pan/Desktop/TeX_MiX/Figures/Metapost_fig/Suites-rec.3}\\
% \noindent%
% \includegraphics{/home/pan/Desktop/TeX_MiX/Figures/Metapost_fig/Suites-rec.4}\quad%  
% \includegraphics{/home/pan/Desktop/TeX_MiX/Figures/Metapost_fig/Suites-rec.5}\quad% 
% \includegraphics{/home/pan/Desktop/TeX_MiX/Figures/Metapost_fig/Suites-rec.7}\\

% \vspace{-2ex}
\setlength{\columnsep}{0.3cm}
% \setlength{\columnseprule}{0.5pt}
\begin{multicols}{3}
\noindent
\includegraphics{/home/pan/Desktop/TeX_MiX/Figures/Metapost_fig/Suites-rec.1}\\
\noindent $f$ croissante, $(u_n)$ d�croissante, convergeant vers le point fixe.\\

\noindent \includegraphics{/home/pan/Desktop/TeX_MiX/Figures/Metapost_fig/Suites-rec.2}\\
\noindent $f$ croissante, $(u_n)$ croissante, convergeant vers le
point fixe attractif, z�ro est un point fixe r�pulsif.\\
%
\noindent \includegraphics{/home/pan/Desktop/TeX_MiX/Figures/Metapost_fig/Suites-rec.5}\\
\noindent $f$ croissante, $(u_n)$ croissante, divergeant vers plus l'infini, le point
fixe est r�pulsif.\\
\end{multicols}%\vspace{-2ex}

\begin{multicols}{3}
\noindent \includegraphics{/home/pan/Desktop/TeX_MiX/Figures/Metapost_fig/Suites-rec.3}\\
\noindent $f$ d�croissante, $(u_n)$ non monotone, convergeant vers le point
fixe (attractif).\\

\noindent \includegraphics{/home/pan/Desktop/TeX_MiX/Figures/Metapost_fig/Suites-rec.4}\\
\noindent $f$ d�croissante, $(u_n)$ non monotone, divergeant (pas de limite ici), le point
fixe est r�pulsif.\\
\noindent \includegraphics{/home/pan/Desktop/TeX_MiX/Figures/Metapost_fig/Suites-rec.7}\\
\noindent La suite est attir�e vers un 2-cycle. Asymptotiquement, elle tend �
�tre 2-p�riodique. Elle ne converge donc pas, le point fixe est
r�pulsif, mais $f\circ f$ admet un point fixe attractif\dots
\end{multicols}%\vspace{-2ex}


\newpage

\subsection{Suite logistique}
\label{sec:logistique}
\begin{small}
  

Ci--dessous des sch�mas en toile d'araign�e pour la famille de fonctions
$f_k$ o�:
\[f_k(x)=k \times x(4-x)\]
Lorsque le param�tre $k$ varie de 0 � 1, on a divers comportements, de
la simple convergence � des ph�nom�nes dits chaotiques. L'�tude de
tels ph�nom�nes constitue le domaine des \og syst�mes dynamiques
discrets\fg{}. Il est intimement li� aux fractales. \\
J'ai rang� les figures par ordre croissant de $k$. Au d�but, z�ro est
le seul point fixe, il est attractif, puis nait un second point fixe
qui devient aussit�t attractif alors que z�ro devient r�pulsif. Ce
nouveau point fixe attire de plus en plus lentement, et devient
r�pulsif, un 2-cycle devient attracteur. Puis c'est un 4-cycle qui
devient attracteur (6\ieme{} figure), il s'en suit une s�rie infinie de doublement de
p�riodes (8-cycles, puis 16-cycles\dots ). Ensuite on trouve malgr�
tout des trois cycles ($k=0,958$) ce qui indique que le chaos va survenir (d'apr�s
le th�or�me de Sarkovski). Sur la derni�re figure (pour $k=1$), on observe que la
suite prend \og presque toutes les valeurs\fg{} dans $\left[0\,;4\right]$.
\[\]
\end{small}
\noindent%
\includegraphics{/home/pan/Desktop/TeX_MiX/Figures/Metapost_fig/Suites-rec.11}\quad%  
\includegraphics{/home/pan/Desktop/TeX_MiX/Figures/Metapost_fig/Suites-rec.13}\quad% 
\includegraphics{/home/pan/Desktop/TeX_MiX/Figures/Metapost_fig/Suites-rec.12}\\[0.2cm]
\noindent%
\includegraphics{/home/pan/Desktop/TeX_MiX/Figures/Metapost_fig/Suites-rec.7}\quad%  
\includegraphics{/home/pan/Desktop/TeX_MiX/Figures/Metapost_fig/Suites-rec.8}\quad% 
\includegraphics{/home/pan/Desktop/TeX_MiX/Figures/Metapost_fig/Suites-rec.6}\\[0.2cm]

\noindent%
\includegraphics{/home/pan/Desktop/TeX_MiX/Figures/Metapost_fig/Suites-rec.9}\quad%  
\includegraphics{/home/pan/Desktop/TeX_MiX/Figures/Metapost_fig/Suites-rec.10}\quad% 
\includegraphics{/home/pan/Desktop/TeX_MiX/Figures/Metapost_fig/Suites-rec.14}%\\[0.2cm]




\section{Th�or�mes de comparaison}
On a les propri�t�s suivantes de comparaison:
\begin{prop}
  Soit $(u_n)$ et $(v_n)$ deux suites convergentes, et $M$ un r�el.
  \begin{dingautolist}{172}
  \item Si pour tout $n$ � partir d'un certain rang on a
    $u_n < M$ alors : $\lim\limits_{ n \to +\infty }
    u_n\leq M$
  \item Si pour tout $n$ � partir d'un certain rang on a
    $u_n < v_n$ alors : $\lim\limits_{ n \to +\infty }
    u_n\leq \lim\limits_{ n \to +\infty } v_n$
  \end{dingautolist}
  
\end{prop}
\begin{prop}
  Si $v_n$ diverge vers $+\infty$ et que pour tout $n$ � partir d'un certain rang on a
    $u_n \geq v_n$ alors : $\lim\limits_{ n \to +\infty }
    u_n=+\infty$
\end{prop}
Utile pour prouver que des suites comme $u_n=2*(-1)^n+n$ tendent
vers $+\infty$.
Ces propri�t�s sont valables pour les fonctions. Par exemple le
th�or�me des gendarmes vu sur les suites en 1\iere{} est valable pour
les fonctions.
\begin{theo}[Th�or�me des gendarmes]
  Si on a trois fonctions $f$, $g$ et $h$ telles que:
  \begin{dingautolist}{172}
 \item Pour tout $x$ sup�rieur � un r�el $x_0$ on a l'encadrement:
\[f(x)\leq g(x)\leq h(x)\]
  \item $f$ et $h$ ont une limite finie identique $\ell$ en $+\infty$
    (par exemple) 
%$\lim\limits_{ x \to \infty } f(x)=\lim\limits_{ x \to \infty} h(x)\ell$
  \end{dingautolist}
Alors $g$ aussi admet une limite en $+\infty$ et c'est aussi~$\ell$.
  
\end{theo}
\begin{figure}[h]
  \centering
  \includegraphics{/home/pan/Desktop/TeX_MiX/Figures/Metapost_fig/Limites.2}
  \caption{Illustration pour la preuve du th�or�me des gendarmes.}
  \label{fig:Theogendarmes}
\end{figure}
\begin{proof}
  Soit $\epsilon\in\R_+^*$.\\
$\lim\limits_{ x \to \infty } f(x)=\ell$ donc il existe $x_1$ tel que:
$x\geq x_1\implies \epsilon \leq f(x)-\ell\leq \epsilon$\\
$\lim\limits_{ x \to \infty } h(x)=\ell$ donc il existe $x_2$ tel que:
$x\geq x_2\implies\epsilon \leq h(x)-\ell\leq \epsilon$\\
Pour $x\geq \max\{x_0, x_1,x_2\}$ on a donc:
\[\epsilon \leq f(x)-\ell\leq g(x)-\ell\leq h(x)-\ell\leq \epsilon\]
Et donc $\epsilon \leq  g(x)-\ell\leq \epsilon$
\end{proof}
\begin{rem}
  Le $+\infty$ peut �tre remplac� par $-\infty$. 
\end{rem}
\section{Suites adjacentes}
\begin{Def}
  On dit que deux suites sont adjacentes si elles
  v�rifient les deux conditions:
  \begin{dingautolist}{172}
  \item L'une est croissante, et l'autre d�croissante.
  \item Leur diff�rence tend vers z�ro.
  \end{dingautolist}
  
\end{Def}
\begin{theo}
\label{theo:suitadj}
  Deux suites adjacentes sont convergentes et convergent vers la m�me
  limite.
\end{theo}
Pour prouver cette propri�t� on prouve d'abord le lemme suivant:
\begin{lemme}
   Si $(u_n)$ et $(v_n)$ sont adjacentes avec $(u_n)$ croissante et
  $(v_n)$ d�croissante alors pour tout $n$, $u_n\leq v_n$
\end{lemme}
\begin{proof}
  \emph{(du lemme.)} 
On pose pour tout $n$: $w_n=u_n-v_n$. Claim: La suite $(w_n)$ est croissante
et tend vers z�ro donc est n�gative. En effet: 
\[ w_{n+1}-w_n=u_{n+1}-u_n-(v_{n+1}-v_n)=\underbrace{u_{n+1}-u_n}_{\geq
  0}+\underbrace{(v_n-v_{n+1})}_{\geq
  0}\geq 0\]
Ceci pour tout $n$ donc $(w_n)$ est croissante.
Si il existait $p\in\N$ tel que $w_p$ soit strictement positif, alors
tous les $w_n$ pour $n$ sup�rieur � $p$ seraient sup�rieurs � $w_p$,
mais cel� est impossible car $(w_n)$ tend vers z�ro donc l'intervalle
$\left]-\epsilon\,;\epsilon\right[$ o� $\epsilon=w_p$ contient tous
les $w_n$ � partir d'un certains rang. Donc c'est absurde, ainsi $w_n$
est toujours n�gatif, \emph{i.e.}  $\forall n \in\mathbb{N},$ $u_n\leq v_n$.
\end{proof}
\begin{proof} du th�or�me~\ref{theo:suitadj}. Soit $(u_n)$ et $(v_n)$  adjacentes avec $(u_n)$ croissante et
  $(v_n)$ d�croissante. D'apr�s le lemme, $\forall n \in\mathbb{N},$
  $u_n\leq v_n$. Comme $(v_n)$ est d�croissante on a donc en
  particulier que: $\forall n \in\mathbb{N},$
  $u_n\leq v_0$. Donc $u_n$ est croissante et major�e, ainsi elle
  converge vers un r�el~$\ell$. De m�me, $(v_n)$ est d�croissante et
  minor�e par $u_0$ donc elle
  converge vers un r�el~$\ell'$. Il reste � voir que $\ell=\ell'$. Or
  $\lim\limits_{n\to\infty } (u_n-v_n)=0=\ell-\ell'$. Donc $\ell=\ell'$.
  
\end{proof}
\begin{prop}
Si deux suites sont adjacentes, leurs termes donnent un encadrement de
leur limite commune. \emph{i.e.} 
  si $(u_n)$ et $(v_n)$ sont adjacentes (avec $(u_n)$ croissante et
  $(v_n)$ d�croissante) et si on note $\ell$ leur limite commune, pour
  tout $n$ on a:
\[u_n\leq \ell \leq v_n\]
\end{prop}
\begin{proof}
  Par un raisonnement identique � celui fait dans le lemme,
  $(u_n-\ell)$ est croissante et tend vers z�ro, donc est n�gative,
  donc $ \forall n \in\mathbb{N}$, $u_n\leq\ell$. De m�me $(v_n-\ell)$ est d�croissante et tend vers z�ro, donc est positive.
\end{proof}
{\bf Application:} \[\sum_{k=0}^{+\infty}\frac{1}{k!}=\e\]
On consid�re les suites:
\[S_n=\sum_{k=0}^{n}\frac{1}{k!}\quad\text{et}\quad
\sigma_n=S_n+\frac{1}{n!}\]
Prouver que les deux suites ont une limite commune dont on donnera une
valeur approch�e � $10^{-3}$ pr�s.
Suite en DM.
\section{Application: preuve du T.V.I. par dichotomie}
\label{sect:preuve.TVI}
\subsection{\'Enonc�s �quivalents au T.V.I.}
Soit $f$ une fonction continue sur un intervalle contenant deux r�els
$a$ et $b$ avec $a<b$. Soit $\lambda$ compris entre $f(a)$ et
$f(b)$. Le th�or�me des valeurs interm�diaires nous dit qu'il existe (au moins) un r�el $c$ dans $\into{a}{b}$ tel que
$f(c)=\lambda$. \ie{}
\begin{equation}
  \label{eq:c}
  \exists c\in\into{a}{b}; f(c)=\lambda
\end{equation}

Simplifions le probl�me:
\begin{enumerate}
\item On pose $g=f-\lambda$. La fonction $g$ v�rifie-t-elle les hypoth�ses du
  th�or�me?
\item Compl�ter: $f(c)=\lambda \iff g(\ldots)=\ldots$.
\item Que peut-on dire des r�els $g(a)$ et $g(b)$?
\item Justifier que le th�or�me des valeurs interm�diaires �quivaut �:
\begin{theo}
Soit $h$ une fonction continue sur un intervalle $\intf{a}{b}$ 
\[h(a)h(b)<0\implies \exists c\in\into{a}{b}; h(c)=0 \]
\end{theo}
\end{enumerate}
\subsection{D�monstration}
Ce th�or�me se prouve par dichotomie. On va \og couper l'intervalle
$\intf{a}{b}$ en deux \fg{}, voir dans quelle moiti�e $h$ change de
signe puis recommencer avec cet intervalle de longueur moiti�e,
jusqu'� isoler la racine cherch�e $c$. 
On d�finit trois suites $(a_n)$, $(b_n)$ et $(c_n)$ sur $\N$ par:


\[ a_0=a\ ; \  b_0=b\ \et{}\ \forall n\in\N,\
c_{n}=\frac{a_n+b_n}{2}\ \text{puis:}\]
\begin{list}{$\bullet$}{}
  \begin{minipage}{0.4\linewidth}
\item {\bf \text{Si:}} $h(a_n)h(c_n)<0$
\item {\bf \text{alors:}} $\left( a_{n+1}=a_n\quad \et{}\quad b_{n+1}=c_n \right)$
\item {\bf \text{sinon:}}$\left( a_{n+1}=c_n\quad \et{}\quad b_{n+1}=b_n \right)$
  \end{minipage}\hfill
  \begin{minipage}{0.5\linewidth}
    \begin{list}{}{}
    \item {\sl \ie{} $h$ change de signe sur $\into{a_n}{c_n}$}
    \item {\sl On se place alors sur l'intervalle $\intf{a_n}{c_n}$}
    \item {\sl On se place alors sur l'intervalle $\intf{c_n}{b_n}$}
    \end{list}
  \end{minipage}
\end{list}
Alors les deux suites $(a_n)$ et $(b_n)$ sont adjacentes, leur limite
commune, notons la~$c$ v�rifie: $h(c)=0$.\\

Exemple graphique pour la compr�hension. Sur ce sch�ma on a trac� une
courbe repr�sentant une fonction $h$ continue sur $\into{a}{b}$ et qui
change de signe sur $\into{a}{b}$.\\
\noindent
\begin{minipage}{9.3cm}
\begin{pspicture}*(-0.2,-4)(9,1.5)
  \psaxes[Dx=1,Ox=0,yAxis=false,labels=none,tickstyle=below](0,0)(8,0)
  \pscurve[linewidth=0.5pt,linecolor=gray]%
  (0,-1)%
  (1.5,0.6)%
  (4,-0.6)%
  (5.6,0)%
  (7,0.8)%
  (8,1)%
\put(-0.1,0.4){$a_0=a$}
\put(7.9,0.4){$b_0=b$}
\put(3.9,0.4){$c_0$}

\psline{-}(4,-2)(8,-2)  \put(3.9,-1.7){$a_1$}  \put(7.9,-1.7){$b_1$}
\psline{-}(4,-3)(6,-3)  \put(3.9,-2.7){$a_2$}  \put(5.9,-2.7){$b_2$}
\psline{-}(5,-4)(6,-4)  \put(4.9,-3.7){$a_3$}  \put(5.9,-3.7){$b_3$}


                    %\psset{xunit=0.5cm,yunit=0.5cm}
                          
                          %\psgrid% griddots=1,%
                          %subgriddiv=2,
                          %gridlabels=0pt,%
                          %xunit=1]
                          %\psaxes{->}(0,0)(0,0)(8,0)
                           %\psaxes{->}(0,0)(-1.5,-2)(1.5,1.5)
                          %\psdots*[dotscale=0.7](-1,1)
                          %\put(-1.5,1){$A$}\put(-0.4,-0.4){$O$}
                          \end{pspicture}
\end{minipage}\hfill%
\begin{minipage}{5cm}
 On a les valeurs suivantes: 
\begin{enumerate}
\item $c_0=\frac{a_0+b_0}{2}$. Bien que $h$ ait des racines sur
  $\into{a_0}{c_0}$ on a $h(a_0)h(c_0)\geq 0$ donc\\ 
$a_1=c_0$ et $b_1=b_0$
\item $c_1=\frac{a_1+b_1}{2}$. Cette fois ci, $h(a_1)h(c_1)< 0$ donc:\\
  $a_2=a_1$ et $b_2=c_1$
\end{enumerate}
\end{minipage}\\
\begin{proof} \emph{(de la convergence de la m�thode)}
  \begin{enumerate}
  \item Justifier que $(a_n)$ est croissante et $(b_n)$ est d�croissante.
  \item Que repr�sente $b_n-a_n$ ? Prouver que: $ \forall n\in\N,\ b_n-a_n=\frac{b-a}{2^n}$
  \item Prouver que $(a_n)$ et $(b_n)$ ont une limite commune que l'on
    notera~$c$.
  \item  Justifier que $h(a_n)$ et $h(b_n)$ convergent
    vers $h(c)$.
  \item On a par construction: $\forall n\in\N,\
    h(a_n)h(b_n)<0$. En d�duire que $h(c)=0$.
 
  \end{enumerate}
\end{proof}
%\newpage
\subsection{Algorithme et programmation}
On utilise la d�monstration pr�c�dente pour d�terminer num�riquement
un encadrement d'amplitude $P$ d'une racine d'une fonction. On va
calculer les diff�rents termes des suites $(a_n)$ et $(b_n)$ jusqu'� ce que la
diff�rence entre $b_n$ et $a_n$ soit inf�rieure �~$P$.  On suppose rentr�e en $Y_1$ dans le menu graph l'expression  de la fonction dont on  cherche une racine.
 Voici l'algorithme\footnote{Un algorithme (d'apr�s le nom d'\textsc{Al Kwarizmi}, p�re de
   l'alg�bre, \emph{al jabr} en arabe) est un nombre fini de
   r�gles � appliquer dans un ordre d�termin� � un nombre fini de
   donn�es, pour arriver en un nombre fini d'�tapes, � un certain
   r�sultat, et cela ind�pendamment des valeurs des donn�es.}:

\vspace{0.3cm}
\noindent
\begin{minipage}{0.6\linewidth}
%%%%%%%%%%%%%%%%%%%%%%%%%%%%%%%%%%%%%%%%%%%%%
\begin{center}
\begin{footnotesize}
\psframebox[linearc=1,cornersize=absolute]{%
  \begin{psmatrix}[rowsep=0.4,colsep=0.5]
    \psovalbox[fillstyle=solid,fillcolor=white]{$A=$? $B=$? $P=$?} \\~&&&\\
    \psframebox{$(A+B)/2\to C$} \\
    \psdiabox[fillstyle=solid,fillcolor=white]{Si $Y_1(A)\star Y_1(C)<0$} &
       \psframebox{Sinon: $C\to A$} \\%& \psframebox{Call to SP2} \\
    \psframebox{Alors: $C\to B$} \\
    ~&\\
\psdiabox[fillstyle=solid,fillcolor=white]{Si $|B-A|\leq P$} &
       \psovalbox{Sinon}& \\ \psovalbox{Alors}\\
    %\psframebox{Action 2} \\
    \psovalbox[fillstyle=solid,fillcolor=yellow]{Afficher $A$ et $B$}
    % Links
    \ncline{->}{1,1}{3,1}
    
    \ncline{->}{3,1}{4,1}
    \ncline{->}{4,1}{4,2}
\ncline{-}{4,2}{6,2} \ncline{->}{6,2}{6,1}
        \ncline{->}{4,1}{5,1}
    \ncline{->}{5,1}{7,1}
    \ncline{->}{7,1}{7,2}
    
\ncline{-}{7,2}{7,3} 
    \ncline{->}{7,1}{8,1}
    \ncline{->}{8,1}{9,1}
        \ncbar[angleA=90,armB=0,nodesepB=0]{->}{7,3}{2,1}%{2,3}
    \end{psmatrix}%
}
\end{footnotesize}
\end{center}
\end{minipage}\hfill%
\begin{minipage}{0.32\linewidth}
\begin{small}
\begin{enumerate}
\item On demande � l'utilisateur les r�els: $A$, $B$, et la pr�cision $P$ souhait�e.
\item On calcule $C$ la moyenne de $A$ et $B$.
\item On teste pour savoir sur quel intervalle la fonction change de signe.
\item On en d�duit les valeurs de $a_{n+1}$ et $b_{n+1}$ qui sont
  encore not�es $A$ et~$B$.
\item Si la pr�cision voulue est atteinte, on affiche \og{\sl La racine
  est encadr�e par:}\fg{} et on affiche $A$ et $B$,
  sinon on recommence.
\end{enumerate}
\end{small}
\end{minipage}
\vspace{0.3cm}\\
Pour demander les valeurs de $A$, $B$ et $P$ on utilise la commande
\emph{Input} ou \emph{Prompt} avec une TI et :? avec une Casio.
Le test \og Si\ldots alors\ldots sinon\fg{} se note:
If\ldots then\ldots else puis End (IfEnd avec une Casio ou EndIf avec
une grosse TI) On utilise aussi la fonction \emph{Goto n} qui renvoit
au label num�rot� $n$ (\emph{Lbl n}). Tester et comprendre le programme
suivant, ensuite taper un programme pour l'algorithme de la~dichotomie.\\[1cm]
%\vspace{0.4cm}
%\newpage
%%%%%%%%%%%%%%
\begin{minipage}{0.4\linewidth}
  \begin{center}
 {\bf Casio} \end{center}
\hrule 
\vspace{0.2cm}
{\small
$1\to I$\\
Int(6$\times$ Rand)+1$\to A$\\
$''$ Donne un entier entre 1 et 6$''$\\
Lbl 1\\
$''N''?\to N$\\
If $N=A$ \\
Then $''$Gagn� en:$''$ $I\blacktriangle$\\
Else $''$Essaye encore$''$\\
$I+1\to I$\\ 
Goto 1\\
IfEnd\\
%Stop
}%%%%%%%%%%%%%%%%%%%%%%
\end{minipage}\hfill
\begin{minipage}{0.4\linewidth}
  \begin{center}
 {\bf TI} \end{center}
\hrule \vspace{0.2cm}
{\small
$1\to I$\\
Int(6$\times$ Rand)+1$\to A$\\
$Disp\ ''$Donne un entier entre 1 et~6$''$\\
Lbl 1\\
$Prompt \ N$\\
If $N=A$ \\
Then $Disp\ ''$Gagn� en:$''$ $Disp\ I$\\
Else $Disp\ ''$Essaye encore$''$\\
$I+1\to I$\\ 
Goto 1\\
End\\
}
\end{minipage}
%%%%%%%%%%%%%%%%%%%%%%
\hrule \vspace{0.3cm} 
\noindent
\begin{small}
  \begin{minipage}{0.45\linewidth}
\noindent
On commence toujours par Shift / Var
\begin{list}{$\bullet$}{}
  \item (?) se trouve en  F4
  \item  < se trouve en F6 / F3
  \item If Then, Else,  IfEnd  dans: COM (F1) 
  \item Goto, Lbl  dans: Jump (F3)
  \item $\blacktriangle$ (pour afficher) en: F5
  \item La fl�che se trouve sur le clavier.
  \end{list}
\end{minipage}\hfill
\begin{minipage}{0.45\linewidth}
  \begin{list}{$\bullet$}{}
  \item If Then, Else, Goto, Lbl, End se trouvent dans: Catalogue / Control  
  \item Disp et Prompt se trouvent dans: Catalogue / $I/O$ (In/Out)
  \item < se trouve dans: Catalogue / Test
  \item La fl�che se trouve sur le clavier (STO comme \emph{store} ou fl�che). 
  \end{list}

\end{minipage}\\
\end{small}

\vspace{2cm} 




\begin{minipage}{0.4\linewidth}
  \begin{center}
 {\bf Casio} \end{center}
\hrule 
\vspace{0.2cm}
{\small
$1\to I$\\
Int(6$\times$ Rand)+1$\to A$\\
$''$ Donne un entier entre 1 et 6$''$\\
Lbl 1\\
$''N''?\to N$\\
If $N=A$ \\
Then $''$Gagn� en:$''$ $I\blacktriangle$\\
Else $''$Essaye encore$''$\\
$I+1\to I$\\ 
Goto 1\\
IfEnd\\
%Stop
}%%%%%%%%%%%%%%%%%%%%%%
\end{minipage}\hfill
\begin{minipage}{0.4\linewidth}
  \begin{center}
 {\bf TI} \end{center}
\hrule \vspace{0.2cm}
{\small
$1\to I$\\
Int(6$\times$ Rand)+1$\to A$\\
$Disp\ ''$Donne un entier entre 1 et~6$''$\\
Lbl 1\\
$Prompt \ N$\\
If $N=A$ \\
Then $Disp\ ''$Gagn� en:$''$ $Disp\ I$\\
Else $Disp\ ''$Essaye encore$''$\\
$I+1\to I$\\ 
Goto 1\\
End\\
}
\end{minipage}
%%%%%%%%%%%%%%%%%%%%%%
\hrule \vspace{0.3cm} 
\noindent
\begin{small}
\begin{minipage}{0.45\linewidth}
\noindent
On commence toujours par Shift / Var
\begin{list}{$\bullet$}{}
  \item (?) se trouve en  F4
  \item  < se trouve en F6 / F3
  \item If Then, Else,  IfEnd  dans: COM (F1) 
  \item Goto, Lbl  dans: Jump (F3)
  \item $\blacktriangle$ (pour afficher) en: F5
  \item La fl�che se trouve sur le clavier
  \end{list}
\end{minipage}\hfill
\begin{minipage}{0.45\linewidth}
  \begin{list}{$\bullet$}{}
  \item If Then, Else, Goto, Lbl, End se trouvent dans: Catalogue / Control  
  \item Disp et Prompt se trouvent dans: Catalogue / $I/O$ (In/Out)
  \item < se trouve dans: Catalogue / Test
  \item La fl�che se trouve sur le clavier (STO comme \emph{store} ou fl�che).
  \end{list}

\end{minipage}
\end{small}
%%%%%%%%%%%%%%%%%%%%%%%%%%%%%%%%%%%%%%%%%%%%%%%%%%%%%%%%
\newpage
\begin{center}
{\large {\bf Programme de dichotomie} }
\end{center}


\begin{minipage}{0.4\linewidth}
  \begin{center}
 {\bf Casio} \\
Pour \texttt{Y1} aller dans Graph/Yvar/Y puis taper 1
\end{center}
\hrule 
\vspace{0.2cm}
{\small
$''A''?\to A$\\
$''B''?\to B$\\
$''P''?\to P$\\
Lbl 1\\
$(A+B)/2\to C$\\
$A\to X$\\
$Y1 \to Y$\\
$C\to X$\\
$Y1 \to Z$\\
If $Y\times Z<0$\\
Then $C\to B$\\
Else \ldots\\% $C\to A$\\
IfEnd\\
If \ldots \\
Then \\
$''$LA RACINE EST ENCADREE PAR:$''$\\
$A\blacktriangle$\\
 $B\blacktriangle$\\
Else Goto 1\\
IfEnd \\
%Stop
}
\end{minipage}\hfill
\begin{minipage}{0.4\linewidth}
  \begin{center}
 {\bf TI}\\
Pour \texttt{Y}$_1$ aller dans VAR/Yvar/$Y_1$ \end{center}
\hrule \vspace{0.2cm}
{\small
Prompt $A$,$B$,$P$\\
Lbl 1\\
$(A+B)/2\to C$\\
If $Y_1(A)\times Y_1(C)<0$\\
Then $C\to B$\\
Else \ldots\\% $C\to A$\\
End\\
If \ldots \\
Then \\
Disp $''$LA RACINE EST ENCADREE PAR:$''$\\
Disp $A$, $B$\\
Else Goto 1\\
End \\
%Stop
}

\end{minipage} %Adjacentes
\chapter{Logarithme n�p�rien}
%\section{}
\section{Premi\`eres propri\'et\'es}
\label{sec:para1}

La fonction $\exp$ est continue et strictement croissante sur
$\R$. Ses limites en $-\infty$ et $+\infty$ sont respectivement $0$ et
$+\infty$. Donc d'apr�s le th�or�me des valeurs interm�diaires, pour
tout r�el strictement positif $y$, l'�quation en $x$:
\begin{equation}
  \label{eq:exy}
  \e^x=y
\end{equation}
a une unique solution que l'on note: $\ln(y)$ appel�e logarithme
n�perien de~$y$. Ainsi $\ln$ est la fonction r�ciproque de
$\exp$. Elle est d�finie sur $\into{0}{+\infty}$.
\begin{equation}
  \label{eq:def}
  \forall y\in \into{0}{+\infty},\quad \e^x=y \iff x=\ln(y)
\end{equation}
\exemples $\e^0=1$ donc $0=\ln(1)$. $\e^1=\e$ donc $1=\ln(\e)$ 
\Def{Pour tout r�el $x$ strictement positif on d�finit le logarithme
n�perien de~$x$ not� $\ln(x)$ comme �tant l'unique r�el $a$ v�rifiant:
\[ \exp(a)=x.\quad \text{On a donc: }\quad  \boxed{\exp(\ln(x))=x}\]}
\begin{prop}
On a ainsi d�fini la fonction logarithme n�perien sur
  $\into{0}{+\infty}$. $\ln$ est la fonction r�ciproque de
$\exp$. Elle  v�rifie:
\begin{eqnarray}
  \label{eq:recip}
  \forall x \in \into{0}{+\infty},\quad \e^{\ln(x)}=x\\
  \forall x \in \R, \quad \ln(\e^x)=x \label{eq:recip2}
\end{eqnarray}
\end{prop}
\begin{proof}
\eqref{eq:recip} c'est la d�finition. On prouve la suivante \eqref{eq:recip2}:
Soit un r�el $x$, on note $y=\e^x$. Alors par d�finition
$x=\ln(y)$. En rempla�ant $y$ dans cette derni�re par $\e^x$ on a le
r�sultat.
\end{proof}%, pour tout r�el $x$: $\quad \ln(\e^x)=x$}
\exercice 
Dresser le tableau de variation de la fonction $\ln$ par un raisonnement
graphique de la r�solution de $\e^x=y$. Prouver que $\ln$ est
croissante sur $\into{0}{+\infty}$.
\exercice R�soudre les �quations suivantes:
\begin{enumerate}
\item $\e^x=2$
\item $\e^{2x}=4$
\item $\e^{\e^x}=3$
\item $\ln(x)=2$
\item $\ln(x^2)=4$
\end{enumerate}
\begin{prop}\label{eqineq_ln}
Comme cons�quence de la croissance de $\ln$ sur
  $\into{0}{+\infty}$ on a pour tous r�els $a$ et $b$ strictement
  positifs:
\[\ln(a)\leq\ln(b)\iff a\leq b \quad\et{}\quad \ln(a)=\ln(b)\iff a= b\]
\end{prop}
\exemple R�soudre $\ln(x-2)\leq 1$ Sol: $ \iff x-2>0\ \et{}\ x-2\leq\e\iff
2<x\leq \e+2$

%\hfill{\stretch{1}}
%\newpage
%\begin{center}
Comme la fonction $\ln$ est la r�ciproque de la fonction exponentielle, leurs courbes sont sym�triques l'une de l'autre par rapport � la droite $\Dr$ d'�quation $y=x$\\
%\vspace{0.5cm}
\psfrag{i}{$\vec{\imath}$} \psfrag{j}{$\vec{\jmath}$} \psfrag{O}{$O$}
\psfrag{F}{$\Cr_{\exp}$}
\psfrag{L}{$\Cr_{\ln}$}
\psfrag{D}{$\Dr:\ y=x$}
\psfrag{e}{$\e$}
\begin{figure}
  \centering
  \includegraphics{/home/pan/Desktop/TeX_MiX/Figures/Metapost_fig/Courbe.5}
  \caption{Courbes de $\exp$ et $\ln$.}
  \label{fig:exp_ln}
\end{figure}
%\includegraphics{/home/pan/Desktop/TeX_MiX/Figures/Metapost_fig/Courbe.5}\\
% \vspace{3cm}
%  Comme la fonction $\ln$ est la r�ciproque de la fonction
%  exponentielle, leurs courbes sont sym�triques l'une de l'autre par
%  rapport � la droite $\Dr$ d'�quation $y=x$\\
% \vspace{0.5cm}
% \includegraphics{/Figures/Courbes/Courbe.5}
% \end{center}
% \newpage
\section{Propri�t�s alg�briques}
cf T.D. pour les preuves:
\subsection{Relation fonctionnelle}
Soient deux r�els $a$ et $b$ strictement positifs.
\begin{enumerate}
\item Simplifier $\e^{\ln(a)}$, $\e^{\ln(b)}$, $\e^{\ln(ab)}$. Ecrire
  le produit  $ab$ de deux mani�res.
\item En d�duire la relation fonctionnelle: $\boxed{\ln(ab)=
    \phantom{\sum \sum \sum \sum }}$
\item L'exponentielle transforme les sommes en produits. Que fait sa
  fonction r�ciproque?
\end{enumerate}
\subsection{Puissance}
\begin{enumerate}
\item Par r�currence on peut montrer que pour tout entier naturel $n$
  on a donc: $\boxed{\ln(a^n)=
    \phantom{\sum \sum \sum  }}$
\item En posant $a=\e^\alpha$ (donc $\alpha=\ldots$) prouver cette
  m�me propri�t� directement.
% en utilisant les propri�t�s de  l'exponentielle.

\item Etablir une formule pour $\boxed{\ln(\sqrt a)=
    \phantom{\sum \sum \sum  }}$
\end{enumerate}
\subsection{Inverse et quotients}
\begin{enumerate}
\item En calculant $\ln(a\times\frac{1}{a})$ de deux mani�res,
  compl�ter la formule: $\boxed{\ln(\frac{1}{a})=
    \phantom{\sum \sum \sum  }}$
\item En d�duire la formule: $\boxed{\ln(\frac{a}{b})=
    \phantom{\sum \sum \sum  }}$
\end{enumerate}
\subsection{Applications}
Exprimer en fonction de $\ln(2)$ uniquement les r�els suivants:% ($n\in\N$):
\begin{enumerate}
\item \qcmabc{$\ln(2^9\times \e^7)$}{$\frac12\ln(16)$}{$\ln(\frac{32}{\e})$}
\item \qcmabc{$\frac{\ln{2^n}}{\ln(2)}$}{$\e^{3\ln(2)}$}{$\ln(-2)$}
\end{enumerate}


\section{Propri�t�s analytiques}
\subsection{D�riv�e}
{\small
La courbe de la fonction $\exp$ et celle de la fonction $\ln$ sont
sym�triques par rapport � la droite d'�quation $y=x$. Ainsi
g�om�triquement il est clair que la courbe de $\ln$ admet une tangente
en tout point.% de $\R^{+*}$ 
(car $\exp$ est d�rivable sur $\R$). On
admet donc que $\ln$ est d�rivable sur  $\R^{+*}$.}
\begin{enumerate}
\item D�terminer l'expression de la d�riv�e de la fonction
  $\ln$. Rappel: $\forall x \in \R^{+*}, \e^{\ln x}=x\qquad$


\item D�terminer la formule pour la d�riv�e d'une compos�e de la forme
  $\ln(u)$ o� $u$ est une fonction d�rivable, strictement positive.
$\boxed{\ln'(x)=\phantom{\sum \sum }}\qquad$  $\boxed{\left(\ln(u)\right)'=\phantom{\sum \sum }}$
\end{enumerate}

\item {\bf Applications}
D�terminer l'ensemble de d�rivabilit� puis d�river les fonctions
suivantes:
\qcmabc{$f(x)=x\ln(x)$}{$g(x)=\frac{\ln(x)}{x}$}{$h(x)=\ln{x^3-1}$}
\subsection{R�soudre des \'equations et in�quations avec $\ln$}
On a vu un outil avec la propri�t�~\ref{eqineq_ln}. Mais cela ne
suffit pas\dots
%\item Donner le tableau de variation puis de signe de $\ln$ sur $\R^{+*}$.
 Pour �tudier le signe d'une d�riv�e qui comporte des $\ln$ on se
  ram�ne � r�soudre des in�quations du type $\ln(X)\leq a$ o�
  $a\in\R$. Compl�ter et justifier: $\boxed{\ln(X)\leq a \iff
     ( X\leq\phantom{\sum \sum } \et{}\ X>\phantom{\sum})}$  
{\bf Application.} R�soudre les in�quations suivantes:
\begin{enumerate}
\item \qcmabc{$1+\ln(x)\leq 0$}{$\ln(x-1)<3$}{$\ln(-2x+3)\geq 2$}
\item \qcmabc{$\ln(2x)-\ln(x^2)<1$}{$\ln x+\ln x^2+\ln x^3 >1$}{$\ln(x^2)-\ln(x+3)<0$}

\item \'Etudier les variations de $f$ sur $\R^{+*}$ o�: $f(x)=x^2\ln(x)$.
\end{enumerate}

\subsubsection{Formalisation}
On veut r�soudre des �quations telles que $\ln(X)=a$ ou $\ln(X)\geq
a$. On d�termine d'abord l'ensemble des nombres r�els
  pour lesquels l'in�quation a un sens, sachant que $\ln(X)$ n'est
d�fini que pour $X>0$. 
\begin{list}{$\bullet$}{}
\item \'Equation. $\ln(X)=a \iff X=\e^a$
\item In�quation. 
\end{list}


\exemples R�soudre:\\
\qcmabc{$\ln(3x-2)=-1$}{$\ln(2x-1)-\ln(x)=-1$}{$\ln(x-1)\geq -3$}\\
\qcmabc{$\ln(x-2)<5$}{$\ln(-2x+1)-\ln(x)\leq
  1$}{$\ln(-2x+1)+\ln(x)=-2$}
\section{\'Etude de la fonction $\ln$}
\subsection{Continuit�, d�riv�e}
\subsection{Limites}
\begin{list}{$\bullet$}{}
\item En $+\infty$. Soit A>0 Pour $x>\e^A$ alors $\ln(x)>A$
\item En $0$. On pose $X=\frac{1}{x}$.
\item Croissance compar�e: 
  \begin{dingautolist}{172}
  \item $\lim\limits_{ x \to \infty } \ln(x)/x=0$\quad On pose $X=\ln x$.
  \item $\lim\limits_{ x \to 0+  }  x\ln(x)=0$
  \end{dingautolist}
\end{list}

\section{Logarithmes et exponentielles de base \textit{a}}
Dans la suite, $a$ d�signe un r�el strictement positif.

\subsection{$a^x$}
\label{sec:expbasea}
\subsubsection{Definition, props}


\begin{Def}
  $\exp_a(x)=\e^{x\ln a}$. On note: $a^x=\e^{x\ln a}$
\end{Def}
\begin{rem}
Pour $a=e$ on retrouve $\e^x$ c'est pourquoi on dit parfois
l'exponentielle de base~e.
\end{rem}
\begin{rem}
  Cette notation est coh�rente, elle �largit la notation uniquement
  valable pour $x\in\mathbb{Z}$. En effet: $\e^{n\ln a}=a^n$
\end{rem}
\begin{rem}
  On donne ainsi sens � des nombres comme $10^{-7,321}$ ou pire $\pi^{\sqrt2}$.
\end{rem}
\DangerZ Ne pas confondre $a^x$ et $x^a$. Toujours partir de $\ln
(a^n)=n\ln a$ pour retrouver l'expression correcte.\\
Prop en vrac: 
\begin{dingautolist}{172}
  \item Les m�me prop alg que exp.
  \item Signe.

  \item D�rivabilit�

  \item \'Etude de la monotonie: Deux cas.
  \end{dingautolist}
Courbes � distribuer.
\exercice R�soudre $2^x=3$, $\left(\frac13\right)^x=2$, $x^3=8$, $x^{\frac12}=3$
\subsubsection{Racine �ni�me}
On peut avec cette nouvelle notion r�soudre l'eq en $x$ (solve for
$x$): $x^n=a$ o� $a\in\mathbb{R}^+_*$ et $n\in\mathbb{N}^*$
\begin{Def}
  Pour $a\in\mathbb{R}^+_*$ et $n\in\mathbb{N}^*$, il existe un unique
  r�el $x$ strictement positif tel que $x^n=a$, c'est
  $x=a^{\frac{1}{n}}$ on l'appelle la racine �ni�me (ou $n$\ieme)
  de~$a$. On note aussi: \[a^{\frac{1}{n}}=\sqrt[n]{a}\]
\end{Def}
\begin{proof} Pour $x$ et $a$ sont strictement positifs:
  \[x^n=a\iff n\ln x=\ln a\iff \ln x=\frac{1}{n}\ln a\iff
  x=\left(\e^{\ln a}\right)^{\frac{1}{n}}\iff x=a^\frac{1}{n}\]
\end{proof}
\begin{rem}
  La racine carr�e de $a$ apparait donc (enfin) comme
  $a^\frac12$. \'Ecrire avec des radicaux $x^{\frac32}$ et
  $x^{-\frac23}$ o� $x\in\mathbb{R}^+_*$
\end{rem}

\exemple R�soudre $x^3=9$. 
\exercice Application: Coefficient multiplicatif moyen. 
Un prix augmente de 10\%, puis de 20\%, puis de 30\%, puis baisse de
30\%. Quel est le pourcentage d'�volution global du prix? Quel est le
pourcentage moyen d'�volution du prix � chaque �tape? Solution:
$CM=1.2012$ augmentation de $20,21\%$. En moyenne:
$1.2012^{0.25}\simeq 1.0469$ soit � chaque �tape une augmentation moyenne de
$4,7\%$.\\
Si on a $n$ r�els strictement positifs $a_1$, \dots $a_n$, alors:
\[ a_1 \cdots a_n= \left( (a_1 \cdots a_n)^\frac{1}{n}\right)^n\]
Alors multiplier par $a_1$, puis $a_2$, puis \dots $a_n$ revient �
multiplier $n$ fois de suite par $(a_1 \cdots a_n)^\frac{1}{n}$.
\subsubsection{$x^\alpha$, g�n�ralisations}
Certains r�sultats  se g�n�ralisent\dots
Pour $\alpha$ r�el:
\paragraph{D�riv�e}
\[\frac{d}{dx}x^\alpha=\alpha x^{\alpha -1}\]
En particulier on retrouve la d�riv�e de $\sqrt x$. Exercice.
\paragraph{Limites}
\begin{dingautolist}{172}
\item Pour $\alpha$ strictement positif, $\lim\limits_{ x \to +\infty } x^\alpha=+\infty$
\item Pour $\alpha$ strictement n�gatif, $\lim\limits_{ x \to +\infty } x^\alpha=0$
\end{dingautolist}

\paragraph{Croissance compar�e pour ln}
(Pour $\alpha$ strictement positif mais surtout int�ressant pour $\alpha\in\left]0\,;1\right[$)
\begin{dingautolist}{172}
  \item $\lim\limits_{ x \to \infty } \ln(x)/x^\alpha=0$\quad On pose $X=x^\alpha$.
  \item $\lim\limits_{ x \to 0+  }  x^\alpha\ln(x)=0$
  \end{dingautolist}
\paragraph{Croissance compar�e pour exp}
(Pour $\alpha$ strictement positif)
\begin{dingautolist}{172}
  \item $\lim\limits_{ x \to +\infty } \exp(x)/x^\alpha=+\infty$
  %\item $\lim\limits_{ x \to -\infty  }  x^\alpha\exp(x)=0$ C'est FAUX!
  \end{dingautolist}
\subsection{$\log_a(x)$}
\label{sec:loga}

On veut maintenant d�finir sur $\mathbb{R}^+_*$ la r�ciproque de $\exp_a$.
\begin{Def}
  $\log_a(x)=\dfrac{\ln x}{\ln a}$
\end{Def}
\begin{rem}
  C'est la r�ciproque de la fct $\exp_a$. En effet: 
\end{rem}
\[x\in\mathbb{R}^+_*\implies \exp_a\circ
  \log_a(x)=\e^{\frac{\ln x}{\ln a}\ln a} =\e^{\ln x}=x\]
\[x\in\mathbb{R}\implies \log_a\circ\exp_a (x)=\dots=x\]
\begin{Def}
  En particulier, le logarithme d�cimal, not� log est le logarithme de
  base~10. $\log(x)=\frac{\ln(x)}{\ln 10}$. C'est la r�ciproque de
  $x\mapsto 10^x$
\end{Def}
\exemple 
\chapter{G�om�trie spatiale: plans de l'espace.}
\section{\'Equation cart�sienne de plan}
\newsavebox{\fmbox}
\newenvironment{fmpage}[1]
     {\begin{lrbox}{\fmbox}\begin{minipage}{#1}}
     {\end{minipage}\end{lrbox}\fbox{\usebox{\fmbox}}}

% Utilisation :
% \begin{fmpage}{3cm}
%    Texte � encadrer dans une bo�te ne d�passant pas 3
%    centim�tres de large.
% \end{fmpage}
%%%% fin exemple %%%%
%\newcommand{\vn}{\vec{n}\,}% mis dans pan_pan.sty
On se place dans un rep�re orthonormal $\oijk$ de l'espace.
%\section{}
%\label{sec:para1}
Soit  $\Para$ le plan passant par $A(x_A;y_A;z_A)$ et de vecteur
normal~$\vn(a;\,b;\,\,c)$. 
\[ M\in\Para\iff\V{AM}\cdot\vn =0 \]
D'o� une �quation cart�sienne de $\Para$:
$a(x-x_A)+b(y-y_A)+c(z-z_A)=0$.
%\begin{prop}%\end{prop}
%\\
%\shadowbox{
%\hrule
%\vspace{0.1cm}
\\
\noindent
\begin{fmpage}{14.8cm}
{\it
Tout plan admettant $\vn(a;\,b\,;\,c)$ comme vecteur normal a une
�quation cart�sienne de la forme \[\boxed{ax+by+cz+d=0}\]
R�ciproquement toute �quation cart�sienne de la forme $ax+by+cz+d=0$
 (avec $a$, $b$, $c$ non tous nuls) est celle d'un plan de vecteur
 normal $\vn(a;\,b\,;\,c)$.
}
\end{fmpage}
\vspace{0.2cm}
%}
%\end{prop}

\noindent
{\bf En pratique: } Pour trouver une �quation d'un plan $\Para$ passant par un
point $A(x_A;y_A;z_A)$, on cherche un
vecteur normal (ce qui donne $a$, $b$ et $c$), puis on trouve $d$ en
utilisant que: 
\[A\in\Para\iff ax_A+by_A+cz_A+d=0\]
\exercice
Soit $\Para$ le plan d'�quation: $x+2y-3z+7=0$
\begin{enumerate}
\item Donner un vecteur normal � $\Para$ ainsi qu'un point appartenant
  � $\Para$.
\item D�terminer l'�quation du plan $\mathscr{Q}$ parall�le � $\Para$,
  passant par $I(1;2;3)$.
\item Soit $\Para'$ et $\mathscr{Q}'$ d'�quations respectives : $5x-y+z+4=0\quad\et{}\quad-2x-4y+6z-1=0$
Prouver que: $\Para\bot\Para'$ et $\Para\parallele\mathscr{Q}'$
\item D�terminer une �quation du plan m�diateur du segment $[AB]$ o�:
  $A(0;-1;2)$ et $B(-2;3;4)$.
\end{enumerate}
\hrule
\vspace{0.2cm}
{\bf Distance d'un point � un plan.} La distance d'un point
$A(x_A;y_A;z_A)$ au plan $\Para$
d'�quation cart�sienne $ax+by+cz+d=0$ est not�e $d(A;\Para)$. Elle
vaut:
\[\boxed{d(A;\Para)=\frac{|ax_A+by_A+cz_A+d|}{\sqrt{a^2+b^2+c^2} }}\]
\begin{proof} $d(A;\Para)=AH$ o� $H$ est le projet� orthogonal de $A$
  sur $\Para$. Soit $B$ un point de~$\Para$. 
  %\begin{enumerate}
  %\item 
Justifier que: $\V{AB}\cdot\vn=\V{AH}\cdot\vn$ puis que:
    \[\boxed{AH=\frac{|\V{AB}\cdot\vn|}{\|\vn\|}}\]
%
%  \item 
Enfin, justifier que: $d=-(ax_B+by_B+cz_B)$ puis conclure. \end{proof}
%  \end{enumerate}\end{proof}
% $H$ est le point de $\Para$ tel que $\V{AH}$ et $\vn$  sont colin�aires. 
%
% \begin{rem}%\end{rem}
% Le d�nominateur est la norme du vecteur normal $\vn(a;\,b;\,c)$ au
% plan $\Para$.
% \end{rem}
\noindent
{\bf Demi-espace.} Toute in�quation de la forme $ax+by+cz+d\geq 0$ ou
$ax+by+cz+d\leq 0$ d�finit un demi-espace ferm� de fronti�re le plan
$\Para$ d'�quation $ax+by+cz+d=0$.
\exercice
On consid�re le plan $\Para$ d'�quation: $x-2y+3z-2=0$ et le point $A(1;3;1)$
\begin{enumerate}
\item Le point $A$ appartient-il � $\Para$? Calculer la distance $d(A;\Para)$.
\item Le plan $\Para$ est la fronti�re de deux demi-espaces. Donner
  une in�quation du demi-espace contenant le point~$A$.
\end{enumerate}
%\section{Angles et distances}
%\subsection{distance entre deux points}
%\subsection{distance d'un point � un plan}
%{\bf Distance d'un point � un plan.} 
%\begin{prop}%\end{prop}

\hrule
\vspace{0.2cm}
\noindent
{\bf Sph�re.}
%\section{\'Equation cart�sienne de sph�re}
 La sph�re $\mathscr{S}$ de centre $\Omega(x_\Omega;\,y_\Omega;\,z_\Omega)$ et de
rayon $R$ ($R\in\R^+$) a pour �quation cart�sienne:\\
$\circ$\hspace{\stretch{1}}
$\boxed{(x-x_\Omega)^2+(y-y_\Omega)^2+(z-z_\Omega)^2=R^2}$ \hspace{\stretch{1}}$\circ$
\begin{proof}
$M\in\mathscr{S}\iff \Omega M=R\iff \Omega M^2=R^2 \iff \V{\Omega M}\cdot\V{\Omega M}=R^2$
\end{proof}
\noindent
L'ensemble des points $M$ de l'espace tels que $\V{MA}\cdot\V{MB}=0$
est la sph�re de diam�tre $[AB]$.
\chapter{Int�gration.}
\section{Aire sous une courbe.}
\subsection{Aire sous la parabole.}
\begin{minipage}{9cm}
On consid�re dans un rep�re orthonorm� la courbe de la fonction carr�,
sur l'intervalle $I=\intf{0}{1}$. On veut calculer l'aire du domaine
entre la courbe et l'axe des abscisses, pour $x$ variant de $0$
�~$1$. Le principe de la m�thode expos�e �tait d�ja connu d'Archim�de,
et a donn� lieu bien plus tard (Newton, Leibniz puis Riemann) � la
th�orie de l'int�gration.\\
  \noindent
{\bf Principe:} On se donne un $n\in\N^*$. On d�coupe l'intervalle $I$ avec une
\emph{subdivision} de $I$ en $n$ intervalles de longueur
$\frac{1}{n}$. On consid�re alors (cf figure pour n=?) les aires de
deux suites de rectangles, de mani�re � obtenir un encadrement de
l'aire cherch�e. On calcule les deux aires, puis on fait tendre $n$
vers l'infini.
\end{minipage}\hfill%
\begin{minipage}{5.5cm}
  \begin{center}
\includegraphics[width=5cm]{/home/pan/Desktop/TeX_MiX/Figures/Courbes/integrale_carre}    
  \end{center}
\end{minipage}
\vspace{0.3cm}
\\
\noindent
Compl�ter la figure en notant les abscisses des sommets des rectangles
qui sont sur l'axe des abscisses.
On note $a_n$ l'aire des rectangles qui sont sous la courbe et
  $A_n$ celle des rectangles qui vont de l'axe des abscisses jusqu'au
  dessus de la courbe. 
\begin{enumerate}
\item Calcul de $a_n$:%
  \begin{enumerate}
  \item L'aire du premier rectangle sous la courbe est nulle, celle du second est
    $\frac{1}{n}\times
    \left(\frac{1}{n}\right)^2=\frac{1}{n^3}$. 
%Combien y a-t-il de   rectangles sous la courbe?
    \item \'Ecrire sans, puis avec symbole sigma l'expression de $a_n$.
  \end{enumerate}
\item Proc�der de m�me pour donner l'expression de $A_n$ en fonction
  d'une somme de carr�s d'entiers.

\item Calculer et donner la limite de $A_n-a_n$.
\item En utilisant la formule: $\displaystyle
  \sum_{k=1}^{n}k^2=\frac{n(n+1)(2n+1)}{6}$ prouver que $A_n$ et $a_n$
  ont une limite commune, et en d�duire l'aire cherch�e que l'on note:
\[\int_0^1 x^2 {\rm d}x\]
\end{enumerate}



\subsection{Fonction aire.}
\begin{minipage}{6.5cm}
  On �tudie un cas un peu plus g�n�ral. On consid�re une
  fonction $f$ d�croissante et continue sur un intervalle . On note
  $C_f$ sa courbe. Sur la
  figure, $f$ est d�croissante sur $\R^+$. On appelle $A(x)$ l'aire entre la
  courbe et l'axe des abscisses pour les abscisses variant de 0 �
  $x$. C'est le domaine gris�. 
\end{minipage}\hfill%
\begin{minipage}{7.3cm}
  \begin{center}
    %\includegraphics[width=7.1cm]{/home/pan/Desktop/TeX_MiX/Figures/Courbes/courbe_integrale} %[width=]
\includegraphics[width=7.1cm]{/home/pan/Desktop/TeX_MiX/Figures/Metapost_fig/Integrale.1} %
  \end{center}
\end{minipage}\\
\begin{enumerate}
\item Donner en utilisant les rectangles
  trac�s un encadrement pour $h>0$ de la diff�rence: 
\[A(x+h)-A(x)\]
en fonction de $f$.
\item En d�duire que la fonction $A$ est d�rivable en $x$, et donner
  l'expression de sa d�riv�e.

\item Combien vaut $A(0)$?
\item Peut-on obtenir le m�me r�sultat si $f$ est croissante?
\item {\bf Application} 
  \begin{enumerate}
  \item V�rifier le calcul de la premi�re
  partie: Trouver une fonction $A$ dont la d�riv�e a pour expression:
  $x^2$ et telle que $A(0)=0$
\item D�terminer l'aire sous la parabole de la fonction carr� mais pour $x$
variant entre $0$ et~$2$.
\item V�rifier qu'avec cette m�thode on retrouve bien l'aire du
  rectangle de c�t�s $a$ et $b$.
  \end{enumerate}
\end{enumerate}

%\subsection{Primitive}
%D�terminer l'aire sous la parabole de la fonction carr� mais pour $x$
%variant entre $-1$ et~$2$. 

\section{Int�grale d'une fonction continue et positive.} %et positive}
Dans cette partie, $f$ est une fonction continue \textbf{positive} sur un
intervalle~$\left[a\,;b\right]$. On se place dans un rep�re orthogonal
$\oij$, on note $\mathscr{C}$ la courbe de~$f$ dans ce rep�re.
\paragraph{Unit� d'aire.}
 On appelle unit� d'aire
(u.a.) l'aire d'un rectangle $ABCD$ o� $\overrightarrow{AB}=\vi$ et
$\overrightarrow{AD}=\vj$. Par exemple si les unit�s choisies sont de
2~cm en abscisse et 3~cm en ordonn�e pour une unit�, alors: $1\text{u.a.}=6~\text{cm}^2$ (car $=2\times 3=6$).% petit dessin.\\


\subsection{Int�grale vue comme une aire.}

%\paragraph{ Int�grale d'une fct cont positive.}

\begin{Def} 
%Soit $f$ continue et positive sur un intervalle $[a;b]$ et $\Cr$ sa courbe dans $\oij$. 
L'int�grale $\int_a^b f(x) {\rm d}x$ est l'aire
  du domaine d�limit� par $\Cr$, $(Ox)$ et les droites d'�quations $x=a$ et $x=b$.
\end{Def}
\begin{rem} On a vu que l'aire s'obtient comme somme d'aires de rectangles de
largeur infinit�simale $\ud x$ et de hauteur $f(x)$ ce qui explique la
notation due � Leibniz (1646-1716). Un peu de vocabulaire:
  \begin{dingautolist}{172}
    \item  $\int$ comme somme. 
    \item La variable $x$ est muette, on met ce qu'on veut: $\int_a^b f(x) {\rm d}x=\int_a^b f(t) {\rm d}t$
    \item On lit int�grale (ou somme) de $a$ � $b$ de $f(x)$ d$x$.
    \item $a$ et $b$ sont les \emph{bornes} de l'int�grale. (\textsl{Lower
      bound et upper bound in english})
       \end{dingautolist}
    
\end{rem}

\exemple $\int_{-2}^{0}3 \ud x=6$ (rectangle) et $\int_{1}^{3} t \ud
t=4$ (trap�ze)

\paragraph{Valeur approch�e d'int�grale.}

Il suffit de d�composer l'aire cherch�e en rectangles, triangles et ou
trap�zes pour obtenir une valeur approch�e voire un encadrement de
l'int�grale cherch�e. 
La calculatrice en mode \texttt{G-solve} permet d'obtenir
graphiquement une valeur approch�e de l'int�grale et un joli coloriage.
L'unit� est l'unit� d'aire (u.a.) Si on veut l'aire en ${\rm cm^2}$ il
faut calculer combien de ${\rm cm^2}$ mesure~1~u.a.



\subsection{Fonction aire, primitives.}
On a vu en activit� que si $f$ est continue, positive et monotone sur
un intervalle $\left[a\,;b \right]$, alors la fonction:
  \begin{equation}
    \label{eq:fctaire}
    F:\ x \longmapsto \int_{a}^{x} f(t) \ud t
  \end{equation}

est une fonction d�rivable sur $\left[a\,;b\right]$ telle que: $F'=f$
et $F(a)=0$. $F(x)$ est l'aire (en u.a.) du domaine compris entre
$\Cr$ et $(Ox)$ pour les abscisses allant de $a$ �~$x$. $F$ est une
fonction croissante (puisque ici $f$ est positive) et postive.

\begin{Def} Soit $f$ une fonction continue sur un intervalle $I$. On appelle
  primitive de $f$ sur $I$ une fonction $F$ d�rivable sur $I$ telle
  que: $F'=f$
\end{Def}
Une fonction n'a pas une unique primitive, on dira bien \og \emph{une}
primitive de~$f$\fg{}. Cependant:
\begin{prop}\label{prop:primetcst}
  Si $F$ et $G$ sont deux primitives de $f$ alors elles diff�rent d'une constante.
\end{prop}

\begin{proof}
  $(F-G)'=f-f=0$ donc $(F-G)$ est une constante.
\end{proof}
Ainsi on pourra dire \og \emph{la} primitive de $f$ qui vaut \dots{} en
\dots{}~\fg{}. Par exemple:
\begin{prop}
  [Unicit�] Soit $f$  continue positive sur un intervalle $\intf{a}{b}$
  alors $f$ admet une unique primitive qui s'annule en $a$, c'est la
  fonction \og aire\fg{} d�finie en~\eqref{eq:fctaire}.
%\[  F(x)=\int_a^x f(t)\ud t\]
\end{prop}
\begin{proof}
 On a prouv� l'existence dans le cas d'une fonction monotone en TD. On l'admet
 dans le cas g�n�ral. La propri�t�~\ref{prop:primetcst} nous garantit l'unicit�.
\end{proof}
%Ainsi, la fonction \og aire\fg{} d�finie par~\eqref{eq:fctaire} est la primitive de $f$ s'annulant en $a$.
\begin{rem} La condition de continuit� est suffisante mais pas
  n�cessaire pour assurer l'existence d'une primitive. Penser � la
  fonction partie enti�re. On peut facilement calculer l'aire sous sa
  courbe, en additionnant des aires de rectangles, et pourtant elle
  n'est pas continue.% \dots
\end{rem}



\subsection{Calcul d'une int�grale � l'aide d'une primitive.}
\begin{prop}\label{prop:intetprim}
  Soit $f$  continue positive sur un intervalle $\intf{a}{b}$ et $F$
  une primitive de $f$ sur $\left[a\,;b\right]$.
  alors:  
\[\int_a^b f(x)\ud x=F(b)-F(a)\]
\end{prop}
\begin{proof}
  Soit $F$  la primitive qui s'annule en $a$, alors le r�sultat est imm�diat. Il faut
  voir que ce calcul ne d�pend pas de la primitive choisie
  pour~$f$:\\
 Si $G$ est une primitive de $f$, alors il existe une
  constante r�elle~$k$ telle que $G=F+k$. Alors
  $G(b)-G(a)=(F(b)+k)-(F(a)+k)=F(b)-F(a).$ 
\end{proof}
\noindent\textbf{Notation: } On note dans les calculs $\left[F(x)\right]_a^b$ pour $F(b)-F(a)$.
\paragraph{Relation de Chasles.}
D'apr�s la notion intuitive d'aire et les d�coupages que l'on peut
faire d'une surface, on a: 
\begin{prop}[Relation de Chasles]
  Soit $c$ un r�el compris entre $a$ et $b$. \end{prop}

\noindent%
\begin{minipage}{1.0\linewidth - 7cm}
\[\int_a^b f(x) {\rm d}x=\int_a^c f(x) {\rm d}x+\int_c^b f(x) {\rm d}x\]
  \begin{proof}
    Soit $F$ une primitive de~$f$. 
\[ (F(c)-F(a))+(F(b)-F(c))=F(b)-F(a) \qedhere\]
  \end{proof}
%On remarque que ce calcul est encore valable si
\end{minipage}\hfill
\begin{minipage}{7cm}

\begin{center}
\includegraphics{/home/pan/Desktop/TeX_MiX/Figures/Metapost_fig/Integrale.2}  
\end{center}
\end{minipage}
\subsection{Calcul de primitives.}
On calcule des primitives de $f$ en reconnaissant en $f$ l'expression d'une
d�riv�e (facile � dire!). On s'aide des propri�t�s de lin�arit� de la d�riv�e pour
obtenir les bons coefficients. En effet:
\begin{prop}[Lin�arit�]\label{prop:linprim}
  Si $f$ et $g$ sont deux fonctions admettant $F$ et $G$
  respectivement comme primitives sur un intervalle, alors pour tous
  r�els~$\lambda$ et $\mu$, on a: $(\lambda F+\mu G)$ qui est une
  primitive de $(\lambda f+\mu g)$.
\end{prop}
\begin{proof}
  $(\lambda F+\mu G)'=(\lambda F)'+(\mu G)'=\lambda F'+\mu G'=\lambda f+\mu g$
\end{proof}
\exemple On cherche une primitive sur $\mathbb{R}$ de $f$ o�
$f(x)=\sin(2x)+x^3$\\
On remarque que: $f(x)=-\frac12 \times (-2\sin(2x))+\frac14 \times (4
x^3)$\\
Donc une primitive de $f$ a pour expression: $F(x)=-\frac12
\cos(2x)+\frac14 x^4$.\\
Si on demande \emph{la} primitive de $f$ s'annulant en~0, alors on
�crit:\\
Les primitives de $f$ ont pour expression: $F(x)=-\frac12
\cos(2x)+\frac14 x^4+k$ o� $k$ est une constante. Alors
$F(0)=-\frac12+k$, ainsi \emph{la} primitive de $f$ s'annulant en~0 a
pour expression: $F(x)=-\frac12
\cos(2x)+\frac14 x^4+\frac12$

\subsection{Existence du logarithme n�p�rien.}
\label{sec:existln}

La fonction inverse est continue et positive sur
$\left]0\,;+\infty\right[$, elle admet donc une unique primitive
s'annulant en~1 qui a pour expression: $\int_1^x \frac{1}{t} {\rm d}t$
ceci pour tout~$x$ de~$\left]0\,;+\infty\right[$. Cela prouve
l'existence du logarithme n�p�rien, donc de sa fonction r�ciproque, la
fonction exponentielle. L'existence de exp avait admise en d�but d'ann�e.
\section{G�n�ralisation de l'int�grale � l'aide d'une primitive.}

\subsection{Int�grale d'une fonction de signe quelconque.}

On admet que toute fonction continue sur un intervalle admet une
primitive, et on d�finit la notion d'int�grale de~$f$ � partir de:
\begin{Def}
 Soit $f$  continue sur un intervalle $\left[a\,;b\right]$ et $F$
  une primitive de $f$ sur $\left[a\,;b\right]$.
  alors:  
\[\boxed{\int_a^b f(x)\ud x= \left[F(x)\right]_a^b= F(b)-F(a)}\] 
\end{Def}
\begin{rem}
  C'est la m�me chose que ce qu'on a �nonc� pour la
  propri�t�~\ref{prop:intetprim}, mais on a ot� la condition de
  positivit� de~$f$. C'est maintenant notre d�finition de
  l'int�grale d'une fonction. \'Evidemment, les deux d�finitions
  co�ncident lorsque $f$ est positive.
\end{rem}

\paragraph{Lin�arit� de l'int�grale.}
D'apr�s la propri�t�~\ref{prop:linprim} on en d�duit:
\begin{prop} Si $f$ et $g$ sont deux fonctions continues sur
  $\left[a\,;b\right]$ et $\lambda$, $\mu$ deux r�els, alors:
\[ \int_a^b \lambda f(x)+\mu g(x)\ud x=\lambda \int_a^b  f(x)\ud x + \mu \int_a^b  g(x)\ud x\]  
\end{prop}

\subsection{Int�grale et signes.}
$\int_a^b f$ et $\int_b^a f$ est-ce la m�me chose? Et bien non, en
observant la d�finition, on voit qu'�changer les bornes revient �
changer le signe de l'int�grale.
\[ \int_b^a f(x)\ud x=-\int_a^b f(x)\ud x\]
%La relation de Chasles est encore valable quelque soit l'ordre. On en d�duit que
 Si $f$ est n�gative, qu'est-ce que cela change? Par lin�arit� on a en
 particulier:
\[ \int_a^b -f(x)\ud x=-\int_a^b f(x)\ud x\]
Donc si $f$ est positive sur $\left[a\,;b\right]$, alors $-f$ est
n�gative et son int�grale est l'oppos�e de l'int�grale de~$f$. 
\begin{prop}[Aire alg�brique] Pour une fonction $f$ continue, de signe
  quelconque sur $\left[a\,;b\right]$, $\displaystyle \int_a^b f(x)\ud
  x$ repr�sente \emph{l'aire alg�brique} du domaine compris entre
  $\Cr$, $(Ox)$ et les droites d'�quations $x=a$ et $x=b$. On compte
  positivement l'aire lorsque $f$ est positive et n�gativement lorsque
  $f$ est n�gative.
  
\end{prop}
\exemple Calculer $\int_0^{2\pi} \sin(x)\ud x$, interpr�ter
g�om�triquement. Quelle est r�ellement l'aire du domaine gris�? 
 
\begin{prop}[Int�grer une in�galit�] Si $f$ et $g$ sont deux fonctions continues
  sur $\left[a\,;b\right]$, telles que: \\
$ \forall x \in\left[a\,;b\right],
  f(x)\geq g(x)$ alors:
\[\int_a^b f(x)\ud x\geq \int_a^b g(x)\ud x\]
  
\end{prop}
\begin{proof}
  Par hypoth�se, la fonction $f-g$ est positive, donc son int�grale
  (vue comme une aire est positive, puis par lin�arit� de l'int�grale,
  on a le r�sultat. 
\end{proof}
\subsection{Formule de la moyenne}
\begin{Def}
  On appelle moyenne de la fonction $f$ sur $\left[a\,;b\right]$ le
  r�el:
$\displaystyle \bar{f}=\frac{1}{b-a}\int_a^b f(x) \ud x$
\end{Def}
\noindent
\begin{minipage}{1.0\linewidth - 7cm}
Pour faire une moyenne de $n$ r�els, on fait leur somme, et on divise
par le nombre $n$ de r�els. Ici pour faire la moyenne des $f(x)$, on
fait leur \og somme\fg{} et on divise par la longueur de l'intervalle
dans lequel varie~$x$. La figure donne une interpr�tation graphique
dans le cas o� $f$ est positive: l'aire du rectangle gris est la m�me que
l'int�grale de $f$ de $a$ �~$b$. 
\end{minipage}\hfill
\begin{minipage}{7cm}

\begin{center}
\includegraphics{/home/pan/Desktop/TeX_MiX/Figures/Metapost_fig/Integrale.3}  
\end{center}
\end{minipage}
\begin{prop}[In�galit�s de la moyenne] Si $f$ admet un minimum $m$ et
  un maximum $M$ sur $\left[a\,;b\right]$, alors on a l'encadrement:
\[ m(b-a) \leq \int_a^b f(x) \ud x\leq M(b-a)\]
Autrement dit la moyenne est comprise entre le minimum et le maximum
de~$f$:\quad $m\leq\bar{f}\leq M$

  
\end{prop}
\subsection{Int�gration par parties.}
\label{sec:IPP}
\begin{prop}[I.P.P.] Si $u$ et $v$ sont deux fonctions d�rivables sur
  un intervalle $\left[a\,;b\right]$, alors la fonction $u'v$ est
  int�grable sur $\left[a\,;b\right]$ et on a la formule:
\[\int_a^b u'(x)v(x) \ud x=\]
  
\end{prop}
\subsection{Aire entre deux courbes}
\noindent Un exemple: D�terminer l'aire en cm$^2$ du domaine gris�
entre les deux paraboles dont l'�quation est donn�e. Les unit�s choisies ont �t�: 1~cm
en abscisse et 0,8~cm en ordonn�e.
\begin{center}
\includegraphics{/home/pan/Desktop/TeX_MiX/Figures/Metapost_fig/Integrale.4}\\

\end{center}
\newpage
\noindent%
\begin{minipage}[t]{7cm}
\begin{center}
\includegraphics{/home/pan/Desktop/TeX_MiX/Figures/Metapost_fig/Integrale.2}\\
Relation de Chasles pour les int�grales.  
\end{center}
\end{minipage}\hfill%
\begin{minipage}[t]{7cm}
\begin{center}
\includegraphics{/home/pan/Desktop/TeX_MiX/Figures/Metapost_fig/Integrale.3}\\
Moyenne d'une fonction sur un intervalle~$\left[a\,;b\right]$.  
\end{center}
\end{minipage}\\[2.5cm]
\noindent%
\begin{minipage}[t]{7cm}
\begin{center}
\includegraphics{/home/pan/Desktop/TeX_MiX/Figures/Metapost_fig/Integrale.2}\\
Relation de Chasles pour les int�grales.  
\end{center}
\end{minipage}\hfill%
\begin{minipage}[t]{7cm}
\begin{center}
\includegraphics{/home/pan/Desktop/TeX_MiX/Figures/Metapost_fig/Integrale.3}\\
Moyenne d'une fonction sur un intervalle~$\left[a\,;b\right]$.  
\end{center}
\end{minipage}\\[2.5cm]
\noindent%
\begin{minipage}[t]{7cm}
\begin{center}
\includegraphics{/home/pan/Desktop/TeX_MiX/Figures/Metapost_fig/Integrale.2}\\
Relation de Chasles pour les int�grales.  
\end{center}
\end{minipage}\hfill%
\begin{minipage}[t]{7cm}
\begin{center}
\includegraphics{/home/pan/Desktop/TeX_MiX/Figures/Metapost_fig/Integrale.3}\\
Moyenne d'une fonction sur un intervalle~$\left[a\,;b\right]$.  
\end{center}
\end{minipage}\\[2.5cm]
\noindent%
\begin{minipage}[t]{7cm}
\begin{center}
\includegraphics{/home/pan/Desktop/TeX_MiX/Figures/Metapost_fig/Integrale.2}\\
Relation de Chasles pour les int�grales.  
\end{center}
\end{minipage}\hfill%
\begin{minipage}[t]{7cm}
\begin{center}
\includegraphics{/home/pan/Desktop/TeX_MiX/Figures/Metapost_fig/Integrale.3}\\
Moyenne d'une fonction sur un intervalle~$\left[a\,;b\right]$.  
\end{center}
\end{minipage}



% \section{Extension � une fonction continue de signe quelconque }
% Par convention, si f est n�gative...



% \subsection{Primitive d'une fonction continue}
% \begin{Def} Soit $f$ une fonction continue sur un intervalle $I$. On appelle
%   primitive de $f$ sur $I$ une fonction $F$ d�rivable sur $I$ telle
%   que: $F'=f$
% \end{Def}
% \exemple $(x^3)'=3x^2$ donc une primitive de $3x^2$ est $x^3$.



% \subsection{G�n�ralisation de l'int�grale � l'aide de primitive}
% Ici on se d�barrasse de la condition de positivit� et de l'ordre de
% $a$ et $b$:
% \begin{Def} $f$ cont sur $I$ contenant $a$ et $b$, $F$ une prim de $f$ alors:
% \[ \int_a^b f(x)\ud x=F(b)-F(a)\]
% \end{Def}

\chapter{Lois de probabilit�s }%et d�nombrement}

%\section{Probabilit�s conditionnelles}
\section{Combinatoire}
{\Large \`A faire plus court}
\subsection{D�nombrer des listes}
Dans ce paragraphe, $E$ d�signe un ensemble � $n$ �l�ments ($n\in
\N^*$). On imagine l'existence d'une urne $U$ contenant $n$ jetons sur
lequels sont inscrits les $n$ �l�ments de $E$. 
\begin{Def} On appelle \emph{liste} de $p$ �l�ments de $E$ une �num�ration
  {\bf ordonn�e} de ces $p$ �l�ments. On la note comme des coordonn�es
  de points.
\end{Def}
\exemple Si $a$, $b$, $c$ sont trois �l�ments de $E$, $(a\,;b\,;c)$ et
$(a\,;c\,;b)$ sont deux listes distinctes avec les trois �l�ments $a$,
$b$ et $c$.
Dans ce paragraphe on s'int�resse donc � d�nombrer dans des situations
o� l'{\bf ordre} des �l�ments compte.
\subsubsection{Permutations d'un ensemble}
% On veut r�pondre aux questions du type:
% \begin{list}{$\bullet$}{}
% \item Combien peut-on �crire de mots distincts avec les lettres A, B, C?
% \item Combien y a-t-il de quint�s avec les chevaux 1, 2, 3, 4, 5?
% \item Combien y a-t-il de listes des $n$ �l�ments de $E$?
% \end{list}
Le mod�le est celui du tirage al�atoire dans l'urne $U$ des $n$
jetons, sans remise, et en notant l'ordre de sortie.
\begin{Def} On
  appelle \emph{permutation} de $E$ une liste de ses $n$ �l�ments.
\end{Def}
\exemple Si $E=\left\{A,\,B,\, C\right\}$ il y a six permutations de
$E$ et donc six mots distincts avec les trois lettres $A$, $B$ et $C$. On les
trouve avec un arbre en distinguant les choix pour le 1\ier{}, le 2\ieme{} et
le 3\ieme{} �l�ment: ABC, ACB, BAC, BCA, CAB, CBA.
\begin{Def} Soit $n\in\N^*$. On note $n!$ l'entier appell�
  \emph{factorielle $n$} d�fini par: $n!=1\times 2\times\cdots\times
  n$.
Par convention on pose $0!=1$
\end{Def}
\begin{prop}
Il y a $n!$ \emph{permutations} des $n$ �l�ments de $E$.
\end{prop}
\begin{proof} On le prouve avec un arbre ou en imaginant remplir des
  cases: Il il a $n$ choix pour le 1\ier{} �l�ment, et pour chacun de
  ces choix il y a $n-1$ choix pour le 2\ieme{} �l�ment \ldots \emph{etc}
  puis plus qu'un choix pour le dernier �l�ment. On trouve donc:
  $n\times(n-1)\times\cdots\times 1=n!$
\end{proof}
\exemple Cinq chevaux font la course. Combien y a-t-il d'arriv�es
possibles (on suppose qu'il ne peut pas y avoir
d'\emph{ex-\ae{}quo}). Une arriv�e est une permutation de l'ensemble
des cinq chevaux. Il y a donc $5!=120$ quint�s possibles avec 5 chevaux donn�s.

 
\subsubsection{Liste sans r�p�tition de $p$ �l�ments parmi $n$}
Le mod�le est celui du tirage al�atoire dans l'urne $U$ de $p$
jetons, {\bf sans remise}, et en notant l'ordre de sortie. O� $p$ est entier,
$1\leq p\leq n$.
\begin{Def} Une \emph{liste sans r�p�tition} de $p$ �l�ments de $E$ est une
  liste o� les $p$ �l�ments sont deux � deux distincts.
\end{Def}
\begin{prop}
Il y a $\dfrac{n!}{(n-p)!}=n\times(n-1)\times\cdots\times (n-(p-1))$  listes sans r�p�tition de $p$ �l�ments de $E$.
\end{prop}
\begin{proof} On le prouve avec un arbre ou en imaginant remplir des
  cases: Il il a $n$ choix pour le 1\ier{} �l�ment, et pour chacun de
  ces choix il y a $n-1$ choix pour le 2\ieme{} �l�ment \ldots \emph{etc}
  puis $(n-(p-1))$ choix pour le $p$\ieme{} et dernier �l�ment.
\end{proof}
\exemple Combien y a-t-il de tierc�s possibles avec $10$ chevaux au
d�part? Ici $E$ est l'esemble des 10 chevaux. Un tierc� est une liste
sans r�p�tition de trois de ces chevaux. Il y a donc: $\dfrac{10!}{7!}=10\times 9\times 8=720$ tierc�s
possibles avec $10$ chevaux au d�part.
\subsubsection{Liste avec r�p�tition de $p$ �l�ments parmi $n$}
Le mod�le est celui du tirage al�atoire dans l'urne $U$ de $p$
jetons, {\bf avec remise}, et en notant l'ordre de sortie. Ici
$p\in\N$. On peut donc avoir $p\geq n$.
\begin{Def} Une \emph{liste avec r�p�tition} de $p$ �l�ments de $E$ est une
  liste o� les $p$ �l�ments ne sont pas n�cessairement deux � deux distincts.
\end{Def}
\og{} Avec r�p�tition \fg{} est donc � comprendre au sens o� \og il peut y
avoir r�p�titon\fg{}. Remarquez que si $p>n$ il y a n�c�ssairement
r�p�tition. (C'est le principe des tiroirs: Si il y a plus de
chaussettes que de tiroirs, il y a au moins un tiroir qui comporte au
moins deux chaussettes.)
\begin{prop}
Il y a $n^p$ listes avec r�p�tition de $p$ �l�ments de $E$.
\end{prop}
\exemple Combien y a-t-il de points de l'espace dont les coordonn�es sont $-1$
ou $1$? Ici $p=3$ et $n=2$ car $E=\{-1;\,1\}$. Il y a deux choix pour
l'abscisse, deux pour l'ordonn�e et deux aussi pour la cote. Soit
$2^3=8$ points en tout. Ce sont les coordonn�es des sommets d'un cube. 

\subsection{Combinaisons}
Maintenant on ne d�nombrera plus des listes mais des sous ensembles de
$E$, l'ordre des �l�ments ne comptera pas. $E$ d�signe encore un ensemble � $n$ �l�ments ($n\in
\N^*$) et $p$ d�signe un entier compris entre $0$ et $n$.\\
\noindent Le mod�le est celui du tirage al�atoire dans l'urne $U$ de $p$
jetons, {\bf sans remise}, et sans tenir compte de l'ordre de sortie.
\subsubsection{Choisir $p$ �l�ments parmi $n$}
\begin{Def} Une \emph{combinaison} de $p$ �l�ments de $E$ est un sous
  ensemble (ou partie) de $E$ qui comporte $p$ �l�ments.
\end{Def}
\begin{Def} Le nombre de combinaisons de $p$ �l�ments d'un  ensemble �
  $n$ �l�ments est not�:$\binom{n}{p}$ et se lit \og{}$p$ parmi $n$\fg{}. (En France
  on le  notait autrefois $\mathrm{C}_n^p$)
\end{Def}

\exemple Si $E=\left\{a\,;\,b\,;\,c\right\}$, alors:
  \begin{list}{$\bullet$}{}
  \item $\binom{3}{3}=1$. Il y a une combinaison de  trois �l�ments de $E$, c'est $E$ lui m�me.
  \item $\binom{3}{2}=3$. Il y a trois combinaisons de deux �l�ments de $E$:
    $\left\{a\,;\,b\right\}$; $\left\{a\,;\,c\right\}$; $\left\{b\,;\,c\right\}$
  \item $\binom{3}{1}=3$. Il y a trois combinaisons d'un �l�ment de $E$:
    $\left\{a\right\}$; $\left\{b\right\}$; $\left\{c\right\}$

  \item $\binom{3}{0}=1$. Il y a une combinaison de 0 �l�ment de $E$ c'est l'ensemble vide: $\emptyset$
  \end{list}
{\bf Calculette.} Pour calculer $\binom{5}{3}$ � la calculette, utiliser la fonction not�e nCr (5 nCr 3 {\sc Exe}). Menu Proba (TI:
Math/Prb, Casio: Optn/Prob). 
\subsubsection{Nombre de combinaisons}

\begin{prop}
$\displaystyle \boxed{\binom{n}{p}=\frac{n!}{p!(n-p)!}}$
\end{prop}
\begin{proof} On sait qu'il y a  $\frac{n!}{(n-p)!}$ listes sans
  r�p�tition de $p$ �l�ments de $E$. Quand on a un ensemble � $p$
  �l�ments il y a $p!$ permutations, soit $p!$ listes sans r�p�tition
  avec ces $p$ �l�ments. C'est pourquoi pour obtenir le nombre de
  parties (\ie{} de combinaisons) de $E$ � $p$ �l�ments on doit
  diviser $\frac{n!}{(n-p)!}$ par $p!$
\end{proof}
\exemple Combien y a-t-il de mains diff�rentes au bridge? (On joue �
quatre au bridge avec un jeu de 52 cartes que l'on distribue
enti�rement). On doit donc compter le nombre de fa�ons de choisir 13
cartes parmi 52, soit $\binom{52}{13}=\frac{52!}{13!(39)!}=1905040678800\approx 1,9.10^{12}$
%%%
\section{Loi de probabilit�s discr�tes}
\subsection{D�finitions}
Soit $\Omega$ un univers fini � $n$ �l�ments ($n\in\N$). On note
$(\omega_i)_{1\leq i\leq n}$ les $n$ �v�nements �l�mentaires qui
constituent $\Omega$. Donc $\Omega=\{\omega_1;\,\omega_2;\ldots ;\omega_n\,\}$
\begin{Def}
  On appelle \emph{variable al�atoire} sur $\Omega$ toute fonction $X$ qui va de
  $\Omega$ dans $\R$. Pour chaque $i=1,2,\ldots, n$ on note
  $x_i=X(\omega_i)$.
  \begin{eqnarray}
    X:&&\Omega \to \R\notag\\
       &&\omega_i \mapsto X(\omega_i)=x_i \notag
  \end{eqnarray}
\end{Def}
L'ensemble des $x_i$ est donc l'ensemble des valeurs possibles pour
$X$. 
\exemple {\sl Une urne contient 3 jetons noirs et 2 jetons blancs. On tire
sans remise deux jetons dans une urne. On note $X$ le nombre de jetons
noirs tir�s.}\\
On peut choisir: $\Omega=\{ \{N;N \};\{N;B \};\{B;B \} \}$\\
 Alors:
$X\left(\{N;N \}\right)=2$ ; $X(\{N;B \})=1$ ; $X(\{B;B\})=0$. \\
On peut aussi choisir: $\Omega'=\{ (N;N);(N;B);(B;N);(B;B) \}$. On
aurait alors: \\
 $X((N;B))=X((B;N))=1$\ldots

\begin{Def} On appelle \emph{loi de probabilit�} d'une variable al�atoire $X$
  la donn�e pour chaque valeur $x_i$ possible pour $X$ de
  $p_i=P(X=x_i)$. On la pr�sente en g�n�ral sous forme de tableau.
 
\end{Def}
 Ce tableau est � rapprocher de ceux qu'on peut dresser en
 statistiques lorsqu'on a une s�rie de valeurs et leur fr�quence
 d'aparition.
 \begin{prop}
   La somme des $p_i=P(X=x_i)$ vaut 1.
 \end{prop}
Cela vient de ce que les \og$X=x_i$\fg{} forment une partition de $\Omega$.
\exemple On reprend l'exemple pr�c�dent. On dresse un arbre 
\verb#(Revoir le code pstricks)#

% \psset{radius=6pt,dotsize=4pt,treefit=loose}
% \begin{center}
% \pstree[treemode=R]
%     {\Tcircle{0}}
%     {\pstree{\Tcircle{N}^{$\frac{3}{5}$}}%_{$\frac{1}{2}$}}
%             {\pstree{\Tcircle[name=XU2,linestyle=dashed]{N}}%^{U}}
%                     {\Tdot~{(N,N)}^{L}
%                      \Tdot~{(4,1)}_{R}}
% \pstree{\Tcircle[name=XD2,linestyle=dashed]{B}}
%                     {\Tdot~{(3,0)}^{L}
%                      \Tdot~{(1,1)}_{R}}}
%     \pstree{\Tcircle{B}_{$\frac{1}{2}$}}%_{Y}}
%            {\pstree{\Tcircle[name=YU2,linestyle=dashed]{N}}
%                     {\Tdot~{(1,2)}^{L}
%                     \Tdot~{(2,0)}_{R}}
%             \pstree{\Tcircle[name=YD2,linestyle=dashed]{B}}
%                     {\Tdot~{(3,3)}^{L}
%                     \Tdot~{(0,1)}_{R}}}}
% \end{center}
\vspace{4cm}
On trouve
$P({N;N})=0,3$; $P({N;B})=0,6$; $P({B;B})=0,1$. D'o� la loi de $X$.
\[
\begin{array}{|c|c|c|c|}
\hline
  x_i&0&1&2 \\ \hline
P(X=x_i)&0,1&0,6&0,3 \\
\hline
\end{array}
\]
\begin{Def} On appelle \emph{fonction de r�partition} d'une variable al�atoire $X$
  la fonction $F$ d�finie sur $R$ par:  $F(x)=P(X\leq x)$. 
\end{Def}
Cette fonction est croissante sur $\R$, varie de 0 � 1 mais n'est
\emph{a priori} pas continue. En effet dans notre cas pr�sent, $X$ ne prend qu'un nombre
fini de valeurs. En chacune de ces valeurs $x_i$, $F$ pr�sente une
discontinuit� et m�me plus pr�cis�ment un saut de hauteur $P(X=x_i)$. Entre deux
valeurs cons�cutives, $F$ est constante.
\exemple On reprend notre exemple. Pour $x<0$, $P(X<x)=0$ et
$P(X=0)=0,1$ donc $P(X\leq 0)=0,1$. Soit $F(0)=0,1$ etc \ldots

La connaissance de $F$ permet de rouver la loi de probabilit� de
$X$. Plus tard, on d�finira des variables al�atoires continues, par leur fonction de r�partition.
\subsection{Esp�rance math�matique, variance}

\begin{Def} $E(X)=\sum \ldots$
\end{Def}
Lin�arit� 

\begin{prop}
  $ E(X+Y)=E(X)+E(Y)$
 \end{prop}
Admis
\begin{prop}
$   E(aX+b)=aE(X)+b$
 \end{prop}

\begin{Def} $V(X)=E\left(\left(X-E(X)\right)^2\right)$
\end{Def}
\begin{prop}
$V(X)=E(X^2)-E(X)^2$
 \end{prop}
\subsection{Loi bin�miale}
\subsubsection{Loi de Bernoulli}
\subsubsection{R�p�tition de $n$ �preuves identiques ind�pendantes}
\section{Lois continues} %Lois de proba
\end{document}